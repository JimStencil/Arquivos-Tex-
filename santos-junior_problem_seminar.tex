\documentclass[11pt, openany]{book}
% inserted by drlf to fix ERROR: No room for a new \count .
% FIXME: Complains of unresolved references
\usepackage{etex}
\reserveinserts{28}
% 
\usepackage{sectsty, fancyhdr, amsmath, multicol,amsmath}
% \usepackage{etex}\reserveinserts
%%%%%%%%%%Assorted packages
\usepackage{makeidx}
\usepackage{hangcaption}
\usepackage[amsmath,thmmarks,standard,thref]{ntheorem}
\usepackage[colorlinks=true,linkcolor=red, pdftitle={junior problem seminar},pdfauthor={david santos},  bookmarksopen]{hyperref}
\usepackage{pdflscape}
\usepackage{thumbpdf}
\usepackage{graphicx}

%%%%%                   Postscript drawing packages
\usepackage{xkeyval, pstricks, pstricks-add}
\usepackage{pst-plot,pst-eucl}



%%%%%%%%%%%%%%CHAPTER HEADINGS
\usepackage[Lenny]{fncychap}
\usepackage{fancyhdr} %%%%for page headers and footers


%%%%%                  FONTS

\usepackage{mathptmx}% instead de package times
\usepackage[scaled=0.92]{helvet} % o  [scaled=0.92], Si you like
\usepackage{courier}
\usepackage[utf8]{inputenc}
\usepackage{t1enc}
%\renewcommand{\rmdefault}{pag}
%\renewcommand{\sfdefault}{phv}
%\renewcommand{\itdefault}{ptm}
\usepackage{pifont} %%para the dingautolists y the proofsymbol
\usepackage{mathrsfs} %%para the fancy cursive mathscr
\usepackage{amsfonts}
\usepackage{amssymb}
\usepackage{marvosym, manfnt,  yhmath,  stmaryrd}
\def\arc#1{{\wideparen{#1}}}
 \DeclareSymbolFont{EulerFraktur}{U}{euf}{m}{n}
\SetSymbolFont{EulerFraktur}{bold}{U}{euf}{b}{n}
\DeclareMathSymbol{!}\mathord  {EulerFraktur}{"21}
\DeclareMathSymbol{(}\mathopen {EulerFraktur}{"28}
\DeclareMathSymbol{)}\mathclose{EulerFraktur}{"29}
\DeclareMathSymbol{+}\mathbin  {EulerFraktur}{"2B}
\DeclareMathSymbol{-}\mathbin  {EulerFraktur}{"2D}
\DeclareMathSymbol{=}\mathrel  {EulerFraktur}{"3D}
\DeclareMathSymbol{[}\mathopen {EulerFraktur}{"5B}
\DeclareMathSymbol{]}\mathclose{EulerFraktur}{"5D}
\DeclareMathSymbol{"}\mathord  {EulerFraktur}{"7D}
\DeclareMathSymbol{&}\mathord  {EulerFraktur}{"26}
\DeclareMathSymbol{:}\mathrel  {EulerFraktur}{"3A}
\DeclareMathSymbol{;}\mathpunct{EulerFraktur}{"3B}
\DeclareMathSymbol{?}\mathord  {EulerFraktur}{"3F}
\DeclareMathSymbol{^}\mathord  {EulerFraktur}{"5E}
\DeclareMathSymbol{`}\mathord  {EulerFraktur}{"12}
\DeclareMathDelimiter{(}{EulerFraktur}{"28}{largesymbols}{"00}
\DeclareMathDelimiter{)}{EulerFraktur}{"29}{largesymbols}{"01}
\DeclareMathDelimiter{[}{EulerFraktur}{"5B}{largesymbols}{"02}
\DeclareMathDelimiter{]}{EulerFraktur}{"5D}{largesymbols}{"03}


%%%%%%%%%%PAGE FORMATTING
%\topmargin .1in \textheight 9in \oddsidemargin -.2in \evensidemargin
%-.2in \textwidth 8in

\usepackage{geometry}
 \geometry{
 a4paper,
 total={180mm,247mm},
 left=10mm,
 right=10mm,
 top=15mm,
 bottom=15mm,
 }


%%%%%FLOAT PLACEMENT

\renewcommand{\topfraction}{.85}
\renewcommand{\bottomfraction}{.7}
\renewcommand{\textfraction}{.15}
\renewcommand{\floatpagefraction}{.66}
\renewcommand{\dbltopfraction}{.66}
\renewcommand{\dblfloatpagefraction}{.66}
%\renewcommand{\rput}{\put}
\setcounter{topnumber}{9} \setcounter{bottomnumber}{9}
\setcounter{totalnumber}{20} \setcounter{dbltopnumber}{9}


%%%%%                    THEOREM-LIKE ENVIRONMENTS

\theorempreskipamount .5cm \theorempostskipamount .5cm
\theoremstyle{change} \theoremheaderfont{\blue\sffamily\bfseries}
\theorembodyfont{\normalfont} \newtheorem{thm}{Theorem}
\newtheorem{cor}[thm]{Corollary}
\newtheorem{lem}[thm]{Lemma}
\newtheorem{axi}[thm]{Axiom}
\newtheorem{rul}[thm]{Rule}
\newtheorem{exa}[thm]{Example}
\newtheorem{df}[thm]{Definition}
\theoremheaderfont{\red\sffamily\bfseries}
%\theorembodyfont{\scriptsize}
\newtheorem{pro}[thm]{Problem}
%%%%%%%%%%%Package para writing the answers at the back de the book
\usepackage{answers}
\Newassociation{answer}{Answer}{probsemans}

\newcommand{\proofsymbol}{\Pisymbol{pzd}{113}}
\theoremstyle{nonumberplain} \theoremheaderfont{\sffamily\bfseries}
\theorembodyfont{\upshape}
\newenvironment{pf}[0]{\itshape\begin{quote}{\bf Proof: \ }}{\proofsymbol\end{quote}}
\newenvironment{f-pf}[0]{\itshape\begin{quote}{\bf First Proof: \ }}{\proofsymbol\end{quote}}
\newenvironment{s-pf}[0]{\itshape\begin{quote}{\bf Second Proof: \ }}{\proofsymbol\end{quote}}
\newenvironment{t-pf}[0]{\itshape\begin{quote}{\bf Third Proof: \ }}{\proofsymbol\end{quote}}
\newenvironment{fo-pf}[0]{\itshape\begin{quote}{\bf Fourth Proof: \ }}{\proofsymbol\end{quote}}
\newenvironment{fi-pf}[0]{\itshape\begin{quote}{\bf Fifth Proof: \ }}{\proofsymbol\end{quote}}
\newenvironment{si-pf}[0]{\itshape\begin{quote}{\bf Sixth Proof: \ }}{\proofsymbol\end{quote}}

\newenvironment{rem}[0]{\begin{quote}{\huge\textcolor{red}{\Pisymbol{pzd}{43}}}\itshape }{\end{quote}}

\widowpenalty=1500



%%%%Title Page

\makeatletter
\def\thickhrulefill{\leavevmode \leaders \hrule height 1pt\hfill \kern \z@}
\renewcommand{\maketitle}{\begin{titlepage}%
    \let\footnotesize\small
    \let\footnoterule\relax
    \parindent \z@
    \reset@font
    \null\vfil
    \begin{flushleft}
     \@title
    \end{flushleft}
    \par
    \hrule height 4pt
    \par
    \begin{flushright}
    \@author \par
    \end{flushright}
    \vskip 60\p@
    \vspace*{\stretch{2}}
    \begin{center}
\Large  \today - Version
    \end{center}
  \end{titlepage}%
  \setcounter{footnote}{0}%
}


\makeatother

%%%%%%%


\setlength{\fboxrule}{1.5pt}


%%%%%%%%%%%%%%%%%INTERVALS
%%%%%%%% lo= left open, rc = right closed, etc.
\def\lcrc#1#2{\left[#1 \ ; #2 \right]}
\def\loro#1#2{ \left]#1 \ ; #2 \right[}
\def\lcro#1#2{\left[#1 \ ; #2 \right. \left[ \right.}
\def\lorc#1#2{\left. \right[#1 \ ; #2 \left.\right]}




%%%%%                Non-standard commands y symbols
\newcommand{\BBZ}{\mathbb{Z}}
\newcommand{\BBR}{\mathbb{R}}
\newcommand{\BBF}{\mathbb{F}}
\newcommand{\BBN}{\mathbb{N}}
\newcommand{\BBC}{\mathbb{C}}
\newcommand{\BBQ}{\mathbb{Q}}
\everymath{\displaystyle}
\def\signum#1{\operatorname{signum}\left( #1 \right)}
\def\absval#1{\left| #1 \right|}


\newcommand{\primes}{\mathbb{P}}
\def\binom#1#2{{#1\choose#2}}
\newcommand{\dis}{\displaystyle}
\newcommand{\í}{\'{\i}}
\def\card#1{\operatorname{card}\left(#1\right)}
\def\fun#1#2#3#4#5{\everymath{\displaystyle}{{#1} : \vspace{1cm}
\begin{array}{lll}{#4} & \rightarrow &
{#5}\\
{#2} &  \mapsto & {#3} \\
\end{array}}}

\def\fractional#1{\left\{ #1 \right\}}
\def\esp#1{\operatorname{Sp}\left(#1\right)}
\def\floor#1{\llfloor #1 \rrfloor}
\def\ceil#1{\llceil #1 \rrceil}
%%%%%%                  And voilà the document !!!
\makeindex

\begin{document}
\Opensolutionfile{probsemans}[probsemans1]

\pagenumbering{roman}
\title{\Large Junior Problem Seminar}
\author{Dr. David A. SANTOS}

\renewcommand{\headrulewidth}{1pt}
\renewcommand{\footrulewidth}{1pt}
\lhead{\nouppercase{\textcolor{blue}{\rightmark}}}
\rhead{\nouppercase{\textcolor{blue}{\leftmark}}}
\lhead[\rm\thepage]{\textcolor{blue}{\it \rightmark}} \rhead[\it
\textcolor{blue}{\leftmark}]{\rm\thepage} \cfoot{}
\thispagestyle{empty} \pagestyle{fancy}


% Front matter may follow here
\begin{frontmatter}
 \maketitle
\tableofcontents
 \end{frontmatter}

\chapter*{Preface}
\markboth{}{} \addcontentsline{toc}{chapter}{Preface}
\markright{Preface} From time to time I get to revise this problem
seminar. Although my chances of addressing the type of students for
which they were originally intended (middle-school, high-school) are
now very remote, I have had very pleasant emails from people around
the world finding this material useful.

\bigskip

I haven't compiled the solutions for the practice problems anywhere.
This is a project that now, having more pressing things to do, I
can't embark. But if you feel you can contribute to these notes,
drop me a line, or even mail me your files!

\hfill \begin{tabular}{l}David A. SANTOS \\
%\href{mailto:dsantos@ccp.edu}
{dsantos@ccp.edu}
\end{tabular}

\bigskip

Throughout the years I have profitted from emails of people who
commend me on the notes, point out typos and errors, etc. Here is
(perhaps incomplete) list of them, in the order in which I have
received emails.
\begin{multicols}{3}
\begin{itemize}
\item Dr. Gerd Schlechtriemen
\item Daniel Wu
\item Young-Soo Lee
\item Rohan Kulkarni
\item Ram Prasad
\item Edward Moy
\item Steve Hoffman
\item Yiwen Yu
\item Tam King Wa
\item Ramji Gannavarapu
\item Jes\'{u}s Benede Garc\'{e}s
\item Linda Scholes
\item Wenceslao Calleja Rodr\'{\i}guez
\item Philip Pennance
\item David Ontaneda
\item Richard A. Smith
\item Danilo Gortaire Jativa
\item Kurt Byron Ang
\item Bharat Narumanchi
\item Chase Hallstrom
\item Cristhian Gonzalo
\item Hikmet Erdogan
\item Maxy Mariasegaram
\item Professor Dennis Guzm\'{a}n
\item Mingjie Zhou
\item Faruk Uygul
\item Marvin D. Hern\'{a}ndez
\item Marco Grassi
\end{itemize}
\end{multicols}


\clearpage



\begin{quote}
    Copyright \copyright{}  2007  David Anthony SANTOS.
    Permission is granted to copy, distribute and/or modify this document
    under the terms of the GNU Free Documentation License, Version 1.2
    or any later version published by the Free Software Foundation;
    with no Invariant Sections, no Front-Cover Texts, and no Back-Cover Texts.
    A copy of the license is included in the section entitled ``GNU
    Free Documentation License''.
\end{quote}
\clearpage
\chapter*{GNU Free Documentation License}
{\tiny
 \begin{center}

       Version 1.2, November 2002


 Copyright \copyright{} 2000,2001,2002  Free Software Foundation, Inc.

 \bigskip

     51 Franklin St, Fifth Floor, Boston, MA  02110-1301  USA

 \bigskip

 Everyone is permitted to copy and distribute verbatim copies
 of this license document, but changing it is not allowed.
\end{center}


\begin{center}
{\bf\large Preamble}
\end{center}

The purpose of this License is to make a manual, textbook, or other
functional and useful document ``free'' in the sense of freedom: to
assure everyone the effective freedom to copy and redistribute it,
with or without modifying it, either commercially or
noncommercially. Secondarily, this License preserves for the author
and publisher a way to get credit for their work, while not being
considered responsible for modifications made by others.

This License is a kind of ``copyleft'', which means that derivative
works of the document must themselves be free in the same sense.  It
complements the GNU General Public License, which is a copyleft
license designed for free software.

We have designed this License in order to use it for manuals for
free software, because free software needs free documentation: a
free program should come with manuals providing the same freedoms
that the software does.  But this License is not limited to software
manuals; it can be used for any textual work, regardless of subject
matter or whether it is published as a printed book.  We recommend
this License principally for works whose purpose is instruction or
reference.


\begin{center}
{\Large\bf 1. APPLICABILITY AND DEFINITIONS\par} \phantomsection
\addcontentsline{toc}{section}{1. APPLICABILITY AND DEFINITIONS}
\end{center}

This License applies to any manual or other work, in any medium,
that contains a notice placed by the copyright holder saying it can
be distributed under the terms of this License.  Such a notice
grants a world-wide, royalty-free license, unlimited in duration, to
use that work under the conditions stated herein.  The
``\textbf{Document}'', below, refers to any such manual or work. Any
member of the public is a licensee, and is addressed as
``\textbf{you}''.  You accept the license if you copy, modify or
distribute the work in a way requiring permission under copyright
law.

A ``\textbf{Modified Version}'' of the Document means any work
containing the Document or a portion of it, either copied verbatim,
or with modifications and/or translated into another language.

A ``\textbf{Secondary Section}'' is a named appendix or a
front-matter section of the Document that deals exclusively with the
relationship of the publishers or authors of the Document to the
Document's overall subject (or to related matters) and contains
nothing that could fall directly within that overall subject. (Thus,
if the Document is in part a textbook of mathematics, a Secondary
Section may not explain any mathematics.)  The relationship could be
a matter of historical connection with the subject or with related
matters, or of legal, commercial, philosophical, ethical or
political position regarding them.

The ``\textbf{Invariant Sections}'' are certain Secondary Sections
whose titles are designated, as being those of Invariant Sections,
in the notice that says that the Document is released under this
License.  If a section does not fit the above definition of
Secondary then it is not allowed to be designated as Invariant.  The
Document may contain zero Invariant Sections.  If the Document does
not identify any Invariant Sections then there are none.

The ``\textbf{Cover Texts}'' are certain short passages of text that
are listed, as Front-Cover Texts or Back-Cover Texts, in the notice
that says that the Document is released under this License.  A
Front-Cover Text may be at most 5 words, and a Back-Cover Text may
be at most 25 words.

A ``\textbf{Transparent}'' copy of the Document means a
machine-readable copy, represented in a format whose specification
is available to the general public, that is suitable for revising
the document straightforwardly with generic text editors or (for
images composed of pixels) generic paint programs or (for drawings)
some widely available drawing editor, and that is suitable for input
to text formatters or for automatic translation to a variety of
formats suitable for input to text formatters.  A copy made in an
otherwise Transparent file format whose markup, or absence of
markup, has been arranged to thwart or discourage subsequent
modification by readers is not Transparent. An image format is not
Transparent if used for any substantial amount of text.  A copy that
is not ``Transparent'' is called ``\textbf{Opaque}''.

Examples of suitable formats for Transparent copies include plain
ASCII without markup, Texinfo input format, LaTeX input format, SGML
or XML using a publicly available DTD, and standard-conforming
simple HTML, PostScript or PDF designed for human modification.
Examples of transparent image formats include PNG, XCF and JPG.
Opaque formats include proprietary formats that can be read and
edited only by proprietary word processors, SGML or XML for which
the DTD and/or processing tools are not generally available, and the
machine-generated HTML, PostScript or PDF produced by some word
processors for output purposes only.

The ``\textbf{Title Page}'' means, for a printed book, the title
page itself, plus such following pages as are needed to hold,
legibly, the material this License requires to appear in the title
page.  For works in formats which do not have any title page as
such, ``Title Page'' means the text near the most prominent
appearance of the work's title, preceding the beginning of the body
of the text.

A section ``\textbf{Entitled XYZ}'' means a named subunit of the
Document whose title either is precisely XYZ or contains XYZ in
parentheses following text that translates XYZ in another language.
(Here XYZ stands for a specific section name mentioned below, such
as ``\textbf{Acknowledgements}'', ``\textbf{Dedications}'',
``\textbf{Endorsements}'', or ``\textbf{History}''.) To
``\textbf{Preserve the Title}'' of such a section when you modify
the Document means that it remains a section ``Entitled XYZ''
according to this definition.

The Document may include Warranty Disclaimers next to the notice
which states that this License applies to the Document.  These
Warranty Disclaimers are considered to be included by reference in
this License, but only as regards disclaiming warranties: any other
implication that these Warranty Disclaimers may have is void and has
no effect on the meaning of this License.


\begin{center}
{\Large\bf 2. VERBATIM COPYING\par} \phantomsection
\addcontentsline{toc}{section}{2. VERBATIM COPYING}
\end{center}

You may copy and distribute the Document in any medium, either
commercially or noncommercially, provided that this License, the
copyright notices, and the license notice saying this License
applies to the Document are reproduced in all copies, and that you
add no other conditions whatsoever to those of this License.  You
may not use technical measures to obstruct or control the reading or
further copying of the copies you make or distribute.  However, you
may accept compensation in exchange for copies.  If you distribute a
large enough number of copies you must also follow the conditions in
section~3.

You may also lend copies, under the same conditions stated above,
and you may publicly display copies.


\begin{center}
{\Large\bf 3. COPYING IN QUANTITY\par} \phantomsection
\addcontentsline{toc}{section}{3. COPYING IN QUANTITY}
\end{center}


If you publish printed copies (or copies in media that commonly have
printed covers) of the Document, numbering more than 100, and the
Document's license notice requires Cover Texts, you must enclose the
copies in covers that carry, clearly and legibly, all these Cover
Texts: Front-Cover Texts on the front cover, and Back-Cover Texts on
the back cover.  Both covers must also clearly and legibly identify
you as the publisher of these copies.  The front cover must present
the full title with all words of the title equally prominent and
visible.  You may add other material on the covers in addition.
Copying with changes limited to the covers, as long as they preserve
the title of the Document and satisfy these conditions, can be
treated as verbatim copying in other respects.

If the required texts for either cover are too voluminous to fit
legibly, you should put the first ones listed (as many as fit
reasonably) on the actual cover, and continue the rest onto adjacent
pages.

If you publish or distribute Opaque copies of the Document numbering
more than 100, you must either include a machine-readable
Transparent copy along with each Opaque copy, or state in or with
each Opaque copy a computer-network location from which the general
network-using public has access to download using public-standard
network protocols a complete Transparent copy of the Document, free
of added material. If you use the latter option, you must take
reasonably prudent steps, when you begin distribution of Opaque
copies in quantity, to ensure that this Transparent copy will remain
thus accessible at the stated location until at least one year after
the last time you distribute an Opaque copy (directly or through
your agents or retailers) of that edition to the public.

It is requested, but not required, that you contact the authors of
the Document well before redistributing any large number of copies,
to give them a chance to provide you with an updated version of the
Document.


\begin{center}
{\Large\bf 4. MODIFICATIONS\par} \phantomsection
\addcontentsline{toc}{section}{4. MODIFICATIONS}
\end{center}

You may copy and distribute a Modified Version of the Document under
the conditions of sections 2 and 3 above, provided that you release
the Modified Version under precisely this License, with the Modified
Version filling the role of the Document, thus licensing
distribution and modification of the Modified Version to whoever
possesses a copy of it.  In addition, you must do these things in
the Modified Version:

\begin{itemize}
\item[A.]
   Use in the Title Page (and on the covers, if any) a title distinct
   from that of the Document, and from those of previous versions
   (which should, if there were any, be listed in the History section
   of the Document).  You may use the same title as a previous version
   if the original publisher of that version gives permission.

\item[B.]
   List on the Title Page, as authors, one or more persons or entities
   responsible for authorship of the modifications in the Modified
   Version, together with at least five of the principal authors of the
   Document (all of its principal authors, if it has fewer than five),
   unless they release you from this requirement.

\item[C.]
   State on the Title page the name of the publisher of the
   Modified Version, as the publisher.

\item[D.]
   Preserve all the copyright notices of the Document.

\item[E.]
   Add an appropriate copyright notice for your modifications
   adjacent to the other copyright notices.

\item[F.]
   Include, immediately after the copyright notices, a license notice
   giving the public permission to use the Modified Version under the
   terms of this License, in the form shown in the Addendum below.

\item[G.]
   Preserve in that license notice the full lists of Invariant Sections
   and required Cover Texts given in the Document's license notice.

\item[H.]
   Include an unaltered copy of this License.

\item[I.]
   Preserve the section Entitled ``History'', Preserve its Title, and add
   to it an item stating at least the title, year, new authors, and
   publisher of the Modified Version as given on the Title Page.  If
   there is no section Entitled ``History'' in the Document, create one
   stating the title, year, authors, and publisher of the Document as
   given on its Title Page, then add an item describing the Modified
   Version as stated in the previous sentence.

\item[J.]
   Preserve the network location, if any, given in the Document for
   public access to a Transparent copy of the Document, and likewise
   the network locations given in the Document for previous versions
   it was based on.  These may be placed in the ``History'' section.
   You may omit a network location for a work that was published at
   least four years before the Document itself, or if the original
   publisher of the version it refers to gives permission.

\item[K.]
   For any section Entitled ``Acknowledgements'' or ``Dedications'',
   Preserve the Title of the section, and preserve in the section all
   the substance and tone of each of the contributor acknowledgements
   and/or dedications given therein.

\item[L.]
   Preserve all the Invariant Sections of the Document,
   unaltered in their text and in their titles.  Section numbers
   or the equivalent are not considered part of the section titles.

\item[M.]
   Delete any section Entitled ``Endorsements''.  Such a section
   may not be included in the Modified Version.

\item[N.]
   Do not retitle any existing section to be Entitled ``Endorsements''
   or to conflict in title with any Invariant Section.

\item[O.]
   Preserve any Warranty Disclaimers.
\end{itemize}

If the Modified Version includes new front-matter sections or
appendices that qualify as Secondary Sections and contain no
material copied from the Document, you may at your option designate
some or all of these sections as invariant.  To do this, add their
titles to the list of Invariant Sections in the Modified Version's
license notice. These titles must be distinct from any other section
titles.

You may add a section Entitled ``Endorsements'', provided it
contains nothing but endorsements of your Modified Version by
various parties--for example, statements of peer review or that the
text has been approved by an organization as the authoritative
definition of a standard.

You may add a passage of up to five words as a Front-Cover Text, and
a passage of up to 25 words as a Back-Cover Text, to the end of the
list of Cover Texts in the Modified Version.  Only one passage of
Front-Cover Text and one of Back-Cover Text may be added by (or
through arrangements made by) any one entity.  If the Document
already includes a cover text for the same cover, previously added
by you or by arrangement made by the same entity you are acting on
behalf of, you may not add another; but you may replace the old one,
on explicit permission from the previous publisher that added the
old one.

The author(s) and publisher(s) of the Document do not by this
License give permission to use their names for publicity for or to
assert or imply endorsement of any Modified Version.


\begin{center}
{\Large\bf 5. COMBINING DOCUMENTS\par} \phantomsection
\addcontentsline{toc}{section}{5. COMBINING DOCUMENTS}
\end{center}


You may combine the Document with other documents released under
this License, under the terms defined in section~4 above for
modified versions, provided that you include in the combination all
of the Invariant Sections of all of the original documents,
unmodified, and list them all as Invariant Sections of your combined
work in its license notice, and that you preserve all their Warranty
Disclaimers.

The combined work need only contain one copy of this License, and
multiple identical Invariant Sections may be replaced with a single
copy.  If there are multiple Invariant Sections with the same name
but different contents, make the title of each such section unique
by adding at the end of it, in parentheses, the name of the original
author or publisher of that section if known, or else a unique
number. Make the same adjustment to the section titles in the list
of Invariant Sections in the license notice of the combined work.

In the combination, you must combine any sections Entitled
``History'' in the various original documents, forming one section
Entitled ``History''; likewise combine any sections Entitled
``Acknowledgements'', and any sections Entitled ``Dedications''. You
must delete all sections Entitled ``Endorsements''.

\begin{center}
{\Large\bf 6. COLLECTIONS OF DOCUMENTS\par} \phantomsection
\addcontentsline{toc}{section}{6. COLLECTIONS OF DOCUMENTS}
\end{center}

You may make a collection consisting of the Document and other
documents released under this License, and replace the individual
copies of this License in the various documents with a single copy
that is included in the collection, provided that you follow the
rules of this License for verbatim copying of each of the documents
in all other respects.

You may extract a single document from such a collection, and
distribute it individually under this License, provided you insert a
copy of this License into the extracted document, and follow this
License in all other respects regarding verbatim copying of that
document.


\begin{center}
{\Large\bf 7. AGGREGATION WITH INDEPENDENT WORKS\par}
\phantomsection \addcontentsline{toc}{section}{7. AGGREGATION WITH
INDEPENDENT WORKS}
\end{center}


A compilation of the Document or its derivatives with other separate
and independent documents or works, in or on a volume of a storage
or distribution medium, is called an ``aggregate'' if the copyright
resulting from the compilation is not used to limit the legal rights
of the compilation's users beyond what the individual works permit.
When the Document is included in an aggregate, this License does not
apply to the other works in the aggregate which are not themselves
derivative works of the Document.

If the Cover Text requirement of section~3 is applicable to these
copies of the Document, then if the Document is less than one half
of the entire aggregate, the Document's Cover Texts may be placed on
covers that bracket the Document within the aggregate, or the
electronic equivalent of covers if the Document is in electronic
form. Otherwise they must appear on printed covers that bracket the
whole aggregate.


\begin{center}
{\Large\bf 8. TRANSLATION\par} \phantomsection
\addcontentsline{toc}{section}{8. TRANSLATION}
\end{center}


Translation is considered a kind of modification, so you may
distribute translations of the Document under the terms of
section~4. Replacing Invariant Sections with translations requires
special permission from their copyright holders, but you may include
translations of some or all Invariant Sections in addition to the
original versions of these Invariant Sections.  You may include a
translation of this License, and all the license notices in the
Document, and any Warranty Disclaimers, provided that you also
include the original English version of this License and the
original versions of those notices and disclaimers.  In case of a
disagreement between the translation and the original version of
this License or a notice or disclaimer, the original version will
prevail.

If a section in the Document is Entitled ``Acknowledgements'',
``Dedications'', or ``History'', the requirement (section~4) to
Preserve its Title (section~1) will typically require changing the
actual title.


\begin{center}
{\Large\bf 9. TERMINATION\par} \phantomsection
\addcontentsline{toc}{section}{9. TERMINATION}
\end{center}


You may not copy, modify, sublicense, or distribute the Document
except as expressly provided for under this License.  Any other
attempt to copy, modify, sublicense or distribute the Document is
void, and will automatically terminate your rights under this
License.  However, parties who have received copies, or rights, from
you under this License will not have their licenses terminated so
long as such parties remain in full compliance.


\begin{center}
{\Large\bf 10. FUTURE REVISIONS OF THIS LICENSE\par} \phantomsection
\addcontentsline{toc}{section}{10. FUTURE REVISIONS OF THIS LICENSE}
\end{center}


The Free Software Foundation may publish new, revised versions of
the GNU Free Documentation License from time to time.  Such new
versions will be similar in spirit to the present version, but may
differ in detail to address new problems or concerns.  See
http://www.gnu.org/copyleft/.

Each version of the License is given a distinguishing version
number. If the Document specifies that a particular numbered version
of this License ``or any later version'' applies to it, you have the
option of following the terms and conditions either of that
specified version or of any later version that has been published
(not as a draft) by the Free Software Foundation.  If the Document
does not specify a version number of this License, you may choose
any version ever published (not as a draft) by the Free Software
Foundation. }






 \clearpage




\renewcommand{\chaptermark}{\markboth{\chaptername\ \thechapter}}
\renewcommand{\sectionmark}{\markright}
\chapter{Essential
Techniques}\section{Reductio ad Absurdum}
\pagenumbering{arabic}\pagestyle{fancy} \setcounter{page}{1}

In this section we will see examples of proofs by contradiction.
That is, in trying to prove a premise, we assume that its negation
is true and deduce incompatible statements from this.

\begin{exa}
Shew, without using a calculator, that $\dis{6 - \sqrt{35} <
\frac{1}{10}}$.
\end{exa}
Solution: Assume that  $\dis{6 - \sqrt{35} \geq \frac{1}{10}}$.
Then $\dis{6 - \frac{1}{10} \geq \sqrt{35}}$ or $59 \geq
10\sqrt{35}$. Squaring both sides we obtain $3481 \geq 3500$,
which is clearly nonsense. Thus it must be the case that $\dis{6 -
\sqrt{35} < \frac{1}{10}}$.
\begin{exa}
Let $a_1, a_2, \ldots , a_n$ be an arbitrary permutation of the
numbers $1, 2, \ldots , n, $ where $n$ is an odd number. Prove
that the product
$$(a_1 - 1)(a_2 - 2)\cdots (a_n - n)$$ is even.
\end{exa}
Solution: First observe that the sum of an odd number of odd
integers is odd. It is enough to prove  that some difference $a_k
- k$ is even. Assume contrariwise that all the differences $a_k -
k$ are odd. Clearly
$$S = (a_1 - 1) + (a_2 - 2) + \cdots + (a_n - n) = 0,$$ since the $a_k$'s are a
reordering of $1, 2, \ldots , n$. $S$ is an odd  number of
summands of odd integers adding to the even integer 0. This is
impossible. Our initial assumption that all the $a_k - k$ are odd
is wrong, so one of these is even and hence the product is even.

\begin{exa}
Prove that $\sqrt{2}$ is irrational.
\end{exa}
Solution: For this proof, we will accept as fact that any positive
integer greater than 1 can be factorised uniquely as the product
of primes (up to the order of the factors).


Assume that $\dis{\sqrt{2} = \frac{a}{b}},$ with positive integers
$a, b$. This yields $2b^2 = a^2.$ Now both $a^2$ and $b^2$ have an
even number of prime factors. So $2b^2$ has an odd numbers of
primes in its factorisation and $a^2$ has an even number of primes
in its factorisation. This is a contradiction.

\begin{exa}
Let $a, b$ be real numbers and assume that for all numbers
$\epsilon > 0$ the following inequality holds:
$$ a < b + \epsilon .$$ Prove that $a \leq b.$
\end{exa}
Solution: Assume contrariwise that $a > b.$ Hence $\dis{\frac{a -
b}{2} > 0}$. Since the inequality $a < b + \epsilon$ holds for
every $\epsilon > 0$ in particular it holds for $\dis{\epsilon =
\frac{a - b}{2}}$. This implies that
$$a < b + \frac{a - b}{2} \ \ {\rm or}  \ \  a < b.$$Thus starting with the assumption that
$a > b$ we reach the incompatible conclusion that $a < b.$ The
original assumption must be wrong. We therefore conclude that $a
\leq b.$
\begin{exa}[Euclid]
Shew that there are infinitely many prime numbers.
\label{exa:inf_primes}\end{exa} Solution:  We need to assume for
this proof that any integer greater than 1 is either a prime or a
product of primes. The following beautiful proof goes back to
Euclid.
\bigskip
    Assume that $\{p_1, p_2, \ldots , p_n\}$ is a list that exhausts all the primes. Consider the number
    $$N = p_1p_2\cdots p_n + 1.$$ This is a positive integer, clearly greater than 1. Observe that none of
    the primes on the list $\{p_1, p_2, \ldots , p_n\}$ divides $N$, since division by any of these primes
    leaves a remainder of 1. Since $N$ is larger than any of the primes on this list, it is either a prime
    or divisible by a prime outside this list. Thus we have shewn that the assumption that any finite list
    of primes leads to the existence of a prime outside this list. This implies that the number of primes
is infinite.
\begin{exa}
Let $n > 1$ be a composite integer. Prove that $n$ has a prime
factor $p \leq \sqrt{n}.$ \label{exa:root_primality}\end{exa}
Solution: Since $n$ is composite, $n$ can be written as $n = ab$
where both $a
> 1, b
> 1$ are integers. Now, if both $a > \sqrt{n}$ and $b > \sqrt{n}$
then $n = ab > \sqrt{n}\sqrt{n} = n$, a contradiction. Thus one of
these factors must be $\leq \sqrt{n}$ and {\em a fortiori} it must
have a prime factor $\leq \sqrt{n}$.


\bigskip

The result in example \ref{exa:root_primality} can be used to test
for primality. For example, to shew that $101$ is prime, we compute
$\llfloor \sqrt{101} \rrfloor = 10$. By the preceding problem,
either $101$ is prime or it is divisible by $2, 3, 5$, or $7$ (the
primes smaller than $10$). Since neither of these primes divides
$101$, we conclude that $101$ is prime.

\begin{exa}\label{exa:squares_mod_4}
 Prove that a sum of two
squares of integers leaves remainder $0$, $1$ or $2$ when divided
by $4$.
\end{exa}
Solution: An integer is either even (of the form $2k$) or odd (of
the form $2k + 1$). We have
$$\begin{array}{lll}(2k)^2 & = & 4(k^2), \\
(2k + 1)^2 & = & 4(k^2 + k) + 1.
  \end{array}$$Thus squares leave remainder $0$ or $1$ when
  divided by $4$ and hence their sum leave remainder $0$, $1$, or
  $2$.

\begin{exa}
Prove that $2003$ is not the sum of two squares by proving that
the sum of any two squares cannot leave remainder $3$ upon
division by $4$.
\end{exa}Solution: $2003$ leaves remainder $3$ upon division by
$4$. But we know from example \ref{exa:squares_mod_4} that sums of
squares do not leave remainder $3$ upon division by $4$, so it is
impossible to write $2003$ as the sum of squares.
\begin{exa}
If $a, b, c$ are odd integers, prove that $ax^2 + bx + c = 0$ does
not have a rational number solution.
\end{exa}Solution: Suppose $\dfrac{p}{q}$ is a rational solution
to the equation. We may assume that $p$ and $q$ have no prime
factors in common, so either $p$ and $q$ are both odd, or one is
odd and the other even. Now
$$a\left(\dfrac{p}{q}\right)^2 + b\left(\dfrac{p}{q}\right) + c = 0 \implies ap^2 + bpq + cq^2 = 0.   $$
If both $p$ and $p$ were odd, then $ap^2 + bpq + cq^2 $ is also
odd and hence $\neq 0$. Similarly if one of them is even and the
other odd then either $ap^2 + bpq$ or $bpq + cq^2 $ is even and
$ap^2 + bpq + cq^2 $ is odd. This contradiction proves that the
equation cannot have a rational root.


\section*{Practice}\addcontentsline{toc}{section}{Practice}
\begin{pro}
The product of $34$ integers is equal to $1$. Shew that their sum
cannot be $0$.
\begin{answer}
Since their product is $1$ the integers must be $\pm 1$ and there
must be an even number of $-1$'s, say $k$ of them. To make the sum
of the numbers $0$ we must have the same number of $1$'s. Thus we
must have $k(1) + k(-1) = 0$, and $k+k = 34$, which means that
$k=17$, which is not even.
\end{answer}

\end{pro}
\begin{pro}
Let $a_1, a_2, \ldots , a_{2000}$ be natural numbers such that
$$\frac{1}{a_1} + \frac{1}{a_2} + \cdots + \frac{1}{a_{2000}} =
1.$$ Prove that at least one of the $a_k$'s is even.
\begin{answer}
Clearing denominators, there are $2000$ summands on the sinistral
side of the form $a_1a_2\cdots a_{i-1}a_{i+1}\cdots a_{2000}$, and
the dextral side we simply have $a_1a_2\cdots a_{i-1}a_{i+1}\cdots
a_{2000}$. If all the $a_k$ were odd, the right hand side would be
odd, and the left hand side would be even, being the sum of $2000$
odd numbers, a contradiction.
\end{answer}
\end{pro}(Hint: Clear  the denominators.)
\begin{pro}
Prove that $\log _2 3$ is irrational.
\begin{answer}
If $log_23 = \dfrac{a}{b}$, with integral $a, b\neq 0$ then $2^a =
3^b$. By uniqueness of factorisation this is impossible unless $a =
b = 0$, which is not an allowed alternative.
\end{answer}

\end{pro}
\begin{pro}
A {\em palindrome} is an integer whose decimal expansion is
symmetric, e.g. $1, 2, 11, 121,$ $15677651$ (but not $010, 0110$)
are palindromes. Prove that there is no positive palindrome which
is divisible by $10.$
\begin{answer}
If the palindrome were divisible by $10$ then it would end in $0$,
and hence, by definition of being a palindrome, it would start in
$0$, which is not allowed in the definition.
\end{answer}
\end{pro}
\begin{pro}
In $\triangle ABC$, $\angle A > \angle B.$ Prove that $BC > AC.$
\begin{answer}
Assume $AC \geq BC$ and locate point $D$ on  the line segment $AC$
such that $AD = BD$. Then $\triangle ADB$ is isosceles at $D$ and we
must have $\angle A = \angle B$, a contradiction.
\end{answer}
\end{pro}
\begin{pro}
Let $0 < \alpha < 1$. Prove that $\sqrt{\alpha} > \alpha$.
\begin{answer}
If $\sqrt{a} \leq \alpha$ then $\alpha \leq \alpha^2$, which implies
that $\alpha (1 - \alpha) \leq 0$, an impossible inequality if $0 <
\alpha < 1$.
\end{answer}
\end{pro}
\begin{pro}
Let $\alpha = 0.999\ldots$ where there are at least $2000$ nines.
Prove that the decimal expansion of $\sqrt{\alpha}$ also starts
with at least $2000$ nines.
\begin{answer}
We have $1 - \dfrac{1}{10^{2000}}<\alpha <1$. Squaring, $$1 -
\dfrac{2}{10^{2000}} + \dfrac{1}{10^{4000}} < \alpha^2. $$Since $-
\dfrac{1}{10^{2000}} + \dfrac{1}{10^{4000}< 0},$ we have $$1 -
\dfrac{1}{10^{2000}} < 1 - \dfrac{1}{10^{2000}} -
\dfrac{1}{10^{2000}} + \dfrac{1}{10^{4000}} < \alpha^2.$$
\end{answer}
\end{pro}
\begin{pro}
Prove that a quadratic equation $$ax^2 + bx + c = 0, \ a \neq  0$$
has at most two solutions.
\end{pro}
\begin{pro}
Prove that if $ax^2 + bx + c = 0$ has real solutions and if $a >
0, b > 0, c > 0$ then both solutions must be negative.
\end{pro}

\section{Pigeonhole Principle} The Pigeonhole
Principle\index{Pigeonhole Principle} states that if $n + 1$
pigeons fly to $n$ holes, there must be a pigeonhole containing at
least two pigeons. This apparently trivial principle is very
powerful. Thus in any group of 13 people, there are always two who
have their birthday on the same month, and if the average human
head has two million hairs, there are at least three people in NYC
with the same number of hairs on their head.



The Pigeonhole Principle is useful in proving {\em existence}
problems, that is, we shew that something exists without actually
identifying it concretely.



Let us see some more examples.

\begin{exa}[Putnam 1978] Let $A$ be any set of twenty integers chosen
from the arithmetic progression $1, 4, \ldots ,100.$ Prove that
there must be two distinct integers in $A$ whose sum is $104$.
\end{exa}
Solution: We partition the thirty four elements of this
progression into nineteen groups $$\{1 \} , \{ 52 \} , \{ 4, 100\}
, \{ 7, 97\} , \{ 10, 94\} , \ldots , \{ 49, 55\}.$$ Since we are
choosing twenty integers and we have nineteen sets, by the
Pigeonhole Principle there must be two integers that belong to one
of the pairs, which add to $104$.
\begin{exa} Shew that amongst any seven distinct positive integers
not exceeding $126$, one can find two of them, say $a$ and $b$,
which satisfy $$ b < a \leq 2b.$$\end{exa} Solution: Split the
numbers $\{ 1, 2, 3, \ldots , 126 \}$ into the six sets $$\{ 1,
2\} , \{ 3, 4, 5, 6\} , \{ 7, 8, \ldots , 13, 14\} , \{ 15, 16,
\ldots , 29, 30\}, $$ $$\{ 31, 32, \ldots , 61, 62\} \ {\rm and} \
\{ 63, 64, \ldots ,126\} .$$By the Pigeonhole Principle, two of
the seven numbers must lie in one of the six sets, and obviously,
any such two will satisfy the stated inequality.

\begin{exa} No matter which fifty five integers may be selected from $$\{ 1, 2, \ldots , 100\},$$
prove that one must select some two that differ by $10$. \end{exa}
Solution: First observe that if we choose $n + 1$ integers from
any string of $2n$ consecutive integers, there will always be some
two that differ by $n$. This is because we can pair the $2n$
consecutive integers
$$ \{ a + 1, a + 2, a + 3, \ldots , a + 2n\}$$ into the $n$ pairs
$$ \{ a + 1, a + n + 1\}, \{ a + 2, a + n + 2\}, \ldots , \{ a + n, a + 2n\},$$and
if $n + 1$ integers are chosen from this, there must be two that
belong to the same group.


So now group the one hundred integers as follows:
$$ \{ 1, 2, \ldots 20 \} , \{ 21, 22, \ldots , 40\} ,$$  $$ \{ 41, 42, \ldots , 60\}, \
 \{ 61, 62, \ldots , 80\} $$
and $$ \{ 81, 82, \ldots , 100\} .$$ If we select fifty five
integers, we must perforce choose eleven from some group. From
that group, by the above observation (let $n = 10$), there must be
two that differ by 10.

\begin{exa}[AHSME 1994] Label one disc ``${\bf 1}$'', two discs
``${\bf 2}$'', three discs ``${\bf 3}$'', \ldots , fifty discs
$``{\bf 50}$''. Put these $1 + 2 + 3 + \cdots + 50 = 1275$ labeled
discs in a box. Discs are then drawn from the box at random
without replacement. What is the minimum number of discs that must
me drawn in order to guarantee drawing at least ten discs with the
same label? \end{exa} Solution: If we draw all the $1 + 2 + \cdots
+ 9 = 45$ labelled ``${\bf 1}$'', \ldots , ``${\bf 9}$'' and any
nine from each of the discs ``${\bf 10}$'', \ldots , ``${\bf
50}$'', we have drawn $45 + 9\cdot 41 = 414$ discs. The 415-th
disc drawn will assure at least ten discs from a label.
\begin{exa}[IMO 1964] Seventeen people correspond by mail with
one another---each one with all the rest. In their letters only
three different topics are discussed. Each pair of correspondents
deals with only one of these topics. Prove that there at least
three people who write to each other about the same topic.
\end{exa} Solution: Choose a particular person of the group, say
Charlie. He corresponds with sixteen others. By the Pigeonhole
Principle, Charlie must write to at least six of the people of one
topic, say topic I. If any pair of these six people corresponds on
topic I, then Charlie and this pair do the trick, and we are done.
Otherwise, these six correspond amongst themselves only on topics
II or III. Choose a particular person from this group of six, say
Eric. By the Pigeonhole Principle, there must be three of the five
remaining that correspond with Eric in one of the topics, say
topic II. If amongst these three there is a pair that corresponds
with each other on topic II, then Eric and this pair correspond on
topic II, and we are done. Otherwise, these three people only
correspond with one another on topic III, and we are done again.

\begin{exa} Given any set of ten natural numbers between $1$ and $99$
inclusive, prove that there are two disjoint nonempty subsets of the
set with equal sums of their elements. \end{exa} Solution: There are
$2^{10} - 1 = 1023$ non-empty subsets that one can form with a given
10-element set. To each of these subsets we associate the sum of its
elements. The maximum value that any such sum can achieve is $90 +
91 + \cdots + 99 = 945 < 1023.$ Therefore, there must be at least
two different subsets $S, T$ that have the same element sum. Then $S
\setminus (S \cap T)$ and $T \setminus (S \cap T)$ also have the
same element sum.
\begin{exa}
Given any $9$ integers whose prime factors lie in the set $\{3, 7,
11 \}$ prove that there must be two whose product is a square.
\end{exa}
Solution: For an integer to be a square, all the exponents of its
prime factorisation must be even. Any integer in the given set has
a prime factorisation of the form $3^a7^b11^c$. Now each triplet
$(a, b, c)$ has one of the following 8 parity patterns: (even,
even, even), (even, even, odd), (even, odd, even), (even, odd,
odd), (odd, even, even), (odd, even, odd), (odd, odd, even), (odd,
odd, odd). In a group of 9 such integers, there must be two with
the same parity patterns in the exponents. Take these two. Their
product is a square, since the sum of each corresponding exponent
will be even.
\section*{Practice}\addcontentsline{toc}{section}{Practice}\markright{Practice}\begin{multicols}{2}\columnseprule 1pt \columnsep 25pt\multicoltolerance=900
\begin{pro}
Prove that among $n+1$ integers, there are always two whose
difference is always divisible by $n$.
\begin{answer}
There are $n$ possible different remainders when an integer is
divided by $n$, so among $n+1$ different integers there must be two
integers in the group leaving the same remainder,  and their
difference is divisible by $n$.
\end{answer}
\end{pro}
\begin{pro}[AHSME 1991] A circular table has exactly sixty chairs
around it. There are $N$ people seated at this table in such a way
that the next person to be seated must sit next to someone. What
is the smallest possible value of $N$?
\begin{answer}$20$
\end{answer}
\end{pro}
\begin{pro} Shew that if any five points are all in, or on, a square
of side $1$, then some pair of them will be at most at distance
$\sqrt{2}/2.$\end{pro} \begin{pro}[Hungarian Math Olympiad, 1947]
Prove that amongst six people in a room there are at least three
who know one another, or at least three who do not know one
another.\end{pro}
\begin{pro} Shew that in any sum of nonnegative real numbers there is always
one number which is at least the average of the numbers and that
there is always one member that it is at most the average of the
numbers.\end{pro}
\begin{pro} We call a set ``sum free'' if no two elements of the set add up
to a third element of the set. What is the maximum size of a sum
free subset of $\{ 1, 2, \ldots , 2n - 1\}$.\end{pro}Hint: Observe
that the set $\{ n + 1, n + 2, \ldots , 2n - 1\}$ of $n + 1$
elements is sum free. Shew that any subset with $n + 2$ elements
is not sum free.
\begin{pro}[MMPC 1992] Suppose that the letters of the English
alphabet are listed in an arbitrary order. \begin{enumerate} \item
Prove that there must be four consecutive consonants. \item Give a
list to shew that there need not be five consecutive consonants.
\item Suppose that all the letters are arranged in a circle. Prove
that there must be five consecutive consonants.
\end{enumerate}
\end{pro}
\begin{pro}[Stanford 1953] Bob has ten pockets and forty four silver
dollars. He wants to put his dollars into his pockets so
distributed that each pocket contains a different number of
dollars.
\begin{enumerate}\item Can he do so? \item Generalise the problem, considering $p$ pockets
and $n$ dollars. The problem is most interesting when $$ n =
\frac{(p - 1)(p - 2)}{2}.$$ Why? \end{enumerate} \end{pro}
\begin{pro}
Let $M$ be a seventeen-digit positive integer and let $N$ be the
number obtained from $M$ by writing the same digits in reversed
order. Prove that at least one digit in the decimal representation
of the number $M + N$ is even.
\end{pro}

\begin{pro} No matter which fifty five integers may be selected from $$\{ 1, 2, \ldots , 100\} , $$
prove that you must select some two that differ by $9$, some two
that differ by $10$, some two that differ by $12$, and some two
that differ by $13$, but that you need not have any two that
differ by $11$. \end{pro}
\begin{pro} Let $mn + 1$ different real numbers be given. Prove that
there is either an increasing sequence with at least $n + 1$
members, or a decreasing sequence with at least $m + 1$
members.\end{pro}
\begin{pro} If the points of the plane are coloured with three colours,
shew that there will always exist two points of the same colour
which are one unit apart.
\end{pro}
\begin{pro} Shew that if the points of the plane are coloured with two
colours, there will always exist an equilateral triangle with all
its vertices of the same colour. There is, however, a colouring of
the points of the plane with two colours for which no equilateral
triangle of side $1$ has all its vertices of the same colour.
\end{pro}
\begin{pro}[USAMO 1979] Nine mathematicians meet at an international conference
and discover that amongst any three of them, at least two speak a
common language. If each of the mathematicians can speak at most
three languages, prove that there are at least three of the
mathematicians who can speak the same language.\end{pro}
\begin{pro}[USAMO 1982] In a party with $1982$ persons, amongst any group
of four there is at least one person who knows each of the other
three. What is the minimum number of people in the party who know
everyone else?\end{pro}
\begin{pro}[USAMO 1985] There are $n$ people at a party. Prove that there are two people such that,
of the remaining $n - 2$ people, there are at least $\dis{\llfloor
n/2\rrfloor - 1}$ of them, each of whom knows both or else knows
neither of the two. Assume that ``knowing'' is a symmetrical
relationship.\end{pro}
\begin{pro}[USAMO 1986] During a certain lecture, each of five mathematicians fell
asleep exactly twice. For each pair of these mathematicians, there
was some moment when both were sleeping simultaneously. Prove
that, at some moment, some three were sleeping simultaneously.
\end{pro}
\end{multicols}



\section{Parity}
\begin{exa}
Two diametrically opposite corners of a chess board are deleted.
Shew that it is impossible to tile the remaining $62$ squares with
$31$ dominoes.
\end{exa}
Solution: Each domino covers one red square and one black squares.
But diametrically opposite corners are of the same colour, hence
this tiling is impossible.

\begin{exa} All the dominoes in a set are laid out in a chain according
to the rules of the game. If one end of the chain is a $6$, what
is at the other end?
\end{exa}
Solution: At the other end there must be a 6 also. Each number of
spots must occur in a pair, so that we may put them end to end.
Since there are eight 6's, this last 6 pairs off with the one at
the beginning of the chain.

\begin{exa}
The numbers $1, 2, \ldots , 10$ are written in a row. Shew that no
matter what choice of sign $\pm$ is put in between them, the sum
will never be $0$.
\end{exa}
Solution: The sum $1 + 2 + \cdots + 10 = 55,$ an odd integer.
Since parity is not affected by the choice of sign, for any choice
of sign $\pm 1 \pm 2 \pm \cdots \pm 10$ will never be even, in
particular it will never be 0.


\begin{df}
A {\em lattice point} $(m, n)$ on the plane is one having integer
coordinates.
\end{df}
\begin{df}
The midpoint of the line joining $(x, y)$ to $(x_1, y_1)$ is the
point
$$\left(\frac{x + x_1}{2}, \frac{y + y_1}{2}\right) .$$
\end{df}


\begin{exa}
Five lattice points are chosen at random. Prove that one can
always find two so that the midpoint of the line joining them is
also a lattice point.
\end{exa}
Solution: There are four parity patterns: (even, even), (even,
odd), (odd, odd), (odd, even). By the Pigeonhole Principle among
five lattice points there must be two having the same parity
pattern. Choose these two. It is clear that their midpoint is an
integer.



\clearpage

For the next few examples we will need to know the names of the
following tetrominoes. \vspace{1cm}

\begin{figure}[h]
\begin{minipage}{5cm}
$$\psset{unit=.6pc} \psline(0,0)(3,0)(3,2)(2,2)(2,0)
\psline(0,0)(0,1)(3,1) \psline(1,0)(1,1) \psline(2,0)(2,1)
 $$\hangcaption{L-tetromino}
\end{minipage}
\hfill
\begin{minipage}{5cm}
$$ \psset{unit=.6pc} \psline(0,0)(3,0)(3,1)(0,1)(0,0)
\psline(1,0)(1,2)(2,2)(2,0)
 $$\hangcaption{T-tetromino}
\end{minipage}
\hfill
\begin{minipage}{5cm}
$$ \psset{unit=.6pc} \psline(0,0)(4,0)(4,1)(0,1)(0,0)
\psline(1,0)(1,1) \psline(2,0)(2,1) \psline(3,0)(3,1)
 $$\hangcaption{Straight-tetromino}
\end{minipage}
\end{figure}
\begin{figure}[h]	
\begin{minipage}{8cm}
$$ \psset{unit=.6pc}
\psline(0,0)(2,0)(2,1)(3,1)(3,2)(1,2)(1,1)(0,1)(0,0)
\psline(1,1)(2,1)(2,2) \psline(0,1)(2,1) \psline(1,0)(1,1)
  $$\hangcaption{Skew-tetromino}
\end{minipage}
\hfill
\begin{minipage}{8cm}
$$ \psset{unit=.6pc} \psline(0,0)(2,0)(2,2)(0,2)(0,) \psline(0,1)(2,1)
\psline(1,0)(1,2) \psline(0,0)(0,1)
  $$\hangcaption{Square-tetromino}
\end{minipage}
\end{figure}
\begin{exa}
A single copy of each of the tetrominoes shewn above is taken.
Shew that no matter how these are arranged, it is impossible to
construct a rectangle.
\end{exa}
Solution: If such a rectangle were possible, it would have 20
squares. Colour the rectangle like a chessboard. Then there are 10
red squares and $10$ black squares. The T-tetromino always covers an
odd number of red squares. The other tetrominoes always cover an
even number of red squares. This means that the number of red
squares covered is odd, a contradiction.
\begin{exa}

Shew that an $8\times 8$ chessboard cannot be tiles with $15$
straight tetrominoes and one L-tetromino.

\end{exa}
Solution: Colour rows $1, 3, 5, 7$ black and colour rows $2, 4, 6$,
and $8$ red. A straight tetromino will always cover an even number
of red boxes and the L-tetromino will always cover an odd number of
red squares. If the tiling were possible, then we would be covering
an odd number of red squares, a contradiction.



\section*{Practice}\addcontentsline{toc}{section}{Practice}\markright{Practice}\begin{multicols}{2}\columnseprule 1pt \columnsep 25pt\multicoltolerance=900
\begin{pro}
Twenty-five boys and girls are seated at a round table. Shew that
both neighbours of at least one student are girls.
\end{pro}
\begin{pro}
A closed path is made of $2001$ line segments. Prove that there is
no line, not passing through a vertex of the path, intersecting
each of the segments of the path.
\end{pro}
\begin{pro}
The numbers $1, 2, \ldots , 2001$ are written on a blackboard. One
starts erasing any two of them and replacing the deleted ones with
their difference. Will a situation arise where all the numbers on
the blackboard be $0$?
\end{pro}
\begin{pro}
Shew that a $10 \times 10$ chessboard cannot be tiled with $25$
straight tetrominoes.
\end{pro}
\begin{pro}
Shew that an $8\times 8$ chess board cannot be tiled with $15$
T-tetrominoes and one square tetromino.
\end{pro}
\end{multicols}
\chapter{Algebra}
\section{Identities with Squares} Recall that
\begin{equation}(x + y)^2 = (x + y)(x + y) = x^2 + y^2 + 2xy \end{equation} If we  substitute $y$ by $y + z$ we obtain
\begin{equation} (x + y + z)^2 = x^2 + y^2 + z^2 + 2xy + 2xz + 2yz \end{equation} If we substitute $z$ by $z + w$ we
obtain
\begin{equation}
(x + y + z + w)^2 = x^2 + y^2 + z^2 + w^2 + 2xy + 2xz + 2xw + 2yz
+ 2yw + 2zw \end{equation}

\begin{exa}
The sum of two numbers is $21$ and their product $-7$. Find (i)
the sum of their squares, (ii) the sum of their reciprocals and
(iii) the sum of their fourth powers.
\end{exa}
Solution: If the two numbers are $a$ and $b$, we are given that $a
+ b = 21$ and $ab = -7.$ Hence $$a^2 + b^2 = (a + b)^2 - 2ab =
21^2 - 2(-7) = 455$$ and
$$\frac{1}{a} + \frac{1}{b} = \frac{b + a}{ab} = \frac{21}{-7} = -3$$
Also
$$a^4 + b^4 = (a^2 + b^2)^2 - 2a^2b^2 = 455^2 - 2(-7)^2 = 357$$

\begin{exa} Find positive integers $a$ and $b$ with
$$\sqrt{5 + \sqrt{24}} = \sqrt{a} + \sqrt{b}.$$\end{exa}
Solution: Observe that
$$5 + \sqrt{24} = 3 + 2\sqrt{2\cdot 3} + 2 = (\sqrt{2} + \sqrt{3})^2.$$
Therefore
$$\sqrt{5 + 2\sqrt{6}} = \sqrt{2} + \sqrt{3}.$$

\begin{exa}Compute
$$\sqrt{(1 000 000)(1 000 001)(1 000 002)(1 000 003) + 1}$$ without using a calculator.
\end{exa}
Solution: Let $x = 1\ 000\ 000 = 10^6$. Then
$$x(x + 1)(x + 2)(x + 3) = x(x + 3)(x + 1)(x + 2) = (x^2 + 3x)(x^2 + 3x + 2).$$
Put $y = x^2 + 3x.$ Then
$$x(x + 1)(x + 2)(x + 3) + 1 =  (x^2 + 3x)(x^2 + 3x + 2) + 1 = y(y + 2) + 1 = (y + 1)^2.$$
Thus
$$\begin{array}{lll}\sqrt{x(x + 1)(x + 2)(x + 3) + 1} & = &  y + 1\\ & =  & x^2 + 3x + 1 \\ & = &
10^{12} + 3\cdot 10^6 + 1  \\
& =  & 1\ 000\ 003\ 000 \ 001.
\end{array}
$$


Another useful identity is the difference of squares:
\begin{equation} x^2 - y^2 = (x - y)(x + y) \end{equation}
\begin{exa} Explain how to compute $123456789^2 - 123456790\times 123456788$ mentally.\end{exa}
Solution: Put $x = 123456789$. Then
$$123456789^2 - 123456790\times 123456788 = x^2 - (x + 1)(x - 1) = 1.$$

\begin{exa}
Shew that
$$1 + x + x^2 + \cdots + x^{1023}
 = (1 + x)(1 + x^2)(1 + x^4)\cdots (1 + x^{256})(1 + x^{512}).$$
\end{exa}
Solution: Put $S = 1 + x + x^2 + \cdots + x^{1023}$. Then $xS = x
+ x^2 + \cdots + x^{1024}$. This gives
$$S - xS = (1 + x + x^2 + \cdots + x^{1023}) -
(x + x^2 + \cdots + x^{1024}) = 1 - x^{1024}$$ or $S(1 - x) = 1 -
x^{1024}$, from where
$$1 + x + x^2 + \cdots + x^{1023} = S = \frac{1 - x^{1024}}{1 - x}.$$
But \renewcommand{\arraystretch}{1.5}
$${\everymath{\dis}\begin{array}{lll}\frac{1 - x^{1024}}{1 - x} & = &  \left(\frac{1 - x^{1024}}{1 - x^{512}}\right)\left(\frac{1 - x^{512}}{1 - x^{256}}\right)
\cdots \left(\frac{1 - x^{4}}{1 - x^2}\right)\left(\frac{1 -
x^{2}}{1 - x}\right)\\ & = &  (1 + x^{512})(1 + x^{256}) \cdots (1
+ x^2)(1 + x), \\ \end{array}}$$proving the assertion.
\begin{exa} Given that
$$\frac{1}{\sqrt{1} + \sqrt{2}} + \frac{1}{\sqrt{2} + \sqrt{3}} + \frac{1}{\sqrt{3} + \sqrt{4}} +
\cdots + \frac{1}{\sqrt{99} + \sqrt{100}}$$is an integer, find
it.\end{exa} Solution: As $1 = n + 1 - n = (\sqrt{n + 1} -
\sqrt{n})(\sqrt{n + 1} + \sqrt{n}),$ we have
$$ \frac{1}{\sqrt{n} + \sqrt{n + 1}} = \sqrt{n + 1} - \sqrt{n}.$$
Therefore
$${\everymath{\dis}\begin{array}{lcl}
\frac{1}{\sqrt{1} + \sqrt{2}} & = & \sqrt{2} - \sqrt{1} \\
\frac{1}{\sqrt{2} + \sqrt{3}} & = & \sqrt{3} - \sqrt{2} \\
\frac{1}{\sqrt{3} + \sqrt{4}} & = & \sqrt{4} - \sqrt{3} \\
\vdots & \vdots & \vdots \\
\frac{1}{\sqrt{99} + \sqrt{100}} & = & \sqrt{100} - \sqrt{99}, \\
\end{array}}$$
and thus
$$\frac{1}{\sqrt{1} + \sqrt{2}} + \frac{1}{\sqrt{2} + \sqrt{3}} + \frac{1}{\sqrt{3} + \sqrt{4}} +
\cdots + \frac{1}{\sqrt{99} + \sqrt{100}} = \sqrt{100} - \sqrt{1}
= 9.$$


Using the difference of squares identity,  $$\begin{array}{lcl}
x^4 + x^2y^2 + y^4 & = & x^4 + 2x^2y^2 + y^4 - x^2y^2 \\
& = & (x^2 + y^2)^2 - (xy)^2  \\
& = & (x^2 - xy + y^2)(x^2 + xy + y^2).

\end{array}$$
The following factorisation is credited to Sophie Germain.
$$\begin{array}{lcl}
a^4 + 4b^4 & = & a^4 + 4a^2b^2 + 4b^4 - 4a^2b^2 \\
& = & (a^2 + 2b^2)^2 - (2ab)^2  \\
& = & (a^2 - 2ab + 2b^2)(a^2 + 2ab + 2b^2)
\end{array}$$
\begin{exa} Prove that $n^4 + 4$ is a prime only when $n = 1$ for $n \in \BBN$. \end{exa}
Solution: Using Sophie Germain's trick,
$$ \begin{array}{lcl} n^4 + 4 & = & n^4 + 4n^2 + 4 - 4n^2 \\
& = & (n^2 + 2)^2 - (2n)^2 \\
&  = & (n^2 + 2 - 2n)(n^2 + 2 + 2n) \\
& = & ((n - 1)^2 + 1)((n + 1)^2 + 1). \end{array}$$Each factor is
greater than 1 for $n > 1,$ and so $n^4 + 4$ cannot be a prime if
$n > 1$.
\begin{exa}Shew that the product of four consecutive integers, none of them $0$,
is never a perfect square.\end{exa} Solution: Let $n - 1, n , n +
1, n + 2$ be four consecutive integers. Then their product $P$ is
$$P = (n - 1)n(n + 1)(n + 2) = (n^3 - n)(n + 2) = n^4 + 2n^3 - n^2
- 2n.$$But
$$(n^2 + n - 1)^2 = n^4 + 2n^3 - n^2 - 2n + 1 = P + 1 > P.$$As $P \neq 0$ and
$P$ is 1 more than a square, $P$ cannot be a square.

\begin{exa} Find infinitely many pairs of integers $(m, n)$ such that $m$  and $n$ share
their prime factors and $(m - 1, n - 1)$ share their prime
factors.\end{exa} Solution: Take $m = 2^k - 1, n = (2^k - 1)^2 , k
= 2, 3, \ldots$. Then $m, n$ obviously share their prime factors
and $m - 1 = 2(2^{k - 1} - 1)$ shares its prime factors with $n -
1 = 2^{k + 1}(2^{k - 1} - 1)$.


\begin{exa}
Prove that if $r \geq s \geq t$ then
\begin{equation}
r^2 - s^2 + t^2  \geq (r - s + t)^2
\label{eq:inequality_diff_squares}
\end{equation}

\end{exa}
Solution:  We have
$$ (r - s + t)^2 - t^2 = (r - s + t - t)(r - s + t + t) = (r - s)(r - s + 2t).$$
Since $t - s \leq 0,$ $r - s + 2t = r + s + 2(t - s) \leq r + s$
and so
$$(r - s + t)^2 - t^2 \leq (r - s)(r + s) = r^2 - s^2$$which gives
$$(r - s + t)^2 \leq r^2 - s^2 + t^2.$$


\section*{Practice}\addcontentsline{toc}{section}{Practice}\markright{Practice}\begin{multicols}{2}\columnseprule 1pt \columnsep 25pt\multicoltolerance=900

\begin{pro} The sum of two numbers is $-7$ and their product $2$. Find (i) the sum of their
reciprocals, (ii) the sum of their squares.
\begin{answer}
(i) $-3.5,$ (ii) $45$
\end{answer}
\end{pro}
\begin{pro}
Write $x^2$ as a sum of  powers of $x + 3.$
\begin{answer}
$x^2 = (x + 3 - 3)^2 = (x + 3)^2 - 6(x + 3) + 9$
\end{answer}
\end{pro}
\begin{pro}
Write $x^2 - 3x + 8$ as a sum of  powers of $x - 1.$
\end{pro}
\begin{pro}
Prove that $3$ is the only prime of the form $n^2 - 1.$
\end{pro}
\begin{pro}
Prove that there are no primes of the form $n^4 - 1.$
\end{pro}
\begin{pro} Prove that $n^4 + 4^n$ is prime only for $n = 1$.
\end{pro}
\begin{pro}
Use Sophie Germain's trick to obtain $$x^4 + x^2 + 1 = (x^2 + x +
1)(x^2 - x + 1),$$ and then find all the primes of the form $n^4 +
n^2 + 1.$
\end{pro}
\begin{pro}
If $a, b$ satisfy $\frac{2}{a + b} = \frac{1}{a} + \frac{1}{b}$,
find $\dis{\frac{a^2}{b^2}}$.
\end{pro}
\begin{pro}
If $\cot x + \tan x = a$, prove that $\cot ^2x + \tan ^2x = a^2 -
2.$ .
\end{pro}
\begin{pro}Prove that if $a, b, c$ are positive integers, then
$$\begin{array}{l}(\sqrt{a} + \sqrt{b} + \sqrt{c})(-\sqrt{a} + \sqrt{b} +
\sqrt{c})\\\quad\cdot (\sqrt{a} - \sqrt{b} + \sqrt{c}) (\sqrt{a} +
\sqrt{b} - \sqrt{c})\end{array}$$is an integer.
\end{pro}
\begin{pro} By direct computation, shew that the product of sums of two squares is itself a
sum of two squares:
\begin{equation}
(a^2 + b^2)(c^2 + d^2) = (ac + bd)^2 + (ad - bc)^2 \end{equation}
\end{pro}
\begin{pro}
 Divide $x^{128} - y^{128}$ by
 $$\begin{array}{l}(x + y)(x^2 + y^2)(x^4 + y^4)(x^8 + y^8)\\ \quad (x^{16} + y^{16})(x^{32} + y^{32})(x^{64} + y^{64}).
 \end{array}$$
 \end{pro}
\begin{pro}
Solve the system
$$x + y = 9,$$$$x^2 + xy + y^2 = 61.$$
\end{pro}
\begin{pro}
Solve the system
$$x - y = 10,$$$$x^2 - 4xy + y^2 = 52.$$
\end{pro}
\begin{pro}
Find the sum of the prime divisors of $2^{16} - 1.$
\end{pro}
\begin{pro}  Find integers $a, b$ with
$$\sqrt{11 + \sqrt{72}} = a + \sqrt{b}.$$\end{pro}
\begin{pro}
Given that the difference $$\sqrt{57 - 40\sqrt{2}} - \sqrt{57 +
40\sqrt{2}}$$is an integer, find it.
\end{pro}
\begin{pro}
Solve the equation
$$\sqrt{x + 3 - 4\sqrt{x - 1}} + \sqrt{x + 8 - 6\sqrt{x - 1}}= 1.$$
\end{pro}
\begin{pro}
Prove that if $a > 0, \ b > 0, a + b > c$, then $\sqrt{a} +
\sqrt{b} > \sqrt{c}$
\end{pro}
\begin{pro}
Prove that if $1 < x < 2,$ then
$$\frac{1}{\sqrt{x + 2\sqrt{x - 1}}} + \frac{1}{\sqrt{x - 2\sqrt{x - 1}}} = \frac{2}{2 - x}.$$
\end{pro}
\begin{pro}
If $x > 0$, from
$$\sqrt{x + 1} - \sqrt{x} = \frac{1}{\sqrt{x + 1} + \sqrt{x}},$$prove that
$$\frac{1}{2\sqrt{x + 1}} < \sqrt{x + 1} - \sqrt{x} < \frac{1}{2\sqrt{x}}.$$
Use this to prove that if  $n > 1$ is a positive integer, then
$$2\sqrt{n + 1} - 2 < 1 + \frac{1}{\sqrt{2}} + \frac{1}{\sqrt{3}} + \cdots + \frac{1}{\sqrt{n}}  < 2\sqrt{n} - 1$$
\end{pro}

\begin{pro} Shew that
$$(1 + x)(1 + x^2)(1 + x^4 )(1 + x^8)\cdots (1 + x^{1024}) = \frac{1 - x^{2048}}{1 - x}.$$
\end{pro}
\begin{pro}
Shew that
$$a^2 + b^2 + c^2 - ab - bc - ca = \frac{1}{2}\left((a - b)^2 + (b - c)^2 + (c - a)^2\right) .$$
\end{pro}
\begin{pro}
Prove that if $r \geq s \geq t \geq u \geq v$ then
\begin{equation}
r^2 - s^2 + t^2 - u^2 + v^2  \geq (r - s + t - u + v)^2
\end{equation}
\begin{answer}
Substitute $t$ by $\sqrt{t^2 - u^2 + v^2}$, in
\ref{eq:inequality_diff_squares}.
\end{answer}
\end{pro}

\begin{pro}[AIME 1987] Compute
$$\frac{(10^4 + 324)(22^4 + 324)(34^4 + 324)(46^4 + 324)(58^4 + 324)}{(4^4 + 324)(16^4 + 324)(28^4 + 324)
(40^4 + 324)(52^4 + 324)}.$$
\end{pro}
\begin{pro}Write $(a^2 + a + 1)^2$ as the sum of three squares.
\end{pro}\end{multicols}
\section{Squares of Real Numbers} If $x$ is a real
number then $x^2 \geq 0$. Thus if $a \geq 0, b \geq 0$ then
$(\sqrt{a} - \sqrt{b})^2 \geq 0$ gives, upon expanding the square,
$a - 2\sqrt{ab} + b \geq 0$, or
$$\sqrt{ab} \leq \frac{a + b}{2}.$$ Since $\dis{\frac{a + b}{2}}$ is the arithmetic mean of
$a, b$ and $\sqrt{ab}$ is the geometric mean of $a, b$ the
inequality
\begin{equation}
\sqrt{ab} \leq \frac{a + b}{2}
\end{equation} is known as the {\em Arithmetic-Mean-Geometric Mean} (AM-GM) Inequality.


\begin{exa}
Let $u_1, u_2, u_3, u_4$ be non-negative real numbers. By applying
the preceding result twice, establish the AM-GM Inequality for
four quantities:
\begin{equation}
(u_1u_2u_3u_4)^{1/4} \leq \frac{u_1 + u_2 + u_3 + u_4}{4}
\end{equation}
\end{exa}
Solution: We have $\sqrt{u_1u_2} \leq \frac{u_1 + u_2}{2}$ and
$\sqrt{u_3u_4} \leq \frac{u_3 + u_4}{2}$. Now, applying the AM-GM
Inequality twice to $\sqrt{u_1u_2}$ and $\sqrt{u_3u_4}$ we obtain
$$\sqrt{\sqrt{u_1u_2}\sqrt{u_3u_4}} \leq \frac{\sqrt{u_1u_2} + \sqrt{u_3u_4}}{2} \leq
\frac{\frac{u_1 + u_2}{2} + \frac{u_3 + u_4}{2}
}{2}.$$Simplification yields the desired result.
\begin{exa}
Let $u, v, w$ be non-negative real numbers. By using the preceding
result on the four quantities $u, v, w,$ and $\frac{u + v +
w}{3}$, establish the AM-GM Inequality for three quantities:
\begin{equation}
(uvw)^{1/3} \leq \frac{u + v + w}{3} \end{equation}
\end{exa}
Solution: By the AM-GM Inequality for four values
$$\left(uvw\left(\frac{u + v + w}{3}\right)\right)^{1/4} \leq \frac{u + v + w + \frac{u + v + w}{3}}{4}.$$Some algebraic
manipulation makes this equivalent to
$$(uvw)^{1/4}\left(\frac{u + v + w}{3}\right)^{1/4} \leq \frac{u + v + w}{4} + \frac{u + v + w}{12}$$ or
upon adding the fraction on the right
$$(uvw)^{1/4}\left(\frac{u + v + w}{3}\right)^{1/4} \leq \frac{u + v + w}{3}. $$
Multiplying both sides by $\dis{\left(\frac{u + v +
w}{3}\right)^{-1/4} }$ we obtain
$$(uvw)^{1/4} \leq \left(\frac{u + v + w}{3}\right)^{3/4},  $$from where the desired inequality
follows.
\begin{exa}
Let $a > 0, b > 0$. Prove the {\em Harmonic-Mean-Geometric-Mean}
Inequality
\begin{equation}\frac{2}{\frac{1}{a} + \frac{1}{b}} \leq \sqrt{ab} \end{equation}
\end{exa}
Solution: By the AM-HM Inequality
$$\sqrt{\frac{1}{a}\cdot\frac{1}{b}} \leq \frac{\frac{1}{a} + \frac{1}{b}}{2},$$from where
the desired inequality follows.
\begin{exa}
Prove that if $a, b, c$ are non-negative real numbers then $$(a +
b)(b + c)(c + a) \geq 8abc.$$
\end{exa}
Solution: The result quickly follows upon multiplying the three
inequalities $a + b \geq 2\sqrt{ab}$, $b + c \geq 2\sqrt{bc}$ and
$c + a \geq 2\sqrt{ca}$.

\begin{exa}
If $a, b, c, d$, are real numbers such that $a^2 + b^2 + c^2 + d^2 =
ab + bc + cd + da,$ prove that $a = b = c = d$.
\end{exa}Solution: Transposing,
$$a^2 - ab + b^2 - bc + c^2 - dc + d^2 - da  = 0,$$or
$$\frac{a^2}{2} - ab + \frac{b^2}{2}  + \frac{b^2}{2} - bc +
\frac{c^2}{2} + \frac{c^2}{2}  - dc + \frac{d^2}{2} +
\frac{d^2}{2} - da + \frac{a^2}{2}  = 0.$$ Factoring,
$$\frac{1}{2}(a - b)^2 + \frac{1}{2}(b - c)^2 +\frac{1}{2}(c - d)^2 +\frac{1}{2}(d - a)^2 = 0.$$
As the sum of non-negative quantities is zero only when the
quantities themselves are zero, we obtain $a = b, b = c, c = d, d
= a,$ which proves the assertion.




We note in passing that from the identity
\begin{equation}
a^2 + b^2 + c^2 - ab - bc - ca = \frac{1}{2}\left((a - b)^2 + (b -
c)^2 + (c - a)^2\right)
\end{equation}
it follows that
\begin{equation}
a^2 + b^2 + c^2  \geq ab + bc + ca
\end{equation}

\begin{exa}
The values of $a, b, c,$ and $d$ are $1, 2, 3$ and $4$ but not
necessarily in that order. What is the largest possible value of
$ab + bc + cd + da$?
\end{exa}
Solution: $$ \begin{array}{lll} ab + bc + cd + da & = & (a + c)(b
+ d) \\
& \leq & \left(\dfrac{a + c + b + d}{2}\right)^2 \\
& = & \left(\dfrac{1 + 2 + 3 + 4}{2}\right)^2 \\
& = & 25,\\
\end{array}$$ by AM-GM. Equality occurs when $a + c = b + d$. Thus one
may choose, for example, $a = 1, c = 4, b = 2, d = 3.$
\section*{Practice}\addcontentsline{toc}{section}{Practice}\markright{Practice}\begin{multicols}{2}\columnseprule 1pt \columnsep 25pt\multicoltolerance=900

\begin{pro}
If $0 < a \leq b$, shew that
$$\frac{1}{8}\cdot\frac{(b - a)^2}{b} \leq \frac{a + b}{2} - \sqrt{ab} \leq \frac{1}{8}\cdot\frac{(b - a)^2}{a} $$
\begin{answer}
Use the fact that $(b - a)^2 = (\sqrt{b} - \sqrt{a})^2(\sqrt{b} +
\sqrt{a})^2$.
\end{answer}
\end{pro}
\begin{pro}
Prove that if $a, b, c$ are non-negative real numbers then $$(a^2
+ 1)(b^2 + 1)(c^2 + 1) \geq 8abc$$
\end{pro}
\begin{pro}
The sum of two positive numbers is $100$. Find their maximum
possible product.
\end{pro}
\begin{pro}
Prove that if $a, b, c$ are positive numbers then
$$\frac{a}{b} + \frac{b}{c} + \frac{c}{a} \geq 3.$$

\end{pro}

\begin{pro}
Prove that of all rectangles with a given perimeter, the square
has the largest area.
\end{pro}
\begin{pro}
Prove that if $0 \leq x \leq 1$ then $x - x^2 \leq \frac{1}{4}$.
\end{pro}
\begin{pro}
Let $0 \leq a, b, c, d \leq 1.$ Prove that at least one of the
products $$a(1 - b), \ \ b(1 - c),\ \  \ \ c(1 - d),\ \  d(1 -
a)$$ is $ \leq \frac{1}{4}$.
\begin{answer}
Suppose that all these products are $ > \frac{1}{4}$. Use the
preceding problem to obtain a contradiction.
\end{answer}
\end{pro}
\begin{pro}
Use the AM-GM Inequality for four non-negative real numbers to
prove a version of the AM-GM for eight non-negative real numbers.
\end{pro}\end{multicols}
\section{Identities with Cubes} By direct
computation we find that
\begin{equation}
(x + y)^3 = (x + y)(x^2 + y^2 + 2xy) = x^3 + y^3 + 3xy(x + y)
\end{equation}
\begin{exa}
The sum of two numbers is $2$ and their product $5$. Find the sum
of their cubes.
\end{exa}
Solution: If the numbers are $x, y$ then $x^3 + y^3 = (x + y)^3 -
3xy(x + y) = 2^3 - 3(5)(2) = -22.$




Two other useful identities are the sum and difference of cubes,
\begin{equation}
 x^3 \pm y^3 = (x \pm y)(x^2 \mp xy + y^2)
 \end{equation}
\begin{exa} Find all the prime numbers of the form $n^3 - 1$, $n$ a positive integer.
\end{exa}
Solution: As $n^3 - 1 = (n - 1)(n^2 + n + 1)$ and as $n^2 + n + 1
> 1,$  it must be the case that $n - 1 = 1$, i.e., $n = 2$.
Therefore, the only prime of this form is  $2^3 - 1 = 7.$
\begin{exa} Prove that
$$1 + x + x^2 + \cdots + x^{80} = (x^{54} + x^{27} + 1)(x^{18} + x^9 + 1)(x^6 + x^3 + 1)(x^2 + x + 1).$$

\end{exa}

Solution: Put $S = 1 + x + x^2 + \cdots + x^{80}.$ Then
$$S - xS = (1 + x + x^2 + \cdots + x^{80}) - (x + x^2 + x^3 + \cdots + x^{80} + x^{81})
= 1 - x^{81},$$or $S(1 - x) = 1 - x^{81}$. Hence
$$1 + x + x^2 + \cdots + x^{80} = \frac{x^{81} - 1}{x - 1}.$$
Therefore
$$\frac{x^{81} - 1}{x - 1} = \frac{x^{81} - 1}{x^{27} - 1}\cdot\frac{x^{27} - 1}{x^{9} - 1}\cdot\frac{x^{9} - 1}{x^3 - 1}\cdot\frac{x^{3} - 1}{x - 1}.$$
Thus
$$1 + x + x^2 + \cdots + x^{80} = (x^{54} + x^{27} + 1)(x^{18} + x^9 + 1)(x^6 + x^3 + 1)(x^2 + x + 1).$$
\begin{exa}
Shew that
\begin{equation}
a^3 + b^3 + c^3 - 3abc =  (a + b + c)(a^2 + b^2 + c^2 - ab - bc -
ca)
\end{equation}
\end{exa}
Solution: We use the identity
$$x^3 + y^3 = (x + y)^3 - 3xy(x + y)$$twice. Then
$$
\begin{array}{lll}
a^3 + b^3 + c^3 - 3abc & = & (a + b)^3 + c^3 - 3ab(a + b) - 3abc \\
& = & (a + b + c)^3 - 3(a + b)c(a + b + c) - 3ab(a + b + c) \\
& = & (a + b + c)((a + b + c)^2 - 3ac - 3bc - 3ab) \\
& = & (a + b + c)(a^2 + b^2 + c^2 - ab - bc - ca)
\end{array}
$$
If $a, b, c$ are non-negative then $a + b + c \geq 0$ and also
$a^2 + b^2 + c^2 - ab - bc - ca \geq 0$ by (2.13). This gives
$$\frac{a^3 + b^3 + c^3}{3} \geq abc.$$Letting $a^3 = x, b^3 = y, c^3 = z$,
for non-negative real numbers $x, y, z,$ we obtain the AM-GM
Inequality for three quantities.
\section*{Practice}\addcontentsline{toc}{section}{Practice}\markright{Practice}\begin{multicols}{2}\columnseprule 1pt \columnsep 25pt\multicoltolerance=900

\begin{pro}
If $a^3 - b^3 = 24, a - b = 2,$  find $(a + b)^2.$
\end{pro}
\begin{pro}
Shew that for integer $n \geq 2,$ the expression
$$\frac{n^3 + (n + 2)^3}{4}$$is a composite integer.
\end{pro}
\begin{pro} If $\tan x + \cot x = a,$  prove that  $\tan ^3 x + \cot ^3 x = a^3 - 3a$. \end{pro}

\begin{pro}[AIME 1986] What is the largest positive integer $n$
for which $$ (n + 10)|(n^3 + 100)?$$ \end{pro}
\begin{pro}Find all the primes of the form $n^3 +
1$.\end{pro}
\begin{pro}
Solve the system
$$x^3 + y^3 = 126,$$$$x^2 - xy + y^2 = 21.$$
\end{pro}
\begin{pro}
Evaluate the sum
$$\begin{array}{l}\frac{1}{\sqrt[3]{1} + \sqrt[3]{2} + \sqrt[3]{4} } + \frac{1}{\sqrt[3]{4} + \sqrt[3]{6} +
\sqrt[3]{9}} \\ \quad + \frac{1}{\sqrt[3]{9} + \sqrt[3]{12} +
\sqrt[3]{16}}.\end{array}$$
\end{pro}
\begin{pro} Find $a^6 + a^{-6}$ given that $a^2 + a^{-2} = 4. $

\begin{answer}
$52$
\end{answer}

 \end{pro}
\begin{pro}
Prove  that
\begin{equation}
(a + b + c)^3 - a^3 - b^3 - c^3 = 3(a + b)(b + c)(c + a)
\end{equation}
\end{pro}
\begin{pro}[ITT 1994] Let $a, b, c, d$ be complex numbers satisfying
 $$ a + b + c + d = a^3 + b^3 + c^3 +
d^3 = 0.$$Prove that a pair of the $a, b, c, d$ must add up to
$0$. \end{pro}
\end{multicols}
\section{Miscellaneous Algebraic Identities} We have seen the identity
\begin{equation}y^2-x^2 = (y-x)(y+x).
\label{eq:diff_of_squares}\end{equation} We would like to deduce a
general identity for $y^n-x^n$, where $n$ is a positive integer. A
few multiplications confirm that
\begin{equation}\label{eq:diff_of_cubes}y^3-x^3 = (y-x)(y^2 + yx +
x^2),
\end{equation}
\begin{equation}\label{eq:diff_of_fourths}y^4-x^4 = (y-x)(y^3 + y^2x + yx^2 + x^3),  \end{equation} and
\begin{equation}\label{eq:diff_of_fifths}y^5-x^5 = (y-x)(y^4 + y^3x + y^2x^2 + yx^3 + x^4).  \end{equation} The general
result is in fact the following theorem.
\begin{thm}\label{thm:binom_difference}If $n$ is a positive integer, then
$$y^n - x^n = (y-x)(y^{n-1} + y^{n-2}x + \cdots + yx^{n-2} + x^{n-1}).   $$
\end{thm}
\begin{pf} We first prove that for $a\neq 1$.
$$1 + a + a^2 + \cdots a^{n-1} = \dfrac{1 - a^{n}}{1 -a}.   $$For,
put $S = 1 + a + a^2 + \cdots + a^{n-1}.$ Then $aS = a + a^2 +
\cdots + a^{n-1} + a^n.$ Thus $S - aS = (1 + a + a^2 + \cdots
+a^{n-1}) - (a + a^2 + \cdots + a^{n-1} + a^n) = 1 - a^n,$ and from
$(1-a)S = S-aS=1 - a^n $ we obtain the result. By making the
substitution $a = \frac{x}{y}$ we see that
$$1 + \frac{x}{y} + \left(\frac{x}{y}\right)^2 + \cdots +
\left(\frac{x}{y}\right)^{n-1} = \dfrac{1-
\left(\frac{x}{y}\right)^n}{1-\frac{x}{y}}   $$ we obtain
$$\left(1-\frac{x}{y}\right)\left(1 + \frac{x}{y} +
\left(\frac{x}{y}\right)^2 + \cdots +
\left(\frac{x}{y}\right)^{n-1}\right) = 1-
\left(\frac{x}{y}\right)^n,   $$ or equivalently,
$$\left(1-\frac{x}{y}\right)\left(1 + \frac{x}{y} +
\frac{x^2}{y^2} + \cdots +\frac{x^{n-1}}{y^{n-1}}\right) = 1-
\frac{x^n}{y^n}.   $$ Multiplying by $y^n$ both sides,
$$y\left(1-\frac{x}{y}\right) y^{n-1}\left(1 + \frac{x}{y} +
\frac{x^2}{y^2} + \cdots +\frac{x^{n-1}}{y^{n-1}}\right) =
y^n\left(1- \frac{x^n}{y^n}\right),   $$ which is
$$y^n - x^n = (y-x)(y^{n-1} + y^{n-2}x + \cdots + yx^{n-2} + x^{n-1}),   $$
yielding the result.
\end{pf}
\begin{rem}
The second factor has $n$ terms and each term has degree (weight)\ \
\
 $n-1$.
\end{rem}
As an easy corollary we deduce
\begin{cor}
If $x, y$ are integers $x \neq y$ and $n$ is a positive integer
then $x - y$ divides $x^n - y^n.$
\end{cor}
Thus without any painful calculation we see that $781 = 1996 -
1215$ divides $1996^5 - 1215^5.$

\begin{exa}[E\H{o}tv\H{o}s 1899] Shew that for any positive integer $n$, the expression
$$2903^n - 803^n - 464^n + 261^n$$is always divisible by $1897$.\end{exa}
Solution: By the theorem above, $2903^n - 803^n$ is divisible by
$2903 - 803 = 2100 = 7\cdot 300$ and $261^n - 464^n$ is divisible
by $-203 = (-29)\cdot 7$. This means that the given expression is
divisible by 7. Furthermore, $2903^n - 464^n$ is divisible by
$2903 - 464 = 2439 = 9\cdot 271$ and $-803^n + 261^n$ is divisible
by $-803  + 261 = -542 = -2\cdot 271$. Therefore as the given
expression is divisible by  7 and by 271 and as these two numbers
have no common factors, we have that $2903^n - 803^n - 464^n +
261^n$ is divisible by $7\cdot 271 = 1897.$

\begin{exa}[$(UM)^2 C^4 \, 1987$] Given that $1002004008016032$ has a
prime factor $p > 250000,$ find it. \end{exa} Solution: If $a =
10^3 , b = 2$ then $$ 1002004008016032 = a^5 + a^4 b + a^3 b^2 +
a^2 b^3 + ab^4 + b^5 = \frac{ a^6 - b^6 }{ a - b }. $$ This last
expression factorises as $$ {\everymath{\displaystyle}
\begin{array}{lcl}\frac{a^6 -
b^6}{a - b} & = & (a + b)(a^2 + ab + b^2)(a^2 - ab + b^2 ) \\
&  = & 1002\cdot 1002004\cdot 998004 \\
& = & 4\cdot 4\cdot 1002\cdot 250501 \cdot k,\end{array} }$$where
$k < 250000$. Therefore $p = 250501$.


Another useful corollary of Theorem \ref{thm:binom_difference} is
the following.
\begin{cor}
If $f(x) = a_0 + a_1x + \cdots + a_nx^n$ is a polynomial with
integral coefficients and if $a, b$ are integers then $b  - a$
divides $f(b) - f(a)$.
\end{cor}

\begin{exa}
Prove that there is no polynomial $p$ with integral coefficients
with $p(2) = 3$ and $p(7) = 17.$
\end{exa}
Solution: If the assertion were true then by the preceding
corollary, $7 - 2 = 5$ would divide $p(7) - p(2) = 17 - 3 = 14,$
which is patently false.



Theorem \ref{thm:binom_difference} also yields  the following
colloraries.
\begin{cor}If  $n$ is an odd positive integer
\begin{equation}
x^n + y^n = (x + y)(x^{n - 1} - x^{n - 2}y + x^{n - 3}y^2 - x^{n -
4}y^3 +  \cdots + x^2y^{n - 3} - xy^{n - 2} + y^{n - 1})
\end{equation}
\end{cor}
\begin{cor}
Let  $x, y$ be integers, $x \neq y$ and let $n$ be an odd positive
number. Then $x + y$ divides  $x^n + y^n.$
\end{cor}
For example $129 = 2^7 + 1$ divides $2^{861} + 1$ and $1001 = 1000
+ 1 = 999 + 2 = \cdots = 500 + 501$ divides
$$1^{1997} + 2^{1997} + \cdots + 1000^{1997}.$$



\begin{exa} Prove the following identity of Catalan:
$$ 1 - \frac{1}{2} + \frac{1}{3} - \frac{1}{4} + \cdots + \frac{1}{2n - 1} - \frac{1}{2n}
= \frac{1}{n + 1} + \frac{1}{n + 2} + \cdots + \frac{1}{2n}.$$
\end{exa}
Solution: The quantity on the sinistral side is
$${\everymath{\displaystyle}\begin{array}{lcl}
\left( 1 + \frac{1}{2} + \frac{1}{3} + \frac{1}{4} + \cdots + \frac{1}{2n - 1} + \frac{1}{2n}\right) & & \\
\qquad - 2\left(\frac{1}{2} + \frac{1}{4} + \frac{1}{6} + \cdots + \frac{1}{2n}\right) & & \\
& = &
\left( 1 + \frac{1}{2} + \frac{1}{3} + \frac{1}{4} + \cdots + \frac{1}{2n - 1} + \frac{1}{2n}\right)  \\
& & \qquad - 2\cdot\frac{1}{2}\left(1 + \frac{1}{2} + \frac{1}{3} +  \frac{1}{4} + \cdots + \frac{1}{n}\right)       \\
& = &
\left( 1 + \frac{1}{2} + \frac{1}{3} + \frac{1}{4} + \cdots + \frac{1}{2n - 1} + \frac{1}{2n}\right)  \\
& & \qquad - \left(1 + \frac{1}{2} + \frac{1}{3} +  \frac{1}{4} + \cdots + \frac{1}{n}\right)      \\
& = & \frac{1}{n + 1} + \frac{1}{n + 2} + \cdots + \frac{1}{2n},
\end{array}}$$
as we wanted to shew.
\section*{Practice}\addcontentsline{toc}{section}{Practice}\markright{Practice}\begin{multicols}{2}\columnseprule 1pt \columnsep 25pt\multicoltolerance=900

\begin{pro}
Shew that $100$ divides $11^{10} - 1.$
\end{pro}
\begin{pro} Shew that  $27195^8 - 10887^8 + 10152^8$ is divisible
by $26460$.\end{pro}
\begin{pro} Shew that $7$ divides
$$2222^{5555} + 5555^{2222}.$$
\begin{answer}
Write $$\begin{array}{lll} 2222^{5555} + 5555^{2222} & = &
(2222^{5555} + 4^{5555})\\
& & \qquad  + (5555^{2222} - 4^{2222}) \\ & & \qquad  - (4^{5555} -
4^{2222}).
\end{array}
$$
\end{answer}
\end{pro}
\begin{pro} Shew that if  $k$ is an odd positive integer
$$1^k + 2^k + \cdots + n^k$$is divisible by
$$1 + 2 + \cdots + n.$$\end{pro}
\begin{pro} Shew that
$$(x + y)^5 - x^5 - y^5 = 5xy(x + y)(x^2 + xy + y^2).$$\end{pro}
\begin{pro} Shew that $$(x + a)^7 - x^7 - a^7 = 7xa(x + a)(x^2 + xa + a^2)^2.$$  \end{pro}
\begin{pro}
Shew that $$A = x^{9999} + x^{8888} + x^{7777} + \cdots + x^{1111}
+ 1$$is divisible by $B = x^9 + x^8 + x^7 + \cdots + x^2 + x + 1$.
\end{pro}
\begin{pro}  Shew that for any natural number $n$, there is another natural number $x$ such that
each term of the sequence
$$ x + 1, x^x + 1, x^{x^x} + 1, \ldots $$ is divisible by $n$.
\begin{answer}
Consider $x = 2n - 1.$
\end{answer}


 \end{pro}

\begin{pro}
Shew that $1492^n - 1770^n - 1863^n + 2141^n$ is divisible by
$1946$ for all positive integers n.
\end{pro}
\begin{pro} Decompose $1 + x + x^{2} + x^3 + \cdots + x^{624}$ into factors.\end{pro}

\begin{pro} Shew that if $2^n - 1$ is prime, then  $n$ must be prime. Primes of this form are
called {\em Mersenne} primes.
\begin{answer}
we have
$$2^n - 1 = 2^{ab} - 1 = (2 ^a - 1)((2^a)^{b - 1} + (2^a)^{b - 2} + \cdots + (2^a)^1 +
1).$$Since $a > 1, 2^a - 1 > 1.$ Since $b > 1$,
$$(2^a)^{b - 1} + (2^a)^{b - 2} + \cdots (2^a)^1 +
1) \geq 2^a + 1 > 1.$$ We have decomposed a prime number (the left
hand side) into the product of two factors, each greater than $1$, a
contradiction. Thus $n$ must be a prime.
\end{answer}
\end{pro}
\begin{pro} Shew that if $2^n + 1$ is a prime, then $n$ must be a power of $2$.
Primes of this form are called {\rm Fermat} primes.
\begin{answer}
We have
$$2^n + 1 = 2^{2^km} + 1 = (2 ^{2^k} + 1)((2^{2^k})^{m - 1} - (2^{2^k})^{m - 2} + \cdots - (2^{2^k})^1
+  1).$$Clearly, $2 ^{2^k} + 1 > 1.$ Also if $m \geq 3$
$$(2^{2^k})^{m - 1} - (2^{2^k})^{m - 2} + \cdots - (2^{2^k})^1
+  1 \geq (2^{2^k})^2 - (2^{2^k})^1 + 1 > 1,$$and so, we have
produced two factors each greater than 1 for the prime $2^n + 1$,
which is nonsense.
\end{answer}


\end{pro}
\begin{pro}
Let $n$  be a positive integer and $x > y$. Prove that
$$\frac{x^n - y^n}{x - y} > ny^{n - 1}.$$By choosing suitable values of $x$ and
$y$, further prove than
$$\left(1 + \frac{1}{n}\right)^n < \left(1 + \frac{1}{n + 1}\right)^{n + 1} $$ and
$$
\left(1 + \frac{1}{n}\right)^{n + 1} > \left(1 + \frac{1}{n +
1}\right)^{n + 2} $$
\end{pro}\end{multicols}
\section{Logarithms}

\begin{df}
Let $a > 0, a \neq 1$ be a real number. A number $x$ is called the
{\em logarithm} of a number $N$ to the base $a$ if $a^x = N$. In
this case we write $x = \log _a N.$
\end{df}

We enumerate some useful properties of  logarithms. We assume
that $a > 0, a \neq 1, M > 0, N > 0.$

\begin{equation} a^{\log _a N} = N \end{equation}
\begin{equation} \log _a MN = \log _a M + \log _a N \end{equation}
\begin{equation}    \log _a \frac{M}{N} = \log _a M - \log _a N \end{equation}
\begin{equation}  \log _a N^\alpha = \alpha\log _a N, \ \ \alpha
{\rm \ any \ real \ number} \end{equation}
\begin{equation} \log _{ a^\beta} N = \frac{1}{\beta} \log _a N, \ \ \beta \neq 0
{\rm \ a \ real \ number} \end{equation}
\begin{equation} (\log _a b)(\log _b a) = 1, \ b > 0, b \neq 1. \end{equation}
\begin{exa}
Given that  $\dis{\log _{8\sqrt{2}} 1024}$ is a rational number,
find it.
\end{exa}
Solution: We have
$$\log _{8\sqrt{2}} 1024 = \log _{2^{7/2}} 1024
= \frac{2}{7}\log _{2} 2^{10} =  \frac{20}{7} $$
\begin{exa} Given that $$ (\log _2 3)\cdot (\log _3 4)\cdot (\log _4 5) \cdots
(\log _{511} 512)$$is an integer, find it.
\end{exa}
Solution: Choose $a > 0, a \neq 1$. Then
$$\begin{array}{lll}(\log _2 3)\cdot (\log _3 4)\cdot (\log _4 5) \cdots (\log _{511} 512)
& = & \frac{\log _a 3}{\log _a 2}\cdot\frac{\log _a 4}{\log _a
3}\cdot \frac{\log _a 5}{\log _a 4}\cdots \frac{\log _a 512}{\log
_a 511}
\\ & = & \frac{\log _a 512}{\log _a 2}. \end{array}$$But
$$\frac{\log _a 512}{\log _a 2} = \log_2 512 = \log _2 2^9 = 9,$$so the integer sought is 9.
\begin{exa} Simplify $$S =  \log \tan 1^\circ + \log \tan 2^\circ + \log \tan 3^\circ +
\cdots + \log \tan 89^\circ .$$ \end{exa} Solution: Observe that
$(90 - k)^\circ + k^\circ = 90^\circ$. Thus adding the $k$th term
to the  $(90 - k)$th term, we obtain
$$\begin{array}{lcl}
S & = & \log (\tan 1^\circ)(\tan 89 ^\circ) + \log (\tan 2^\circ )(\tan 88^\circ ) \\
& & \qquad + \log (\tan 3^\circ)(\tan 87^\circ ) + \cdots + \log
(\tan 44^\circ )(\tan 46^\circ ) + \log \tan 45^\circ .
\end{array}$$As
$\tan k^\circ = 1/ \tan (90 - k)^\circ$, we get
$$S = \log 1 + \log 1 + \cdots + \log 1 + \log \tan 45^\circ .$$ Finally, as
$\tan 45^\circ = 1,$ we gather that
$$S = \log 1 + \log 1 + \cdots + \log 1 = 0.$$
\begin{exa} Which is greater $\log _5 7$ or $\log _{8} 3$?\end{exa}
Solution: Clearly $\log _5 7 > 1 > \log _8 3$.
\begin{exa}
Solve the system
$$5\left(\log _x y + \log _y x\right) = 26$$
$$xy = 64$$
\end{exa}
Solution: Clearly we need $x > 0, y > 0, x \neq 1, y \neq 1.$ The
first equation may be written as $\dis{5\left(\log _x y +
\frac{1}{\log _x y}\right) = 26}$ which is the same as $(\log _x y
- 5)(\log _y x - \frac{1}{5}) = 0$. Thus the system splits into
the two equivalent systems (I) $\log _x y = 5, xy = 64$ and (II)
$\log _x y = 1/5, xy = 64.$ Using the conditions $x > 0, y > 0, x
\neq 1, y \neq 1$ we obtain the two sets of solutions $x = 2, y =
32$ or $x = 32, y = 2.$

\begin{exa} Let $\llfloor x \rrfloor$ be the unique integer satisfying
$x - 1 < \llfloor x \rrfloor \leq x.$ For example $\llfloor 2.9
\rrfloor = 2, \llfloor -\pi \rrfloor = -4.$ Find
$$\llfloor \log _2 1 \rrfloor + \llfloor \log _2 2 \rrfloor + \llfloor \log _2 3 \rrfloor
+ \cdots + \llfloor \log _2 1000 \rrfloor . $$
\end{exa} Solution: First observe that $2^9 = 512 < 1000 < 1024 = 2^{10}.$
We decompose the interval $[1; 1000]$ into dyadic blocks
$$[1; 1000] = [1; 2[ \ \bigcup \ [2; 2^2[ \ \bigcup \ [2^2; 2^3[ \ \bigcup \  \cdots
\ \bigcup \ [2^8; 2^9[ \ \bigcup\ [2^9; 1000]. $$ If $x \in [2^k,
2^{k + 1}[$ then  $\llfloor \log _2 x \rrfloor = k.$ If $a, b$ are
integers, the interval $[a; b[$ contains $b - a$ integers. Thus
$$\begin{array}{lll}
\llfloor \log _2 1 \rrfloor + \llfloor \log _2 2 \rrfloor + \llfloor
\log _2 3 \rrfloor + \cdots + \llfloor \log _2 1000 \rrfloor  & = &
(2^1 - 2^0)0 +
(2^2 - 2^1)1\\ & & \qquad + (2^3 - 2^2)2 + \cdots \\
& & \qquad + (2^9 - 2^8)8\\ & & \qquad + (1000 - 2^9)9 \\
& = & 0 + 2\cdot 1 +  4\cdot 2 + 8\cdot 3\\ & & \qquad + 16\cdot 4 + 32\cdot 5 + \\
& & \qquad + 64\cdot 6 + 128\cdot 7 \\ \qquad & & + 256\cdot 8 + 489\cdot 9 \\
& = & 7987  \\
\end{array}
$$
(the last interval has $1000 - 512 + 1 = 489$ integers).
\section*{Practice}\addcontentsline{toc}{section}{Practice}\markright{Practice}\begin{multicols}{2}\columnseprule 1pt \columnsep 25pt\multicoltolerance=900

\begin{pro} Find the exact value of
$$\begin{array}{l}\frac{1}{\log _2 1996!} + \frac{1}{\log _3 1996!} +\frac{1}{\log _4 1996!}\\ \quad +
\cdots + \frac{1}{\log _{1996} 1996!}.\end{array}$$
\begin{answer}
$1$

\end{answer}
\end{pro}
\begin{pro}
Shew that $\dis{\log _{1/2} x > \log _{1/3} x}$ only when $0 < x <
1$.
\end{pro}
\begin{pro}
Prove that $\log _3 \pi + \log _\pi 3 > 2.$
\end{pro}
\begin{pro}
Let $a > 1.$ Shew that $\dis{\frac{1}{\log _a x} > 1}$ only when
$1 < x < a.$
\end{pro}
\begin{pro}
Let $A = \log _6 16, B = \log _{12} 27$. Find integers $a, b, c$
such that $(A + a)(B + b) = c$.
\end{pro}
\begin{pro} Solve the equation
$$\log _{1/3}\left( \cos x + \frac{\sqrt{5}}{6}\right) + \log _{1/3} \left(\cos x - \frac{\sqrt{5}}{6}\right) = 2.$$\end{pro}
\begin{pro} Solve
$$\log _2 x + \log _4 y + \log _4 z = 2,$$
$$\log _3 x + \log _9 y + \log _9 z = 2,$$
$$\log _4 x + \log _{16} y + \log _{16} z = 2.$$
\end{pro}
\begin{pro} Solve the equation
$$x^{0.5\log _{\sqrt{x}} (x^2 - x)} = 3^{\log _9 4}.$$\end{pro}
\begin{pro}
Given that $\log _{ab} a = 4$, find
$$\log _{ab} \frac{\sqrt[3]{a}}{\sqrt{b}}.$$
\end{pro}\end{multicols}
\section{Complex Numbers} We use the symbol $i$ to
denote $i = \sqrt{-1}$. Then $i^2 = -1.$ Clearly $i^0 = 1, i^1 =
1, i^2 = -1, i^3 = -i, i^4 = 1, i^5 = i,$ etc., and so the powers
of $i$ repeat themselves cyclically in a cycle of period 4.
\begin{exa}
Find $i^{1934}$.
\end{exa}
Solution: Observe that $1934 = 4(483) + 2$ and so $i^{1934} = i^2
= -1$.





Complex numbers occur naturally in the solution of quadratic
equations.
\begin{exa} Solve $2x^2 + 6x + 5 = 0$ \end{exa}
Solution: Completing squares,
$$
\begin{array}{lcl}
2x^2 + 6x + 5 & = & 2x^2 + 6x + \frac{9}{2} + \frac{1}{2} \\
& = & (\sqrt{2}x + \frac{{3}}{\sqrt{2}})^2 - (i\frac{1}{\sqrt{2}})^2 \\
& = & (\sqrt{2}x + \frac{{3}}{\sqrt{2}} - i\frac{{1}}{\sqrt{2}})
(\sqrt{2}x + \frac{{3}}{\sqrt{2}} + i\frac{{1}}{\sqrt{2}}). \\
\end{array}$$
Then  $x = -\frac{3}{{2}} \pm i\frac{1}{{2}}$.

If $a, b$ are real numbers then the object $a + bi$ is called a
{\em complex number}. If $a + bi, c + di$ are complex numbers,
then the sum of them is naturally defined as
\begin{equation} (a + bi) + (c + di) = (a + c) + (b + d)i \end{equation}



The product of $a + bi$ and $c + di$ is obtained by multiplying
the binomials:
\begin{equation}
(a + bi)(c + di) = ac + adi + bci + bdi^2 = (ac - bd) + (ad + bc)i
\end{equation}

\begin{df}
If $a, b$ are real numbers, then the {\em conjugate} $\overline{a
+ bi}$ of  $a + bi$ is defined by
\begin{equation}
\overline{a + bi} = a - bi
\end{equation} The {\em norm} $|a + bi|$ of $a + bi$ is
defined by
\begin{equation}
|a + bi| = \sqrt{(a + bi)(\overline{a + bi})} = \sqrt{a^2 + b^2}
\end{equation}
\end{df}



\begin{exa}
Find $|7 + 3i|$.
\end{exa}
Solution: $|7 + 3i| = \sqrt{(7 + 3i)(7 - 3i)} = \sqrt{7^2 + 3^2} =
\sqrt{58}$.
\begin{exa}
Express the quotient $\dis{\frac{2 + 3i}{3 - 5i}}$ in the form $a
+ bi$.
\end{exa}
Solution: We have
$$
\frac{2 + 3i}{3 - 5i} = \frac{2 + 3i}{3 - 5i}\cdot \frac{3 + 5i}{3
+ 5i} = = \frac{-9 + 19i}{34} = \frac{-9}{34} + \frac{19i}{34}$$



If $z_1, z_2$ are complex numbers, then their norms are
multiplicative.
\begin{equation} |z_1z_2| = |z_1||z_2|\end{equation}

\begin{exa}
Write $(2^2 + 3^2)(5^2 + 7^2)$ as the sum of two squares.
\end{exa}
Solution: The idea is to write $2^2 + 3^2 = |2 + 3i|^2, 5^2 + 7^2
= |5 + 7i|^2$ and use the multiplicativity of the norm.
Now \\
\begin{center}
\begin{tabular}{lll}
$(2^2 + 3^2)(5^2 + 7^2)$ &  $=$ &  $|2 + 3i|^2|5 + 7i|^2$ \\
&  $=$ & $|(2 + 3i)(5 + 7i)|^2$ \\
&  $=$ & $|-11 + 29i|^2$ \\
& $=$ & $11^2 + 29^2$
\end{tabular}
 \end{center}
\begin{exa}
Find the roots of $x^3 - 1 = 0$.

\end{exa}
Solution: $x^3 - 1 = (x - 1)(x^2 + x + 1)$. If $x \neq 1$, the two
solutions to $x^2 + x + 1 = 0$ can be obtained using the quadratic
formula, getting $\dis{x = 1/2 \pm i\sqrt{3}/2}$. Traditionally
one denotes $\omega = 1/2 + i\sqrt{3}/2$ and hence $\omega ^2 =
1/2 - i\sqrt{3}/2$. Clearly $\omega ^3 = 1$ and $\omega ^2 +
\omega + 1 = 0.$

\begin{exa}[AHSME 1992]
Find the product of the real parts of the roots of $z^2 - z = 5 -
5i.$
\end{exa}
Solution: By the quadratic formula,
\renewcommand{\arraystretch}{1.8}
$$
\begin{array}{lll}
z & = & \frac{1}{2} \pm \frac{1}{2}\sqrt{21 - 20i} \\
& = & \frac{1}{2} \pm \frac{1}{2}\sqrt{21 - 2\sqrt{-100}} \\
& = & \frac{1}{2} \pm \frac{1}{2}\sqrt{25 - 2\sqrt{(25)(-4)} - 4} \\
& = & \frac{1}{2} \pm \frac{1}{2}\sqrt{(5 - 2i)^2} \\
& = & \frac{1}{2} \pm \frac{5 - 2i}{2} \\

\end{array}
$$
The roots are thus $3 - i$ and $-2 + i$. The product of their real
parts is therefore
$(3)(-2) = -6.$\\
\begin{rem} Had we chosen  to write $21 - 20i = (-5 + 2i)^2$, we
would have still gotten the same values of $z$.
\end{rem}
\clearpage
\section*{Practice}\addcontentsline{toc}{section}{Practice}\markright{Practice}\begin{multicols}{2}\columnseprule 1pt \columnsep 25pt\multicoltolerance=900

\begin{pro} Simplify
$$\frac{(1 + i)^{2004}}{(1 - i)^{2000}}.$$
\begin{answer}
Observe that $(1+i)^{2004} = ((1+i)^2)^{1002} = (2i)^{1002}$, etc.
\end{answer}

\end{pro}
\begin{pro} Prove that
$$\begin{array}{l}1 + 2i + 3i^2 + 4i^{3} \\
\quad + \cdots + 1995i^{1994} + 1996i^{1995}  \\  \qquad  =  -998 -
998i.\end{array}$$
\begin{answer}
Group the summands in groups of four terms and observe that
$$
\begin{array}{lll}ki^{k+1} + (k+1)i^{k+2}+ & & \\ \qquad (k+2)i^{k+3}+ (k+4)i^{k+4} & &  \\  &  =  & i^{k+1}(k
+(k+1)i -(k+2)-(k+3)i) \\  & =  & -2-2i. \end{array}$$
\end{answer}

\end{pro}
\begin{pro} Let
$$(1 + x + x^2)^{1000} = a_0 + a_1x + \cdots + a_{2000}x^{2000}.$$Find
$$a_0 + a_4 + a_8 + \cdots + a_{2000}.$$
\begin{answer}
If $k$ is an integer, $i^k + i^{k + 1} + i^{k + 2} + i^{k + 3} =
i^{k}(1 + i + i^2 + i^3) = 0$.
\end{answer}
\end{pro}

\end{multicols}
\chapter{Arithmetic}
\section{Division Algorithm}
\begin{df}
If $a \neq 0, b$ are integers, we say that $a$ {\em divides} $b$
if there is an integer $c$ such that $ac = b.$ We write this as
$a|b.$ \end{df} If $a$ does not divide $b$ we write $a\not |b.$ It
should be clear that if $a|b$ and $b \neq 0$ then $1 \leq |a| \leq
|b|.$



\begin{thm} The following are properties of divisibility.\begin{itemize}
\item If $c$ divides $a$ and $b$ then $c$ divides any linear
combination of $a$ and $b$. That is, if $a, b, c, m, n$ are
integers with $c|a, c|b$, then $c|(am + nb)$. \item Division by an
integer is transitive. That is, if $x, y, z$ are integers with
$x|y , y|z$ then $x|z$.
\end{itemize} \end{thm}
\begin{pf} There are integers $s, t$ with $sc = a, tc = b$. Thus
$$am + nb = c(sm + tn), $$giving $c|(am + bn).$ Also, there are integers $u, v$ with $xu = y, yv = z.$ Hence $xuv = z$, giving
$x|z$. \end{pf}




\par A very useful property of the integers is the following:\\

\begin{thm}[Division Algorithm] Let  $a, b$ be integers, $b > 0$. There exist unique integers
 $q$ and $r$ satisfying
\begin{equation}
a = bq + r, \ \ 0 \leq r < b \end{equation}\end{thm}
\begin{pf}
The set $S = \{a - bs: s\in\BBZ, b - as \geq 0\}$ is non-empty,
since $a - b(-a^2) \geq 0$. Since $S$ is a non-empty set of
non-negative integers, it must contain a least element, say $r = a
- bq.$ To prove uniqueness, assume $a = bq + r = bq' + r'$ with $0
\leq r' < b$. Then $b(q - q') = r' - r$. This means that $b|(r' -
r)$. Since $0 \leq |r' - r| < b$, we must have $r' = r$. But this
also implies $q = q'$.
\end{pf}


For example, $39 = 4\cdot 9 + 3$. The Division Algorithm thus
discriminates integers according to the remainder they leave upon
division by $a$. For example, if $a = 2$, then according to the
Division Algorithm, the integers may be decomposed into the two
families
$$A_0 = \{  \ldots -4, -2, 0, 2, 4, \ldots \},$$
$$A_1 = \{ \ldots, -5, -3, -1, 1, 3, 5, \ldots \}.$$


Therefore, all integers have one of the forms $2k$ or $2k + 1.$ We
mention in passing that every integer of the form  $2k + 1$ is
also of the form  $2t - 1,$ for $2k + 1 = 2(k + 1) - 1,$ so it
suffices to take $t = k + 1$.



If $a = 4$ we may decompose the integers into the four families
$$B_0 = \{\ldots , -8, -4, 0, 4, 8, \ldots \},$$
$$B_1 = \{  \ldots , -7, -3, 1, 5, 9, \ldots \},$$
$$B_2 = \{  \ldots , -6, -2, 2, 6, 10, \ldots \},$$
$$B_3 = \{  \ldots , -5, -1, 3, 7, 11, \ldots \}.$$
Therefore any integer will take one of the forms $4k, 4k + 1, 4k +
2$ or $4k + 3.$ Again, any integer of the form $4k + 1$ is also of
the form $4t - 3$ and any integer of the form  $4k + 3$ is also of
the form $4t - 1$.


\begin{exa} Shew that the square of any integer is of the form $4k$ or of the form
$4k + 1$. That is, the square of any integer is either divisible
by $4$ or leaves remainder $1$ upon division by $4$.\end{exa}
Solution: If $n$ is even, that is $ n = 2a,$ then $n^2 =(2a)^2
=4a^2,$ which is of the form $4k.$ If $n$ is odd, say $n = 2t +
1,$ then  $n^2 = (2t + 1)^2 = 4(t^2 + t) + 1,$ which is of the
form $4k + 1$.
\begin{exa} Shew that no integer in the sequence
$$11, 111, 1111, 11111, \ldots$$ is a perfect square.
\end{exa}
Solution: Clearly 11 is not a square, so assume,  that an integer
of this sequence has
 $n > 2$ digits. If $n > 2$,
$$\underbrace{11\ldots 1}_{n \ 1{\rm 's}} = \underbrace{11\ldots 11}_{n - 2 \ 1{\rm 's}}00 + 12 - 1
= 100 \cdot \underbrace{11\ldots 11}_{n - 2 \ 1{\rm 's}} + 12 - 1.
$$ Hence any integer in this sequence is of the form  $4k - 1$. By
the preceding problem, no integer of the form $4k - 1$ can be a
square. This finishes the proof.
\begin{exa} Shew that $n^2 + 23$ is divisible by  $24$ for infinitely many values of  $n$.
\end{exa}
Solution: Observe that $n^2 + 23 = n^2 - 1 + 24 = (n - 1)(n + 1) +
24$. Therefore the families of integers $n = 24m \pm 1, m = 0, \pm
1, \pm 2, \pm 3, \ldots$ produce infinitely many values such that
$n^2 + 23$ is divisible by 24.
\begin{exa} Shew that the square of any prime greater than $3$ leaves remainder
$1$ upon division by $12.$\end{exa} Solution: If $p > 3$ is prime,
then $p$ is of one of the forms $6k \pm 1$.



Now,
$$(6k \pm 1)^2 = 12(3k^2 \pm k) + 1,$$
proving the assertion.


\begin{exa}
Prove that if $p$  is a prime, then one of   $8p - 1$ and  $8p +
1$ is a prime and the other is composite.
\end{exa}
Solution: If $p = 3, \  8p - 1 = 23$ and $8p + 1 = 25,$ then the
assertion is true for $p = 3$. If $p > 3$, then either  $p = 3k +
1$ or $p = 3k + 2.$ If $p = 3k + 1, \ 8p - 1 = 24k - 7$ and $8p +
1 = 24k - 6,$ which is divisible by 6 and hence not prime. If $p =
3k + 2, \ 8p - 1 = 24k - 15$ is not a prime, .
\begin{exa}
Shew that if $3n + 1$ is a square, then $n + 1$ is the sum of
three squares.
\end{exa}
Solution: Clearly $3n + 1$ is not a multiple of 3, and so $3n + 1
= (3k \pm 1)^2$. Therefore
$$n + 1 = \frac{(3k \pm 1)^2 - 1}{3} + 1 = 3k^2 \pm 2k + 1 = k^2 + k^2 + (k \pm 1)^2,$$
as we wanted to shew.
\begin{exa}[AHSME 1976]
Let  $r$ be the common remainder when $1059, 1417$ and $2312$ are
divided by $d > 1.$ Find $d - r$.
\end{exa}
Solution: By the division algorithm there are integers $q_1, q_2,
q_3$ with $1059 = dq_1 + r, 1417 = dq_2 + r$ and $2312 = dq_3 +
r$. Subtracting we get $1253 = d(q_3 - q_1), 895 = d(q_3 - q_2)$
and $358 = d(q_2 - q_1)$. Notice that $d$ is a common divisor of
$1253, 895,$ and $358$. As $1253 = 7\cdot 179,\ $ $895 = 5 \cdot
179,$ and  $358 = 2\cdot 179$, we see that 179 is the common
divisor greater than 1 of all three quantities, and so $d = 179.$
Since $1059 = 179q_1 + r,$ and $1059 = 5\cdot 179 + 164,$ we
deduce that $r = 164.$ Finally, $d - r = 15.$
\begin{exa} Shew that from any three integers, one can always choose two so that $a^3 b - ab^3$
is divisible by $10$. \end{exa} Solution: It is clear that $a^3 b -
ab^3 = ab(a - b)(a + b)$ is always even, no matter which integers
are substituted. If one of the three integers is of the form $5k$,
then we are done. If not, we are choosing three integers that lie in
the residue classes $5k \pm 1$ or $5k \pm 2$. By the Pigeonhole
Principle, two of them must lie in one of these two groups, and so
there must be two whose sum or whose difference is divisible by 5.
The assertion follows.
\section*{Practice}\addcontentsline{toc}{section}{Practice}\markright{Practice}
\begin{multicols}{2}\columnseprule 1pt \columnsep 25pt\multicoltolerance=900
\begin{pro}  Find all positive integers $n$ for which $$ n + 1|n^2 + 1.$$ \end{pro}
\begin{pro}If $7|3x + 2$ prove that $7|(15x^2 - 11x -14.)$.\end{pro}
\begin{pro}
Shew that the square of any integer is of the form $3k$ or $3k +
1.$
\end{pro}
\begin{pro} Prove that if $3|(a^2 + b^2 )$, then $3|a$ and $3|b$
\begin{answer}
Argue by contradiction. Assume $a = 3k \pm 1$ or $b = 3m \pm 1$.
\end{answer}

\end{pro}
\begin{pro}
Shew that if the sides of a right triangle are all integers, then
$3$ divides one of the lengths of a side.
\end{pro}
\begin{pro} Given that $5$ divides $(n + 2)$, which of the following are
divisible by $5$
$$ n^2 - 4, \ n^2 + 8n + 7, \ n^4 - 1, \ n^2 - 2n ?$$\end{pro}
\begin{pro} Prove that there is no prime triplet of the form $p, p
+ 2, p + 4$, except for $3, 5, 7.$\end{pro}
\begin{pro}
Find the largest positive integer $n$ such that
$$(n + 1)(n^4 + 2n) + 3(n^3 + 57)$$be divisible by $n^2 + 2.$
\begin{answer}
 $13$
\end{answer}
\end{pro}

\begin{pro}
Demonstrate that if n is a positive integer such that $2n + 1$ is
a square, then $n + 1$ is the sum of two consecutive squares.
\end{pro}
\begin{pro}
Shew that the product of two integers of the form $4n + 1$ is
again of this form. Use this fact and an argument by contradiction
similar to Euclid's to prove that there are infinitely many primes
of the form $4n - 1$.\end{pro}
\begin{pro}
Prove that there are infinitely many primes of the form $6n - 1.$
\end{pro}
\begin{pro}
Prove that there are infinitely many primes $p$ such that $p - 2$
is not prime.
\end{pro}
\begin{pro}Demonstrate that there are no three consecutive odd integers such that
each is the sum of two squares greater than zero.\end{pro}
\begin{pro} Let $n > 1$ be a positive integer. Prove that if one of the
numbers $2^n - 1, 2^n + 1$ is prime, then the other is composite.
\end{pro}
\begin{pro} Prove that there are infinitely many integers $n$ such that
$4n^2 + 1$ is divisible by both $13$ and $5$.\end{pro}
\begin{pro} Prove that any integer $n > 11$ is the sum of two
positive composite numbers.
\begin{answer}
Think of $n - 6$ if $n$ is even and $n - 9$ if $n$ is odd.
\end{answer}

\end{pro}
\begin{pro}
Prove that $3$ never divides $n^2 + 1.$\end{pro}

\begin{pro}  Shew the existence of infinitely many natural numbers
$x, y$ such that $x(x + 1)|y(y + 1)$ but
$$ x\not |y \,\, {\rm and}\,\, (x + 1)\not |y, $$and also
$$ x\not |(y + 1)\,\, {\rm and}\,\, (x + 1) \not |(y + 1).$$
\begin{answer}
Try $x = 36k + 14, y = (12k + 5)(18k + 7)$.
\end{answer}

\end{pro}
\end{multicols}
\section{The Decimal Scale} Any natural number $n$
can be written in the form
$$
n = a_010^k + a_110^{k - 1} + a_210^{k - 2} + \cdots + a_{k - 1}10
+ a_k
$$
where $1 \leq a_0 \leq 9, 0 \leq a_j \leq 9, j \geq 1.$ This is
the {\em decimal} representation of $n$. For example
$$65789 =
6\cdot 10^4 + 5\cdot 10^3 + 7 \cdot 10^2 + 8 \cdot 10 + 9.$$
\begin{exa}
Find a reduced fraction equivalent to the repeating decimal
$0.\overline{123} = 0.123123123\ldots .$
\end{exa}
Solution: Let $N = 0.123123123\ldots.$ Then $1000N =
123.123123123\ldots .$ Hence $1000N - N = 123,$ whence $\dis{N =
\frac{123}{999} = \frac{41}{333}}$.
\begin{exa}
What are all the two-digit positive integers in which the
difference between the integer and the product of its two digits
is $12$?
\end{exa}
Solution: Let such an integer be $10a + b$, where $a, b$ are
digits. Solve $10a + b - ab = 12$ for $a$ getting $$a = \frac{12 -
b}{10 - b} = 1 + \frac{2}{10 - b}.$$Since $a$ is an integer, $10 -
b$ must be a positive integer that divides 2. This gives  $b = 8,
a = 2$ or $b  = 9, a = 3.$ Thus 28 and 39 are the only such
integers.

\begin{exa} Find all the integers with initial digit $6$ such that if this initial integer is
suppressed, the resulting number is $1/25$ of the original
number.\end{exa} Solution: Let $x$ be the integer sought. Then $x
= 6\cdot 10^n + y$ where $y$ is a positive integer. The given
condition stipulates that
$$ y = \frac{1}{25}\left(6\cdot 10^n + y\right) ,$$that is,
$$y = \frac{10^{n}}{4} = 25\cdot 10^{n - 2}.$$This requires $n \geq 2$, whence
$y = 25, 250, 2500, 25000,$ etc.. Therefore $x = 625, 6250, 62500,
625000,$ etc..
\begin{exa}[IMO 1968] Find all natural numbers $x$ such that the product of
their digits (in decimal notation) equals $x^2 - 10x - 22$.
\end{exa}
Solution: Let $x$ have the form
$$x = a_0 + a_110 + a_210^2 + \cdots + a_{n}10^n, \ a_k \leq 9, \ a_n \neq 0.$$
Let $P(x)$ be the product of the digits of $x, P(x) = x^2 - 10x -
22.$ Now $P(x) = a_0a_1 \cdots a_n \leq 9^na_n < 10^na_n \leq x$
(strict inequality occurs when $x$ has more than one digit). This
means that $x^2 - 10x - 22 \leq x$ which entails that $x < 13$,
whence $x$ has one digit or $x = 10, 11$ or $12.$  Since $x^2 -
10x - 22 = x$ has no integral solutions, $x$ cannot have one
digit. If $x = 10, P(x) = 0,$ but $x^2 - 10x - 22 \neq 0.$ If $x =
11, P(x) = 1,$ but $x^2 - 10x - 22 \neq 1.$ The only solution is
seen to be $x = 12$.
\begin{exa}
A whole number decreases an integral number of times when its last
digit is deleted. Find all such numbers.
\end{exa}
Solution: Let $0 \leq y \leq 9,$ and $10x + y = mx$, where $m, x$
are natural numbers. This requires $\dis{10 + \frac{y}{x} = m},$
an integer. Hence, $x$ must divide $y$. If $y = 0,$ any natural
number $x$ will do, as we obtain multiples of 10. If $y = 1$ then
$x = 1$, and we obtain $11.$ Continuing in this fashion, the
sought number are the multiples of 10, together with the numbers
11, 12, 13, 14, 15, 16, 17, 18, 19, 22, 24, 26, 28, 33, 36, 39,
44, 55, 77, 88, and 99.
\begin{exa} Shew that all integers in the sequence
$$49, 4489, 444889, 44448889, \underbrace{44\ldots 44}_{n \ 4{\rm 's}}\underbrace{88 \ldots 88}_{n - 1 \ 8{\rm 's}}9 $$
are perfect squares.
\end{exa}
Solution: Observe that
$$
\begin{array}{lcl}\underbrace{44\ldots 44}_{n \ 4{\rm 's}}\underbrace{88 \ldots 88}_{n - 1 \ 8{\rm 's}}9
& =  & \underbrace{44\ldots 44}_{n \ 4{\rm 's}}\cdot 10^{n} + \underbrace{88 \ldots 88}_{n - 1 \ 8{\rm 's}}\cdot 10 + 9 \\
& = & \frac{4}{9}\cdot (10^{n} - 1)\cdot 10^{n} + \frac{8}{9}\cdot (10^{n - 1} - 1)\cdot 10 + 9 \\
& = & \frac{4}{9}\cdot 10^{2n}  + \frac{4}{9}\cdot 10^n +  \frac{1}{9}  \\
& = & \frac{1}{9}\left( 2\cdot 10^n  + 1 \right)^2  \\
& = & \left(\frac{2\cdot 10^n + 1}{3}\right)^2
\end{array}$$
We must shew that this last quantity is an integer, that is, that
3 divides $2\cdot 10^n + 1 = 2\underbrace{00\ldots 00}_{n - 1 \
0{\rm 's}}1$. But the sum of the digits of this last quantity is
3, which makes it divisible by 3. In fact, $\dis{\frac{2\cdot 10^n
+ 1}{3} = \underbrace{6\ldots 6}_{n - 1 \ 6{\rm 's}}7}$
\begin{exa}[AIME 1987] An ordered pair $(m, n)$ of non-negative integers is
called {\em simple} if the addition $m + n$ requires no carrying.
Find the number of simple ordered pairs of non-negative integers
that add to $1492$.
\end{exa}
Solution: Observe that there are $d + 1$ solutions to $x + y = d,$
where $x, y$ are positive integers and $d$ is a digit. These are

$$(0 + d), \ (1 + d - 1), \ (2 + d - 2), \ldots , \ (d + 0) $$
Since there is no carrying, we search for the numbers of solutions
of this form to $x + y = 1,$ $u + v = 4,$ $s + t = 9,$ and $a + b
= 2$. Since each separate solution may combine with any other, the
total number of simple pairs is
$$(1 + 1)(4 + 1)(9 + 1)(2 + 1) = 300.$$
\begin{exa}[AIME 1992] For how many pairs of consecutive integers in
$$\{1000, 1001, \ldots , 2000\}$$is no carrying required when the two integers are added?
\end{exa}
Solution: Other than 2000, a number on this list has the form $n =
1000 + 100a + 10b + c,$ where $a, b, c$ are digits. If there is no
carrying in $n + n + 1$ then $n$ has the form
$$1999, \ 1000 + 100a + 10b + 9, \ 1000 + 100a + 99, 1000 + 100a + 10b + c$$with
$0 \leq a, b, c \leq 4$, i.e., five possible digits. There are
$5^3 = 125$ integers of the form $1000 + 100a + 10b + c, 0 \leq a,
b, c \leq 4$, $5^2 = 25$ integers of the form  $1000 + 100a + 10b
+ 9, 0 \leq a, b \leq 4$, and 5 integers of the form $1000 + 100a
+ 99, 0 \leq a \leq 4$. The total of integers sought is thus $125
+ 25 + 5 + 1 = 156.$

\begin{exa}[AIME 1994] Given a positive integer $n$, let $p(n)$ be
the product of the non-zero digits of $n$. (If $n$ has only one
digit, then $p(n)$ is equal to that digit.) Let
$$ S = p(1) + p(2) + \cdots + p(999).$$ Find $S$. \end{exa}
Solution:  If $x = 0$, put $m(x) = 1$, otherwise put $m(x) = x.$
We use three digits to label all the integers, from 000 to 999 If
$a, b, c$ are digits, then clearly $p(100a + 10b + c) =
m(a)m(b)m(c).$ Thus
$$
\begin{array}{lll}
p(000) +  p(001) + p(002) + \cdots + p(999)
& = & m(0)m(0)m(0) + m(0)m(0)m(1)\\
&& + m(0)m(0)m(2) + \cdots + m(9)m(9)m(9) \\
 & = & (m(0) + m(1) + \cdots + m(9))^3 \\
& = & (1 + 1 + 2 + \cdots + 9)^3 \\
& = & 46^3 \\
& = & 97336.  \\
\end{array}
$$

Hence$$ \begin{array}{lll} S &  = & p(001) + p(002) + \cdots +
p(999)\\
 & = &  97336 - p(000)\\
 & = & 97336 - m(0)m(0)m(0)\\
 & = &  97335.\\
 \end{array}$$

\begin{exa}[AIME 1992] Let $S$ be the set of all rational numbers $r$, $0 < r < 1,$
that have a repeating decimal expansion of the form
$$0.abcabcabc\ldots = 0.\overline{abc},$$where the digits $a, b, c$ are not necessarily distinct. To
write the elements of $S$ as fractions in lowest terms, how many
different numerators are required?
\end{exa}
Solution: Observe that $0.abcabcabc\ldots = \dis{\frac{abc}{999}}$,
and that $999 = 3^3\cdot 37.$ If $abc$ is neither divisible by $3$
nor by $37$, the fraction is already in lowest terms. By
Inclusion-Exclusion there are
$$999 - \left(\frac{999}{3} + \frac{999}{37}\right) + \frac{999}{3\cdot 37} = 648$$
such fractions. Also, fractions of the form $\dis{\frac{s}{37}}$
where  $s$ is divisible by $3$ but not by $37$ are in $S$. There are
12 fractions of this kind (with s = 3, 6, 9, 12, \ldots , 36). We do
not consider fractions of the form $\dis{\frac{l}{3^t}, t \leq 3}$
with $l$ divisible by $37$ but not by $3$, because these fractions
are $> 1$ and hence not in $S$. The total number of distinct
numerators in the set of reduced fractions is thus $640 + 12 = 660.$

\clearpage

\section*{Practice}\addcontentsline{toc}{section}{Practice}\markright{Practice}\begin{multicols}{2}\columnseprule 1pt \columnsep 25pt\multicoltolerance=900
\begin{pro} Find an equivalent fraction for the repeating decimal $0.31\overline{72}$.\end{pro}
\begin{pro} A two-digit number is divided by the sum of its digits. What is the largest possible remainder?
\end{pro}
\begin{pro} Shew that the integer
$$\underbrace{11\ldots 11}_{221 \ 1{\rm 's}}$$ is a composite number.
\end{pro}

\begin{pro}
Let $a$ and $b$ be the integers
$$a = \underbrace{111\ldots 1}_{m \ 1{\rm 's}}$$
$$b = 1\underbrace{000\ldots 0}_{m - 1 \ 0{\rm 's}}5.$$Shew that
$ab + 1$ is a perfect square.
\end{pro}
\begin{pro}  What digits appear on the product
$$ \underbrace{3\ldots 3}_{666 \ 3{\rm 's}}\cdot\underbrace{6\ldots 6}_{666 \ 6{\rm 's}} ?$$\end{pro}
\begin{pro}
Shew that there exist no integers with the following property: if
the initial digit is suppressed, the resulting integer is $1/35$
of the original number.
\end{pro}
\begin{pro}Shew that the sum of all the integers of  n digits, $n \geq 3$, is
$$494\underbrace{99\ldots 9}_{n - 3 \ 9{\rm 's}}55\underbrace{00\ldots 0}_{n - 2 \ 0{\rm 's}}.$$\end{pro}
\begin{pro}
Shew that for any positive integer  $n$,
$$\underbrace{11\ldots 1}_{2n \ 1{\rm 's}} - \underbrace{22\ldots 2}_{n \ 2{\rm 's}}$$is a
perfect square.
\end{pro}
\begin{pro}
A whole number is equal to the arithmetic mean of all the numbers
obtained from the given number with the aid of all possible
permutation of its digits. Find all whole numbers with that
property.
\end{pro}
\begin{pro}
The integer $n$ is the smallest multiple of $15$ such that every
digit of $n$ is either $0$ or $8$. Compute $\dis{\frac{n}{15}}$.
\end{pro}
\begin{pro}
Shew that {\em Champernowne's number }
$$0.12345678910111213141516171819202122\ldots ,$$ which is the sequence of natural numbers written
after the decimal point, is irrational.
\end{pro}
\begin{pro}
Given that $$\frac{1}{49} =
0.020408163265306122448979591836734693877551,$$ find the last
thousand digits of
$$1 + 50 + 50^2 + \cdots + 50^{999}.$$
\end{pro}

\begin{pro} Let $t$ be a positive real number. Prove that there is a positive integer $n$ such that
the decimal expansion of $nt$ contains a $7$.
\end{pro}
\begin{pro}[AIME 1989] Suppose that $n$ is a positive integer and $d$ is a single
digit in base-ten. Find $n$ if
$$\frac{n}{810} = 0.d25d25d25d25d25\ldots$$
\end{pro}
\begin{pro}[AIME 1988] Find the smallest positive integer whose cube ends in
$888$.
\end{pro}
\begin{pro}[AIME 1986] In the parlour game, the ``magician'' asks one of the
participants to think of a three-digit number $abc$, where $a, b,
c$ represent the digits of the number in the order indicated. The
magician asks his victim to form the numbers
$$acb, bac, cab, cba,$$to add these numbers and to reveal their sum $N$. If told the value of $N$, the
magician can identify $abc$. Play the magician and determine $abc$
if $N = 319$.
\end{pro}
\begin{pro}[AIME 1988] For any positive integer $k$, let $f_1(k)$ denote the square
of the sums of the digits of $k.$ For $n \geq 2,$ let $f_n(k) =
f_1(f_{n - 1}(k)).$ Find $f_{1988}(11).$
\end{pro}
\begin{pro}[IMO 1969] Determine all three-digit numbers $N$ that are divisible
by $11$ and such that $\dis{\frac{N}{11}}$ equals the sum of the
squares of the digits of $N$.
\end{pro}
\begin{pro}[IMO 1962] Find the smallest natural number having the last digit $6$
and if this $6$ is erased and put in from of the other digits, the
resulting number is four times as large as the original number.
\end{pro}
\end{multicols}
\section{Non-decimal Scales} The fact that most
people have ten fingers has fixed our scale of notation to the
decimal. Given any positive integer $r > 1$, we can, however,
express any number  $x$ in base $r$.



If $n$ is a positive integer, and $r > 1$ is an integer, then $n$
has the base-$r$ representation
$$n = a_0 + a_1r + a_2r^2 + \cdots + a_kr^k, \ 0 \leq a_t \leq r - 1, \ a_k \neq 0, \ r^k \leq n < r^{k + 1}.$$


We use the convention that we shall refer to a decimal number
without referring to its base, and to a base-$r$ number by using
the subindex $_r$.

\begin{exa}
Express the decimal number $5213$ in base-seven.
\end{exa}
Solution: Observe that $5213 < 7^5.$ We thus want to find $0 \leq
a_0, \ldots , a_4 \leq 6, a_4 \neq 0$ such that
$$5213 = a_47^4 + a_37^3 + a_27^2 + a_17 + a_0.$$Dividing by $7^4$, we obtain
$2 + $ proper fraction $= a_4 + $ proper fraction. This means that
$a_4 = 2.$ Thus $5213 = 2\cdot7^4 + a_37^3 + a_27^2 + a_17 + a_0$
or $411 = 5213 = a_37^3 + a_27^2 + a_17 + a_0$. Dividing by $7^3$
this last equality we obtain $1 + $ proper fraction $= a_3 + $
proper fraction, and so $a_3 = 1.$ Continuing in this way we
deduce that $5213 = 21125_7.$


The method of successive divisions used in the preceding problem
can be conveniently displayed as
\begin{center}
\begin{tabular}{l|l|l}
7 & 5212 & 5 \\
\hline
7 & 744 & 2 \\
\hline
7 & 106 & 1 \\
\hline
7 & 15 & 1 \\
\hline
7 & 2 & 2 \\

\end{tabular}
 \end{center}
The central column contains the successive quotients and the
rightmost column contains the corresponding remainders. Reading
from the last remainder up, we recover $5213 = 21125_7.$
\begin{exa}
Write $562_7$ in base-five.
\end{exa}
Solution: $562_7 = 5\cdot 7^2 + 6\cdot 7 + 2 = $ in decimal scale,
so the problem reduces to convert $289$ to base-five. Doing
successive divisions,
\begin{center}
\begin{tabular}{l|l|l}
5 & 289 & 4 \\
\hline
5 & 57 & 2 \\
\hline
5 & 11 & 1 \\
\hline
5 & 2 & 2 \\

\end{tabular}
 \end{center}Thus $562_7 = 289 = 2124_5.$
\begin{exa}
Express the fraction $\dis{\frac{13}{16}}$ in base-six.
\end{exa}
Solution: Write
$$\frac{13}{16} = \frac{a_1}{6} + \frac{a_2}{6^2} +  \frac{a_3}{6^3} +  \frac{a_4}{6^4} + \cdots  $$
Multiplying by 6, we obtain $4 + $ proper fraction $ = a_1 + $
proper fraction, so $a_1 = 4.$ Hence
$$\frac{13}{16} - \frac{4}{6} = \frac{7}{48} = \frac{a_2}{6^2} +  \frac{a_3}{6^3} +  \frac{a_4}{6^4} + \cdots $$
Multiply by $6^2$ we obtain $5 + $ proper fraction $= a_2 + $
proper fraction, and so $a_2 = 5.$ Continuing in this fashion
$$\frac{13}{16} = \frac{4}{6} + \frac{5}{6^2} +  \frac{1}{6^3} +  \frac{3}{6^4} = 0.4513_6.$$


We may simplify this procedure of successive multiplications by
recurring to the following display:
\renewcommand{\arraystretch}{2}
$$
\begin{array}{l|l|l}
6 & \frac{13}{16} & 4 \\ \hline
6 & \frac{7}{8} & 5 \\
\hline
6 & \frac{1}{4} & 1 \\
\hline
6 & \frac{1}{2} & 3 \\
\end{array}
$$
The third column contains the integral part of the products of the
first column and the second column. Each term of the second column
from the second on is the fractional part of the product obtained
in the preceding row. Thus $6\cdot\frac{13}{16} - 4 =
\frac{7}{8}$, $6\cdot\frac{7}{8} - 5 = \frac{1}{4}$, etc..
\begin{exa}
Prove that $4.41_r$ is a perfect square in any scale of notation.
\end{exa}
Solution:
$$4.41_r = 4 + \frac{4}{r} + \frac{4}{r^2} = \left(2 + \frac{1}{r}\right)^2$$
\begin{exa}[AIME 1986] The increasing sequence
$$1, 3, 4, 9, 10, 12, 13, \ldots$$consists of all those positive integers which are powers of $3$ or
sums of distinct powers or $3$. Find the hundredth term of the
sequence.
\end{exa}
Solution: If the terms of the sequence are written in base-three,
they comprise the positive integers which do not contain the digit
2. Thus the terms of the sequence in ascending order are
$$1_3, 10_3, 11_3, 100_3, 101_3, 110_3, 111_3, \ldots$$In the {\em binary} scale these numbers are, of course,
the ascending natural numbers $1, 2, 3, 4, \ldots$. Therefore to
obtain the 100th term of the sequence we write 100 in binary and
then translate this into ternary: $100 = 1100100_2$ and $1100100_3
= 3^6 + 3^5 + 3^2 = 981.$
\begin{exa}[AHSME 1993] Given $0 \leq x_0 < 1$, let
$$x_n = \left\{ \begin{array}{lll}
2x_{n - 1} & {\rm if} & 2x_{n - 1} < 1, \\
2x_{n - 1} - 1 & {\rm if} & 2x_{n - 1} \geq 1.\\
\end{array}
\right.$$for all integers $n > 0$. For how many $x_0$ is it true
that $x_0 = x_5$?
\end{exa}
Solution: Write $x_0$ in binary,
$$x_0 = \sum _{k = 1} ^\infty \frac{a_k}{2^k}, \ \ \ a_k = 0 \ \ {\rm or} \ \ 1.$$
The algorithm given moves the binary point one unit to the right.
For $x_0$ to equal $x_5$ we need $(0.a_1a_2a_3a_4a_5a_6a_7\ldots
)_2 = (0.a_6a_7a_8a_9a_{10}a_{11}a_{12}\ldots )_2.$ This will
happen if and only if $x_0$ has a repeating expansion with
$a_1a_2a_3a_4a_5$ as the repeating block. There are $2^5 = 32$
such blocks. But if $a_1 = a_2 = \cdots = a_5 = 1$ then $x_0 = 1$,
which lies outside $]0, 1[$. The total number of values for which
$x_0 = x_5$ is therefore $32 - 1 = 31.$
\section*{Practice}\addcontentsline{toc}{section}{Practice}\markright{Practice}\begin{multicols}{2}\columnseprule 1pt \columnsep 25pt\multicoltolerance=900
\begin{pro}
Express the decimal number $12345$ in every scale from binary to
base-nine.
\end{pro}
\begin{pro}
Distribute the $27$ weights of $1^2, 2^2, 3^2, \ldots , 27^2$ lbs
each into three separate piles, each of equal weight.
\end{pro}
\begin{pro}
Let ${\cal C}$ denote the class of positive integers which, when
written in base-three, do not require the digit $2.$ Prove that no
three integers in ${\cal C}$ are in arithmetic progression.
\end{pro}
\begin{pro}
What is the largest integer that I should be permitted to choose
so that you may determine my number in twenty ``yes'' or ``no''
questions?
\end{pro}
\begin{pro}
Let $\llfloor x \rrfloor$ denote the greatest integer less than or
equal to $x$. Does the equation
$$\llfloor x\rrfloor  + \llfloor 2x\rrfloor + \llfloor 4x\rrfloor + \llfloor 8x\rrfloor + \llfloor 16x\rrfloor
+ \llfloor 32x\rrfloor = 12345$$have a solution?
\end{pro}
\end{multicols}
\section{Well-Ordering Principle} The set $\BBN =
\{0, 1, 2, 3, 4, \ldots  \}$ of natural numbers is endowed with
two operations, addition and multiplication, that satisfy the
following properties for natural number $a, b,$ and $c$:
\begin{enumerate}
\item {\bf Closure:} $a + b$ and $ab$ are also natural numbers,
\item {\bf Commutativity:} $a + b = b + a$ and $ab = ba,$ \item
{\bf Associative Laws:} $(a + b) + c = a + (b + c)$ and $(ab)c =
a(bc)$, \item {\bf Distributive Law:} $a(b + c) = ab + ac$ \item
{\bf Additive Identity:} $0 + a = a.$ \item {\bf Multiplicative
Identity:} $1a = a.$
\end{enumerate}

One further property of the natural numbers is the following.\\

\flushleft{{\bf Well-Ordering Axiom:}} {\em Every non-empty subset
${\cal S}$ of the natural numbers has a least element.}

As an example of the use of the Well-Ordering Axiom let us prove
that there is no integer between 0 and 1.

\begin{exa}
Prove that there is no integer in the open interval $]0; 1[$.
\end{exa}
Solution: Assume to the contrary that the set $\cal{S}$ of
integers in $]0; 1[$ is non-empty. As a set of positive integers,
by Well-Ordering it must contain a least element, say $m$. Since
$0 < m < 1,$ we have $0 < m^2 < m < 1.$ But this last string of
inequalities says that $m^2$ is an integer in $]0; 1[$ which is
smaller than $m$, the smallest integer in $]0; 1[$. This
contradiction shews that $m$ cannot exist.



Recall that an {\em irrational} number is one that cannot be
represented as the ratio of two integers.

\begin{exa}
Prove that $\sqrt{2}$ is irrational.
\end{exa}
Solution: The proof is by contradiction. Suppose that $\sqrt{2}$
were rational, i.e., that $\sqrt{2} = \frac{a}{b}$ for some
integers $a, b, b \neq 0.$ This implies that the set
$${\cal A} = \{ n\sqrt{2}: \ {\rm both} \ n \ {\rm and} \ n\sqrt{2} \ {\rm positive} \ {\rm integers}\}$$
is non-empty since it contains $a.$ By Well-Ordering, ${\cal A}$
has a smallest element, say $j = k\sqrt{2}$. As $\sqrt{2} - 1 >
0,$ $j(\sqrt{2} - 1) = j\sqrt{2} - k\sqrt{2} = \sqrt{2}(j - k)$,
is a positive integer. Since $2 < 2\sqrt{2}$ implies $2 - \sqrt{2}
< \sqrt{2}$ and also $j\sqrt{2} = 2k,$ we see that
$$(j - k)\sqrt{2} = k(2 - \sqrt{2}) < k\sqrt{2} = j.$$Thus $(j - k)\sqrt{2}$ is a positive integer
in ${\cal A}$ which is smaller than $j$. This contradicts the
choice of $j$ as the smallest integer in ${\cal A}$ and hence,
finishes the proof.

\begin{exa}
Let $a, b, c$ be integers such that $a^6 + 2b^6 = 4c^6$. Shew that
$a = b = c = 0.$
\end{exa}
Solution: Clearly we can restrict ourselves to non-negative
numbers. Choose a triplet of non-negative integers $a, b, c$
satisfying this equation and with
$$\max (a, b, c) > 0$$as small as possible. If $a^6 + 2b^6 = 4c^6$, then $a$ must be even,
$a = 2a_1.$ This leads to $32a_1 ^6 + b^6 = 2c^6.$ This implies
that $b$ is even, $b = 2b_1$ and so $16a_1 ^6 + 32b_1 ^6 = c^6$.
This implies that $c$ is even, $c = 2c_1$ and so $a_1 ^6 + 2b_1 ^6
= 4c_1 ^6$. But clearly $\max (a_1, b_1, c_1) < \max (a, b, c)$.
We have produce a triplet of integers with a maximum smaller than
the smallest possible maximum, a contradiction.
\begin{exa}[IMO 1988] If $a, b$ are positive integers such that
$\dis{\frac{a^2 + b^2}{1 + ab}}$ is an integer, then shew that
$\dis{\frac{a^2 + b^2}{1 + ab}}$ must be a square.
\end{exa}
Solution: Suppose that $\dis{\frac{a^2 + b^2}{1 + ab} = k}$ is a
counterexample of an integer which is not a perfect square, with
$\max (a, b)$ as small as possible. We may assume without loss of
generality that $a < b$ for if $a = b$ then
$$0 < k = \frac{2a^2}{a^2 + 1} = 2 - \frac{2}{a^2 + 1} < 2,$$
which forces $k = 1$, a square.

\bigskip

Now, $a^2 + b^2 - k(ab + 1) = 0$ is a quadratic in $b$ with sum of
roots $ka$ and product of roots $a^2 - k.$ Let $b_1, b$ be its
roots, so $b_1 + b = ka, bb_1 = a^2 - k.$

\bigskip

As $a, k$ are positive integers, supposing $b_1 < 0$ is
incompatible with $a^2 + b_1 ^2 = k(ab_1 + 1)$. As $k$ is not a
perfect square, supposing $b_1 = 0$ is incompatible with $a^2 +
0^2 = k(0\cdot a + 1)$. Also
$$b_1 = \frac{a^2 - k}{b} < \frac{b^2 - k}{b} = b - \frac{k}{b} < b.$$Thus we have shewn $b_1$ to
be a positive integer with $\dis{\frac{a^2 + b_1 ^2}{1 + ab_1} = k}$
smaller than $b$. This is a contradiction to the choice of $b$. Such
a counterexample $k$ cannot exist, and so $\dis{\frac{a^2 + b^2}{1 +
ab}}$ must be a perfect square. In fact, it can be shewn that
$\dis{\frac{a^2 + b^2}{1 + ab}}$ is the square of the greatest
common divisor of $a$ and $b$.

\clearpage

\section*{Practice}\addcontentsline{toc}{section}{Practice}\markright{Practice}\begin{multicols}{2}\columnseprule 1pt \columnsep 25pt\multicoltolerance=900
\begin{pro}
Find all integers solutions of $a^3 + 2b^3 = 4c^3$.
\end{pro}
\begin{pro}
Prove that the equality $x^2 + y^2 + z^2 = 2xyz$ can hold for
whole numbers $x, y, z$ only when $x = y = z = 0$.
\end{pro}
\begin{pro}
Shew that the series of integral squares does not contain an
infinite arithmetic progression.
\end{pro}
\begin{pro}
Prove that $x^2 + y^2 = 3(z^2 + w^2)$ does not have a positive
integer solution.
\end{pro}

\end{multicols}




\section{Mathematical Induction} The Principle of
Mathematical Induction is based on the following fairly intuitive
observation. Suppose that we are to perform a task that involves a
certain finite number of steps. Suppose that these steps are
sequential. Finally, suppose that we know how to perform the
$n$-th step provided we have accomplished the $n - 1$-th step.
Thus if we are ever able to start the task (that is, if we have a
base case), then we should be able to finish it (because starting
with the base we go to the next case, and then to the case
following that, etc.).

We formulate the Principle of Mathematical Induction (PMI) as follows: \\
\flushleft{{\bf Principle of Mathematical Induction}} Suppose we
have an assertion $P(n)$ concerning natural numbers satisfying the
following two properties:
\begin{enumerate}
\item[] ({\bf PMI I}) $P(k_0)$ is true for some natural number
$k_0$, \item[] ({\bf PMI II}) If $P(n - 1)$ is true then $P(n)$ is
true.
\end{enumerate}
Then the assertion $P(n)$ is true for every $n \geq k_0.$


\begin{exa}
Prove that the expression $3^{3n + 3} - 26n - 27$ is a multiple of
$169$ for all natural numbers $n$.
\end{exa}
Let $P(n)$ be the assertion ``$3^{3n + 3} - 26n - 27$ is a
multiple of $169$.'' Observe that $3^{3(1) + 3} - 26(1) - 27 = 676
= 4(169)$ so $P(1)$ is true. Assume the truth of $P(n - 1)$, that
is, that  there is an integer $M$ such that
$$3^{3(n - 1) + 3} - 26(n - 1) - 27 = 169M.$$ This entails
$$3^{3n} - 26n - 1 = 169M.$$
Now
$$
\begin{array}{lll}
3^{3n + 3} - 26n - 27 & = & 27\cdot 3^{3n} - 26n - 27 \\
&  = & 27(3^{3n} - 26n - 1) + 676n \\
& = & 27(169M) + 169\cdot 4n \\
& = & 169(27M + 4n),
\end{array}
$$and so  the truth of $P(n - 1)$ implies the truth of $P(n)$. The assertion then follows for all
$n \geq 1$ by PMI.
\begin{exa}
Prove that
$$(1 + \sqrt{2})^{2n} + (1 - \sqrt{2})^{2n}$$ is an even integer and that
$$(1 + \sqrt{2})^{2n} - (1 - \sqrt{2})^{2n} = b\sqrt{2}$$for some positive integer
$b$, for all integers $n \geq 1$.
\end{exa}
Solution: Let $P(n)$ be the assertion: ``$$(1 + \sqrt{2})^{2n} +
(1 - \sqrt{2})^{2n}$$ is an even integer and that
$$(1 + \sqrt{2})^{2n} - (1 - \sqrt{2})^{2n} = b\sqrt{2}$$for some positive integer
$b$.'' We see that $P(1)$ is true since
$$(1 + \sqrt{2})^2 + (1 - \sqrt{2})^2 = 6,$$and
$$(1 + \sqrt{2})^2 - (1 - \sqrt{2})^2 = 4\sqrt{2}.$$Assume now that $P(n - 1)$, i.e., assume that
$$(1 + \sqrt{2})^{2(n - 1)} + (1 - \sqrt{2})^{2(n - 1)} = 2N$$ for some integer $N$ and that
$$(1 + \sqrt{2})^{2(n - 1)} - (1 - \sqrt{2})^{2(n - 1)} = a\sqrt{2}$$for some positive integer
$a$. Consider now the quantity
$$
\begin{array}{lll}
(1 + \sqrt{2})^{2n} + (1 - \sqrt{2})^{2n} & = &
(1 + \sqrt{2})^2(1 + \sqrt{2})^{2n - 2} + (1 - \sqrt{2})^2(1 - \sqrt{2})^{2n - 2} \\
& = & (3 + 2\sqrt{2})(1 + \sqrt{2})^{2n - 2} + (3 - 2\sqrt{2})(1 - \sqrt{2})^{2n - 2} \\
& = & 12N + 4a \\
& = & 2(6n + 2a),
\end{array}
$$an even integer. Similarly
$$
\begin{array}{lll}
(1 + \sqrt{2})^{2n} - (1 - \sqrt{2})^{2n} & = &
(1 + \sqrt{2})^2(1 + \sqrt{2})^{2n - 2} - (1 - \sqrt{2})^2(1 - \sqrt{2})^{2n - 2} \\
& = & (3 + 2\sqrt{2})(1 + \sqrt{2})^{2n - 2} - (3 - 2\sqrt{2})(1 - \sqrt{2})^{2n - 2} \\
& = & 3a\sqrt{2} + 2\sqrt{2}(2N) \\
& = & (3a + 4N)\sqrt{2},
\end{array}
$$which is of the form $b\sqrt{2}$. This implies that $P(n)$ is true. The statement of the
problem follows by PMI.
\begin{exa}
Prove that if $k$ is odd, then $2^{n + 2}$ divides
$$k^{2^n} - 1$$for all natural numbers $n$.

\end{exa}
Solution: The statement is evident for $n = 1,$ as $k^2 - 1 = (k -
1)(k + 1)$ is divisible by 8 for any odd natural number $k$ since
$k - 1$ and $k + 1$ are consecutive even integers. Assume that
$2^{n + 2}a = k^{2^n} - 1$ for some integer $a$. Then
$$k^{2^{n + 1}} - 1 = (k^{2^n} - 1)(k^{2^n} + 1) = 2^{n + 2}a(k^{2^n} + 1).$$Since $k$ is odd,
$k^{2^n} + 1$ is even and so $k^{2^n} + 1 = 2b$ for some integer
$b$. This gives
$$k^{2^{n + 1}} - 1 = 2^{n + 2}a(k^{2^n} + 1)  = 2^{n + 3}ab,$$and so the assertion follows by
PMI.
\begin{exa}
Let $s$ be a positive integer. Prove that every interval $[s, 2s]$
contains a power of $2$.
\end{exa}
Solution: If $s$ is a power of 2, then there is nothing to prove.
If $s$ is not a power of 2 then it must lie between two
consecutive powers of 2, say $2^r < s < 2^{r + 1}.$ This yields
$2^{r + 1} < 2s.$ Hence $s < 2^{r + 1} < 2s,$ which yields the
result.

\begin{df}
The {\em Fibonacci Numbers} are given by $f_0 = 0, \ f_1 = 1, \
f_{n + 1} = f_n + f_{n - 1}, n \geq 1$, that is every number after
the second one is the sum of the preceding two.
\end{df}

The Fibonacci sequence then goes like $0, 1, 1, 2, 3, 5, 8, 13,
21, \ldots .$
\begin{exa}
Prove that for integer $n \geq 1,$
$$f_{n - 1}f_{n + 1} = f_n ^2 + (-1)^{n + 1}.$$
\end{exa}
Solution: If $n = 1,$ then $2 = f_0f_2 = 1^2 + (-1)^2 = f_1 ^2 +
(-1)^{1 + 1}.$ If $f_{n - 1}f_{n + 1} = f_n ^2 + (-1)^{n + 1}$
then using the fact that $f_{n + 2} = f_n + f_{n + 1}$,
$$
\begin{array}{lll}
f_nf_{n + 2} & = & f_n(f_n + f_{n + 1}) \\
& = & f_n ^2 + f_n f_{n + 1} \\
& = & f_{n - 1}f_{n + 1} - (-1)^{n + 1} + f_n f_{n + 1} \\
& = & f_{n + 1}(f_{n - 1} + f_n) + (-1)^{n + 2} \\
& = & f_{n + 1} ^2 + (-1)^{n + 2}, \\

\end{array}
$$
which establishes the assertion by induction.

\begin{exa}\label{exa:splitting_square}
Prove that a given square can be decomposed into $n$ squares, not
necessarily of the same size, for all $n = 4, 6, 7,  8, \ldots$.
\end{exa}
Solution: A quartering of a subsquare increases the number of
squares by three (four new squares are gained but the original
square is lost). Figure \ref{fig:quartering} below shews that $n =
4$ is achievable.\vspace{1cm}
\begin{figure}[h]
\begin{minipage}{5cm}
$$\psset{unit=1pc} \psline(-2,-2)(-2,2)(2,2)(2,-2)(-2,-2)
\psline(0,-2)(0,2)
\psline(-2,0)(2,0)$$\vspace{1cm}\footnotesize\hangcaption{Example
\ref{exa:splitting_square}.}\label{fig:quartering}
\end{minipage} \hfill
\begin{minipage}{5cm}
$$
\psset{unit=1pc} \psline(-2,-2)(-2,2)(2,2)(2,-2)(-2,-2)
\psline(-2,.6666667)(2,.6666667) \psline(.6666667,-2)(.6666667,2)
\psline(-.6666667,.6666667)(-.6666667,2)
\psline(.6666667,-.6666667)(2,-.6666667)
 $$\vspace{1cm}\footnotesize\hangcaption{Example \ref{exa:splitting_square}.}\label{fig:six_parts}
\end{minipage}
\hfill
\begin{minipage}{5cm}
$$
\psset{unit=1pc} \psline(-2,-2)(-2,2)(2,2)(2,-2)(-2,-2)
\psline(-2,1)(2,1) \psline(1,-2)(1,2) \psline(0,1)(0,2)
\psline(-1,1)(-1,2) \psline(1,0)(2,0) \psline(1,-1)(2,-1)
$$\vspace{1cm}\footnotesize\hangcaption{Example \ref{exa:splitting_square}.} \label{fig:eight_parts}
 \end{minipage}

 \end{figure}
If $n$ were achievable, a quartering would make  $\{ n, n + 3, n +
6, n + 9, \ldots \}$ also achievable. We will shew now that $n = 6$
and $n = 8$ are achievable. But this is easily seen from  figures
\ref{fig:six_parts} and \ref{fig:eight_parts}, and this finishes the
proof.

\bigskip

Sometimes it is useful to use the following version of PMI, known
as the Principle of Strong
Mathematical Induction (PSMI). \\





\flushleft{{\bf Principle of Strong Mathematical Induction}}
Suppose we have an assertion $P(n)$ concerning natural numbers
satisfying the following two properties:
\begin{itemize}
\item ({\bf PSMI I}) $P(k_0)$ is true for some natural number
$k_0$, \item ({\bf PSMI II}) If $m < n$ and $P(m), P(m + 1),
\ldots , P(n - 1)$ are true then $P(n)$ is true.
\end{itemize}
Then the assertion $P(n)$ is true for every $n \geq k_0.$



\begin{exa}
In the country of SmallPesia coins only come in values of $3$ and
$5$ pesos. Shew that any quantity of pesos greater than or equal
to $8$ can be paid using the available coins.
\end{exa}
Solution: We use PSMI. Observe that $8 = 3 + 5, 9 = 3 + 3 + 3, 10
= 5 + 5$, so, we can pay $8, 9,$ or $10$ pesos with the available
coinage. Assume that we are able to pay $n - 3, n - 2,$ and $n -
1$ pesos, that is, that $3x + 5y = k$ has non-negative solutions
for $k = n - 3, n - 2$ and $n - 1$. We will shew that we may also
obtain solutions for $3x + 5y = k$ for $k = n, n + 1$ and $n + 2$.
Now
$$ 3x + 5y = n - 3 \Longrightarrow 3(x + 1) + 5y = n,$$
$$ 3x_1 + 5y_1 = n - 2 \Longrightarrow 3(x_1 + 1) + 5y_1 = n + 1,$$
$$ 3x_2 + 5y_2 = n - 1 \Longrightarrow 3(x_2 + 1) + 5y_2 = n + 2,$$ and so if the amounts
$n - 3, n - 2, n - 1$ can be paid so can $n, n + 1, n + 2.$ The
statement of the problem now follows from PSMI.


\begin{exa}[USAMO 1978] An integer $n$ will be called {\em good} if we can
write
$$n = a_1 + a_2 + \cdots + a_k,$$where the integers $a_1, a_2, \ldots , a_k$ are positive
integers (not necessarily distinct) satisfying
$$\frac{1}{a_1} + \frac{1}{a_2} + \cdots \frac{1}{a_k} = 1.$$Given the information that the integers
$33$ through $73$ are good, prove that every integer $\geq 33$ is
good.
\end{exa}
Solution: We first prove that if $n$ is good, then $2n + 8$ and
$2n + 9$ are also good. For assume that $n = a_1 + a_2 + \cdots +
a_k,$ and
$$\frac{1}{a_1} + \frac{1}{a_2} + \cdots \frac{1}{a_k} = 1.$$
Then $2n + 8 = 2(a_1 + a_2 + \cdots + a_k) + 4 + 4$ and
$$\frac{1}{2a_1} + \frac{1}{2a_2} + \cdots \frac{1}{2a_k} + \frac{1}{4} + \frac{1}{4}
= \frac{1}{2} + \frac{1}{4} + \frac{1}{4} = 1.$$ Also $2n + 9 =
2(a_1 + a_2 + \cdots + a_k) + 3 + 6$ and
$$\frac{1}{2a_1} + \frac{1}{2a_2} + \cdots \frac{1}{2a_k} + \frac{1}{3} + \frac{1}{6}
= \frac{1}{2} + \frac{1}{3} + \frac{1}{6} = 1.$$ Therefore
\begin{center}
if $n$ is good then $2n + 8$ and $2n + 9$ are good \ (*)
 \end{center}
We now establish the truth of the assertion of the problem by
induction on $n$. Let $P(n)$ be the proposition ``all the integers
$n, n + 1, n + 2, \ldots , 2n + 7$'' are good. By the statement of
the problem, we see that $P(33)$ is true. But (*) implies the
truth of $P(n + 1)$ whenever $P(n)$ is true. The assertion is thus
proved by induction.

\clearpage

\section*{Practice}\addcontentsline{toc}{section}{Practice}\markright{Practice}\begin{multicols}{2}\columnseprule 1pt \columnsep 25pt\multicoltolerance=900
\begin{pro}
Use Sophie Germain's trick to shew that $x^4 + x^2 + 1 = (x^2 - x
+ 1)(x^2 + x + 1)$. Use this to shew that if $n$ is a positive
integer then
$$2^{2^{n + 1}} + 2^{2^n} + 1$$has at least $n$ different prime factors.
\end{pro}
\begin{pro}
Prove that $n^3 + (n + 1)^3 + (n + 2)^3$ is divisible by $9$.
\end{pro}
\begin{pro}
Let $n \in \BBN$. Prove the inequality
$$\frac{1}{n + 1} + \frac{1}{n + 2} + \cdots + \frac{1}{3n + 1} > 1.$$
\end{pro}
\begin{pro}
Prove that for all positive integers $n$ and all real numbers $x$,
\begin{equation}
|\sin n x| \leq n |\sin x|
\end{equation}
\end{pro}
\begin{pro}
Prove that
$$\underbrace{\sqrt{2 + \sqrt{2 + \sqrt{2 + \cdots + \sqrt{2}}}}}_{n \
{\rm radical\ signs}} = 2\cos \frac{\pi}{2^{n + 1}}$$for $n \in
\BBN$.
\end{pro}

\begin{pro}
Let $a_1 = 3, b_1 = 4,$ and $a_n = 3^{a_{n - 1}}, \ b_n = 4^{b_{n
- 1}}$ when $n > 1$. Prove that $a_{1000} > b_{999}$.
\end{pro}
\begin{pro}
Let $n\in \BBN , n > 1$. Prove that
$$\frac{1\cdot 3\cdot 5\cdots (2n - 1)}{2\cdot 4\cdot 6\cdots (2n)} < \frac{1}{\sqrt{3n + 1}}.$$
\end{pro}
\begin{pro}
Prove that for all natural number $n > 1$,
$$\frac{4^n}{n + 1} < \frac{(2n)!}{(n!)^2}.$$
\end{pro}
\begin{pro} Let $k$ be a positive integer
Prove that if $\dis{x + \frac{1}{x}}$ is an integer then $\dis{x^k
+ \frac{1}{x^k}}$ is also an integer.
\end{pro}
\begin{pro}
Prove that for all natural numbers $n > 1$,
$$\frac{1}{1^2} + \frac{1}{2^2} + \frac{1}{3^2} + \cdots + \frac{1}{n^2} < 2 - \frac{1}{n}.$$
\end{pro}

\begin{pro} Let $n \geq 2$ be an integer.
Prove that $f_1 + f_2 + \cdots + f_n = f_{n + 2} - 1.$
\end{pro}
\begin{pro} Let $n, m\geq 0$ be  integers.
Prove that
\begin{equation}
f_{n + m} = f_{n - 1}f_m + f_nf_{m + 1}
\end{equation}
\end{pro}

\begin{pro}
This problem uses the argument of A. Cauchy's to prove the AM-GM
Inequality. It consists in shewing that AM-GM is true for all powers
of $2$ and then deducing its truth for the numbers between two
consecutive powers of $2$. Let $a_1, a_2, \ldots , a_l$ be
non-negative real numbers. Let $P(l)$ be the assertion the AM-GM
Inequality
$$\frac{a_1 + a_2 + \cdots + a_l}{l} \geq \sqrt[l]{a_1a_2\cdots a_l}$$holds for
the $l$ given numbers.
\begin{enumerate}
\item Prove that $P(2)$ is true. \item Prove that the truth of
$P(2^{k - 1})$  implies that of $P(2^k)$. \item Let $2^{k - 1} < n
< 2^k$. By considering the $2^k$ quantities
$$a_1 = y_1, a_2 = y_2, \ldots , a_n = y_n,$$
$$a_{n + 1} = a_{n + 1} = \cdots = a_{2^k} =\frac{y_1 + y_2 + \cdots
+ y_n}{n},$$prove that $P(n)$ is true.
\end{enumerate}
\end{pro}\end{multicols}
\section{Congruences}

\begin{df}
The notation $a \equiv b \mod n$ is due to Gau\ss , and it means
that $n|(a - b).$
\end{df}


Thus if $a \equiv b\mod n$  then $a$ and $b$ leave the same
remainder upon division by $n$.  For example, since $8$ and $13$
leave the same remainder upon division by $5$, we have $8 \equiv
13 \mod 5 $. Also observe that $5|(8 - 13).$  As a further
example, observe that $-8 \equiv -1 \equiv 6 \equiv 13 \mod 7$.



Consider all the integers and arrange them in five columns as
follows.
$$\begin{array}{ccccc}
\ldots & \ldots & \ldots & \ldots & \ldots \\
-10 & -9 & -8 & -7 & -6 \\
-5 & -4 & -3 & -2 & -1 \\
0 & 1 & 2 & 3 & 4 \\
5 & 6 & 7 & 8 & 9 \\
\ldots & \ldots & \ldots & \ldots & \ldots
\end{array}$$
The arrangement above shews that any integer comes in one of 5
flavours: those leaving remainder 0 upon division by 5, those
leaving remainder 1 upon division by 5, etc..





Since $n|(a - b)$ implies that $\exists k \in \BBZ$ such that $nk
= a - b$, we deduce that $a \equiv b \mod n$ if and only if there
is an integer $k$ such that $a = b + nk.$



The following theorem is quite useful.
\begin{thm} Let $n \geq 2$  be an integer. If $x \equiv y  \mod n$ and
$u \equiv v \mod n$ then
$$ax + bu \equiv ay + bv \mod n.$$
\label{thm:congruences}\end{thm}
\begin{pf}As $n|(x - y), \ n|(u - v)$ then there are integers $s, t$
with $ns = x - y, \ nt = u - v$. This implies that
$$a(x - y) + b(u - v) = n(as + bt),$$ which entails that,
$$n|(ax + bu - ay - bv).$$This last assertion is equivalent to saying
$$ax + bu \equiv ay + bv  \mod n.$$ This finishes the proof.
\end{pf}
\begin{cor} Let $n \geq 2$  be an integer. If $x \equiv y \mod n$ and
$u \equiv v \mod n$ then

$$xu \equiv  yv \mod n.$$
\label{cor:cong_1}\end{cor}
\begin{pf} Let $a = u, b = y$ in Theorem \ref{thm:congruences}.
\end{pf}
\begin{cor} Let $n > 1$ be an integer, $x \equiv y  \mod n$ and
$j$ a positive integer. Then $x^j \equiv y^j \mod  n.$
\label{cor:cong_2}\end{cor} \begin{pf}  Use repeteadly Corollary
\ref{cor:cong_1} with $u = x, v = y.$
\end{pf}
\begin{cor} Let $n > 1$ be an integer, $x \equiv y  \mod n$.
If $f$ is a polynomial with integral coefficients then $f(x)
\equiv f(y) \mod n$.
\end{cor}






\begin{exa} Find the remainder when $6^{1987}$ is divided by
$37$. \end{exa} Solution: $6^2 \equiv -1 \mod 37$. Thus $$6^{1987}
\equiv 6\cdot 6^{1986} \equiv 6(6^2 )^{993} \equiv 6(-1)^{993}
\equiv -6 \equiv 31  \mod 37$$ and the remainder sought is $31$.
\begin{exa} Find the remainder when
$$12233\cdot 455679 + 87653^3$$

is divided by $4$. \end{exa} Solution: $12233 = 12200 + 32 + 1
\equiv 1 \mod 4$. Similarly, $455679 = 455600 + 76 + 3 \equiv 3 $,
$87653 = 87600 + 52 + 1 \equiv 1$ $\mod 4$. Thus
$$12233\cdot 455679 + 87653^3 \equiv 1\cdot 3 + 1^3 \equiv 4 \equiv 0 \ \mod 4.$$This
means that $12233\cdot 455679 + 87653^3$ is divisible by 4.

\begin{exa} Prove that $7$ divides $3^{2n + 1} + 2^{n + 2}$ for all natural numbers $n$. \end{exa}
Solution: Observe that $$3^{2n + 1} \equiv 3\cdot 9^n \equiv
3\cdot 2^n \mod 7$$ and $$2^{n + 2} \equiv 4\cdot 2^n \mod 7$$.
Hence
$$ 3^{2n + 1} + 2^{n + 2} \equiv 7\cdot 2^n \equiv 0 \ \mod \ 7,$$for all
natural numbers $n.$
\begin{exa} Prove the following result of Euler: $641|(2^{32} + 1).$\end{exa}
Solution: Observe that $641 = 2^7\cdot 5 + 1 = 2^4 + 5^4.$ Hence
$2^7\cdot 5 \equiv -1 \mod 641$ and $5^4 \equiv -2^4 \mod 641$.
Now, $2^7\cdot 5 \equiv -1 \mod 641$ yields $$5^4 \cdot 2^{28} =
(5\cdot 2^7)^4 \equiv (-1)^4 \equiv 1 \mod 641.$$ This last
congruence and $$5^4 \equiv -2^4 \mod 641$$ yield $$-2^{4}\cdot
2^{28} \equiv 1\mod 641,$$ which means that $641|(2^{32} + 1).$
\begin{exa}Prove that $7|(2222^{5555} + 5555^{2222}).$\end{exa}
Solution: $2222 \equiv 3 \mod 7$, $5555 \equiv 4 \mod 7$ and $3^5
\equiv 5 \mod 7$. Now $$2222^{5555} + 5555^{2222} \equiv 3^{5555}
+ 4^{2222} \equiv (3^5)^{1111} + (4^2)^{1111} \equiv 5^{1111} -
5^{1111} \equiv 0 \mod 7.$$
\begin{exa} Find the units digit of $7^{7^7}$.\end{exa}
Solution: We must find $7^{7^7} \mod 10$. Now, $7^2 \equiv - 1
\mod 10$, and so $7^3 \equiv 7^2 \cdot 7 \equiv -7 \equiv 3 \mod
10$ and $7^4 \equiv (7^2)^2 \equiv 1$ mod 10. Also, $7^2 \equiv 1
\mod 4$ and so $7^7 \equiv (7^2)^3 \cdot 7 \equiv 3 \mod 4$, which
means that there is an integer $t$ such that $7^7 = 3 + 4t.$ Upon
assembling all this,
$$ 7^{7^7} \equiv 7^{4t + 3} \equiv (7^4)^t \cdot 7^3 \equiv 1^t\cdot 3
\equiv 3 \ \mod \ 10. $$Thus the last digit is 3.

\begin{exa} Find infinitely many integers $n$ such that
$2^n + 27$ is divisible by $7$.\end{exa} Solution: Observe that
$2^1 \equiv 2, 2^2 \equiv 4, 2^3 \equiv 1, 2^4 \equiv 2, 2^5
\equiv 4, 2^6 \equiv 1 \mod 7$ and so $2^{3k} \equiv 1 \mod 3$ for
all positive integers $k$. Hence $2^{3k} + 27 \equiv 1 + 27 \equiv
0 \mod 7$ for all positive integers $k$. This produces the
infinitely many values sought.
\begin{exa} Prove that $2^k - 5, k = 0, 1, 2, \ldots$ never leaves remainder
$1$ when divided by $7$.\end{exa} Solution: $2^1 \equiv 2, 2^2
\equiv 4, 2^3 \equiv 1 \mod 7$, and this cycle of three repeats.
Thus $2^k - 5$ can leave only remainders 3, 4, or 6 upon division
by 7.
\begin{exa}[AIME 1994] The increasing sequence $$3, 15, 24, 48, \ldots ,$$ consists of those positive multiples
of $3$ that are one less than a perfect square. What is the
remainder when the $1994$-th term of the sequence is divided by
$1000$? \end{exa} Solution: We want $3|n^2 - 1 = (n - 1)(n + 1)$.
Since 3 is prime, this requires $n = 3k + 1$ or $n = 3k - 1, k =
1, 2, 3, \ldots $. The sequence $3k + 1, k = 1, 2, \ldots$
produces the terms $n^2 - 1 = (3k + 1)^2 - 1$ which are the terms
at even places of the sequence of $3, 15, 24, 48, \ldots $. The
sequence $3k - 1, k = 1, 2, \ldots$ produces the terms $n^2 - 1 =
(3k - 1)^2 - 1$ which are the terms at odd places of the sequence
$3, 15, 24, 48, \ldots $. We must find the 997th term of the
sequence $3k + 1, k = 1, 2, \ldots$. Finally, the term sought is
$(3(997) + 1)^2 - 1 \equiv (3(-3) + 1)^2 - 1 \equiv 8^2 - 1 \equiv
63 \mod 1000.$ The remainder sought is 63.

\section*{Practice}\addcontentsline{toc}{section}{Practice}\markright{Practice}\begin{multicols}{2}\columnseprule 1pt \columnsep 25pt\multicoltolerance=900

\begin{pro} Prove that $0, 1, 3, 4, 9, 10,$ and $12$ are the only perfect squares
modulo $13$.  \end{pro} (Hint: It is enough to consider $0^2, 1^2,
2^2, \ldots , 12^2$. In fact, by observing that $r^2 \equiv (13 -
r)^2 \mod n$, you only have to go half way.)
\begin{pro} Prove that there are no
integers with $x^2 - 5y^2 = 2.$ \end{pro} (Hint: Find all the
perfect squares mod 5.)
\begin{pro}Which digits must we substitute for a and b in $30a0b03$ so that the resulting
integer be divisible by $13$?\end{pro}

\begin{pro}  Find the number of all $n, 1 \leq n \leq 25$
such that $n^2 + 15n + 122$ is divisible by $6.$
\end{pro}(Hint: $n^2 + 15n + 122 \equiv n^2 + 3n + 2 = (n + 1)(n + 2)$ mod 6.)


\begin{pro}[AIME 1983] Let $a_n = 6^n + 8^n$. Determine the remainder
when $a_{83}$ is divided by $49.$ \end{pro}
\begin{pro}[Polish Mathematical Olympiad] What digits should be put instead of $x$ and $y$
in $30x0y03$ in order to give a number divisible by $13$?\end{pro}
\begin{pro} Prove that if $9|(a^3 + b^3 + c^3)$, then $3|abc$, for
integers $a, b, c.$ \end{pro}
\begin{pro}Describe all integers $n$ such that $10|n^{10} + 1.$\end{pro}
\begin{pro} Find the last digit of $3^{100}$. \end{pro}
\begin{pro}[AHSME 1992] What is the size of the largest
subset S of $\{ 1, 2, \ldots , 50\}$ such that no pair of distinct
elements of S has a sum divisible by $7$?\end{pro}
\begin{pro} Prove that there are no integer solutions to the equation $x^2 - 7y = 3.$ \end{pro}
\begin{pro} Prove that if $7|a^2 + b^2$ then $7|a$ and $7|b$. \end{pro}

\begin{pro} Prove that there are no integers with $$ 800000007 = x^2 + y^2 + z^2 .$$\end{pro}
\begin{pro} Prove that the sum of the decimal digits of a perfect square cannot be equal
to $1991$. \end{pro}
\begin{pro} Prove that $$7|4^{2^n} + 2^{2^n} + 1$$for all natural numbers
n.\end{pro}
\begin{pro} Find the last two digits of $3^{100}$. \end{pro}
\begin{pro}[USAMO 1986] What is the smallest integer $n > 1$, for
which the root-mean-square of the first $n$ positive integers is
an integer? \end{pro} {\bf Note.} {\footnotesize The root mean
square of $n$ numbers $a_1 , a_2 , \ldots , a_n$ is defined to be
$$ \left(\frac{a_1 ^2 + a_2 ^2 + \cdots + a_n
^2}{n}\right)^{1/2}.$$}


\begin{pro} If $62ab427$ is a  multiple of  $99$, find the digits
$a$ and $b$.\end{pro}
\begin{pro} Shew that an integer is divisible by $2^n, n = 1, 2, 3, \ldots$ if the number formed
by its last $n$ digits is divisible by $2^n$.\end{pro}
\begin{pro} Find the last digit of
$$2333333334\cdot 9987737 + 12\cdot 21327 + 12123\cdot 99987.$$ \end{pro}

\begin{pro}[AIME 1994] The increasing sequence $$3, 15, 24, 48, \ldots ,$$
consists of all those multiples of  $3$ which are one less than a
square. Find the remainder when the $1994$th term is divided by
$1000$.
\begin{answer}$63$  \end{answer}
\end{pro}

\begin{pro}[AIME 1983] Let $a_n = 6^n + 8^n$. Find the remainder when $a_{83}$ is
divided by $49.$ \end{pro}
\begin{pro} Shew that if  $9|(a^3 + b^3 + c^3)$, then $3|abc$, for the integers
$a, b, c.$ \end{pro}
\end{multicols}
\section{Miscellaneous Problems Involving
Integers}

Recall that $\llfloor x\rrfloor$ is the unique integer satisfying
\begin{equation}
x - 1 < \llfloor x\rrfloor \leq x
\end{equation}
Thus $\llfloor x\rrfloor$ is $x$ is an integer, or the integer just
to the left of $x$ if $x$ is not an integer. For example $\llfloor
3.9\rrfloor = 1, \llfloor -3.1\rrfloor = -4.$



     Let $p$ be a prime and $n$ a positive integer. In the product
$n! = 1\cdot 2\cdot 3\cdots n$ the number of factors contributing a
factor of of $p$ is $\llfloor \frac{n}{p}\rrfloor $ the number of
factors contributing a factor of $p^2$ is $\llfloor
\frac{n}{p^2}\rrfloor$, etc.. This proves the following theorem.

\begin{thm}[De Polignac-Legendre] The highest power of a prime $p$ diving $n!$
is given by
\begin{equation}
\sum _{k = 1} ^\infty \llfloor \frac{n}{p^k}\rrfloor
\end{equation}
\end{thm}


\begin{exa} How many zeroes are there at the end of $999! = 1\cdot 2\cdot 3\cdot 4\cdots 998\cdot 999$? \end{exa}
Solution: The number of zeroes is determined by the highest power
of 10 dividing $999!$. As there are fewer multiples of 5 amongst
$\{1, 2, \ldots , 999\}$ that multiples of 2, the number of zeroes
is the determined by the highest power of 5 dividing $999!$. But
the highest power of 5 dividing $999!$ is given by
$$\llfloor\frac{999}{5}\rrfloor + \llfloor\frac{999}{5^2}\rrfloor + \llfloor\frac{999}{5^3}\rrfloor
+ \llfloor\frac{999}{5^4} \rrfloor = 199 + 39 + 7 + 1 =
246.$$Therefore 999! ends in  246 zeroes.
\begin{exa}
Let $m, n$ be non-negative integers. Prove that
\begin{equation}
\frac{(m + n)!}{m!n!} \ \ \  {\rm is \ an \ integer.}
\end{equation}
\end{exa}
Solution: Let $p$ be a prime and $k$ a positive integer. By the De
Polignac-Legendre Theorem, it suffices to shew that
$$\llfloor\frac{m  + n}{p^k}\rrfloor \geq \llfloor\frac{m}{p^k}\rrfloor + \llfloor\frac{n}{p^k}\rrfloor .$$
This inequality in turn will follow from the inequality
\begin{equation}
\llfloor \alpha\rrfloor + \llfloor\beta\rrfloor \leq \llfloor\alpha
+ \beta\rrfloor
\end{equation}which we will shew  valid for all real numbers $\alpha, \beta$.


Adding the inequalities $\llfloor \alpha \rrfloor \leq \alpha, \
\llfloor \beta \rrfloor \leq \beta,$ we obtain $\llfloor \alpha
\rrfloor + \llfloor \beta \rrfloor \leq \alpha + \beta .$ Since
$\llfloor \alpha \rrfloor + \llfloor \beta \rrfloor$ is an integer
less than or equal to $\alpha + \beta$, it must be less than or
equal than the integral part of $\alpha + \beta$, that is $\llfloor
\alpha\rrfloor + \llfloor\beta\rrfloor \leq \llfloor\alpha +
\beta\rrfloor$, as we wanted to shew.





Observe that $(m + n)! = m!(m + 1)(m + 2)\cdots (m + n)$. Thus
cancelling a factor of $m!$,
$$\frac{(m + n)!}{m!n!}
= \frac{(m + 1)(m + 2)\cdots (m + n)}{n!}$$ we see that the product
of $n$ consecutive positive integers is divisible by $n!$. If all
the integers are negative, we may factor out a $(-1)^n$, or if they
include $0$, their product is $0$. This gives the following theorem.
\begin{thm}
The product of $n$ consecutive integers is divisible by $n!$.
\end{thm}



\begin{exa} Prove that $n^5 - 5n^3 + 4n$ is always divisible by $120$ for all integers $n$. \end{exa}
Solution: We have
$$n^5 - 5n^3 + 4n = n(n^2 - 4)(n^2 - 1) = (n - 2)(n - 1)(n)(n + 1)(n + 2),$$
the product of 5 consecutive integers and hence divisible by $5! =
120.$

\begin{exa} Let $A$ be a positive integer and let $A'$ be the resulting integer after a specific
permutation of the digits of $A$. Shew that if $A + A' = 10^{10}$
then $A$ s divisible by $10$.\end{exa} Solution: Clearly, $A$ and
$A'$ must have 10 digits each. Put
$$A = \overline{a_{10}a_9a_8\ldots a_1}$$and $$A' = \overline{b_{10}b_9b_8\ldots b_1},$$
where $a_k, b_k, k = 1, 2, \ldots , 10$ are the  digits of $A$ and
$A'$ respectively. As $A + A' = 10000000000$, we must have $a_1 +
b_1 = a_2 + b_2 = \cdots = a_i + b_i  = 0$ and
$$a_{i + 1} + b_{i + 1}= 10,   a_{i + 2} + b_{i + 2} = \cdots = a_{10} + b_{10} = 9, $$
for some subindex $i, 0 \leq i \leq 9$. Notice that if $i = 9$
there are no sums $a_{i + 2} + b_{i + 2}, a_{i + 3} + b_{i + 3},
\ldots$ and if $i = 0$ there are no sums $a_1 + b_1, \ldots , a_i
+ b_i.$


Adding,
$$a_1 + b_1 + a_2 + b_2 + \cdots + a_i + b_i + a_{i + 1} + b_{i + 1} + \cdots + a_{10} + b_{10}
= 10 + 9(9 - i). $$If  $i$ is even, $10 + 9(9 - i)$ is odd and if
$i$ is odd $10 + 9(9 - i)$ is even. As
$$a_1 + a_2 + \cdots + a_{10} = b_1 + b_2 + \cdots + b_{10},$$ we have
$$a_1 + b_1 + a_2 + b_2 + \cdots + a_i + b_i + a_{i + 1} + b_{i + 1} + \cdots + a_{10} + b_{10}
= 2(a_1 + a_2 + \cdots + a_{10}), $$an even integer. We gather
that $i$ is odd, which entails that $a_1 = b_1 = 0,$ that is , $A$
and $A'$ are both divisible by 10.
\begin{exa}[Putnam 1956] Prove that every positive integer has a multiple
whose decimal representation involves all $10$ digits.
\end{exa}
Solution: Let $n$ be an arbitrary positive integer with $k$
digits. Let $m = 1234567890\cdot 10^{k + 1}$. Then all of the $n$
consecutive integers
$$m + 1, m + 2, \ldots, m + n$$begin with $1234567890$ and one of them is divisible by $n$.

\begin{exa}[Putnam 1966] Let $0 < a_1 < a_2 < \ldots < a_{mn +
1}$ be $mn + 1$ integers. Prove that you can find either $m + 1$
of them no one of which divides any other, or $n + 1$ of them,
each dividing the following. \end{exa} Solution: Let, for each $1
\leq k \leq mn + 1, n_k$ denote the length of the longest chain,
starting with $a_k$ and each dividing the following one, that can
be selected from $a_k , a_{k + 1}, \ldots , a_{mn + 1}$. If no
$n_k $ is greater than $n$, then the are at least $m + 1 \ n_k$'s
that are the same. However, the integers $a_k$ corresponding to
these $n_k$'s cannot divide each other, because $a_k |a_l$ implies
that $n_k \geq n_l + 1.$
\begin{thm} If $k|n$ then $f_k|f_n$.\end{thm}
{\bf Proof} Letting $s = kn, t = n$ in the identity $f_{s + t} =
f_{s - 1}f_t + f_{s}f_{t + 1}$ we obtain
$$ f_{(k + 1)n} = f_{kn + n} = f_{n - 1}f_{kn} + f_nf_{kn + 1}.$$It is clear
that if $f_n|f_{kn}$ then $f_n|f_{(k + 1)n}$. Since $f_n|f_{n\cdot
1}$, the assertion follows.


\begin{exa} Prove that if $p$ is an odd prime and if $$ \frac{a}{b} = 1 + 1/2 + \cdots + 1/(p - 1),$$
then $p$ divides $a$. \end{exa} Solution: Arrange the sum as $$ 1
+ \frac{1}{p - 1} + \frac{1}{2} + \frac{1}{p - 2} + \cdots +
\frac{1}{(p - 1)/2} + \frac{1}{(p + 1)/2}.$$ After summing
consecutive pairs, the numerator of the resulting fractions is
$p$. Each term in the denominator is $< p$. Since $p$ is a prime,
the $p$ on the numerator will not be thus cancelled out.

\begin{exa} The sum of  some positive integers is $1996$. What is their maximum product?\end{exa}
Solution: We are given some positive integers $a_1, a_2, \ldots ,
a_n$ with $a_1 + a_2 + \cdots + a_n = 1996.$ To maximise
$a_1a_2\cdots a_n$, none of the $a_k$'s can be 1. Let us shew that
to maximise this product, we make as many possible $a_k = 3$ and
at most two $a_j = 2.$



Suppose that $a_j > 4.$ Substituting $a_j$ by the two terms $a_j -
3$ and 3 the sum is not changed, but the product increases since
$a_j < 3(a_j - 3).$ Thus the  $a_k$'s must equal 2, 3 or 4. But $2
+ 2 + 2 = 3 + 3$ and $2 \times 2 \times 2 < 3\times 3,$ thus if
there are more than two 2's we may substitute them by 3's. As
$1996 = 3(665) + 1 = 3(664) + 4,$ the maximum product sought is
$3^{664}\times 4.$
\begin{exa} Find all the positive integers of the form
$$r + \frac{1}{r},$$where $r$ is a rational number.\end{exa}
Solution: We will shew that the expression $r + 1/r$ is a positive
integer only if  $r = 1,$ in which case $r + 1/r = 2.$ Let
$$r + \frac{1}{r} = k,$$ $k$ a positive integer. Then
$$r = \frac{k \pm \sqrt{k^2 - 4}}{2}.$$Since $k$ is an integer, $r$ will be an integer
if and only $k^2 - 4$ is a square of the same parity as $k$. Now,
if $k \geq 3,$
$$ (k - 1)^2 < k^2 - 4 < k^2,$$that is, $k^2 - 4$ is strictly between two consecutive squares and so
it cannot be itself a square. If $k = 1,$ $\sqrt{k^2 - 4}$ is not
a real number. If $k = 2,$ $k^2 - 4 = 0$. Therefore, $r + 1/r =
2,$ that is, $r = 1.$ This finishes the proof.
\begin{exa}
For how many integers $n$ in $\{1, 2, 3, \ldots , 100\}$ is the
tens digit of $n^2$ odd?
\end{exa}
Solution: In the subset $\{1, 2, \ldots 10\}$ there are only two
values of $n$ (4 and 6) for which the digits of the tens of $n^2$
is odd. Now, the tens digit of $(n + 10)^2 = n^2 + 20n + 100$ has
the same parity as the tens digit of $n^2$. Thus there are only 20
$n$ for which the prescribed condition is verified.
\section*{Practice}\addcontentsline{toc}{section}{Practice}\markright{Practice}\begin{multicols}{2}\columnseprule 1pt \columnsep 25pt\multicoltolerance=900
\begin{pro} Find the sum
$$5 + 55 + 555 + \cdots + \underbrace{5\ldots 5}_{n \ 5{\rm 's}}. $$\end{pro}
\begin{pro} Shew that for all numbers $a \neq 0, a \neq \pm i\sqrt{3}$
the following formula of Reyley (1825) holds.
$$\begin{array}{l}
a = \left(\frac{a^6 + 45a^5 - 81a^2 + 27}{6a(a^2 + 3)^2}\right)^3
\\ \qquad + \left(\frac{-a^2 + 30a^2 - 9}{6a(a^2 + 3)}\right)^3\\ \qquad  +
\left(\frac{-6a^3 + 18a}{(a^2 + 3)^2}\right) ^3. \end{array}
$$ If $a$ is rational this shews that
every rational number is expressible as the sum of the cubes of
three rational numbers.
\end{pro}
\begin{pro} What is the largest power of $7$ that divides $1000!$?\end{pro}
\begin{pro}Demonstrate that for all integer values $n$, $$ n^9 - 6n^7 + 9n^5 - 4n^3$$
is divisible by $8640$.\end{pro}
\begin{pro} Prove that if $n > 4$ is composite, then n divides $(n - 1)!$. \\
\begin{answer} Consider, separately, the cases when $n$ is and is not a
perfect square.\end{answer}
\end{pro}
\begin{pro}
Find all real numbers satisfying the equation
$$\llfloor x^2 - x - 2\rrfloor = \llfloor x\rrfloor.$$
\end{pro}
\begin{pro}
Solve the equation
$$\llfloor \frac{x}{1999}\rrfloor = \llfloor \frac{x}{2000}\rrfloor$$
\end{pro}
\begin{pro}[Putnam 1948]
Let $n$ be a positive integer. Prove that
$$\llfloor \sqrt{n} + \sqrt{n + 1} \rrfloor = \llfloor \sqrt{4n + 2} \rrfloor $$
\end{pro}
(Hint: Prove that $\sqrt{4n + 1} < \sqrt{n} + \sqrt{n + 1} <
\sqrt{4n + 3}$. Argue that neither $4n + 2$ nor $4n + 3$ are
perfect squares.)
\begin{pro} Prove that $6|n^3 - n$, for all integers $n$. \end{pro}

\begin{pro}[Polish Mathematical Olympiad] Prove that if n is an
even natural number, then the number $13^n + 6$ is divisible by
$7$.\end{pro}
\begin{pro} Find, with proof, the unique square which is the product
of four consecutive odd numbers.\end{pro}

\begin{pro}[Putnam 1989] How many primes amongst the positive
integers, written as usual in base-ten are such that their digits
are alternating $1$'s and $0$'s, beginning and ending in $1$?
\end{pro}
\begin{pro} Let $a, b, c$ be the lengths of the sides of a triangle. Shew
that
$$ 3(ab + bc + ca) \leq (a + b + c)^2 \leq 4(ab + bc + ca).$$\end{pro}

\begin{pro} Let $k \geq 2$ be an integer. Shew that if $n$ is a positive integer, then
$n^k$ can be represented as the sum of $n$ successive odd
numbers.\end{pro}
\begin{pro}[IMO 1979] If $a, b$ are natural numbers such that
$$ \frac{a}{b} = 1 - \frac{1}{2} + \frac{1}{3} - \frac{1}{4} + \cdots - \frac{1}{1318} + \frac{1}{1319},$$
prove that $1979|a$.\end{pro}
\begin{pro}[Polish Mathematical Olympiad] A {\em triangular number} is one of the form $1 + 2 + \ldots + n,
n\in \BBN$. Prove that none of the digits $2, 4, 7, 9$ can be the
last digit of a triangular number.\end{pro}
\begin{pro}Demonstrate that there are infinitely many square triangular numbers.\end{pro}
\begin{pro}[Putnam 1975] Supposing that an integer $n$ is the sum
of two triangular numbers, $$ n = \frac{a^2 + a}{2} + \frac{b^2 +
b}{2},$$write $4n + 1$ as the sum of two squares, $4n + 1 = x^2 +
y^2$ where $x$ and $y$ are expressed in terms of $a$ and $b$.


Conversely, shew that if $4n + 1 = x^2 + y^2 ,$ then $n$ is the
sum of two triangular numbers.\end{pro}
\begin{pro}[Polish Mathematical Olympiad] Prove that \\
amongst ten successive natural numbers, there are always at least
one and at most four numbers that are not divisible by any of the
numbers $2, 3, 5, 7.$\end{pro}

\begin{pro}Are there five consecutive positive integers such that the sum of the first
four, each raised to the fourth power, equals the fifth raised to
the fourth power?\end{pro}
\begin{pro} Prove that $$ \frac{(2m)!(3n)!}{(m!)^2 (n!)^3}$$is always an integer.\end{pro}

\begin{pro} Prove that for $n \in \BBN ,$ $(n!)!$ is divisible by $n!^{(n - 1)!}$\end{pro}
\begin{pro}[Olimp\'{\i}ada matem\'{a}tica espa\~{n}ola, 1985]
If $n$ is a positive integer, prove that $(n + 1)(n + 2)\cdots
(2n)$  is divisible by $2^n$.\end{pro}
\end{multicols}
\chapter{Sums, Products, and Recursions}
\section{Telescopic cancellation} We could sum the series

$$a_1 + a_2 + a_3 + \cdots + a_n$$
if we were able to find $\{v_k\}$ satisfying $a_k = v_k - v_{k -
1}.$ For
$$a_1 + a_2 + a_3 + \cdots + a_n = v_1 - v_0 + v_2 - v_1 + \cdots + v_{n - 1} - v_{n - 2}
+ v_n - v_{n - 1} = v_n - v_0.$$ If such sequence $v_n$ exists, we
say that
 $a_1+ a_2 + \cdots + a_n$  is a {\em telescopic series.}


\begin{exa} Simplify
$$ \left( 1 + \frac{1}{2}\right) \cdot \left( 1 + \frac{1}{3}\right) \cdot
\left( 1 + \frac{1}{4}\right) \cdots \left( 1 +
\frac{1}{99}\right) . $$
\end{exa}
Solution: Adding each fraction:
$$\frac{3}{2} \cdot \frac{4}{3}\cdot\frac{5}{4}\cdots\frac{100}{99}, $$ which simplifies to
$100/2 = 50.$


\begin{exa} Find integers $a, b$ so that
$$ \left(2 + 1\right)\cdot \left(2^2 + 1\right)\cdot\left(2^{2^{2}} + 1\right)
\cdot\left(2^{2^{3}} + 1\right)\cdots \left(2^{2^{99}} + 1\right)
= 2^a + b .$$ \end{exa} Solution: Using the identity $x^2 - y^2 =
(x - y)(x + y)$ and letting $P$ be the sought product:
$${\everymath{\displaystyle}\begin{array}{lcr}(2 - 1)P & = & (2 - 1)\left(2 + 1\right)\cdot \left(2^2
+ 1\right)\cdot\left(2^{2^{2}} + 1\right)
\cdot\left(2^{2^{3}} + 1\right)\cdots \left(2^{2^{99}} + 1\right) \\
& = & \left(2^2 - 1\right)\cdot \left(2^2 +
1\right)\cdot\left(2^{2^{2}} + 1\right)
\cdot\left(2^{2^{3}} + 1\right)\cdots \left(2^{2^{99}} + 1\right) \\
& = & \left(2^{2^{2}} - 1\right)\cdot\left(2^{2^{2}} + 1\right)
\cdot\left(2^{2^{3}} + 1\right)\cdots \left(2^{2^{99}} + 1\right) \\
& = & \left(2^{2^{3}} - 1\right)\cdot\left(2^{2^{3}} + 1\right)
\cdot\left(2^{2^{4}} + 1\right)\cdots \left(2^{2^{99}} + 1\right) \\
& \vdots & \vdots \\
& = & (2^{2^{99}} - 1)(2^{2^{99}} + 1) \\
& = & 2^{2^{100}} - 1,
\end{array}}$$ whence
$$P = 2^{2^{100}} - 1.$$

\begin{exa} Find the exact value of the product
$$ P = \cos \frac{\pi}{7}\cdot\cos \frac{2\pi}{7}\cdot\cos \frac{4\pi}{7}.$$
\end{exa}
Solution: Multiplying both sides by $\sin \frac{\pi}{7}$ and using
$\sin 2x = 2 \sin x \cos x$ we obtain
$${\everymath{\displaystyle}\begin{array}{lcr}
\sin \frac{\pi}{7}P & = & (\sin \frac{\pi}{7}\cos \frac{\pi}{7})\cdot\cos \frac{2\pi}{7}\cdot\cos \frac{4\pi}{7} \\
& = & \frac{1}{2}(\sin \frac{2\pi}{7}\cos \frac{2\pi}{7})\cdot\cos \frac{4\pi}{7} \\
& = & \frac{1}{4}(\sin \frac{4\pi}{7}\cos \frac{4\pi}{7}) \\
& = & \frac{1}{8}\sin \frac{8\pi}{7}.


\end{array}}$$
As $\sin \frac{\pi}{7} = -\sin \frac{8\pi}{7}$, we deduce that
$$P = - \frac{1}{8}.$$

\begin{exa}
Shew that $$\frac{1}{2}\cdot\frac{3}{4}\cdot\frac{5}{6}\cdots
\frac{9999}{10000} < \frac{1}{100}.$$
\end{exa}
Solution: Let
$$A = \frac{1}{2}\cdot\frac{3}{4}\cdot\frac{5}{6}\cdots \frac{9999}{10000} $$
and
$$ B =\frac{2}{3}\cdot\frac{4}{5}\cdot\frac{6}{7}\cdots\frac{10000}{10001}.$$


Clearly, $x^2 - 1 < x^2$  for all real numbers $x$. This implies
that
$$\frac{x - 1}{x} < \frac{x}{x + 1}$$ whenever these four quantities are
positive. Hence
$${\everymath{\displaystyle}\begin{array}{ccc}
{1}/{2} & < & {2}/{3} \\
{3}/{4} & < & {4}/{5} \\
{5}/{6} & < & {6}/{7} \\
\vdots & \vdots & \vdots \\
{9999}/{10000} & < & {10000}/{10001} \\
\end{array} }$$
As all the numbers involved are positive, we multiply both columns
to obtain
$$\frac{1}{2}\cdot\frac{3}{4}\cdot\frac{5}{6}\cdots \frac{9999}{10000}
<
\frac{2}{3}\cdot\frac{4}{5}\cdot\frac{6}{7}\cdots\frac{10000}{10001},$$
or $A < B.$ This yields $A^2 = A\cdot A < A\cdot B.$ Now
$$A\cdot B = \frac{1}{2}\cdot\frac{2}{3}\cdot\frac{3}{4}\cdot\frac{4}{5}\cdot\frac{5}{6}\cdot\frac{6}{7}
\cdot\frac{7}{8}\cdots\frac{9999}{10000}\cdot\frac{10000}{10001} =
\frac{1}{10001},$$ and consequently, $A^2 < A\cdot B = 1/10001.$
We deduce that $A < 1/\sqrt{10001} < 1/100.$

For the next example we recall that $n!$ ($n$ {\em factorial})
means
$$n! = 1\cdot 2\cdot 3\cdots n.$$For example, $1! = 1, 2!= 1\cdot 2 = 2,$
$3! = 1\cdot 2 \cdot 3 = 6, 4! = 1\cdot 2\cdot 3\cdot 4 = 24.$
Observe that $(k + 1)! = (k + 1)k!.$ We make the convention $0! =
1.$
\begin{exa} Sum
$$ 1\cdot 1! + 2\cdot 2! + 3\cdot 3! + \cdots + 99\cdot 99!.$$
\end{exa}
Solution: From $(k + 1)! = (k + 1)k! = k\cdot k! + k!$ we deduce
$(k + 1)! - k! = k\cdot k!.$ Thus
$$
\begin{array}{lcl}
1\cdot 1! & = &  2! - 1! \\
2\cdot 2! & = & 3! - 2! \\
3\cdot 3! & = & 4! - 3! \\
\vdots & \vdots & \vdots \\
98\cdot 98 & = & 99! - 98! \\
99\cdot 99! & = &  100! - 99!
\end{array}
$$
Adding both columns,
$$1\cdot 1! + 2\cdot 2! + 3\cdot 3! + \cdots + 99\cdot 99! = 100! - 1! = 100! - 1.$$
\section*{Practice}\addcontentsline{toc}{section}{Practice}\markright{Practice}\begin{multicols}{2}\columnseprule 1pt \columnsep 25pt\multicoltolerance=900

\begin{pro} Find a closed formula for
$$D_n = 1 - 2 + 3 - 4 + \cdots + (-1)^{n - 1}n.$$ \end{pro}

\begin{pro} Simplify
$$ \left( 1 - \frac{1}{2^2}\right) \cdot \left( 1 - \frac{1}{3^2}\right) \cdot
\left( 1 - \frac{1}{4^2}\right) \cdots \left( 1 -
\frac{1}{99^2}\right) .
$$
\begin{answer}
$\dfrac{50}{99}$
\end{answer}
\end{pro}

\begin{pro} Simplify
$$\begin{array}{l}\log _2 \left( 1 + \frac{1}{2}\right) + \log _2 \left( 1 + \frac{1}{3}\right) \\
\quad + \log _2 \left( 1 + \frac{1}{4}\right) + \cdots + \log _2
\left( 1 + \frac{1}{1023}\right) .\end{array}$$
\begin{answer}
$9$
\end{answer}
\end{pro}
\begin{pro}
Prove that for all positive integers $n$, $2^{2^n} + 1$ divides
$$ 2^{2^{2^n} + 1} - 2.$$ \end{pro}
\end{multicols}
\section{Arithmetic Sums}

An {\em arithmetic progression} is one of the form
$$a, a + d, a + 2d, a + 3d, \ldots, a + (n - 1)d, \ldots$$


One important arithmetic sum is

$$1 + 2 + \cdots + n = \frac{n(n + 1)}{2}.$$

To obtain a closed form, we utilise
Gauss' trick: \\
If
$$ A_n =  1 + 2 + 3 + \cdots + n $$
 then$$ A_n =  n + (n - 1) +  \cdots + 1.$$Adding these two quantities,
$$ \begin{array}{lcccccccc} A_n & = & 1 & + &  2 &  + &  \cdots & + & n \\
{A_n} & {=} & n & + & (n - 1) & + & \cdots & + & 1 \\
2A_n & =  & (n + 1) & + & (n + 1) & + & \cdots & + & (n + 1) \\
  & = & n(n + 1), & & & & & \end{array}$$since there are $n$ summands. This
  gives $A_n = \dis{\frac{n(n + 1)}{2}}$, that is,
\begin{equation} 1 + 2 + \cdots + n = \frac{n(n + 1)}{2}.\end{equation}


For example,
$$1 + 2 + 3 + \cdots + 100 = \frac{100(101)}{2} = 5050.$$


Applying Gauss's trick to the general arithmetic sum
$$ (a) + (a + d) + (a + 2d) + \cdots + (a + (n - 1)d) $$we obtain
\begin{equation}
(a) + (a + d) + (a + 2d) + \cdots + (a + (n - 1)d) = \frac{n(2a +
(n - 1)d)}{2}
\end{equation}


\begin{exa} Find the sum of all the integers from $1$ to $1000$ inclusive, which are not
multiples of $3$ or $5$.
\end{exa}
Solution: One computes the sum of all integers from 1 to 1000 and
weeds out the sum of the multiples of 3 and the sum of the
multiples of 5, but puts back the multiples of 15, which one has
counted twice. Put
$$A_n = 1 + 2 + 3 + \cdots + n,$$
$$B = 3 + 6 + 9 + \cdots + 999 = 3A_{333},$$
$$C = 5 + 10 + 15 + \cdots + 1000 = 5A_{200},$$
$$D= 15 + 30 + 45 +
\cdots + 990 = 15A_{66}.$$ The desired sum is
$$
\begin{array}{lll}
A_{1000} - B - C + D
 & = & A_{1000} - 3A_{333} - 5A_{200} + 15A_{66} \\
& = & 500500 - 3\cdot 55611  - 5\cdot 20100 + 15\cdot 2211 \\
&  = &  266332. \end{array}$$
\begin{exa}

Each element of the set $\{10, 11, 12, \ldots , 19, 20\}$ is
multiplied by each element of the set $\{21, 22, 23, \ldots , 29,
30\}$. If all these products are added, what is the resulting sum?
\end{exa}
Solution: This is asking for the product $(10 + 11 + \cdots +
20)(21 + 22 + \cdots + 30)$ after all the terms are multiplied.
But
$$10 + 11 + \cdots + 20 = \frac{(20 + 10)(11)}{2} = 165$$ and
$$21 + 22 + \cdots +
30 = \frac{(30 + 21)(10)}{2} = 255.$$ The required  total is
$(165)(255) = 42075$.
\begin{exa}
The sum of a certain number of consecutive positive integers is
$1000$. Find these integers.\end{exa} Solution: Let the the sum of
integers  be $S = (l + 1) +  (l + 2) + (l + n)$. Using Gauss' trick
we obtain $S = \dis{\frac{n(2l + n + 1)}{2}}$. As $S = 1000,$ $2000
= n(2l + n + 1)$. Now $2000 = n^2 + 2ln + n > n^2$, whence $n \leq
\llfloor\sqrt{2000}\rrfloor = 44$. Moreover, $n$ and $2l + n + 1$
divisors of 2000 and are of opposite parity. Since $2000 = 2^45^3$,
the odd factors of 2000 are 1, 5, 25, and 125. We then see that the
problem has te following solutions:
$$n = 1, \ l = 999,$$
$$n = 5, \ l = 197,$$
$$n = 16, \ l = 54,$$
$$n = 25, \ l = 27.$$


\begin{exa}
Find the sum of all integers between $1$ and $100$ that leave
remainder $2$ upon division by $6$.
\end{exa}
Solution: We want the sum of the integers of the form $6r + 2, r =
0, 1, \ldots , 16.$ But this is
$$\sum _{r = 0} ^{16}(6r + 2) = 6\sum _{r = 0} ^{16} r + \sum _{r = 0} ^{16}2 = 6\frac{16(17)}{2} + 2(17)
= 850.$$
\section*{Practice}\addcontentsline{toc}{section}{Practice}\markright{Practice}\begin{multicols}{2}\columnseprule 1pt \columnsep 25pt\multicoltolerance=900

\begin{pro} Shew that
$$ 1 + 2 + 3 + \cdots + (n^2 - 1) + n^2 = \frac{n^2(n^2 + 1)}{2}.$$\end{pro}
\begin{pro} Shew that
$$1 + 3 + 5 + \cdots + 2n - 1 = n^2.$$\end{pro}
\begin{pro}[AHSME 1994] Sum the series
$$20 + 20\frac{1}{5} + 20\frac{2}{5} + \cdots + 40.$$
\begin{answer}
$ 3030$
\end{answer}
\end{pro}
\begin{pro}
Shew that $$\frac{1}{1996} + \frac{2}{1996} + \frac{3}{1996} +
\cdots + \frac{1995}{1996} $$ is an integer.
\end{pro}
\begin{pro}[AHSME 1991] Let $T_n = 1 + 2 + 3 + \cdots  + n$ and
$$P_n = \frac{T_2}{T_2 - 1}\cdot\frac{T_3}{T_3 - 1}\cdot\frac{T_4}{T_4 - 1}\cdots \frac{T_n}{T_n - 1}.$$
Find $P_{1991}.$
\begin{answer}
$\frac{5973}{1993}$
\end{answer}
\end{pro}
\begin{pro}
Given that
$$\frac{1}{a + b}, \frac{1}{b + c}, \frac{1}{c + a}$$are consecutive terms in an arithmetic progression, prove that
$$b^2, a^2, c^2$$are also consecutive terms in an arithmetic progression.
\end{pro}
\begin{pro} Consider the following table:
$$1 = 1$$
$$2 + 3 + 4 = 1 + 8$$
$$ 5 + 6 + 7 + 8 + 9 = 8 + 27$$
$$10 + 11 + 12 + 13 + 14 + 15 + 16 = 27 + 64$$
Conjecture the law of formation and prove your answer.\end{pro}
\begin{pro} The odd natural numbers are arranged as follows:
$$(1)$$
$$(3, 5)$$
$$(7, 9, 11)$$
$$(13, 15, 17, 19)$$
$$(21, 23, 25, 27, 29) $$
$$ ...............................$$
Find the sum of the $n$th row.  \end{pro}
\begin{pro}
Sum
$$1000^2 - 999^2 + 998^2 - 997^2 + \cdots + 4^4 - 3^2 + 2^2 - 1^1.$$
\end{pro}
\begin{pro}
The first term of an arithmetic progression is $14$ and its
$100$th term is $-16.$ Find (i) its $30$th term and (ii) the sum
of all the terms from the first to the $100$th.
\end{pro}
\end{multicols}
\section{Geometric Sums} A {\em geometric
progression} is one of the form

$$a, ar, ar^2, ar^3, \ldots, ar^{n - 1}, \ldots, $$

\begin{exa} Find the following geometric sum:
$$ 1 + 2 + 4 + \cdots + 1024.$$\end{exa}
Solution: Let
$$S = 1 + 2 + 4 + \cdots + 1024.$$ Then
$$2S = 2 + 4 + 8 + \cdots + 1024 + 2048.$$Hence
$$S = 2S - S = (2 + 4 + 8 \cdots + 2048) - (1 + 2 + 4 + \cdots + 1024) = 2048 - 1 = 2047.$$

\begin{exa}  Find the geometric sum
$$x = \frac{1}{3} + \frac{1}{3^2} + \frac{1}{3^3} + \cdots + \frac{1}{3^{99}}.$$\end{exa}
Solution: We have
$$ \frac{1}{3}x = \frac{1}{3^2} + \frac{1}{3^3} + \cdots + \frac{1}{3^{99}} + \frac{1}{3^{100}}.$$
Then
$$\begin{array}{lcl}
\frac{2}{3}x & = & x - \frac{1}{3}x \\
& = & (\frac{1}{3} + \frac{1}{3^2} + \frac{1}{3^3} + \cdots + \frac{1}{3^{99}}) \\
&  & \qquad  - (\frac{1}{3^2} + \frac{1}{3^3} + \cdots + \frac{1}{3^{99}} + \frac{1}{3^{100}}) \\
& = & \frac{1}{3} - \frac{1}{3^{100}}.\end{array}$$From which we
gather
$$ x = \frac{1}{2} - \frac{1}{2\cdot 3^{99}}.$$

The following example presents an {\em arithmetic-geometric} sum.
\begin{exa} Sum $$a = 1 + 2\cdot 4 + 3\cdot 4^2 + \cdots + 10\cdot 4^{9}.$$  \end{exa}
Solution: We have
$$
4a  =  4  +  2\cdot 4^2  +  3\cdot 4^3  +  \cdots +  9\cdot 4^9 +
10\cdot 4^{10}. $$ Now, $4a - a$ yields
$$3a  =  - 1  - 4  - 4^2  - 4^3  -  \cdots  - 4^9 +  10\cdot 4^{10}. $$
Adding this last geometric series,
$$a = \frac{10\cdot 4^{10}}{3} - \frac{4^{10} - 1}{9}.$$

\begin{exa} Find the sum
$$ S_n = 1 + 1/2 + 1/4 + \cdots + 1/2^n.$$Interpret your result as $n \rightarrow \infty$.  \end{exa}
Solution: We have
$$ S_n - \frac{1}{2}S_n = (1 + 1/2 + 1/4 + \cdots + 1/2^n) - (1/2 + 1/4 + \cdots + 1/2^n + 1/2^{n + 1}) =
1 - 1/{2^n}.$$Whence $$ S_n = 2 - 1/2^n.$$So as $n$ varies, we
have:
$$\begin{array}{lcll}
S_1 & = 2 - 1/2^0 & = & 1 \\
S_2 & = 2 - 1/2 &  = & 1.5 \\
S_3 & = 2 - 1/2^2 & = & 1.875 \\
S_4 & = 2 - 1/2^3 & = & 1.875 \\
S_5 & = 2 - 1/2^4 & = & 1.9375 \\
S_6 & = 2 - 1/2^5 & = & 1.96875 \\
S_{10} & = 2 - 1/2^9 & = & 1.998046875 \\
\end{array}$$Thus the farther we go in the series, the closer we get to 2.




Let us sum now the geometric series
$$S = a + ar + ar^2 + \cdots + ar^{n - 1}.$$
Plainly, if $r = 1$ then $S = na$, so we may assume that $r \neq
1$. We have
$$rS = ar + ar^2 + \cdots + ar^n.$$ Hence
$$S - rS = a + ar + ar^2 + \cdots + ar^{n - 1} - ar - ar^2 - \cdots - ar^n = a - ar^n.$$
From this we deduce that $$S = \frac{a - ar^n}{1 - r},$$that is,
\begin{equation} a + ar + \cdots + ar^{n - 1} =  \frac{a - ar^n}{1 - r}
\end{equation}


If $|r| < 1$ then $r^n \rightarrow 0$ as $n \rightarrow \infty$. \\


For $|r| < 1$, we obtain the sum of the infinite geometric series
\begin{equation}  a + ar + ar^2 + \cdots   =  \frac{a}{1 - r}   \end{equation}

\begin{exa} A fly starts at the origin and goes $1$ unit up, $1/2$ unit right, $1/4$ unit down, $1/8$ unit left,
$1/16$ unit up, etc., {\em ad infinitum.} In what coordinates does
it end up?\end{exa} Solution: Its $x$ coordinate is
$$\frac{1}{2} - \frac{1}{8} + \frac{1}{32} - \cdots
= \frac{\frac{1}{2}}{1 - \frac{-1}{4}} = \frac{2}{5}.$$ Its $y$
coordinate is
$$1 - \frac{1}{4} + \frac{1}{16} - \cdots = \frac{1}{1 - \frac{-1}{4}} = \frac{4}{5}.$$Therefore, the
fly ends up in $(\frac{2}{5}, \frac{4}{5}).$

\section*{Practice}\addcontentsline{toc}{section}{Practice}\markright{Practice}\begin{multicols}{2}\columnseprule 1pt \columnsep 25pt\multicoltolerance=900

\begin{pro}
The $6$th term of a geometric progression is $20$ and the $10$th
is $320$. Find (i) its $15$th term, (ii) the sum of its first $30$
terms.
\end{pro}
\end{multicols}
\section{Fundamental Sums}

In this section we compute several sums using telescoping
cancellation.


We start with the sum of the first $n$ positive integers, which we
have already computed using Gauss' trick.

\begin{exa} Find a closed formula for $$A_n = 1 + 2 + \cdots + n.$$
\end{exa}
Solution: Observe that
$$ k^2 - (k - 1)^2 = 2k - 1.$$
From this
$$
\begin{array}{lcl}
1^2 - 0^2 & = & 2\cdot 1 - 1 \\
2^2 - 1^2 & = & 2\cdot 2 - 1 \\
3^2 - 2^2 & = & 2\cdot 3 - 1 \\

\vdots & \vdots & \vdots \\
n^2 - (n - 1)^2 & = & 2\cdot n - 1
\end{array}
$$
Adding both columns,
$$n^2 - 0^2 = 2(1 + 2 + 3 + \cdots  + n) - n.$$
Solving for the sum,
$$1 + 2 + 3 + \cdots + n =  n^2/2 +  {n}/{2} = \frac{n(n + 1)}{2}.$$



\begin{exa} Find the sum
$$1^2 + 2^2 + 3^2 + \cdots + n^2.$$ \end{exa}
Solution: Observe that
$$ k^3 - (k - 1)^3 = 3k^2 - 3k + 1.$$
Hence
$$
\begin{array}{lcl}
1^3 - 0^3 & = & 3\cdot 1^2 - 3\cdot 1 + 1 \\
2^3 - 1^3 & = & 3\cdot 2^2 - 3\cdot 2 + 1 \\
3^3 - 2^3 & = & 3\cdot 3^2 - 3\cdot 3 + 1 \\
\vdots & \vdots & \vdots \\
n^3 - (n - 1)^3 & = & 3\cdot n^2 - 3\cdot n + 1
\end{array}
$$
Adding both columns,
$$n^3 - 0^3 = 3(1^2 + 2^2 + 3^2 + \cdots  + n^2) - 3(1 + 2 + 3 + \cdots + n) + n.$$
From the preceding example $1 + 2 + 3 + \cdots + n = \cdot n^2/2 +
{n}/{2} = \frac{n(n + 1)}{2}$ so

$$n^3 - 0^3 = 3(1^2 + 2^2 + 3^2 + \cdots  + n^2) - \frac{3}{2}\cdot n(n + 1) + n.$$
Solving for the sum,
$$1^2 + 2^2 + 3^2 + \cdots + n^2 = \frac{n^3}{3}  + \frac{1}{2}\cdot n(n + 1) - \frac{n}{3}.$$
After simplifying we obtain
\begin{equation}
1^2 + 2^2 + 3^2 + \cdots + n^2 = \frac{n(n + 1)(2n + 1)}{6}
\end{equation}



\begin{exa} Add the series
$$\frac{1}{1\cdot 2} + \frac{1}{2\cdot 3} + \frac{1}{3\cdot 4} + \cdots
+ \frac{1}{99\cdot 100} .$$ \end{exa} Solution: Observe that
$$ \frac{1}{k(k + 1)} = \frac{1}{k} - \frac{1}{k + 1}.$$Thus
$$
\begin{array}{lcl}
\frac{1}{1\cdot 2} & =  &\frac{1}{1} - \frac{1}{2} \\
\frac{1}{2\cdot 3} & =  &\frac{1}{2} - \frac{1}{3} \\
\frac{1}{3\cdot 4} & =  &\frac{1}{3} - \frac{1}{4} \\
\vdots & \vdots & \vdots \\
\frac{1}{99\cdot 100} & =  &\frac{1}{99} - \frac{1}{100}
\end{array}
$$
Adding both columns,
$$\frac{1}{1\cdot 2} + \frac{1}{2\cdot 3} + \frac{1}{3\cdot 4} + \cdots
+ \frac{1}{99\cdot 100} = 1 - \frac{1}{100} = \frac{99}{100}.$$
\begin{exa} Add
$$ \frac{1}{1\cdot 4} + \frac{1}{4\cdot 7} + \frac{1}{7\cdot 10} + \cdots + \frac{1}{31\cdot 34}.$$
\end{exa}
Solution: Observe that
$$ \frac{1}{(3n + 1)\cdot (3n + 4)} = \frac{1}{3}\cdot \frac{1}{3n + 1} - \frac{1}{3}\cdot\frac{1}{3n + 4}.$$
Thus
$$
\begin{array}{lcl}
\frac{1}{1\cdot 4} & = & \frac{1}{3} - \frac{1}{12} \\
\frac{1}{4\cdot 7} & =  &\frac{1}{12} - \frac{1}{21} \\
\frac{1}{7\cdot 10} & =  &\frac{1}{21} - \frac{1}{30} \\
\frac{1}{10\cdot 13} & =  &\frac{1}{30} - \frac{1}{39} \\
\vdots & \vdots & \vdots \\
\frac{1}{34\cdot 37} & =  &\frac{1}{102} - \frac{1}{111}
\end{array}
$$
Summing both columns,
$$\frac{1}{1\cdot 4} + \frac{1}{4\cdot 7} + \frac{1}{7\cdot 10} + \cdots + \frac{1}{31\cdot 34}
= \frac{1}{3} - \frac{1}{111} = \frac{12}{37} .$$
\begin{exa} Sum
$$ \frac{1}{1\cdot 4\cdot 7} + \frac{1}{4\cdot 7\cdot 10} + \frac{1}{7\cdot 10 \cdot 13} + \cdots + \frac{1}{25\cdot 28\cdot 31}.$$
\end{exa}
Solution: Observe that
$$ \frac{1}{(3n + 1)\cdot (3n + 4)\cdot (3n + 7)} = \frac{1}{6}\cdot \frac{1}{(3n + 1)(3n + 4)} - \frac{1}{6}\cdot\frac{1}{(3n + 4)(3n + 7)}.$$
Therefore
$$
\begin{array}{lcl}
\frac{1}{1\cdot 4\cdot 7} & = & \frac{1}{6\cdot 1\cdot 4} - \frac{1}{6\cdot 4\cdot 7} \\
\frac{1}{4\cdot 7\cdot 10} & =  &\frac{1}{6\cdot 4\cdot 7} - \frac{1}{6\cdot 7\cdot 10} \\
\frac{1}{7\cdot 10\cdot 13} & =  &\frac{1}{6\cdot 7\cdot 10} - \frac{1}{6\cdot 10\cdot 13} \\

\vdots & \vdots & \vdots \\
\frac{1}{25\cdot 28\cdot 31} & =  &\frac{1}{6\cdot 25\cdot 28} -
\frac{1}{6\cdot 28\cdot 31}
\end{array}
$$
Adding each column,
$$ \frac{1}{1\cdot 4\cdot 7} + \frac{1}{4\cdot 7\cdot 10} + \frac{1}{7\cdot 10 \cdot 13} + \cdots
+ \frac{1}{25\cdot 28\cdot 31} = \frac{1}{6\cdot 1\cdot 4} -
\frac{1}{6\cdot 28\cdot 31} = \frac{9}{217}.$$
\begin{exa} Find the sum
$$ 1\cdot 2 + 2\cdot 3 + 3\cdot 4 + \cdots + 99\cdot 100.$$ \end{exa}
Solution: Observe that
$$ k(k + 1) = \frac{1}{3}(k)(k + 1)(k + 2) - \frac{1}{3}(k - 1)(k)(k + 1).$$Therefore

$$
\begin{array}{lcl}
1\cdot 2 & = &  \frac{1}{3}\cdot 1\cdot 2 \cdot 3 - \frac{1}{3}\cdot 0\cdot 1\cdot 2 \\
2\cdot 3 & = &  \frac{1}{3}\cdot 2\cdot 3 \cdot 4 - \frac{1}{3}\cdot 1\cdot 2\cdot 3 \\
3\cdot 4 & = &  \frac{1}{3}\cdot 3\cdot  4\cdot 5 - \frac{1}{3}\cdot 2\cdot 3\cdot 4 \\
\vdots & \vdots & \vdots \\
99\cdot 100 & = &  \frac{1}{3}\cdot 99\cdot 100 \cdot 101 -
\frac{1}{3}\cdot 98\cdot 99\cdot 100
\end{array}
$$
Adding each column,
$$1\cdot 2 + 2\cdot 3 + 3\cdot 4 + \cdots + 99\cdot 100 = \frac{1}{3}\cdot 99\cdot
100\cdot 101 - \frac{1}{3}\cdot 0\cdot 1\cdot 2 = 333300.$$


\section*{Practice}\addcontentsline{toc}{section}{Practice}\markright{Practice}\begin{multicols}{2}\columnseprule 1pt \columnsep 25pt\multicoltolerance=900

\begin{pro} Shew that
\begin{equation}
1^3 + 2^3 + 3^3 + \cdots + n^3 = \left(\frac{n(n + 1)}{2}\right)
^2
\end{equation}
\end{pro}
\begin{pro}Let $a_1, a_2, \ldots , a_n$ be arbitrary numbers. Shew that

$$
\begin{array}{l}
a_1 + a_2(1 + a_1) + a_3(1 + a_1)(1 + a_2) \\ \quad + a_4(1 + a_1)(1 + a_2)(1 + a_3)  + \cdots \\
\quad + a_{n - 1}(1 + a_1)(1 + a_2)(1 + a_3)\cdots (1 + a_{n - 2})  \\
\qquad\ = (1 + a_1)(1 + a_2)(1 + a_3)\cdots (1 + a_n) - 1.
\end{array}$$
\end{pro}
\begin{pro} Shew that
$$\csc 2 + \csc 4 + \csc 8 + \cdots + \csc 2^n = \cot 1 - \cot 2^n.$$
\begin{answer}
Shew first that $\csc 2x = \cot x - \cot 2x.$

\end{answer}
\end{pro}
\begin{pro} Let $0 < x < 1.$ Shew that
$$\sum _{n= 1} ^\infty \frac{x^{2^n}}{1 - x^{2^{n + 1}}} = \frac{x}{1 - x}.$$
\begin{answer}
Observe that
$$ \frac{y}{1 - y^2} = \frac{1}{1 - y} - \frac{1}{1 - y^2}.
$$
\end{answer}
\end{pro}
\begin{pro} Shew that
$$\begin{array}{l}\tan \frac{\pi}{2^{100}} + 2\tan \frac{\pi}{2^{99}} \\ \quad + 2^{2}\tan \frac{\pi}{2^{2^{98}}} +
\cdots + 2^{98}\tan \frac{\pi}{2^2}\\ \qquad = \cot
\frac{\pi}{2^{100}}.\end{array}$$
\end{pro}
\begin{pro} Shew that
$$\sum _{k = 1} ^n \frac{k}{k^4 + k^2 + 1} = \frac{1}{2}\cdot\frac{n^2 + n}{n^2 + n + 1}.$$
\end{pro}
\begin{pro} Evaluate
$$\left(\frac{1\cdot 2\cdot 4 + 2 \cdot 4\cdot 8 + 3\cdot 6\cdot 12 + \cdots}{
1\cdot 3 \cdot 9 + 2\cdot 6\cdot 18 + 3\cdot 9\cdot 27 +
\cdots}\right) ^{1/3}.$$
\end{pro}
\begin{pro} Shew that $$\sum _{n = 1} ^\infty \arctan \frac{1}{1 + n + n^2} = \frac{\pi}{4}.$$
\end{pro}
Hint: From $$\tan x - \tan y = \frac{\tan x - \tan y}{1 + \tan
x\tan y}$$deduce that $$\arctan a - \arctan b = \arctan \frac{a -
b}{1 + ab}$$for suitable $a$ and $b$.
\begin{pro} Prove the following result due to Gramm
$$ \prod _{n = 2} ^\infty \frac{n^3 - 1}{n^3 + 1} = \frac{2}{3}.$$
\end{pro}
\end{multicols}
\section{First Order Recursions} We have already
seen the Fibonacci numbers, defined by the recursion $f_0 = 0, f_1
= 1$ and
$$f_{n + 1} = f_n + f_{n - 1}, \ n \geq 1.$$



The {\em order} of the recurrence is  the difference between the
highest and the lowest subscripts. For example
$$u_{n + 2} - u_{n + 1} = 2$$ is of the first order, and
$$u_{n + 4} + 9u^2 _n = n^5$$is of the fourth order.




A recurrence is {\em linear} if the subscripted letters appear
only to the first power. For example

$$u_{n + 2} - u_{n + 1} = 2$$ is a linear recurrence and
$$x ^2 _n + nx_{n - 1} = 1 \ \ {\rm and } \ \ x_n + 2^{x_{n - 1}} = 3 $$are not linear
recurrences.



A recursion is {\em homogeneous} if all its terms contain the
subscripted variable to the same power. Thus
$$x_{m + 3} + 8x_{m + 2} - 9x_m = 0$$ is homogeneous. The equation
$$x_{m + 3} + 8x_{m + 2} - 9x_m = m^2 - 3$$is not homogeneous.



A {\em closed form} of a recurrence is a formula that permits us
to find the $n$-th term of the recurrence without having to know a
priori the terms preceding it.



We outline a method for solving first order linear recurrence
relations of the form
$$x_n = ax_{n - 1} + f(n), a \neq 1,$$ where $f$ is a polynomial. \\
\begin{enumerate}
\item First solve the homogeneous recurrence $x_n = ax_{n - 1}$ by
``raising the subscripts'' in the form $x^n = ax^{n - 1}$. This we
call the {\em characteristic equation}. Cancelling this gives $x =
a.$ The solution to the homogeneous equation $x_n = ax_{n - 1}$
will be of the form $x_n = Aa^n$, where $A$ is a constant to be
determined. \item Test a solution of the form $x_n = Aa^n + g(n),$
where $g$ is a polynomial of the same degree as $f$.
\end{enumerate}




\begin{exa} Let $x_0 = 7$ and $x_n = 2x_{n - 1}, n \geq 1.$ Find a closed form for $x_n.$\end{exa}
Solution: Raising subscripts we have the characteristic equation
$x^n = 2x^{n - 1}$. Cancelling, $x = 2$. Thus we try a solution of
the form $x_n = A2^n$, were $A$ is a constant. But $7 = x_0 =
A2^0$ and so $A = 7.$
The solution is thus $x_n = 7(2)^n$. \\
{\em Aliter:} We have
$$
\begin{array}{lcl}
x_0 & = & 7 \\
x_1 & = & 2x_0 \\
x_2 & = & 2x_1 \\
x_3 & = & 2x_2 \\
\vdots & \vdots & \vdots \\
x_n & = & 2x_{n-1} \\
\end{array}
$$
Multiplying both columns,
$$x_0x_1\cdots x_n = 7\cdot 2^nx_0x_1x_2\cdots x_{n - 1}.$$Cancelling the common factors on both sides of
the equality,
$$x_n = 7\cdot 2^n.$$

\begin{exa} Let $x_0 = 7$ and $x_n = 2x_{n - 1} + 1, n \geq 1.$ Find a closed form for $x_n.$  \end{exa}
Solution: By raising the subscripts in the homogeneous equation
we obtain $x^n = 2x^{n - 1}$ or $x = 2$. A solution to the
homogeneous equation will be of the form $x_n = A(2)^n$. Now $f(n)
= 1$ is a polynomial of degree 0 (a constant) and so we test a
particular constant solution $C$. The general solution will have
the form $x_n = A2^n + B$. Now, $7 = x_0 = A2^0 + B = A + B$.
Also, $x_1 = 2x_0 + 7 = 15$ and so $15 = x_1 = 2A + B$. Solving
the simultaneous equations
$$A + B = 7,$$
$$2A + B = 15,$$we find $A = 8, B = -1.$ So the solution is $x_n = 8(2^n) - 1 = 2^{n + 3} - 1.$ \\
{\em Aliter:} We have:
$$
\begin{array}{lcl}
x_0 & = & 7 \\
x_1 & = & 2x_0  + 1\\
x_2 & = & 2x_1  + 1\\
x_3 & = & 2x_2  + 1\\
\vdots & \vdots & \vdots \\
x_{n - 1} & = & 2x_{n - 2} + 1 \\
x_n & = & 2x_{n-1} + 1\\
\end{array}
$$
Multiply the  $k$th row by $2^{n - k}$. We  obtain
$$
\begin{array}{lcl}
2^nx_0 & = & 2^n\cdot 7 \\
2^{n - 1}x_1 & = & 2^nx_0  + 2^{n - 1}\\
2^{n - 2}x_2 & = & 2^{n - 1}x_1  + 2^{n - 2}\\
2^{n - 3}x_3 & = & 2^{n - 2}x_2  + 2^{n - 3}\\
\vdots & \vdots & \vdots \\
2^2x_{n - 2} & = & 2^3x_{n - 3} + 2^2 \\
2x_{n - 1} & = & 2^2x_{n - 2} + 2 \\
x_n & = & 2x_{n-1} + 1\\
\end{array}
$$Adding both columns, cancelling, and adding the geometric sum,
$$x_n = 7\cdot 2^n + (1 + 2 + 2^2 + \cdots + 2^{n - 1}) = 7\cdot 2^n + 2^n - 1 = 2^{n + 3} - 1.$$
{\em Aliter:} Let $u_n = x_n + 1 = 2x_{n - 1} + 2 = 2(x_{n - 1} +
1) = 2u_{n - 1}.$  We solve the recursion $u_n = 2u_{n - 1}$ as we
did on our first example: $u_n = 2^nu_0 = 2^n(x_0 + 1) = 2^n\cdot
8 = 2^{n + 3}. $ Finally, $x_n = u_n - 1 = 2^{n + 3} - 1.$
\begin{exa}
Let $x_0 = 2, x_n = 9x_{n - 1} - 56n + 63$. Find a closed form for
this recursion.
\end{exa}
Solution: By raising the subscripts in the homogeneous equation
we obtain the characteristic equation $x^n = 9x^{n - 1}$ or $x =
9$. A solution to the homogeneous equation will be of the form
$x_n = A(9)^n$. Now $f(n) = -56n + 63$ is a polynomial of degree 1
and so we test a particular solution of the form $Bn + C$. The
general solution will have the form $x_n = A9^n + Bn + C$. Now
$x_0 = 2, x_1 = 9(2) - 56 + 63 = 25, x_2 = 9(25) - 56(2) + 63 =
176$. We thus solve the system
$$2 = A + C,$$
$$25 = 9A + B + C,$$
$$176 = 81A + 2B + C.$$
We find $A = 2, B = 7, C = 0.$ The general solution is $x_n =
2(9^n) + 7n.$
\begin{exa}
Let $x_0 = 1, x_n = 3x_{n - 1} - 2n^2 + 6n - 3$. Find a closed
form for this recursion.
\end{exa}
Solution: By raising the subscripts in the homogeneous equation
we obtain the characteristic equation $x^n = 3x^{n - 1}$ or $x =
9$. A solution to the homogeneous equation will be of the form
$x_n = A(3)^n$. Now $f(n) = - 2n^2 + 6n - 3$ is a polynomial of
degree 2 and so we test a particular solution of the form $Bn^2 +
Cn + D$. The general solution will have the form $x_n = A3^n +
Bn^2 + Cn + D$. Now $x_0 = 1, x_1 = 3(1) - 2 + 6 - 3 = 4, x_2 =
3(4) - 2(2)^2 + 6(2) - 3 = 13, x_3 = 3(13) - 2(3)^2 + 6(3) - 3 =
36$. We thus solve the system
$$1 = A + D,$$
$$4 = 3A + B + C + D,$$
$$13 = 9A + 4B + 2C + D,$$
$$36 = 27A + 9B + 3C + D.$$
We find $A = B = 1, C = D = 0.$ The general solution is $x_n = 3^n
+ n^2.$

\begin{exa}
Find a closed form for $x_{n} = 2x_{n - 1} + 3^{n - 1}, x_0 = 2.$
\end{exa}
Solution: We test a solution of the form $x_n = A2^n + B3^n$. Then
$x_0 = 2, x_1 = 2(2) + 3^0 = 5.$ We solve the system
$$2 = A + B,$$
$$7 = 2A + 3B.$$We find $A = 1, B = 1.$ The general solution is $x_n = 2^n + 3^n.$




We now tackle the case when $a = 1.$ In this case, we simply
consider a polynomial $g$ of degree 1 higher than the degree of
$f$.



\begin{exa} Let $x_0 = 7$ and $x_n = x_{n - 1} + n, n \geq 1.$ Find a closed formula for $x_n.$ \end{exa}
Solution: By raising the subscripts in the homogeneous equation
we obtain the characteristic equation $x^n = x^{n - 1}$ or $x =
1$. A solution to the homogeneous equation will be of the form
$x_n = A(1)^n = A$, a constant. Now $f(n) = n$ is a polynomial of
degree 1 and so we test a particular solution of the form $Bn^2 +
Cn + D$, one more degree than that of $f$. The general solution
will have the form $x_n = A + Bn^2  + Cn + D$. Since $A$ and $D$
are constants, we may combine them to obtain $x_n = Bn^2 + Cn +
E.$ Now, $x_0 = 7, x_1 = 7 + 1 = 8, x_2 = 8 + 2 = 10.$ So we solve
the system
$$7 = E,$$
$$8 = B + C + E,$$
$$10 = 4B + 2C + E.$$ We find $\dis{B = C = \frac{1}{2}, E = 7}$. The general solution is
$\dis{x_n = \frac{n^2}{2} + \frac{n}{2} + 7}$. \\
{\em Aliter:} We have
$$
\begin{array}{lcl}
x_0 & = & 7 \\
x_1 & = & x_0 + 1 \\
x_2 & = & x_1  + 2\\
x_3 & = & x_2  + 3\\
\vdots & \vdots & \vdots \\
x_n & = & x_{n-1} + n \\
\end{array}
$$
Adding both columns,
$$x_0 + x_1 + x_2 + \cdots + x_n = 7 + x_0 + x_2 + \cdots + x_{n - 1} + (1 + 2 + 3 + \cdots + n).$$
Cancelling and using the fact that $\dis{1 + 2 + \cdots + n =
\frac{n(n + 1)}{2}}$,
$$x_n = 7 + \frac{n(n + 1)}{2}.$$




Some non-linear first order recursions maybe reduced to a linear
first order recursion by a suitable transformation.
\begin{exa}
A recursion satisfies $u_0 = 3, u_{n + 1} ^2 = u_{n}, n \geq 1.$
Find a closed form for this recursion.

\end{exa}
Solution: Let $v_n = \log u_n.$ Then $v_{n } = \log u_{n } = \log
u_{n - 1} ^{1/2} = \frac{1}{2} \log u_{n - 1} = \frac{v_{n -
1}}{2}.$ As $v_n = v_{n - 1}/2,$ we have $v_n = v_0/2^n$, that is,
$\log u_n = (\log u_0)/2^n$. Therefore, $u_n = 3^{1/2^n}.$

\section*{Practice}\addcontentsline{toc}{section}{Practice}\markright{Practice}\begin{multicols}{2}\columnseprule 1pt \columnsep 25pt\multicoltolerance=900

\begin{pro}
Find a closed form for $\dis{x_0 = 3, x_n = \frac{x_{n - 1} +
4}{3}.}$
\begin{answer}
 $x_n = \dis{\frac{1}{3^n} + 2}$.
\end{answer}
\end{pro}
\begin{pro}
Find a closed form for $\dis{x_0 = 1, x_n = 5x_{n - 1} - 20n +
25.}$
\begin{answer}
$x_n = 5^n + 5n.$
\end{answer}
\end{pro}
\begin{pro}
Find a closed form for $\dis{x_0 = 1, x_n = x_{n - 1} + 12n.}$
\begin{answer}
$x_n = 6n^2 + 6n + 1.$
\end{answer}
\end{pro}
\begin{pro}
Find a closed form for $x_n = 2x_{n - 1} + 9(5^{n - 1}), x_0 = 5.$
\begin{answer}
$x_n = 2^n + 3(5^n).$
\end{answer}
\end{pro}
\begin{pro}
Find a closed form for
$$a_0 = 5, a_{j + 1} = a_j ^2 + 2a_j, j \geq 0.$$
\begin{answer}
$a_{j + 1} = 6^{2^j} - 1.$
\end{answer}
\end{pro}


\begin{pro}[AIME, 1994] If $n \geq 1$,

$$x_n + x_{n - 1} = n^2.$$Given that $x_{19} = 94,$ find the remainder when $x_{94}$ is divided by
$1000.$
\end{pro}

\begin{pro} Find a closed form for
$$x_0 = -1; \ x_n = x_{n - 1} + n^2, n > 0.$$
\end{pro}

\begin{pro}
If $u_0 = 1/3$ and $u_{n + 1} = 2u_n ^2 - 1,$ find a closed form
for $u_n$.
\begin{answer}
Let $u_n = \cos v_n.$
\end{answer}
\end{pro}
\begin{pro}
Let $x_1 = 1, x_{n + 1} = x_n ^2 - x_n + 1, n > 0.$ Shew that
$$\sum _{n = 1} ^\infty \frac{1}{x_n} = 1.$$
\end{pro}

\end{multicols}
\section{Second Order Recursions} All the
recursions that we have so far examined are first order
recursions, that is, we find the next term of the sequence given
the preceding one. Let us now briefly examine how to solve some
second order recursions.



We now outline a method for solving second order homogeneous
linear recurrence relations of the form
$$x_n = ax_{n - 1} + bx_{n - 2}.$$ \\
\begin{enumerate}
\item Find the characteristic equation  by ``raising the
subscripts'' in the form $x^n = ax^{n - 1}  + bx^{n - 2}$.
Cancelling this gives $x^2 - ax - b = 0.$ This equation has two
roots $r_1$ and $r_2.$ \item If the roots are different, the
solution will be of the form $x_n = A(r_1)^n + B(r_2)^n$, where
$A, B$ are  constants. \item If the roots are identical, the
solution will be of the form $x_n = A(r_1)^n + Bn(r_1)^n$.


\end{enumerate}
\begin{exa}
Let $x_0 = 1, x_1 = -1, x_{n + 2} + 5x_{n + 1} + 6x_n = 0$.

\end{exa}
Solution: The characteristic equation is $x^2 + 5x + 6 = (x + 3)(x
+ 2) = 0$. Thus we test a solution of the form $x_n = A(-2)^n +
B(-3)^n.$ Since $1 = x_0 = A + B, -1 = -2A - 3B$, we quickly find
$A = 2, B = -1.$ Thus the solution is $x_{n} = 2(-2)^n -(-3)^n.$
\begin{exa}
Find a closed form for the Fibonacci recursion $f_0 = 0, f_1 = 1,
f_n = f_{n - 1} + f_{n - 2}$.
\end{exa}
Solution: The characteristic equation is $f^2 - f - 1 = 0$, whence
a solution will have the form $\dis{f_n = A\left(\frac{1 +
\sqrt{5}}{2}\right)^n + B\left(\frac{1 - \sqrt{5}}{2}\right)^n}$.
The initial conditions give
$$0 = A + B,$$
$$1 = A\left(\frac{1 + \sqrt{5}}{2}\right) + B\left(\frac{1 - \sqrt{5}}{2}\right)
= \frac{1}{2}\left(A + B\right) + \frac{\sqrt{5}}{2}\left(A -
B\right) = \frac{\sqrt{5}}{2}\left(A - B\right)$$ This gives
$\dis{A = \frac{1}{\sqrt{5}}, B = -\frac{1}{\sqrt{5}}}$. We thus
have the {\em Cauchy-Binet Formula:}
\begin{equation}
f_n = \frac{1}{\sqrt{5}}\left(\frac{1 + \sqrt{5}}{2}\right)^n -
\frac{1}{\sqrt{5}}\left(\frac{1 - \sqrt{5}}{2}\right)^n
\end{equation}
\begin{exa}

Solve the recursion $x_0 = 1, x_1 = 4, x_n = 4x_{n - 1}  - 4x_{n -
2} = 0.$
\end{exa}
Solution: The characteristic equation is $x^2 - 4x + 4 = (x - 2)^2
= 0$. There is a multiple root and so we must test a solution of
the form $x_n = A2^n + Bn2^n.$ The initial conditions give
$$1 = A,$$
$$4 = 2A + 2B.$$ This solves to $A = 1, B = 1.$ The solution is thus $x_n = 2^n + n2^n$.
\clearpage
\section*{Practice}\addcontentsline{toc}{section}{Practice}\markright{Practice}
\begin{multicols}{2}\columnseprule 1pt \columnsep
25pt\multicoltolerance=900
\begin{pro}

Solve the recursion $x_0 = 0, x_1 = 1, x_n = 10x_{n - 1}  - 21x_{n
- 2}$.
\end{pro}
\begin{pro}

Solve the recursion $x_0 = 0, x_1 = 1, x_n = 10x_{n - 1}  - 25x_{n
- 2}$.
\end{pro}
\begin{pro}

Solve the recursion $x_0 = 0, x_1 = 1, x_n = 10x_{n - 1}  - 21x_{n
- 2} + n$.
\end{pro}
\begin{pro}

Solve the recursion $x_0 = 0, x_1 = 1, x_n = 10x_{n - 1}  - 21x_{n
- 2} + 2^n$.
\end{pro}
\end{multicols}




\section{Applications of Recursions}

\begin{exa} Find the recurrence relation for the number of $n$ digit binary
sequences with no pair of consecutive $1$'s. \end{exa} Solution:
It is quite easy to see that $a_1 = 2, a_2 = 3.$ To form $a_n, n
\geq 3,$ we condition on the last digit. If it is 0, the number of
sequences sought is $a_{n - 1}$. If it is 1, the penultimate digit
must be 0, and the number of sequences sought is $a_{n - 2}$. Thus
$$ a_n = a_{n - 1} + a_{n - 2}, \vspace{5mm} a_1 = 2, \ a_2 = 3.$$
\begin{exa} Let there be drawn $n$ ovals on the plane. If an oval intersects
each of the other ovals at exactly two points and no three ovals
intersect at the same point, find a recurrence relation for the
number of regions into which the plane is divided. \end{exa}
Solution: Let this number be $a_n$. Plainly $a_1 = 2$.  After the
$n - 1$th stage, the $n$th oval intersects the previous ovals at
$2(n - 1)$ points, i.e. the $n$th oval is divided into $2(n - 1)$
arcs. This adds $2(n - 1)$ regions to the $a_{n - 1}$ previously
existing. Thus $$ a_n = a_{n - 1} + 2(n -1), \ a_1 = 2.$$
\begin{exa} Find a recurrence relation for the number of regions into which
the plane is divided by $n$ straight lines if every pair of lines
intersect, but no three lines intersect. \end{exa} Solution: Let
$a_n$ be this number. Clearly $a_1 = 2.$ The $n$th line is cut by
he previous $n - 1$ lines at $n - 1$ points, adding $n$ new
regions to the previously existing $a_{n -1}.$ Hence $$ a_n = a_{n
- 1} + n, \ a_1 = 2 .$$
\begin{exa} {\em (Derangements)} An absent-minded secretary is filling $n$
envelopes with $n$ letters. Find a recursion for the number $D_n$
of ways in which she never stuffs the right letter into the right
envelope.\end{exa} Solution: Number the envelopes $1, 2, 3, \cdots
, n$. We condition on the last envelope.  Two events might happen.
Either $n$ and $r (1 \leq r \leq n - 1)$ trade places or they do
not.



In the first case, the two letters $r$ and $n$ are misplaced. Our
task is just to misplace the other $n - 2$ letters, $(1, 2, \cdots
, r - 1, r + 1, \cdots , n - 1)$ in the slots $(1, 2, \cdots , r -
1, r + 1, \cdots , n - 1).$ This can be done in $D_{n - 2}$ ways.
Since $r$ can
be chosen in $n - 1$ ways, the first case can happen in $(n - 1)D_{n - 2}$ ways.\\




In the second case, let us say that letter $r$, $(1 \leq r \leq n
- 1)$ moves to the $n$-th position but $n$ moves not to the $r$-th
position.  Since $r$ has been misplaced, we can just ignore it.
Since $n$ is not going to the $r$-th position, we may relabel $n$
as $r$.  We now have $n - 1$ numbers to misplace, and this can be
done in
$D_{n - 1}$ ways.\\
As $r$ can be chosen in $n - 1$ ways, the total number of ways for
the second case is $(n - 1)D_{n - 1}.$  Thus $D_n = (n - 1)D_{n -
2} + (n - 1)D_{n - 1}.$

\begin{exa}
There are two urns, one is full of water and the other is empty.
On the first stage, half of the contains of urn I is passed into
urn II. On the second stage 1/3 of the contains of urn II is
passed into urn I. On stage three, 1/4 of the contains of urn I is
passed into urn II. On stage four 1/5 of the contains of urn II is
passed into urn I, and so on. What fraction of water remains in
urn I after the 1978th stage?

\end{exa}
Solution: Let $x_n, y_n, n = 0, 1, 2, \ldots$ denote the fraction
of water in urns I and II respectively at stage $n$. Observe that
$x_n + y_n = 1$ and that
\renewcommand{\arraystretch}{2}

$$
\begin{array}{ll}
x_0 = 1;  y_0 = 0 &  \\
& x_1 = x_0 - \frac{1}{2}x_0 = \frac{1}{2}; y_1 = y_1 + \frac{1}{2}x_0 = \frac{1}{2}\\
& x_2 = x_1 + \frac{1}{3}y_1 = \frac{2}{3}; y_2 = y_1 - \frac{1}{3}y_1 = \frac{1}{3} \\
& x_3 = x_2 - \frac{1}{4}x_2 = \frac{1}{2}; y_1 = y_1 + \frac{1}{4}x_2 = \frac{1}{2} \\
& x_4 = x_3 + \frac{1}{5}y_3 = \frac{3}{5}; y_1 = y_1 - \frac{1}{5}y_3 = \frac{2}{5} \\
& x_5 = x_4 - \frac{1}{6}x_4 = \frac{1}{2}; y_1 = y_1 + \frac{1}{6}x_4 = \frac{1}{2} \\
& x_6 = x_5 + \frac{1}{7}y_5 = \frac{4}{7}; y_1 = y_1 - \frac{1}{7}y_5 = \frac{3}{7} \\
& x_7 = x_6 - \frac{1}{8}x_6 = \frac{1}{2}; y_1 = y_1 + \frac{1}{8}x_6 = \frac{1}{2} \\
& x_8 = x_7 + \frac{1}{9}y_7 = \frac{5}{9}; y_1 = y_1 - \frac{1}{9}y_7 = \frac{4}{9} \\
\end{array}
$$
A pattern emerges (which may be proved by induction) that at each
odd stage $n$ we have $x_n = y_n = \frac{1}{2}$ and that at each
even stage we have (if $n = 2k$) $x_{2k} = \frac{k + 1}{2k + 1},
y_{2k} = \frac{k}{2k + 1}$. Since $\frac{1978}{2} = 989$ we have
$x_{1978} = \frac{990}{1979}$.



\section*{Practice}\addcontentsline{toc}{section}{Practice}\markright{Practice}\begin{multicols}{2}\columnseprule 1pt \columnsep 25pt\multicoltolerance=900

\begin{pro} At the {\em Golem Gambling Casino Research Institute} an
experiment is performed by rolling a die until two odd numbers
have appeared (and then the experiment stops). The tireless
researchers wanted to find a recurrence relation for the number of
ways to do this. Help them!
\begin{answer}
$a_0 = 0, \ a_n = a_{n - 1} + (n - 1)3^n .$
\end{answer}

\end{pro}
\begin{pro} Mrs. Rosenberg has $\$ 8 \ 000 \ 000$ in one of her five savings
accounts. In this one, she earns $15 \%$ interest per year. Find a
recurrence relation for the amount of money after $n$ years.
\end{pro}
\begin{pro} Find a recurrence relation for the number of ternary $n$-digit
sequences with no consecutive $2$'s. \end{pro}
\end{multicols}
\chapter{Counting}
\section{Inclusion-Exclusion}
 In this section we investigate a tool for counting unions of events. It is known as {\em The Principle
  of Inclusion-Exclusion} \index{inclusion-exclusion principle}or Sylvester-Poincar\'{e} Principle.
\begin{thm}[Two set Inclusion-Exclusion] \label{thm:two_set_incl_excl}
$$\card{A \cup B} = \card{A} + \card{B} - \card{A \cap B}
$$\end{thm}
\begin{pf} We have  $$A\cup B = (A\setminus B) \cup (B\setminus A) \cup (A\cap B),$$
and this last expression is a union of disjoint sets. Hence
$$\card{A\cup B} = \card{A\setminus B}+ \card{B\setminus A} +\card{A\cap B}.$$
But $$ A\setminus B = A \setminus (A\cap B) \implies
\card{A\setminus B} = \card{A} - \card{A\cap B},$$ $$B\setminus A =
B \setminus (A\cap B) \implies \card{B\setminus A} = \card{B} -
\card{A\cap B},
$$from where we deduce the result.
\end{pf}
In the Venn diagram \ref{fig:2_set_incl_excl}, we mark by $R_1$ the
number of elements which  are simultaneously in both sets (i.e., in
$A \cap B$), by $R_2$ the number of elements which are in $A$ but
not in $B$ (i.e., in $A \setminus B$), and by $R_3$ the number of
elements which are $B$  but not in $A$ (i.e., in $B \setminus A$).
We have $R_1 + R_2 + R_3 = \card{A \cup B}$, which illustrates the
theorem.
 \vspace{2cm}
\begin{figure}[h]
\centering
\begin{minipage}{7cm}
$$
\psset{unit=1.5pc} \pscircle(-1,0){2} \pscircle(1,0){2}
\rput(0,0){R_1} \rput(-2, 0){R_2} \rput(2, 0){R_3}
\rput(-3, 1.5){A} \rput(3, 1.5){B}$$ \vspace{.5in}
\footnotesize\hangcaption{Two-set Inclusion-Exclusion}
\label{fig:2_set_incl_excl}
\end{minipage}
\begin{minipage}{7cm}
$$
\psset{unit=1.5pc} \pscircle(-1,0){2} \pscircle(1,0){2}
\rput(0,0){\tiny 10} \rput(-2, 0){\tiny 18} \rput(2, 0){\tiny 6}
\rput(2.5, 2.5){\tiny 6} \rput(-3, 1.5){\tiny A} \rput(3, 1.5){\tiny
B} $$\vspace{1cm} \footnotesize\footnotesize\hangcaption{Example
\ref{exa:incl_excl_1}.}\label{fig:incl_excl_1}\end{minipage}
\end{figure}
\begin{exa}
Of  $40$ people, $28$ smoke and  $16$ chew tobacco. It is also known
that  $10$ both smoke and chew. How many among the $40$ neither
smoke nor chew? \label{exa:incl_excl_1}\end{exa} Solution: Let $A$
denote the set of smokers and $B$ the set of chewers. Then
$$\card{A \cup B} = \card{A} + \card{B} - \card{A \cap B} = 28 + 16 - 10 = 34,$$
meaning that there are 34 people that either smoke or chew (or
possibly both). Therefore the number of people that neither smoke
nor chew is $40 - 34 = 6.$ \\
{\em Aliter}: We fill up the Venn diagram in figure
\ref{fig:incl_excl_1} as follows. Since $\card{A \cap B} = 10$, we
put a $10$ in the intersection. Then we put a $28 - 10 = 18$ in the
part that $A$ does not overlap $B$ and a $16 - 10 = 6$ in the part
of $B$ that does not overlap $A$. We have accounted for $10 + 18 + 6
= 34$ people that are in at least one of the set. The remaining $40
- 34 = 6$ are outside these sets.
\begin{exa}
How many integers between $1$ and $1000$ inclusive, do not share a
common factor with $1000$, that is, are relatively prime to $1000$?
\end{exa}
Solution:   Observe that $1000 = 2^35^3$, and thus from the $1000$
integers we must weed out those that have a factor of $2$ or of $5$
in their prime factorisation. If $A_2$ denotes the set of those
integers divisible by 2 in the interval $[1; 1000]$ then clearly
$\dis{ \card{A_2} = \llfloor \frac{1000}{2} \rrfloor} = 500$.
Similarly, if $A_5$ denotes the set of those integers divisible by
$5$ then $\dis{ \card{A_5} = \llfloor \frac{1000}{5} \rrfloor} =
200$. Also $\dis{ \card{A_2 \cap A_5} = \llfloor \frac{1000}{10}
\rrfloor} = 100$. This means that there are $\card{A_2 \cup A_5} =
500 + 200 - 100 = 600$ integers  in the interval $[1; 1000]$ sharing
at least a factor with $1000$, thus there are $1000 - 600 = 400$
integers in $[1; 1000]$ that do not share a factor prime factor with
$1000$.

\bigskip

We now deduce a formula for counting the number of elements of a
union of three events. \vspace{4cm}
\begin{figure}[h]
\begin{center}
\psset{unit=2pc} \pscircle(-1,0){2} \pscircle(1,0){2}
\pscircle(0,1.41421356){2} \rput(2,0){$R_1$} \rput(-2,
0){$R_2$} \rput(0,0.707106781){$R_3$} \rput(0,
2.3){$R_4$} \rput(0,-0.8){$R_5$} \rput(-1.43,1.43){$R_6$} \rput(1.43,1.43){$R_7$} \rput(-4.5, 0){$A$}
\rput(4.5, 0){$B$} \rput(0, 4.2){$C$}
\end{center}
\vspace{2cm} \footnotesize\footnotesize\hangcaption{Three-set
Inclusion-Exclusion} \label{fig:3set_incl_excl}\end{figure}
\begin{thm}[Three set
Inclusion-Exclusion]\label{thm:three_set_incl_excl} Let $A, B, C$ be
events of the same sample space $\Omega$. Then
$$
\begin{array}{lll}
\card{A \cup B \cup C} & = & \card{A} + \card{B} + \card{C}\\ & & -
\card{A \cap B} - \card{B \cap C }   - \card{C \cap A} \\ & & +
\card{A \cap B \cap C}
\end{array}
$$

\end{thm}
\begin{pf}
Using the associativity and distributivity of unions of sets, we see
that
$$
\begin{array}{lll}
\card{A \cup B \cup C} & = & \card{A \cup (B \cup C)} \\
& = & \card{A} + \card{B \cup C} - \card{A \cap (B \cup C)} \\
& = & \card{A} + \card{B \cup C} - \card{(A \cap B) \cup (A \cap C)}  \\
& = & \card{A} + \card{B} + \card{C} - \card{B \cap C} \\
& &  - \card{A \cap B} - \card{A \cap C}\\ & & + \card{(A\cap B)
\cap (A \cap C)}  \\
& = & \card{A} + \card{B} + \card{C} - \card{B \cap C}  \\
& & - \left( \card{A \cap B} + \card{A \cap C}  - \card{A\cap B \cap C}\right) \\
& = & \card{A} + \card{B} + \card{C} \\ & & \quad - \card{A \cap B}
- \card{B \cap C } - \card{C \cap A}\\ & &  \quad + \card{A \cap B
\cap C} .\
\end{array}
$$
This gives the Inclusion-Exclusion Formula for three sets. See also
figure \ref{fig:3set_incl_excl}.

\end{pf}
\begin{rem} In the Venn diagram in figure
\ref{fig:3set_incl_excl} there are $8$ disjoint regions: the 7 that
form $A \cup B \cup C$ and the outside region, devoid of any element
belonging to $A \cup B \cup C$.
\end{rem}
\begin{exa}
How many integers between $1$ and $600$ inclusive are not divisible
by neither $3,$ nor $5$, nor $7$?
\end{exa}
Solution: Let $A_k$ denote the numbers in $[1; 600]$ which are
divisible by $k$. Then
$$
\begin{array}{lllll}
\card{A_3} & = & \llfloor\frac{600}{3}\rrfloor        & = & 200, \\
\card{A_5} & = & \llfloor\frac{600}{5}\rrfloor        & = & 120, \\
\card{A_7} & = & \llfloor \frac{600}{7}\rrfloor       & = & 85, \\
\card{A_{15}} &  = & \llfloor \frac{600}{15}\rrfloor  & = & 40 \\
\card{A_{21}} &  = & \llfloor \frac{600}{21}\rrfloor  & = & 28\\
\card{A_{35}} &  = & \llfloor \frac{600}{35}\rrfloor  & = & 17 \\
\card{A_{105}} &  = &\llfloor \frac{600}{105}\rrfloor & = & 5\\
\end{array}
$$
By Inclusion-Exclusion there are $200 + 120 + 85 - 40 - 28 - 17 + 5
= 325$ integers in $[1; 600]$ divisible by at least one of $3$, $5$,
or $7$. Those not divisible by these numbers are a total of $600 -
325 = 275.$
\begin{exa}
In a group of $30$ people, $8$ speak English, $12$ speak Spanish and
$10$ speak French. It is known that $5$ speak English and Spanish,
$5$ Spanish and French, and $7$ English and French. The number of
people speaking all three languages is $3$. How many do not speak
any of these languages?

\label{exa:3set_incl_excl}\end{exa} Solution: Let $A$ be the set of
all English speakers, $B$ the set of Spanish speakers and $C$ the
set of French speakers in our group. We fill-up the  Venn diagram in
figure \ref{fig:3set_incl_excl_exa} successively. In the
intersection of all three we put 8. In the region common to $A$ and
$B$ which is not filled up we put $5 - 2 = 3$. In the region common
to $A$ and $C$ which is not already filled up we put $5 - 3 = 2$. In
the region common to $B$ and $C$ which is not already filled up, we
put $7 - 3 = 4.$ In the remaining part of $A$ we put $8 - 2 - 3 - 2
= 1,$ in the remaining part of $B$ we put $12 - 4 - 3 - 2 = 3$, and
in the remaining part of $C$ we put $10 - 2 - 3 - 4 = 1$. Each of
the mutually disjoint regions comprise a total of $1 + 2 + 3 + 4 + 1
+ 2 + 3 = 16$ persons. Those outside these three sets are then $30 -
16 = 14.$ \vspace{2cm}
\begin{figure}[h]
\centering
\begin{minipage}{7cm}
$$\psset{unit=1.5pc} \pscircle(-1,0){2} \pscircle(1,0){2}
\pscircle(0,1.41421356){2} \rput(2,0){\tiny 3} \rput(-2, 0){\tiny 1}
\rput(0,0.707106781){\tiny 3} \rput(0, 2.3){\tiny 1}
\rput(0,-0.8){\tiny 2} \rput(-1.43,1.43){\tiny 2}
\rput(1.43,1.43){\tiny 4} \rput(-3.8, 0){\tiny A} \rput(3.8,
0){\tiny B} \rput(0, 3.8){\tiny C}$$
\vspace{1cm}\footnotesize\footnotesize\hangcaption{Example
\ref{exa:3set_incl_excl}.} \label{fig:3set_incl_excl_exa}
\end{minipage}
\begin{minipage}{7cm}
$$\psset{unit=1.5pc} \pscircle(-1,0){2} \pscircle(1,0){2}
\pscircle(0,1.41421356){2} \rput(2.1,-0.1){\tiny 15 } \rput(-2.1,
-0.1){\tiny 20 } \rput(0,0.707106781){\tiny x} \rput(0, 2.3){\tiny
u} \rput(0,-1){\tiny y} \rput(-1.43,1.43){\tiny z}
\rput(1.43,1.43){\tiny t} \uput[l](-3.8, 0){\mathrm{Movies}}
\uput[r](4, 0){\tiny \mathrm{Reading}} \uput[r](0, 3.8){\tiny
\mathrm{Sports}}$$
\vspace{1cm}\footnotesize\footnotesize\hangcaption{Example
\ref{exa:inclusion_exclusion}.} \label{fig:inclusion_exclusion}
\end{minipage}
\end{figure}

\begin{exa}
\label{exa:inclusion_exclusion} A survey shews that $90\%$ of
high-schoolers in Philadelphia like at least one of the following
activities: going to the movies, playing sports, or reading. It is
known that $45\%$ like the movies, $48\%$ like sports, and $35\%$
like reading. Also, it is known that $12\%$ like both the movies and
reading, $20\%$ like only the movies, and $15\%$ only reading. What
percent of high-schoolers like all three activities?
\end{exa}
Solution: We make the Venn diagram in as in figure
\ref{fig:inclusion_exclusion}. From it we gather the following
system of equations
$$\begin{array}{ccccccccccccccc}x & + & y & + &z & &  & & & & & + &20     & = & 45 \\
x &  &  & + &z &+ &t &+ &u & & & & & = & 48 \\
x & + & y &  & &+ &t & &  &+ &15 & & & = & 35 \\
x & + & y &  & & & &  & & &  & & & = & 12 \\
x & + & y & + &z &+ &t &+ &u &+ &15 &+ &20& = & 90 \\
\end{array}$$
The solution of this system is seen to be $x = 5,$ $y = 7$, $z =
13$, $t = 8$, $u = 22$. Thus the percent wanted is $5\%$.

\section*{Practice}\addcontentsline{toc}{section}{Practice}\markright{Practice}
\begin{multicols}{2}\columnseprule 1pt \columnsep 25pt\multicoltolerance=900


\begin{pro}
Consider the set $$A = \{2,4,6, \ldots , 114\} . $$
\begin{dingautolist}{202}
\item How many elements are there in $A$?\item How many are
divisible by $3$? \item How many are divisible by $5$?  \item How
many are divisible by $15$? \item How many are divisible by either
$3$, $5$ or both? \item How many are neither divisible by $3$ nor
$5$? \item How many are divisible by exactly one of $3$ or $5$?
\end{dingautolist}
\begin{answer} Let $A_k\subseteq A$ be the set of those integers divisible by
$k$.
\begin{dingautolist}{202}
\item Notice that the elements are $2 = 2(1)$, $4 = 2(2)$, \ldots
, $114 = 2(57)$. Thus $\card{A} = 57.$ \item There are $\llfloor
\frac{57}{3}\rrfloor = 19$ integers in $A$ divisible by $3$. They
are
$$\{6, 12, 18, \ldots , 114\}.$$ Notice that $114 = 6(19)$. Thus
$\card{A_3} = 19$.  \item There are $\llfloor \frac{57}{5}\rrfloor =
11$ integers in $A$ divisible by $5$. They are $$\{10, 20, 30,
\ldots , 110\}.$$ Notice that $110 = 10(11)$. Thus $\card{A_{5}} =
11$\item There are $\llfloor \frac{57}{15}\rrfloor = 3$  integers in
$A$ divisible by $15$. They are $\{30, 60, 90\}.$ Notice that $90 =
30(3)$. Thus $\card{A_{15}} = 3 $, and observe that by Theorem
\ref{thm:intersection_of_ideals} we have $\card{A_{15}} = \card{A_3
\cap A_5}$.\item We want $\card{A_3 \cup A_5} = 19 + 11 = 30.$ \item
We want
$$\begin{array}{lll}\card{A \setminus (A_3 \cup A_5)} & = &
 \card{A} - \card{A_3 \cup
A_5} \\ &  = &  57 - 30 \\ &  = &  27.\end{array}$$ \item We want
$$\begin{array}{lll}\card{(A_3 \cup A_5) \setminus (A_3 \cap A_5)}
& =  & \card{(A_3 \cup A_5)} \\
& & \qquad - \card{A_3 \cap A_5}\\ &  = &  30 - 3
\\ & = &  27.\end{array}$$
\end{dingautolist}
\end{answer}
\end{pro}
\begin{pro}
Consider the set of the first $100$ positive integers: $$ A =
\{1,2,3,\ldots , 100\}.$$  \begin{dingautolist}{202}\item  How many
are divisible by $2$? \item  How many are divisible by $3$?
\item How many are divisible by $7$? \item  How many are divisible
by $6$?  \item  How many are divisible by $14$? \item  How many are
divisible by $21$? \item How many are divisible by $42$? \item
 How many are relatively prime to $42$?
 \item How many are divisible by $2$ and $3$ but not by $7$?
  \item How many are divisible by  exactly one of $2$, $3$ and $7$?
\end{dingautolist}
\begin{answer}
We have
\begin{dingautolist}{202}\item $\floor{\dfrac{100}{2}} = 50$
\item $\floor{\dfrac{100}{3}} = 33$ \item  $\floor{\dfrac{100}{7}} = 14$ \item  $\floor{\dfrac{100}{6}} = 16
$   \item  $\floor{\dfrac{100}{14}} = 7$  \item
$\floor{\dfrac{100}{21}} = 4$
\item $\floor{\dfrac{100}{42}} = 2$  \item
$100-50-33-14+15+7+4-2 = 27$
 \item $16-2=14$
  \item $ 52$
\end{dingautolist}
\end{answer}
     \end{pro}
          \begin{pro}
A survey of a group's viewing habits over the last year revealed the
following information:
\begin{dingautolist}{202}
\item  $28\%$ watched gymnastics \item  $29\%$ watched baseball
\item $19\%$ watched soccer \item  $14\%$ watched gymnastics and
baseball \item $12\%$ watched baseball and soccer \item  $10\%$
watched gymnastics and soccer \item  $8\%$ watched all three sports.
\end{dingautolist}
Calculate the percentage of the group that watched none of the three
sports during the last year.
\begin{answer}
$52\%$
\end{answer}
    \end{pro}
           \begin{pro}
Out of $40$ children, $30$ can swim, $27$ can play chess, and only
$5$ can do neither. How many children can swim and play chess?
\begin{answer}$22$
\end{answer}
    \end{pro}
\begin{pro} At {\em Medieval High} there are forty students. Amongst
them, fourteen like Mathematics, sixteen like theology, and eleven
like alchemy. It is also known that seven like Mathematics and
theology, eight like theology and alchemy and five like Mathematics
and alchemy. All three subjects are favoured by four students. How
many students like neither Mathematics, nor theology, nor alchemy?
\begin{answer} Let $A$ be the set of students liking Mathematics, $B$ the set
of students liking theology, and $C$ be the set of students liking
alchemy. We are given that $$ \card{A} = 14, \card{B} = 16,$$ $$
\card{C} = 11, \card{A\cap B} = 7, \card{B\cap C} = 8, \card{A\cap
C} = 5,$$ and
$$ \card{A\cap B\cap C} = 4.$$By the Principle of Inclusion-Exclusion,
$$\begin{array}{lll} \card{ A^c \cap  B^c \cap  C^c}  &= &  40 - \card{A} - \card{B} - \card{C}\\
& &  + \card{A\cap B} + \card{A\cap C} + \card{B\cap C} \\ & &  -
\card{A\cap B \cap C} . \end{array}$$ Substituting the numerical
values of these cardinalities $$ 40 - 14 - 16 - 11 + 7 + 5 + 8 - 4 =
15.
$$
\end{answer}
\end{pro}
         \begin{pro}
How many strictly positive integers less than or equal to $1000$ are
\begin{dingautolist}{202}
\item perfect squares? \item perfect cubes? \item perfect fifth
powers? \item perfect sixth powers? \item perfect tenth powers?
\item perfect fifteenth powers? \item perfect thirtieth powers?
\item neither perfect squares, perfect cubes, perfect fifth
powers?
\end{dingautolist}
\begin{answer}We have
\begin{dingautolist}{202}
\item  $31$ \item  $10$ \item  $3$ \item  $3$ \item  $1$ \item  $1$  \item  $1$ \item  $960$
\end{dingautolist}
\end{answer}
   \end{pro}


\begin{pro}
An auto insurance company has $10,000$ policyholders. Each policy
holder is classified as
\begin{center} \begin{itemize}\item young or old, \item male or female, and \item married or single. \end{itemize}\end{center}
Of these policyholders, $3000$ are young, $4600$ are male, and
$7000$ are married. The policyholders can also be classified as
$1320$ young males, $3010$ married males, and $1400$ young married
persons. Finally, $600$ of the policyholders are young married
males. How many of the company's policyholders are young, female,
and single? \begin{answer} Let $Y, F, S, M$ stand for young, female,
single, male, respectively, and let $H$ stand for
married.\footnote{Or $H$ for {\em hanged}, if you prefer.} We have
$$\begin{array}{lll}\card{Y\cap F \cap S} & = & \card{Y\cap F} - \card{Y\cap F \cap H} \\
& =  &\card{Y} - \card{Y\cap M}  \\
& & \qquad - (\card{Y\cap H}  - \card{Y\cap H
\cap M}) \\
& = & 3000 - 1320 - (1400 - 600) \\
& = & 880.
\end{array}
$$
\end{answer}
\end{pro}
   \begin{pro}[AHSME 1988] $X$, $Y$, and $Z$ are pairwise disjoint
sets of people. The average ages of people in the sets $X$, $Y$,
$Z$, $X\cup Y$, $X\cup Y$, and $Y\cup Z$ are given below:
$$\begin{array}{|l|l|l|l|l|l|l|}\hline \mathrm{Set} & X & Y & Z & X\cup Y & X\cup Z & Y \cup Z \\
\hline \mathrm{Average\ Age} & 37 & 23 & 41 & 29 & 39.5 & 33 \\
\hline
\end{array}$$
What is  the average age of the people in the set $X\cup Y \cup
Z$?\begin{answer}$34$
\end{answer}
      \end{pro}
    \begin{pro}
Each of the students in the maths class twice attended a concert. It
is known that $25, 12,$ and $23$ students attended concerts A, B,
and C respectively. How many students are there in the maths class?
How many of them went to concerts A and B, B and C, or B and C?
\begin{answer}$30; 7; 5; 18$
\end{answer}
      \end{pro}
   \begin{pro}
The films A, B, and C were shewn in the cinema for a week. Out of
$40$ students (each of which saw either all the three films, or one
of them, $13$ students saw film A, $16$ students saw film B, and 19
students saw film C. How many students saw all three films?
\begin{answer}$4$
\end{answer}
\end{pro}
\begin{pro}
Would you believe a market investigator that reports that of $1000$
people, $816$ like candy, $723$ like ice cream, $645$ cake, while
$562$ like both candy and ice cream, $463$ like both candy and cake,
$470$ both ice cream and cake, while $310$ like all three? State
your reasons! \begin{answer} Let $C$ denote the set of people who
like candy, $I$ the set of people who like ice cream, and $K$ denote
the set of people who like cake. We are given that $\card{C} = 816$,
$\card{I} = 723$, $\card{K} = 645$, $\card{C\cap I} = 562$,
$\card{C\cap K} = 463$, $\card{I\cap K} = 470$, and $\card{C\cap
I\cap K} = 310$. By Inclusion-Exclusion we have
$$\begin{array}{lll}
\card{C \cup I \cup K} & = & \card{C} + \card{I} + \card{K} \\ & & -
\card{C \cap I} - \card{C \cap K }
- \card{I \cap C} \\ & &  + \card{C \cap I \cap K} \\
& = & 816 + 723 + 645 - 562 - 463 - 470 + 310 \\ & = & 999.
\end{array}$$
The investigator miscounted, or probably did not report one person
who may not have liked any of the three things.
\end{answer}
\end{pro}
\begin{pro}[AHSME 1991]  For a set $S$, let $\card{2^S}$ denote the
number of subsets of $S$. If $A, B, C$, are sets for which $$
\card{2^A} + \card{2^B} + \card{2^C} = \card{2^{A\cup B\cup C}}$$and
$$\card{A} = \card{B} = 100,$$ then what is the minimum possible value of
$\card{A\cap B\cap C}$? \begin{answer} A set with $k$ elements has
$2^k$ different subsets. We are given
$$ 2^{100} + 2^{100} + 2^{\card{C}} = 2^{\card{A\cup B\cup C}}.$$ This forces
$\card{C} = 101$, as $1 + 2^{\card{C} - 101}$ is larger than $1$ and
a power of $2$. Hence $\card{A\cup B \cup C} = 102$. Using the
Principle Inclusion-Exclusion, since $\card{A} + \card{B} + \card{C}
- \card{A \cup B \cup C} = 199,$
$$ \begin{array}{lcl}\card{A\cap B \cap C} & = & \card{A\cap B} + \card{A\cap C} + \card{B \cap
C} - 199 \\
& = & (\card{A} + \card{B} - \card{A\cup B}) \\ & & \qquad + (\card{A} + \card{C} \\
& & \quad - \card{A\cup C}) + \card{B} + \card{C}\\ & & \qquad  - \card{B\cup C} - 199 \\
& = & 403 - \card{A\cup B} - \card{A \cup C} - \card{B\cup
C}.\end{array}$$As $A\cup B, A\cup C, B\cup C \subseteq A\cup B \cup
C,$ the cardinalities of all these sets are $\leq 102.$ Thus
$$\begin{array}{lll} \card{A\cap B \cap C} & = & 403 - \card{A\cup B} - \card{A\cup C}\\ & & \qquad  - \card{B \cup C} \geq 403 -
3\cdot 102\\ &  = &  97. \end{array}$$ By letting $$A = \{ 1, 2,
\ldots , 100\} , B = \{ 3, 4, \ldots , 102\} ,$$ and $$C = \{ 1, 2,
3, 4, 5, 6, \ldots, 101, 102\}$$ we see that the bound $\card{A\cap
B \cap C} = \card{\{ 4, 5, 6, \ldots , 100\}} = 97$ is achievable.
\end{answer}
\end{pro}
\begin{pro}[Lewis Carroll in {\em A Tangled Tale.}]
In a very hotly fought battle, at least $70\%$ of the combatants
lost an eye, at least $75\%$ an ear, at least $80\%$ an arm, and at
least $85\%$ a leg. What can be said about the percentage who lost
all four members? \begin{answer} Let $A$ denote the set of those who
lost an eye, $B$ denote those who lost an ear, $C$ denote those who
lost an arm and $D$ denote those losing a leg. Suppose there are $n$
combatants.
 Then
$$\begin{array}{lll}n & \geq & \card{ A \cup B} \\ & = &  \card{ A} + \card{ B} - \card{ A \cap B} \\ & = & .7n + .75n - \card{ A\cap B}, \end{array}$$
$$\begin{array}{lll}n  &\geq & \card{ C \cup D} \\ & = & \card{ C} + \card{ D} - \card{ C \cap D} \\ & = & .8n + .85n - \card{ C\cap
D}. \end{array}$$This gives
$$\card{ A \cap B} \geq .45n,$$
$$\card{ C \cap D} \geq .65n.$$This means that
$$\begin{array}{lll}n &\geq & \card{ (A\cap B) \cup (C \cap D)} \\ & = & \card{ A \cap B} + \card{ C \cap D} - \card{ A \cap B \cap C \cap D}\\
& \geq &  .45n + .65n - \card{ A \cap B \cap C \cap D},\end{array}$$
whence
$$\card{ A \cap B \cap C \cap D} \geq .45 + .65n - n = .1n.$$This means
that at least $10\%$ of the combatants lost all four members.
\end{answer}
\end{pro}

\begin{pro}
Let $x, y$ be real numbers. Prove that
$$
x + y = \min (x, y) + \max (x, y)
$$
\end{pro}
\begin{pro}
Let $x, y , z$ be real numbers. Prove that
$$ \begin{array}{lll}
\max (x, y, z)  & = &  x + y + z - \min (x, y) - \min (y, z)\\ & &
\qquad  - \min (z, x) + \min (x, y , z) \end{array}$$
\end{pro}

\end{multicols}
\section{The Product Rule}
 \begin{rul}[Product Rule] Suppose that an experiment $E$ can be performed in $k$ stages: $E_1$ first, $E_2$ second,
 \ldots, $E_k$ last.  Suppose moreover that $E_i$ can be done in $n_i$ different
ways, and that the number of ways of performing $E_i$ is not
influenced by any predecessors $E_1, E_2, \ldots , E_{i - 1}$. Then
$E_1$ {\bf and} $E_2$ {\bf and} \ldots {\bf and} $E_k$  can occur
simultaneously in $n_1n_2\cdots n_k$ ways.
\end{rul}
\begin{exa}
In a group of $8$ men and $9$ women we can pick one man {\bf and}
one woman in $8\cdot 9 = 72$ ways. Notice that we are choosing two
persons.
\end{exa}

\begin{exa}
A red die and a blue die are tossed. In how many ways can they land?
\end{exa}
Solution: If we view the outcomes as an ordered pair $(r, b)$ then
by the multiplication principle we have the $6\cdot 6 = 36$ possible
outcomes
$$\begin{array}{llllll} (1,1) & (1,2) & (1,3) & (1,4) & (1,5) & (1,6) \\
(2,1) & (2,2) & (2,3) & (2,4) & (2,5) & (2,6) \\
(3,1) & (3,2) & (3,3) & (3,4) & (3,5) & (3,6) \\
(4,1) & (4,2) & (4,3) & (4,4) & (4,5) & (4,6) \\
(5,1) & (5,2) & (5,3) & (5,4) & (5,5) & (5,6) \\
(6,1) & (6,2) & (6,3) & (6,4) & (6,5) & (6,6) \\  \end{array} $$
The red die can land in any of $6$ ways, $$\begin{array}{|c|c|}\hline 6  & \hspace{.5cm}  \\
\hline
\end{array}  $$
and also, the blue die may land in any of $6$ ways
$$\begin{array}{|c|c|}\hline 6  & 6  \\ \hline \end{array}.  $$
\begin{exa}
A multiple-choice test consists of $20$ questions, each one with $4$
choices. There are $4$ ways of answering the first question, $4$
ways of answering the second question, etc., hence there are $4^{20}
= 1099511627776$ ways of answering the exam.
\end{exa}


\begin{exa}\label{exa:3_digit_integers}
There are $9\cdot 10\cdot 10 =900$ positive 3-digit integers:
$$100, 101, 102, \ldots , 998, 999.        $$
For, the leftmost integer cannot be $0$ and so there are only $9$
choices $\{1,2,3,4,5,6,7,8,9\}$ for it,
$$\begin{array}{|c|c|c|}\hline 9  & \hspace{.5cm} & \hspace{.5cm}  \\ \hline \end{array}.  $$
There are $10$ choices for the second digit
$$\begin{array}{|c|c|c|}\hline 9  & 10 & \hspace{.5cm}  \\ \hline \end{array},  $$
and also $10$ choices for the last digit
$$\begin{array}{|c|c|c|}\hline 9  & 10 & 10  \\ \hline \end{array}.  $$
\end{exa}

\begin{exa}\label{exa:3_digit_integers_even}
There are $9\cdot 10\cdot 5 =450$ even positive 3-digit integers:
$$100, 102, 104, \ldots , 996, 998.        $$
For, the leftmost integer cannot be $0$ and so there are only $9$
choices $\{1,2,3,4,5,6,7,8,9\}$ for it,
$$\begin{array}{|c|c|c|}\hline 9  & \hspace{.5cm} & \hspace{.5cm}  \\ \hline \end{array}.  $$
There are $10$ choices for the second digit
$$\begin{array}{|c|c|c|}\hline 9  & 10 & \hspace{.5cm}  \\ \hline \end{array}.  $$
Since the integer must be even, the last digit must be one of the
$5$ choices $\{0,2,4,6,8\}$
$$\begin{array}{|c|c|c|}\hline 9  & 10 & 5  \\ \hline \end{array}.  $$
\end{exa}


\begin{df}
A {\em palindromic integer} or {\em palindrome} is a positive
integer whose decimal expansion is symmetric and that is not
divisible by $10$. In other words, one reads the same integer
backwards or forwards.\footnote{A palindrome in common parlance, is
a word or phrase that reads the same backwards to forwards. The
Philadelphia street name {\em Camac} is a palindrome. So are the
phrases (if we ignore punctuation) (a) ``A man, a plan, a canal,
Panama!'' (b) ``Sit on a potato pan!, Otis.'' (c) ``Able was I ere I
saw Elba.'' This last one is attributed to Napoleon, though it is
doubtful that he knew enough English to form it. }
\end{df}
\begin{exa}
The following integers are all palindromes:
$$ 1,8, 11, 99, 101, 131, 999, 1234321, 9987899.         $$
\end{exa}
\begin{exa}\label{exa:5digit_palindromes}
How many palindromes  are there of $5$ digits? \\
Solution: There are $9$ ways of choosing the leftmost digit.

$$\begin{array}{|c|c|c|c|c|}\hline 9 & \hspace{.5cm} & \hspace{.5cm} & \hspace{.5cm} & \hspace{.5cm}  \\ \hline \end{array}.  $$
Once the leftmost digit is chosen, the last digit must be identical
to it, so we have
$$\begin{array}{|c|c|c|c|c|}\hline 9 & \hspace{.5cm} & \hspace{.5cm} & \hspace{.5cm} & 1  \\ \hline \end{array} . $$
There are $10$ choices for the second digit from the left
$$\begin{array}{|c|c|c|c|c|}\hline 9 & 10 & \hspace{.5cm} & \hspace{.5cm} & 1  \\ \hline \end{array} . $$
Once this digit is chosen, the second digit from the right must be
identical to it, so we have only $1$ choice for it,
$$\begin{array}{|c|c|c|c|c|}\hline 9 & 10 & \hspace{.5cm} & 1 & 1  \\ \hline \end{array} . $$
Finally, there are $10$ choices for the third digit from the right,
$$\begin{array}{|c|c|c|c|c|}\hline 9 & 10 & 10 & 1 & 1  \\ \hline \end{array}, $$
which give us $900$ palindromes of $5$-digits.
\end{exa}

\begin{exa}\label{exa:5digit_even_palindromes} How many palindromes of $5$ digits are even? \end{exa}
Solution: A five digit even palindrome has the form $ABCBA$, where
$A$ belongs to $ \{2, 4, 6, 8\}$, and $B,C$ belong to
$\{0,1,2,3,4,5,6,7,8,9\}$. Thus there are $4$ choices for the first
digit, $10$ for the second, and $10$ for the third. Once these
digits are chosen, the palindrome is completely determined.
Therefore, there are $4\times 10 \times 10 = 400$ even palindromes
of $5$ digits.
\begin{exa} How many positive divisors does $300$
have? \end{exa}Solution: We have $300 = 3\cdot 2^25^2$. Thus every
factor of $300$ is of the form $3^a2^b5^c $, where $0 \leq a \leq
1,$ $0 \leq b \leq 2,$ and $0 \leq c \leq 2$. Thus there are $2$
choices for $a$, $3$ for $b$ and $3$ for $c$. This gives $2\cdot 3
\cdot 3 = 18$ positive divisors.
\begin{exa}
\label{pro:paths}How many paths consisting of a sequence of
horizontal and/or vertical line segments, each segment connecting a
pair of adjacent letters in figure \ref{fig:paths} spell $BIPOLAR$?
\end{exa}
\begin{figure}[h]
\begin{minipage}{7cm}$$
\begin{array}{ccccccccccccc}
 &  & &  &  &  & B &  &  &  &  &  &  \\
 &  & &  &  & B & I & B &  &  &  &  &  \\
 &  & &  & B & I & P & I & B &  &  &  &  \\
 &  & & B & I & P & O & P & I & B &  &  &  \\
 &  & B& I & P & O & L & O & P & I & B &  &  \\
& B & I& P & O & L & A & L & O & P & I & B &  \\
B & I & P& O & L & A & R & A & L & O & P & I & B \\    \end{array}
$$
 \footnotesize\hangcaption{Problem
\ref{pro:paths}.}\label{fig:paths}
\end{minipage}
\hfill
\begin{minipage}{7cm}
$$
\begin{array}{ccccccc}
 &  & &  &  &  & B  \\
 &  & &  &  & B & I   \\
 &  & &  & B & I & P   \\
 &  & & B & I & P & O  \\
 &  & B& I & P & O & L   \\
& B & I& P & O & L & A  \\
B & I & P& O & L & A & R  \\    \end{array}
$$
 \footnotesize\hangcaption{Problem
\ref{pro:paths}.}\label{fig:paths2}
\end{minipage}
\end{figure}
Solution: Split the diagram, as in figure \ref{fig:paths2}. Since
every required path must use the $R$, we count paths starting from
$R$ and reaching up to a $B$. Since there are six more rows that we
can travel to, and since at each stage we can go either up or left,
we have $2^6 = 64$ paths. The other half of the figure will provide
$64$ more paths. Since the middle column is shared by both halves,
we have a total of $64 + 64 - 1 = 127$ paths.

\bigskip

We now prove that if a set $A$ has $n$ elements, then it has $2^n$
subsets. To motivate the proof, consider the set $\{a, b, c\}$. To
each element we attach a binary code of length $3$. We write $0$ if
a particular element is not in the set and $1$ if it is.  We then
have the following associations:
\begin{multicols}{2}\columnseprule 1pt \columnsep 25pt\multicoltolerance=900
\begin{enumerate}
\item[] $ \varnothing \leftrightarrow 000,    $
\item[] $ \{a\} \leftrightarrow 100,    $
\item[] $ \{b\} \leftrightarrow 010,    $
\item[] $ \{c\} \leftrightarrow 001,    $
\item[] $ \{a,b\} \leftrightarrow 110,    $
\item[] $ \{a,c\} \leftrightarrow 101,    $
\item[] $ \{b,c\} \leftrightarrow 011,    $
\item[] $ \{a,b,c\} \leftrightarrow 111.    $
\end{enumerate}
\end{multicols}
Thus there is a one-to-one correspondence between the subsets of a
finite set of $3$ elements and binary sequences of length $3$.
\begin{thm}[Cardinality of the Power Set]\label{thm:cardinality_of_power_set} Let $A$ be a finite set
with $\card{A} = n$. Then $A$ has $2^n$ subsets.
\end{thm}
\begin{pf}
We attach a binary code to each element of the subset, $1$ if the
element is in the subset and $0$ if the element is not in the
subset.  The total number of subsets is the total number of such
binary codes, and there are $2^n$ in number.
\end{pf}

\section*{Homework}\addcontentsline{toc}{section}{Homework}\markright{Homework}
\begin{multicols}{2}\columnseprule 1pt \columnsep 25pt\multicoltolerance=900
\begin{pro}
A true or false exam has ten questions. How many possible answer
keys are there?
\begin{answer}$2^{10}=1024$
\end{answer}
  \end{pro}

\begin{pro} Out of nine different pairs of shoes, in how many ways could I choose a
right shoe and a left shoe, {which should not form a pair}?
\begin{answer} I can choose a right shoe in any of nine ways, once this
has been done, I can choose a non-matching left shoe in eight ways,
and thus I have
72 choices.\\
{\em Aliter}:  I can choose any pair in $9 \times 9 = 81$ ways. Of
these, 9 are matching pairs, so the number of non-matching pairs is
$81 - 9 = 72$.
\end{answer}
\end{pro}
  \begin{pro} In how many ways can the following prizes be given away to a
class of twenty boys: first and second Classical, first and second
Mathematical, first Science, and first French? \begin{answer}
$=(20)(19)(20)(19)(20)(20)= 57760000$
\end{answer}
   \end{pro}
     \begin{pro}
Under old hardware, a certain programme accepted passwords of the
form $$eell $$where $$ e\in \{0,2,4,6,8\},   \qquad
l\in\{a,b,c,d,u,v,w,x,y,z\}.$$ The hardware was changed and now the
software accepts passwords of the form $$eeelll.   $$ How many more
passwords of the latter kind are there than of the former kind?
\begin{answer}$10^35^3-10^25^2= 122500$
\end{answer}
 \end{pro}

\begin{pro}\label{exa:prod_rule_1}
A license plate is to be made according to the following provision:
it has four characters, the first two characters can be any letter
of the English alphabet and the last two characters can be any
digit. One is allowed to repeat letters and digits. How many
different license plates can be made?
\begin{answer} The number of different license plates is the number of different
four-tuples (Letter $_1$, Letter $_2$, Digit $_1$, Digit $_2$). The
first letter can be chosen in $26$ ways, and so we have
$$\begin{array}{|c|c|c|c|}\hline 26 & \hspace{.5cm} & \hspace{.5cm} & \hspace{.5cm}    \\ \hline \end{array} . $$
The second letter can be chosen in any of $26$ ways:
$$\begin{array}{|c|c|c|c|}\hline 26 & 26 & \hspace{.5cm} & \hspace{.5cm}    \\ \hline \end{array} . $$
The first digit can be chosen in $10$ ways:
$$\begin{array}{|c|c|c|c|}\hline 26 & 26 & 10 & \hspace{.5cm}    \\ \hline \end{array} . $$
Finally, the last digit can be chosen in $10$ ways:
$$\begin{array}{|c|c|c|c|}\hline 26 & 26 & 10 & 10    \\ \hline \end{array} . $$
By the multiplication principle, the number of different four-tuples
is $26\cdot 26\cdot 10 \cdot 10 = 67600.$
\end{answer}
\end{pro}
\begin{pro}
In problem \ref{exa:prod_rule_1}, how many different license plates
can you make if (i) you may repeat letters but not digits?, (ii) you
may repeat digits but not letters?, (iii) you may  repeat neither
letters nor digits? \begin{answer} (i) In this case we have a grid
like
$$\begin{array}{|c|c|c|c|}\hline 26 & 26 & 10 & 9    \\ \hline \end{array} , $$
since after a digit has been used for the third position, it cannot
be used again.
Thus this can be done in $26\cdot 26 \cdot 10\cdot 9 = 60840 $ ways.\\
(ii) In this case we have a grid like
$$\begin{array}{|c|c|c|c|}\hline 26 & 25 & 10 & 10    \\ \hline \end{array} , $$
since after a letter has been used for the first position, it
cannot be used again. Thus this can be done in $26\cdot 25 \cdot 10\cdot 10 = 65000$ ways. \\
(iii) After a similar reasoning, we obtain a grid like
$$\begin{array}{|c|c|c|c|}\hline 26 & 25 & 10 & 9    \\ \hline \end{array} . $$
Thus this can be done in $26\cdot 25 \cdot 10\cdot 9 = 58500$ ways.
\end{answer}
\end{pro}
  \begin{pro}
An alphabet consists of the {\bf five} consonants \{p, v, t, s, k\}
and the {\bf three} vowels \{a, e, o\}. A license plate is to be
made using {\bf four} letters of this
alphabet.\begin{dingautolist}{202}\item How many letters does this
alphabet have? \item If a license plate is of the form $CCVV$ where
$C$ denotes a consonant and $V$ denotes a vowel, how many possible
license plates are there, assuming that you may repeat both
consonants and vowels?
 \item If a license plate is of the form $CCVV$ where
$C$ denotes a consonant and $V$ denotes a vowel, how many possible
license plates are there, assuming that you may repeat  consonants
but not vowels? \item If a license plate is of the form $CCVV$ where
$C$ denotes a consonant and $V$ denotes a vowel, how many possible
license plates are there, assuming that you may repeat vowels but
not consonants?  \item If a license plate is of the form $LLLL$
where $L$ denotes any letter of the alphabet, how many possible
license plates are there, assuming that you may not repeat letters?
\end{dingautolist}
\begin{answer} [1] $8$, [2] $5^23^2 = 225$, [3] $5^2\cdot 3 \cdot 2 =
150$, [4] $5\cdot 4 \cdot 3^2 = 180$, [5] $8\cdot 7 \cdot 6 \cdot 5
= 1680$.
\end{answer}
\end{pro}
     \begin{pro}
A man lives within reach of three boys' schools and four girls'
schools. In how many ways can he send his three sons and two
daughters to school?
\begin{answer}$ 432 $
\end{answer}
 \end{pro}
\begin{pro} How many distinct four-letter words can be made with the
letters of the set $\{ c, i, k, t\}$ \begin{dingautolist}{202}
\item if the letters are not to be repeated? \item if the letters
can be repeated? \end{dingautolist} \begin{answer} Solution:
\begin{dingautolist}{202}\item  The first letter can be one of any
4. After choosing the first letter, we
have 3 choices for the second letter, etc.. The total number of words is thus $4\cdot 3 \cdot 2 \cdot 1 = 24$. \\
\item The first letter can be one of any 4. Since we are allowed
repetitions, the second letter can also be one of any 4, etc.. The
total number of words so formed is thus $4^4 = 256.$
\end{dingautolist}
\end{answer}
\end{pro}
\begin{pro} How many distinct six-digit numbers that are multiples of $5$
can be formed from the list of digits $\{ 1, 2, 3, 4, 5, 6\}$ if we
allow repetition? \begin{answer} The last digit must perforce be
$5$. The other five digits can be filled with any of the six digits
on the list: the total number is thus $6^5$.
\end{answer}
\end{pro}
\begin{pro}
Telephone numbers in {\em Land of the Flying Camels} have $7$
digits, and the only digits available are $\{0,1, 2,3,4,5, 7, 8\}$.
No telephone number may begin in $0$, $1$ or $5$. Find the number of
telephone numbers possible that meet the following criteria:
\begin{dingautolist}{202}
\item You may repeat all digits.
 \item You may not repeat any of the digits.
 \item You may repeat the digits, but the
phone number must be even. \item You may repeat the digits, but the
phone number must be odd. \item You may not repeat the digits and
the phone numbers must be odd.

\end{dingautolist}
\begin{answer}We have \begin{dingautolist}{202} \item This is $ 5\cdot 8^6 =
1310720$.
 \item This is $5\cdot 7\cdot 6 \cdot 5 \cdot 4 \cdot 3 \cdot 2
= 25200  $. \item  This is $5\cdot 8^5 \cdot 4 = 655360 $.\item This
is $5\cdot 8^5 \cdot 4 = 655360 $. \item  We condition on the last
digit. If the last digit were $1$ or $5$ then we would have $5$
choices for the first digit, and so we would have $$5\cdot 6 \cdot
5\cdot 4 \cdot 3\cdot 2 \cdot 2 = 7200
$$ phone numbers. If the last digit were either $3$ or $7$, then
we would have $4$ choices for the last digit and so we would
have$$4\cdot 6\cdot 5 \cdot 4 \cdot 3\cdot 2 \cdot 2  = 5760   $$
phone numbers. Thus the total number of phone numbers is $$7200 +
5760 = 12960.
$$

\end{dingautolist}
\end{answer}
\end{pro}


       \begin{pro}
How many $5$-lettered words can be made out of $26$ letters,
repetitions allowed, but not consecutive repetitions (that is, a
letter may not follow itself in the same word)?

\begin{answer}$26\cdot 25^4 = 10156250$

\end{answer}
    \end{pro}
\begin{pro}How many positive integers are there having $n\geq 1$
digits?
\begin{answer} For the leftmost digit cannot be $0$ and so we have only
the nine choices
$$\{1,2,3,4,5,6,7,8,9\}$$ for this digit. The other $n - 1$ digits
can be filled out in $10$ ways, and so there are
$$ 9\cdot\underbrace{10\cdots 10}_{n-1 \ 10'{\rm s}} = 9\cdot 10^{n - 1}. $$
\end{answer}
\end{pro}

\begin{pro}How many $n$-digits integers ($n \geq 1$) are there
which are even? \begin{answer}The leftmost digit cannot be $0$ and
so we have only the nine choices
$$\{1,2,3,4,5,6,7,8,9\}$$ for this digit. If the integer is going
to be even, the last digit can be only one of the five
$\{0,2,4,6,8\}$. The other $n - 2$ digits can be filled out in $10$
ways, and so there are
$$ 9\cdot \underbrace{10\cdots 10}_{n-2 \ 10'{\rm s}}\cdot 5 = 45\cdot 10^{n - 2}. $$
\end{answer}
\end{pro}

\begin{pro} How many $n$-digit nonnegative integers do not contain the digit
$5$? \begin{answer} $9$ $1$-digit numbers and $8\cdot 9^{n - 1}$
$n$-digit numbers $n \geq 2$.
\end{answer}
\end{pro}
\begin{pro} How many $n$-digit numbers do not have the digit $0$?
\begin{answer} One can choose the last digit in 9 ways, one can choose the
penultimate digit in 9 ways, etc. and one can choose the second
digit in 9 ways, and finally one can choose the first digit in 9
ways. The total number of ways is thus $9^n$.
\end{answer}
\end{pro}
            \begin{pro}
There are $m$ different roads from town A to town B. In how many
ways can Dwayne travel from town A to town B and back if (a) he may
come back the way he went?, (b) he must use a different road of
return?
\begin{answer}$m^2$, $m(m-1)$
\end{answer}
    \end{pro}

\begin{pro} How many positive divisors does $2^{8}3^95^{2}$ have? What is the sum of these divisors?
\begin{answer} We will assume that the positive integers may be factorised in
a unique manner as the product of primes. Expanding the product
$$(1 + 2 + 2^2 + \cdots + 2^8)(1 + 3 + 3^2 + \cdots + 3^9)(1 + 5 + 5^2)$$
each factor of $2^83^95^{2}$ appears and only the factors of this
number appear. There are then, as many factors as terms in this
product. This means that there are $(1 + 8)(1 + 9)(1 + 3) = 320$
factors.
\bigskip

The sum of the divisors of this number may be obtained by adding up
each geometric series in parentheses. The desired sum is then
$$\frac{2^9 - 1}{2 - 1}\cdot\frac{3^{10} - 1}{3 - 1}\cdot\frac{5^{3} - 1}{5 - 1}
= 467689684.$$
\begin{rem}A similar argument gives the following.
Let $p_1, p_2, \ldots , p_k$ be different primes. Then the integer
$$n = p_1 ^{a_1} p_2 ^{a_2} \cdots  p_k  ^{a_k}  $$has
$${\rm d}(n) = (a_1 + 1)(a_2 + 1)\cdots (a_k + 1)
$$positive divisors. Also, if  $\sigma (n)$ denotes the sum of all positive divisors of $n$, then $$\sigma (n) =
\frac{p_1 ^{a_1 + 1} - 1}{p_1 - 1}\cdot\frac{p_2 ^{a_2 + 1} - 1}{p_2
- 1}\cdots \frac{p_k ^{a_k + 1} - 1}{p_k - 1}.$$
\end{rem}
\end{answer}
\end{pro}
\begin{pro} How many factors of $2^{95}$ are larger than $1,000,000$?
\begin{answer} The $96$ factors of $2^{95}$ are  $1, 2, 2^2, \ldots ,
2^{95}$. Observe that $2^{10} = 1024$ and so $2^{20} = 1048576$.
Hence
$$2^{19} = 524288 < 1000000 < 1048576 = 2^{20}.$$ The factors
greater than $1,000,000$ are thus $2^{20}, 2^{21}, \ldots 2^{95}$.
This makes for $96 - 20 = 76$ factors.
\end{answer}
\end{pro}


            \begin{pro}
How many positive divisors does $360$ have? How many are even? How
many are odd? How many are perfect squares?
\begin{answer}$(1+3)(1+2)(1+1)=24$; $18$; $ 6$; $4$. \end{answer}
     \end{pro}
         \begin{pro}[AHSME 1988]
 At the end of a professional bowling tournament, the top 5
bowlers have a play-off. First \# 5 bowls \#4. The loser receives
the 5th prize and the winner bowls \# 3 in another game. The loser
of this game receives the 4th prize and the winner bowls \# 2. The
loser of this game receives the 3rd prize and the winner bowls \# 1.
The loser of this game
 receives the 2nd prize and the winner the 1st prize.  In how many orders can bowlers \#1 through \#5 receive the prizes?
\begin{answer}$16$
\end{answer}
\end{pro}

             \begin{pro}
The password of the anti-theft device of a car is a four digit
number, where one can use any digit in the set
$$\{0,1,2,3,4,5,6,7,8,9\}.$$
\begin{enumerate} \item[A.]
\begin{dingautolist}{202} \item How many such passwords are possible?
\item How many of the passwords have all their digits distinct?
\end{dingautolist}
\item[B.] After an electrical failure, the owner must reintroduce
the password in order to deactivate the anti-theft device. He knows
that the four digits of the code are $2,0,0,3$ but does not recall
the order.
\begin{dingautolist}{202} \item How many such passwords are possible using only these digits?
\item If the first attempt at the password fails, the owner must
wait two minutes before a second attempt, if the second attempt
fails he must wait four minutes before a third attempt, if the third
attempt fails he must wait eight minutes before a fourth attempt,
etc. (the time doubles from one attempt to the next). How many
passwords can the owner attempt in a period of $24$ hours?
\end{dingautolist}
\end{enumerate}
\begin{answer}A. [1] $10000$, [2] $5040$, B. [1] $12$ , [2] $10$
\end{answer}
   \end{pro}
          \begin{pro}
The number $3$ can be expressed as a sum of one or more positive
integers in four ways, namely, as $3$, $1+2$, $2+1$, and $1+1+1$.
Shew that any positive integer $n$ can be so expressed in $2^{n-1}$
ways.
\begin{answer} $n = \underbrace{1 + 1 + \cdots + 1}_{n-1 \ +\mathrm{'s}}$. One either erases or keeps a plus sign.\end{answer}
\end{pro}
\begin{pro}
 Let $n = 2^{31}3^{19}$. How many positive integer divisors of $n^2$ are less
than $n$ but do not divide $n$? \begin{answer} There are $589$ such
values. The easiest way to see this  is to observe that there is a
bijection between the divisors of $n^2$ which are $> n$ and those $<
n$. For if $n^2 = ab,$ with $a > n$, then $b < n$, because otherwise
$n^2 = ab > n\cdot n = n^2$, a contradiction. Also, there is exactly
one decomposition $n^2 = n\cdot n.$ Thus the desired number is
$$\llfloor\frac{d(n^2)}{2}\rrfloor + 1 - d(n) =
\llfloor\frac{(63)(39)}{2}\rrfloor + 1 - (32)(20) = 589.$$
\end{answer}
\end{pro}
\begin{pro}Let $n \geq 3.$ Find the number of $n$-digit ternary sequences
that contain at least one $0$, one $1$ and one $2$.
\begin{answer} The total number of sequences is $3^n$. There are
$2^n$ sequences that contain no $0$, $1$ or $2$. There is only one
sequence that contains only $1$'s, one that contains only $2$'s, and
one that contains only $0$'s. Obviously, there is no ternary
sequence that contains no $0$'s or $1$'s or $2$'s. By the Principle
of Inclusion-Exclusion, the number required is
$$ 3^n - (2^n + 2^n + 2^n) + (1 + 1 + 1) = 3^n - 3\cdot 2^n + 3.$$
\end{answer}
\end{pro}
\begin{pro}
In how many ways can one decompose the set $$\{1, 2, 3, \ldots ,
100\}$$ into subsets $A, B, C$ satisfying
$$ A \cup B \cup C = \{1, 2, 3, \ldots , 100\} \ \ \ {\rm and} \ \ \ A \cap B \cap C = \varnothing$$
\begin{answer} The conditions of the problem stipulate that both the region
outside the circles in diagram \ref{fig:3set_incl_excl} and $R_3$
will be empty. We are thus left with $6$ regions to distribute $100$
numbers. To each of the $100$ numbers we may thus assign one of $6$
labels. The number of sets thus required is $6^{100}$.
\end{answer}
\end{pro}
\end{multicols}


\section{The Sum Rule}
\begin{rul}[Sum Rule: Disjunctive Form] Let $E_1, E_2, \ldots, E_k$, be pairwise mutually exclusive events. If $E_i$ can
occur in $n_i$ ways, then either $E_1$ {\bf or} $E_2$ {\bf or},
\ldots, {\bf or} $E_k$ can occur in $$ n_1 + n_2 + \cdots n_k
$$ways.
\end{rul}
\begin{rem}
Notice that the ``{\bf or}'' here is exclusive.
\end{rem}
\begin{exa}
In a group of $8$ men and $9$ women we can pick one man {\bf or} one
woman in $8+  9 = 17$ ways. Notice that we are choosing one person.
\end{exa}


\begin{exa} There are five Golden retrievers, six Irish setters, and eight
Poodles at the pound. How many ways can two dogs be chosen if they
are not the same kind. \end{exa} Solution: We choose: a Golden
retriever {\bf and} an Irish setter {\bf or} a Golden retriever {\bf
and} a Poodle {\bf or} an Irish setter {\bf and} a Poodle.

\bigskip
One Golden retriever and one Irish setter can be chosen in $5 \cdot
6 = 30$ ways; one Golden retriever and one Poodle can be chosen in
$5 \cdot 8 = 40$ ways; one Irish setter and one Poodle can be chosen
in $6 \cdot 8 = 48$ ways.  By the sum rule, there are $30 + 40 + 48
= 118$ combinations.
\begin{exa} To write a book $1890$ digits were utilised. How many pages does the book have?
\end{exa}
Solution: A total of $$1\cdot 9 + 2\cdot 90  = 189$$digits are used
to write pages $1$ to $99$, inclusive. We have of $1890 - 189 =
1701$ digits at our disposition which is enough for $1701/3  = 567$
extra pages (starting from page $100$). The book has $99 + 567 =
666$ pages.

\begin{exa}
 The sequence of palindromes, starting with $1$ is
written in ascending order
$$1, 2, 3, 4, 5, 6, 7, 8, 9, 11, 22, 33, \ldots $$
Find the $1984$-th positive palindrome.
\end{exa}Solution: It is easy to see that there are $9$ palindromes of $1$-digit, $9$ palindromes with $2$-digits,
$90$ with $3$-digits, $9$0 with $4$-digits, $900$ with $5$-digits
and $900$ with $6$-digits. The last palindrome with $6$ digits,
$999999$, constitutes the $9 + 9 + 90 + 90 + 900 + 900 = 1998$th
palindrome. Hence, the $1997$th palindrome is $998899$, the $1996$th
palindrome is $997799$, the $1995$th palindrome is $996699$, the
$1994$th is $995599$, etc., until we find the $1984$th palindrome to
be $985589$.

\begin{exa}
The integers from $1$ to $1000$ are written in succession. Find the
sum of all the digits.\end{exa} Solution: When writing the integers
from $000$ to $999$ (with three digits), $3\times 1000 = 3000$
digits are used. Each of the $10$ digits is used an equal number of
times, so each digit is used $300$ times. The the sum of the digits
in the interval $000$ to $999$ is thus
$$(0 + 1 + 2 + 3+ 4 + 5+ 6+ 7 + 8 + 9)(300) = 13500.
$$Therefore, the sum of the digits when writing the integers from
$000$ to $1000$ is $13500 + 1 = 13501$.
\begin{exa}How many 4-digit integers can be formed with the set
of digits $\{0, 1, 2, 3, 4, 5\}$ such that no digit is repeated and
the resulting integer is a multiple of $3$? \end{exa} Solution: The
integers desired have the form $D_1D_2D_3D_4$ with $D_1 \neq 0$.
Under the stipulated constraints, we must have
$$D_1 + D_2 + D_3 + D_4\in\{6, 9, 12\}.$$ We thus consider three
cases.

\bigskip

Case I: $D_1 + D_2 + D_3 + D_4 = 6$. Here we have  $\{D_1, D_2, D_3,
D_4\} = \{0, 1, 2, 3, 4\}, D_1 \neq 0$. There are then $3$ choices
for $D_1$. After $D_1$ is chosen, $D_2$ can be chosen in $3$ ways,
$D_3$ in $2$ ways, and $D_1$ in $1$ way. There are thus $3\times 3
\times 2 \times 1 = 3\cdot 3! = 18$ integers satisfying case I.

\bigskip

Case II: $D_1 + D_2 + D_3 + D_4 = 9$. Here we have $\{D_1, D_2, D_3,
D_4\} = \{0, 2, 3, 4\}, D_1 \neq 0$ or $\{D_1, D_2, D_3, D_4\} =
\{0, 1, 3, 5\}, D_1 \neq 0$. Like before, there are $ 3\cdot 3! =
18$ numbers in each possibility, thus we have $2\times 18 = 36$
numbers in case II.

\bigskip

Case III: $D_1 + D_2 + D_3 + D_4 = 12$. Here we have $\{D_1, D_2,
D_3, D_4\} = \{0, 3,4,5\}, D_1 \neq 0$ or $\{D_1, D_2, D_3, D_4\} =
\{1, 2, 4, 5\}$. In the first possibility there are $ 3\cdot 3! =
18$ numbers, and in the second there are $4! = 24$. Thus we have $18
+ 24 = 42$ numbers in case III.

\bigskip
The desired number is finally $18 + 36 + 42 = 96.$




\section*{Homework}\addcontentsline{toc}{section}{Homework}\markright{Homework}
\begin{multicols}{2}\columnseprule 1pt \columnsep 25pt\multicoltolerance=900
\begin{pro}
How many different sums can be thrown with two dice, the faces of
each die being numbered $0, 1, 3, 7, 15, 31$?
\begin{answer}$21$
\end{answer}
 \end{pro}
         \begin{pro}
How many different sums can be thrown with three dice, the faces of
each die being numbered $1, 4, 13, 40, 121, 364$?
\begin{answer}$56$
\end{answer}
      \end{pro}
            \begin{pro}
How many two or three letter initials for people are available if at
least one of the letters must be a D and  one allows repetitions?
\begin{answer} $(26^2 - 25^2) + (26^3 - 25^3) = 2002$\end{answer}
\end{pro}         \begin{pro} How many strictly positive integers have all
their digits distinct? \begin{answer}$$\begin{array}{l}9 +
9\cdot 9   \\ \qquad +9\cdot9\cdot 8 +9\cdot9\cdot8\cdot7    \\
\qquad +9\cdot9\cdot8\cdot7\cdot6+9\cdot9\cdot8\cdot7\cdot6\cdot 5
\\  \qquad + 9\cdot9\cdot8\cdot7\cdot6\cdot5\cdot 4
+9\cdot9\cdot8\cdot7\cdot6\cdot5\cdot4\cdot 3    \\ \qquad +
9\cdot9\cdot8\cdot7\cdot6\cdot5\cdot4\cdot3\cdot 2   \\
\qquad + 9\cdot9\cdot8\cdot7\cdot6\cdot5\cdot4\cdot3\cdot2\cdot
1    \\
  \qquad =
8877690
\end{array}
$$

\end{answer}
    \end{pro}
     \begin{pro} The Morse code consists of points and dashes. How many letters
can be in the Morse code if no letter contains more than four signs,
but all must have at least one?\begin{answer} $2 + 4 + 8 + 16 = 30.$
\end{answer}
 \end{pro}
         \begin{pro}
An $n\times n \times n$ wooden cube is painted blue and then cut
into $n^3$ $1\times 1 \times 1$ cubes. How many cubes (a) are
painted on exactly three sides, (b) are painted in exactly two
sides, (c) are painted in exactly one side, (d) are not painted?
\begin{answer}$8$; $12(n - 2)$; $6(n - 2)^2$; $(n - 2)^3$\\
Comment: This proves that $n^3 = (n-2)^3 + 6(n-2)^2 + 12(n-2) + 8$.
\end{answer}
  \end{pro}
\begin{pro}[AIME 1993] How many even integers between $4000$ and
$7000$ have four different digits? \begin{answer} We condition on
the first digit, which can be $4, 5$, or $6$. If the number starts
with $4$, in order to satisfy the conditions of the problem, we must
choose the last digit from the set $\{0, 2, 6, 8\}$. Thus we have
four choices for the last digit. Once this last digit is chosen, we
have $8$ choices for the penultimate digit and $7$ choices for the
antepenultimate digit. There are thus $4\times 8\times 7 = 224$ even
numbers which have their digits distinct and start with a $4$.
Similarly, there are $224$ even numbers will all digits distinct and
starting with a $6$. When they start with a 5, we have $5$ choices
for the last digit, $8$ for the penultimate and $7$ for the
antepenultimate. This gives $5\times 8\times 7 = 280$ ways. The
total number is thus $224 + 224 + 280 =  728$.
\end{answer}
\end{pro}
\begin{pro}
All the natural numbers, starting with $1$, are listed consecutively
$$123456789101112131415161718192021\ldots$$Which digit occupies the $1002$nd place?
\begin{answer} When the number $99$ is written down, we have used
$$1\cdot 9 + 2\cdot 90 = 189$$ digits. If we were able to write
999, we would have used
$$1\cdot 9 + 2\cdot 90  + 3\cdot 900 =  2889$$digits, which is more than 1002 digits. The 1002nd
digit must be among the three-digit positive integers. We have $1002
- 189 = 813$ digits at our disposal, from which we can make
$\llfloor\frac{813}{3}\rrfloor = 271$ three-digit integers, from 100
to 270. When the $0$ in $270$ is written, we have used $189 + 3\cdot
271 = 1002$ digits. The $1002$nd digit is the $0$ in $270$.
\end{answer}
\end{pro}
\begin{pro} All the positive integers are written in succession.
$$ 123456789101112131415161718192021222324 \ldots $$
Which digit occupies the  $206790$th place?
\begin{answer}$4$
\end{answer}
\end{pro}
\begin{pro}
All the positive integers with initial digit $2$ are written in
succession:
$$2, 20, 21, 22, 23, 24, 25, 26, 27, 28, 29, 200, 201, \ldots,    $$
Find the $1978$-th digit written.
\begin{answer}
 There is $1$ such number with $1$ digit, $10$ such numbers with
$2$ digits, $100$ with three digits, $1000$ with four digits, etc.
Starting with $2$ and finishing with $299$ we have used $1\cdot 1 +
2\cdot 10 + 3\cdot 100 = 321$ digits. We need $1978 - 321 = 1657$
more digits from among the $4$-digit integers starting with $2$. Now
$ \llfloor \frac{1657}{4} \rrfloor = 414$, so we look at the $414$th
4-digit integer starting with $2$, namely, at $2413$. Since the $3$
in $2413$ constitutes the $321 + 4\cdot 414 = 1977$-th digit used,
the $1978$-th digit must be the $2$ starting $2414$.
\end{answer}
\end{pro}
         \begin{pro}[AHSME 1998] Call a  $7$-digit telephone number
$d_1d_2d_3-d_4d_5d_6d_7$ {\em memorable} if the prefix sequence
$d_1d_2d_3$ is exactly the same as either of the sequences
$d_4d_5d_6$ or $d_5d_6d_7$ or possibly both. Assuming that each
$d_i$ can be any of the ten decimal digits $0,1,2,\ldots , 9,$ find
the number of different memorable telephone numbers.
\begin{answer} $19990$
\end{answer}
    \end{pro}
           \begin{pro}
Three-digit numbers are made using the digits $\{1, 3,7,8,9\}$.
\begin{dingautolist}{202}
\item How many of these integers are there? \item How many are
even? \item How many are palindromes? \item How many are divisible
by $3$?
\end{dingautolist}
\begin{answer} [1] $125$, [2] $25$, [3] $25$, [4] $5 + 2\cdot 3+  3\cdot 6
=29$.
\end{answer}
    \end{pro}
      \begin{pro}[AHSME 1989] Five people are sitting at a round table.
Let $f \geq 0$ be the number of people sitting next to at least one
female, and let $m \geq 0$ be the number of people sitting next to
at least one male. Find the number of possible ordered pairs $(f,
m)$. \begin{answer} $8$
\end{answer}
   \end{pro}
     \begin{pro} How many integers less than $10000$ can be made with the eight
digits $0, 1, 2, 3, 4, 5, 6, 7?$ \begin{answer} $4095$

\end{answer}
 \end{pro}
         \begin{pro}[ARML 1999]
In how many ways can one arrange the numbers $21, 31, 41, 51, 61,
71$, and $81$ such that the sum of every four consecutive numbers is
divisible by $3$?
\begin{answer}$144 $
\end{answer}
\end{pro}
\begin{pro}
Let $S$ be the set of all natural numbers whose digits are chosen
from the set $\{1, 3, 5, 7\}$ such that no digits are repeated. Find
the sum of the elements of $S$.\begin{answer} First observe that $1
+ 7 = 3 + 5 = 8$. The numbers formed have either one, two, three or
four digits. The sum of the numbers of $1$ digit is clearly $1 + 7 +
3 + 5 = 16$.

\bigskip

There are $4\times 3 = 12$ numbers formed using $2$ digits, and
hence $6$ pairs adding to $8$ in the units and the tens. The sum of
the $2$ digits formed is $6((8)(10) + 8) = 6\times 88 = 528$.

\bigskip

There are $4\times 3\times 2 = 24$ numbers formed using $3$ digits,
and hence $12$ pairs adding to $8$ in the units, the tens, and the
hundreds. The sum of the $3$ digits formed is $12(8(100) + (8)(10) +
8) = 12\times 888 = 10656$.

\bigskip

There are $4\times 3\times 2\cdot 1 = 24$ numbers formed using $4$
digits, and hence $12$ pairs adding to $8$ in the units, the tens
the hundreds, and the thousands. The sum of the $4$ digits formed is
$12(8(1000) + 8(100) + (8)(10) + 8) = 12\times 8888 = 106656$.

\bigskip

The desired sum is finally
$$16 + 528 + 10656 + 106656 =  117856.$$
\end{answer}
\end{pro}
\begin{pro}
Find the number of ways to choose a pair $\{a , b\}$ of distinct
numbers from the set $\{1, 2, \ldots , 50\}$ such that

\begin{dingautolist}{202}
\item  $|a - b| = 5$ \item $|a - b| \leq 5$.

\end{dingautolist}
\begin{answer} Observe that
\begin{dingautolist}{202}
\item We find the pairs
$$\{1, 6\}, \{2, 7\}, \{3, 8\}, \ldots , \{45, 50\}, $$so there
are 45 in total. (Note: the pair $\{a, b\}$ is indistinguishable
from the pair $\{b, a\}$. \\
\item  If $|a - b| = 1$, then we have
$$\{1, 2\}, \{2, 3\}, \{3, 4\}, \ldots , \{49, 50\},$$or 49 pairs.
If $|a - b| = 2$, then we have
$$\{1, 3\}, \{2, 4\}, \{3, 5\}, \ldots , \{48, 50\},$$or 48 pairs.
If $|a - b| = 3$, then we have
$$\{1, 4\}, \{2, 5\}, \{3, 6\}, \ldots , \{47, 50\},$$or 47 pairs.
If $|a - b| = 4$, then we have
$$\{1, 5\}, \{2, 6\}, \{3, 7\}, \ldots , \{46, 50\},$$or 46 pairs.
If $|a - b| = 5$, then we have
$$\{1, 6\}, \{2, 7\}, \{3, 8\}, \ldots , \{45, 50\},$$or 45 pairs.

The total required is thus
$$49 + 48 + 47 + 46 + 45 = 235.$$
\end{dingautolist}
\end{answer}
\end{pro}
\begin{pro}[AIME 1994] Given a positive integer $n$, let $p(n)$ be
the product of the non-zero digits of $n$. (If $n$ has only one
digit, then $p(n)$ is equal to that digit.) Let
$$ S = p(1) + p(2) + \cdots + p(999).$$ Find $S$.
\begin{answer}  If $x = 0$, put $m(x) = 1$, otherwise put $m(x) = x.$ We use
three digits to label all the integers, from 000 to 999 If $a, b, c$
are digits, then clearly $p(100a + 10b + c) = m(a)m(b)m(c).$ Thus
$$ p(000) + \cdots + p(999)
 = m(0)m(0)m(0) +\cdots + m(9)m(9)m(9),  $$
which in turn
$$
\begin{array}{ll}
   = & (m(0) + m(1) + \cdots + m(9))^3 \\
 = & (1 + 1 + 2 + \cdots + 9)^3 \\
 = & 46^3 \\
 = & 97336.  \\
\end{array}
$$

Hence$$ \begin{array}{lll} S &  = & p(001) + p(002) + \cdots +
p(999)\\
 & = &  97336 - p(000)\\
 & = & 97336 - m(0)m(0)m(0)\\
 & = &  97335.\\
 \end{array}$$
\end{answer}
\end{pro}


\end{multicols}
\section{Permutations without Repetitions}
\begin{df}
We define the symbol $!$ (factorial), as follows: $0! = 1$, and for
integer $n \geq 1$, $$n! = 1\cdot 2 \cdot 3 \cdots n. $$ $n!$ is
read $n$ {\em factorial}.
\end{df}
\begin{exa}We have
$$\begin{array}{lll} 1! &=& 1, \\
2!     &=& 1\cdot 2 = 2, \\
3! & = & 1\cdot 2\cdot 3 = 6, \\
4!& = &1\cdot 2\cdot 3\cdot 4 = 24, \\
5! & = & 1\cdot 2\cdot 3 \cdot 4 \cdot 5 = 120. \end{array}$$
\end{exa}
\begin{exa}We have
\renewcommand{\arraystretch}{3}
$$\begin{array}{lll}  \frac{7!}{4!} &=& \frac{7\cdot 6\cdot 5 \cdot
4!}{4!} = 210,    \\ \frac{(n + 2)!}{n!}& =& \frac{(n + 2)(n +
1)n!}{n!} = (n + 2)(n + 1),    \\ \frac{(n - 2)!}{(n + 1)!} & = &
\frac{(n - 2)!}{(n + 1)(n)(n - 1)(n - 2)!} = \frac{1}{(n + 1)(n)(n -
1)}. \end{array}
$$
\end{exa}
\begin{df}
Let $x_1, x_2, \ldots , x_n$ be $n$ distinct objects. A {\em
permutation} of these objects is simply a rearrangement of them.
\end{df}
\begin{exa}
There are $24$ permutations of the letters in $MATH$, namely
$$\begin{array}{llllll} MATH & MAHT & MTAH & MTHA & MHTA &  MHAT\\
 AMTH & AMHT & ATMH & ATHM & AHTM &  AHMT\\
  TAMH & TAHM & TMAH & TMHA & THMA &  THAM\\
   HATM & HAMT & HTAM & HTMA & HMTA &  HMAT\\       \end{array}   $$

\end{exa}
\begin{thm}\label{thm:counting_permutations}
Let $x_1, x_2, \ldots , x_n$ be $n$ distinct objects. Then there are
$n!$ permutations of them.
\end{thm}
\begin{pf}
The first position can be chosen in $n$ ways, the second object in
$n - 1$ ways, the third in $n - 2$, etc. This gives $$n(n-1)(n-2)
\cdots 2\cdot 1 = n!.
$$
\end{pf}
\begin{exa}
The number of permutations of the letters of the word $RETICULA$ is
$8! = 40320$.
\end{exa}
\begin{exa}
A bookshelf contains $5$ German books, $7$ Spanish books and $8$
French books. Each book is different from one another.
\begin{multicols}{2}\columnseprule 1pt \columnsep 25pt\multicoltolerance=900
\begin{dingautolist}{202}
\item  How many different arrangements can be done of these books?
\item How many different arrangements can be done of these books
if books of each language must be next to each other? \item How many
different arrangements can be done of these books if all the French
books must be next to each other? \item How many different
arrangements can be done of these books if no two French books must
be next to each other?
\end{dingautolist}
\end{multicols}
\end{exa}
Solution:
\begin{multicols}{2}\columnseprule 1pt \columnsep 25pt\multicoltolerance=900
\begin{dingautolist}{202} \item We are permuting $5 + 7 + 8 = 20$ objects. Thus the number
of arrangements sought is $20! = 2432902008176640000$. \item
``Glue'' the books by language, this will assure that books of the
same language are together. We permute the $3$ languages in $3!$
ways. We permute the German books in $5!$ ways, the Spanish books in
$7!$ ways and the French books in $8!$ ways. Hence the total number
of ways is $3!5!7!8! = 146313216000$. \item Align the German books
and the Spanish books first. Putting these $5 + 7 = 12$ books
creates $12 + 1 = 13$ spaces (we count the space before the first
book, the spaces between books and the space after the last book).
To assure that all the French books are next each other, we ``glue''
them together and put them in one of these spaces.  Now, the French
books can be permuted in $8!$ ways and the non-French books can be
permuted in $12!$ ways. Thus the total number of permutations is
$$ (13)8!12! =  251073478656000.$$
\item Align the German books and the Spanish books first. Putting
these $5 + 7 = 12$ books creates $12 + 1 = 13$ spaces (we count the
space before the first book, the spaces between books and the space
after the last book). To assure that no two French books are next to
each other, we put them into these spaces. The first French book can
be put into any of $13$ spaces, the second into any of $12$, etc.,
the eighth French book can be put into any $6$ spaces. Now, the
non-French books can be permuted in  $12!$ ways. Thus the total
number of permutations is
$$ (13)(12)(11)(10)(9)(8)(7)(6)12!,$$which is $24856274386944000.$
\end{dingautolist}
\end{multicols}






\section*{Homework}\addcontentsline{toc}{section}{Homework}\markright{Homework}
\begin{multicols}{2}\columnseprule 1pt \columnsep 25pt\multicoltolerance=900
\begin{pro}
How many changes can be rung with a peal of five bells?
\begin{answer}$120$
\end{answer}
 \end{pro}
       \begin{pro}A bookshelf contains $3$
Russian novels, $4$ German novels, and $5$ Spanish novels. In how
many ways may we align them if\begin{dingautolist}{202} \item there
are no constraints as to grouping? \item
 all the Spanish novels must be together?  \item no
two Spanish novels are next to one  another?

\end{dingautolist}
\begin{answer}
$479001600$; $4838400$; $33868800$
\end{answer}
      \end{pro}
          \begin{pro}
How many permutations of the word {\bf IMPURE} are there? How many
permutations start with {\bf P} and end in {\bf U}?
 How many permutations are there if the {\bf P} and the {\bf U} must always be together in the order {\bf PU}?
 How many permutations are there in which no two vowels ({\bf I, U, E}) are adjacent?

\begin{answer}$720$; $24$; $120$; $144$

\end{answer}
     \end{pro}
           \begin{pro}
How many arrangements can be made of out of the letters of the word
{\bf DRAUGHT}, the vowels never separated?
\begin{answer}$1440$
\end{answer}
   \end{pro}
     \begin{pro}[AIME 1991]
 Given a rational number, write it as a fraction in lowest terms
and calculate the product of the resulting numerator and
denominator. For how many rational numbers between $0$ and $1$ will
$20!$ be the resulting product?
\begin{answer}$128$
\end{answer}
   \end{pro}
          \begin{pro}[AMC12 2001]
 A spider has one sock and one shoe for each of its eight legs.
In how many different orders can the spider put on its socks and
shoes, assuming that, on each leg, the sock must be put on before
the shoe?
\begin{answer}$81729648000$
\end{answer}
  \end{pro}
    \begin{pro}
How many trailing $0$'s are there when $1000!$ is multiplied out?
\begin{answer}$249 $
\end{answer}
\end{pro}


\begin{pro}
In how many ways can $8$ people be seated in a row if
\begin{dingautolist}{202}
\item  there are no constraints as to their seating arrangement?
\item persons $X$ and $Y$ must sit next to one another? \item
there are $4$ women and $4$ men and no $2$ men or $2$ women can sit
next to each other? \item there are $4$ married couples and each
couple must sit together? \item
there are $4$ men and they must sit next to each other?\\

\end{dingautolist}
\begin{answer}We have
\begin{dingautolist}{202}
\item   This is $8!$. \item  Permute $XY$ in $2!$ and put them in
any of the $7$ spaces created by the remaining $6$ people. Permute
the remaining $6$ people. This is $2!\cdot 7\cdot 6!$.\item In this
case, we alternate between sexes. Either we start with a man or a
woman (giving $2$ ways), and then we permute the men and the women.
This is $2\cdot 4!4!$.\item  Glue the couples into $4$ separate
blocks. Permute the blocks in $4!$ ways. Then permute each of the
$4$ blocks in $2!$. This is $4!(2!)^4$. \item
 Sit the women first, creating $5$ spaces in between.
Glue the men together and put them in any of the $5$ spaces. Permute
the men in $4!$ ways and the women in $4!$. This is $5\cdot 4!4!$.
\end{dingautolist}
\end{answer}
\end{pro}
\end{multicols}

\section{Permutations with Repetitions} We now
consider permutations with repeated objects.
\begin{exa}\label{exa:permu_repetitions}
In how many ways may the letters of the word $$MASSACHUSETTS$$ be
permuted?
\end{exa}Solution: We put subscripts on the repeats forming
$$MA_1S_1S_2A_2CHUS_3ET_1T_2S_4.$$There are now $13$
distinguishable objects, which can be permuted in $13!$ different
ways by Theorem \ref{thm:counting_permutations}. For each of these
$13!$ permutations, $A_1A_2$ can be permuted in $2!$ ways,
$S_1S_2S_3S_4$ can be permuted in $4!$ ways, and $T_1T_2$ can be
permuted in $2!$ ways. Thus the over count $13!$ is corrected by the
total actual count
$$\frac{13!}{2!4!2!} = 64864800.
$$
A reasoning analogous to the one of example
\ref{exa:permu_repetitions}, we may prove
\begin{thm}
Let there be $k$ types of objects: $n_1$ of type $1$;  $n_2$ of type
$2$; etc. Then the number of ways in which these $n_1 + n_2 + \cdots
+n_k$ objects can be rearranged is $$ \frac{(n_1 + n_2 + \cdots
+n_k)!}{n_1!n_2! \cdots n_k!}. $$
\end{thm}


\begin{exa} In how many ways may we permute the letters of the word
$MASSACHUSETTS$ in such a way that $MASS$ is always together, in
this order?
\end{exa}
Solution: The particle $MASS$  can be considered as one block and
the $9$ letters $A,$ $C,$ $H,$ $U,$ $S,$ $E,$ $T,$ $T,$ $S$. In $A,$
$C,$ $H,$ $U,$ $S,$ $E,$ $T,$ $T,$ $S$ there are four $S$'s and two
$T$'s and so the total number of permutations sought is
$$\frac{10!}{2!2!} = 907200.
$$

\begin{exa}\label{exa:summingto9}
In how many ways may we write the number $9$ as the sum of three
positive integer summands? Here order counts, so, for example, $1 +
7 + 1$ is to be regarded different from $7 + 1 + 1$.
\end{exa}
Solution: We first look for answers with $$a + b + c = 9, 1 \leq a
\leq b \leq c \leq 7$$ and we find the permutations of each triplet.
We have
$$\begin{array}{|l|l|}\hline (a,b,c) & \mathrm{Number\ of\ permutations} \\
\hline (1,1,7) & \dfrac{3!}{2!} = 3 \\
\hline (1,2,6) & 3! = 6 \\
\hline (1,3,5) & 3! = 6  \\
\hline (1,4,4) & \dfrac{3!}{2!} = 3 \\
\hline (2,2,5) & \dfrac{3!}{2!} = 3 \\
\hline (2,3,4) & 3! = 6 \\
\hline (3,3,3) & \dfrac{3!}{3!} = 1 \\
\hline
   \end{array}$$
Thus the number desired is $$ 3 + 6 + 6 +3 + 3 + 6 + 1 =   28.$$





\begin{exa}
In how many ways can the letters of the word {\bf MURMUR} be
arranged without letting two letters which are alike come together?
\end{exa}
Solution: If we started with, say , {\bf MU} then the {\bf R} could
be arranged as follows:
$$\begin{array}{|c|c|c|c|c|c|}\hline {\bf M} & {\bf U} & {\bf R} & \hspace{.5cm}  & {\bf R} & \hspace{.5cm}    \\ \hline \end{array}\ ,  $$
$$\begin{array}{|c|c|c|c|c|c|}\hline {\bf M} & {\bf U} & {\bf R} & \hspace{.5cm}  & \hspace{.5cm} & {\bf R}    \\ \hline \end{array}\ ,  $$
$$\begin{array}{|c|c|c|c|c|c|}\hline {\bf M} & {\bf U} & \hspace{.5cm} & {\bf R}  & \hspace{.5cm} & {\bf R}    \\ \hline \end{array}\ .  $$
In the first case there are $2! = 2$ of putting the remaining {\bf
M} and {\bf U}, in the second there are $2!=2$ and in the third
there is only $1!$. Thus starting the word with {\bf MU} gives $2 +
2 + 1 = 5$ possible arrangements. In the general case, we can choose
the first letter of the word in $3$ ways, and the second in $2$
ways. Thus the number of ways sought is $3\cdot 2 \cdot 5 = 30$.
\begin{exa}
In how many ways can the letters of the word {\bf AFFECTION} be
arranged, keeping the vowels in their natural order and not letting
the two {\bf F}'s come together?
\end{exa}
Solution: There are $\dfrac{9!}{2!}$ ways of permuting the letters
of {\bf AFFECTION}. The $4$ vowels can be permuted in  $4!$ ways,
and in only one of these will they be in their natural order. Thus
there are $\dfrac{9!}{2!4!}$ ways of permuting the letters of {\bf
AFFECTION} in which their vowels keep their natural order.

\bigskip

Now, put the $7$ letters of {\bf AFFECTION} which are not the two
{\bf F}'s. This creates $8$ spaces in between them where we put the
two {\bf F}'s. This means that there are $8\cdot 7!$ permutations of
{\bf AFFECTION} that keep the two {\bf F}'s together. Hence there
are $\dfrac{8\cdot 7!}{4!}$ permutations of {\bf AFFECTION} where
the vowels occur in their natural order.

\bigskip

In conclusion, the number of permutations sought is
$$ \dfrac{9!}{2!4!} - \dfrac{8\cdot 7!}{4!} = \dfrac{8!}{4!}\left(\dfrac{9}{2} -1\right) = \dfrac{8\cdot 7\cdot 6 \cdot 5 \cdot 4!}{4!}\cdot\dfrac{7}{2}  = 5880 $$



\begin{exa}
How many arrangements of five letters can be made of the letters of
the word {\bf PALLMALL}?
\end{exa}
Solution: We consider the following cases:
\begin{dingautolist}{202}
\item there are four {\bf L}'s and a different letter. The
different letter can be chosen in $3$ ways, so there are
$\dfrac{3\cdot 5!}{4!} = 15$ permutations in this case. \item there
are three {\bf L}'s and two {\bf A}'s. There are $\dfrac{5!}{3!2!} =
10$ permutations in this case. \item there are three {\bf L}'s and
two different letters. The different letters can be chosen in $3$
ways ( either {\bf P} and {\bf A}; or {\bf P} and {\bf M}; or {\bf
A} and {\bf M}), so there are $\dfrac{3\cdot 5!}{3!} = 60$
permutations in this case. \item there are two {\bf L}'s, two {\bf
A}'s and a different letter from these two. The different letter can
be chosen in $2$ ways. There are $\dfrac{2\cdot 5!}{2!2!} = 60$
permutations in this case. \item there are two {\bf L}'s and three
different letters. The different letters can be chosen in $1$ way.
There are $\dfrac{1\cdot 5!}{2!} = 60$ permutations in this case.
\item there is one  {\bf L}. This forces having two {\bf A}'s and
two other different letters. The different letters can be chosen in
$1$ way. There are $\dfrac{1\cdot 5!}{2!} = 60$ permutations in this
case.

\end{dingautolist}
The total number of permutations is thus seen to be
$$15 + 10 + 60 + 60 + 60 + 60 = 265.   $$








\section*{Homework}\addcontentsline{toc}{section}{Homework}\markright{Homework}
\begin{multicols}{2}\columnseprule 1pt \columnsep 25pt\multicoltolerance=900
\begin{pro}
In how many ways may one permute the letters of the word {\bf
MEPHISTOPHELES}? \begin{answer}$1816214400$
\end{answer}
   \end{pro}
   \begin{pro}
How many arrangements of four letters can be made out of the letters
of {\bf KAFFEEKANNE} without letting the three {\bf E}'s come
together?
\begin{answer}$548$
\end{answer}
  \end{pro}
   \begin{pro}
How many numbers can be formed with the digits $$1,2,3,4,3,2,1$$ so
that the odd digits occupy the odd places?
\begin{answer}$18$
\end{answer}
  \end{pro}

\begin{pro}
In this problem you will determine how many different signals, each
consisting
 of $10$ flags hung in a line, can be made from a set of $4$ white
 flags, $3$ red flags, $2$ blue flags, and $1$ orange flag, if
 flags of the same colour are identical.
\begin{dingautolist}{202}
\item How many are there if there are no constraints on the order?
 \item How many are there if the orange flag must always be
first? \item How many are there if there must be a white flag at the
beginning and another white flag at the end?
\end{dingautolist}
\begin{answer}We have
\begin{dingautolist}{202}
\item This is $$\dfrac{10!}{4!3!2!}   $$
 \item This is $$\dfrac{9!}{4!3!2!}   $$\item  This is
$$\dfrac{8!}{2!3!2!}   $$
\end{dingautolist}
\end{answer}
\end{pro}

     \begin{pro}\label{pro:summingto10}
In how many ways may we write the number $10$ as the sum of three
positive integer summands? Here order counts, so, for example, $1 +
8 + 1$ is to be regarded different from $8 + 1 + 1$.
\begin{answer}$36$
\end{answer}
  \end{pro}
     \begin{pro}
Three distinguishable dice are thrown. In how many ways can they
land and give a sum of $9$?
\begin{answer}$25$
\end{answer}
 \end{pro}
  \begin{pro}
In how many ways can $15$ different recruits be divided into three
equal groups? In how many ways can they be drafted into three
different regiments?
\begin{answer}$126126$; $756756$
\end{answer}
\end{pro}
\end{multicols}
\section{Combinations without Repetitions}
\begin{df}
Let $n, k$ be  non-negative integers with $0\leq k \leq n$. The
symbol $\dis{\binom{n}{k}}$ (read ``$n$ {\em choose} $k$'') is
defined and denoted by
$$\binom{n}{k} = \frac{n!}{k!(n - k)!} = \frac{n\cdot (n - 1) \cdot (n - 2) \cdots (n - k + 1)}{1\cdot 2 \cdot 3 \cdots k}.  $$
\end{df}
\begin{rem}
Observe that in the last fraction, there are $k$ factors in both the
numerator and denominator. Also, observe the boundary conditions
$$\binom{n}{0} = \binom{n}{n} = 1,\ \ \ \ \binom{n}{1} =
\binom{n}{n-1} = n.
$$
\end{rem}
\begin{exa}We have
$$ \begin{array}{lll}\binom{6}{3} & =  &\frac{6\cdot 5 \cdot 4}{1\cdot 2 \cdot 3} = 20,  \\
 \binom{11}{2}  & =  & \frac{11\cdot 10}{1\cdot 2 } = 55,  \\ \binom{12}{7} & =  &
 \frac{12\cdot 11 \cdot 10\cdot 9 \cdot 8 \cdot 7
\cdot 6}{1\cdot 2 \cdot 3\cdot 4\cdot 5 \cdot 6 \cdot 7} = 792,  \\
\binom{110}{109} &   =  & 110, \\ \binom{110}{0} &   =  & 1.
\end{array}
$$
\end{exa}
\begin{rem}
Since $n - (n - k) = k  $, we have for integer $n, k$, $0 \leq k
\leq n$, the symmetry identity
\begin{center}\fcolorbox{blue}{yellow}{$\dis{\binom{n}{k} = \frac{n!}{k!(n - k)!} = \frac{n!}{(n-k)!(n -
(n-k))!} = \binom{n}{n - k}}$}.
\end{center} This can be interpreted as follows: if there are $n$ different tickets in a hat, choosing $k$ of them out of the hat is the same
as choosing $n - k$ of them to remain in the hat.\end{rem}
\begin{exa}
$$\binom{11}{9} = \binom{11}{2} = 55, $$
$$ \binom{12}{5} = \binom{12}{7} = 792. $$
\end{exa}
\begin{df}
Let there be $n$ distinguishable objects. A $k$-{\em combination} is
a selection of $k$, ($0 \leq k \leq n$) objects from the $n$ made
without regards to order.
\end{df}
\begin{exa} The
2-combinations from the list $\{X, Y, Z, W\}$ are
$$XY, XZ, XW, YZ, YW, WZ.    $$
\end{exa}
\begin{exa} The
3-combinations from the list $\{X, Y, Z,W\}$ are
$$XYZ, XYW, XZW,  YWZ.    $$
\end{exa}


\begin{thm}
Let there be $n$ distinguishable objects, and let $k$, $0 \leq k
\leq n$. Then the numbers of $k$-combinations of these $n$ objects
is $\dis{\binom{n}{k}}$.
\end{thm}
\begin{pf}
Pick any of the $k$ objects. They can be ordered in $n(n - 1)(n - 2)
\cdots (n - k + 1)$, since there are $n$ ways of choosing the {\em
first}, $n - 1$ ways of choosing the {\em second}, etc. This
particular choice of $k$ objects can be permuted in $k!$ ways. Hence
the total number of $k$-combinations is $$ \frac{n(n - 1)(n - 2)
\cdots (n - k + 1)}{k!}=  \binom{n}{k}.$$
\end{pf}
\begin{exa}
From a group of $10$ people, we may choose a committee of $4$ in
$\dis{\binom{10}{4} = 210} $ ways.
\end{exa}


\begin{exa}
In a group of $2$ camels, $3$ goats, and $10$ sheep in how many ways
may one choose $6$ animals if
\begin{multicols}{2}\columnseprule 1pt \columnsep 25pt\multicoltolerance=900

\begin{dingautolist}{202}
\item there are no constraints in species? \item the two camels
must be included? \item the two camels must be excluded? \item there
must be at least $3$ sheep? \item there must be at most $2$ sheep?
\item Joe Camel, Billy Goat and Samuel Sheep hate each other and
they will not work in the same group. How many compatible committees
are there? \end{dingautolist}
\end{multicols}
\end{exa}
Solution:
\begin{multicols}{2}\columnseprule 1pt \columnsep 25pt\multicoltolerance=900

\begin{dingautolist}{202}
\item  There are $2 +3+10=15$ animals and we must choose $6$, whence $\binom{15}{6} = 5005$
\item Since the $2$ camels are included, we must choose $6-2 = 4$
more animals from a list of $15-2 = 13$ animals, so
 $\binom{13}{4} = 715$\item Since the $2$ camels must be excluded, we must choose $6$ animals from a list of $15-2 = 13$, so
  $\binom{13}{6} = 1716$
\item If $k$ sheep are chosen from the $10$ sheep, $6-k$ animals must be chosen from the remaining $5$ animals, hence
 $$\binom{10}{3}\binom{5}{3} + \binom{10}{4}\binom{5}{2} + \binom{10}{5}\binom{5}{1} + \binom{10}{6}\binom{5}{0}, $$ which simplifies
 to $
 4770$.
\item  First observe that there cannot be $0$ sheep, since that would mean choosing $6$ other animals. Hence, there must be either $1$ or $2$ sheep,
and so $3$ or $4$  of the other animals. The total number is thus
$$\binom{10}{2}\binom{5}{4} + \binom{10}{1}\binom{5}{5}  = 235. $$
\item A compatible group will either exclude all these three
animals or include exactly one of them. This can be done in
$$\binom{12}{6}+ \binom{3}{1}\binom{12}{5} = 3300 $$ways.
\end{dingautolist}
\end{multicols}




\begin{exa}\label{exa:routes}
To count the  number of shortest routes from $A$ to $B$ in figure
\ref{fig:routes} observe that any shortest path must consist of $6$
horizontal moves and $3$ vertical ones for a total of $6 + 3 = 9$
moves. Of these $9$ moves once we choose the $6$ horizontal ones the
$3$ vertical ones are determined. Thus there are $\binom{9}{6} = 84$
paths.
\end{exa}
\begin{exa}\label{exa:routes2}
To count the  number of shortest routes from $A$ to $B$ in figure
\ref{fig:routes2} that pass through point $O$ we count the number of
paths from $A$ to $O$ (of which there are $\binom{5}{3} = 20$) and
the number of paths from $O$ to $B$ (of which there are
$\binom{4}{3} = 4$). Thus the desired number of paths is
$\binom{5}{3}\binom{4}{3} = (20)(4) = 80$.
\end{exa}

\vspace{2cm}
\begin{figure}[h]
\centering
\begin{minipage}{5cm}
$$\rput(-2,0){\psset{unit=2pc}\psset{gridwidth=1pt,gridlabels=0, subgriddiv=0}\psgrid(0,0)(0,0)(6,3)
\psline[linewidth=2pt, linecolor=red](0,0)(4,0)(4,2)(6,2)(6,3)
\uput[l](0,0){A} \uput[r](6,3){B} }  $$ \vspace{1cm}
\footnotesize\footnotesize\hangcaption{Example \ref{exa:routes}.}
\label{fig:routes}
\end{minipage}
\hfill
\begin{minipage}{5cm}
$$\rput(-2,0){\psset{unit=2pc}\psset{gridwidth=1pt,gridlabels=0, subgriddiv=0}\psgrid(0,0)(0,0)(6,3)
\psdots[dotstyle=*, dotscale=1](3,2)\psline[linewidth=2pt,
linecolor=red](0,0)(3,0)(3,2)(6,2)(6,3)
\uput[l](0,0){A}\uput[ur](3,2){O} \uput[r](6,3){B}}   $$
\vspace{1cm} \footnotesize\footnotesize\hangcaption{Example
\ref{exa:routes2}.} \label{fig:routes2}
\end{minipage} \hfill
\begin{minipage}{5cm}
$$
\psset{unit=2pc} \pscircle(-1,0){2} \pscircle(1,0){2}
\pscircle(0,1.41421356){2}{\rput(2,0){9550} \rput(-2, 0){9550}
\rput(0,0.707106781){14406} \rput(0, 2.3){9550}
\rput(0,-0.72){14266} \rput(-1.43,1.43){14266}
\rput(1.43,1.43){14266} \uput[dl](-1, -2){\mathrm{without\ a}\ 7}
\uput[dr](1, -2){\mathrm{without\ an}\ 8} \uput[u](0,
3.5){\mathrm{without\ a}\ 9}}
$$\vspace{1cm}\footnotesize\hangcaption{Example
\ref{exa:5digit_inclusion-exclusion}.}\label{fig:5digit_inclusion-exclusion}
\end{minipage}


\end{figure}


\begin{exa}\label{exa:5digit_inclusion-exclusion}
Consider the set of $5$-digit  positive integers written in decimal
notation.
\begin{multicols}{2}\columnseprule 1pt \columnsep 25pt\multicoltolerance=900

\begin{enumerate}
\item How many are there?

  \item How many do not have a $9$ in
their decimal representation?
  \item How many have at least one
$9$ in their decimal representation?
 \item How many have
exactly one $9$?

  \item How many have exactly two $9$'s?

\item How many have exactly three $9$'s?

 \item How many have
exactly four $9$'s?

  \item How many have exactly five $9$'s?

\item How many have neither an $8$ nor a $9$ in their decimal
representation?


\item How many have neither a $7$, nor an $8$, nor a $9$ in their
decimal representation?



\item How many have either a $7$, an $8$, or a $9$ in their
decimal representation?


\end{enumerate}

\end{multicols}

 \end{exa} Solution: \begin{multicols}{2}\columnseprule 1pt \columnsep 25pt\multicoltolerance=900

\begin{enumerate}
\item  There are $9$ possible choices for the first digit and $10$
possible choices for the remaining digits. The number of choices is
thus $9\cdot10^4 = 90000$.

  \item There are $8$ possible choices for the first digit and $9$
possible choices for the remaining digits. The number of choices is
thus $8\cdot 9^4 = 52488$.
  \item The difference $90000 - 52488 = 37512.$
 \item We condition on the first digit. If the first digit is a $9$
then the other four remaining digits must be different from $9$,
giving $9^4 = 6561$ such numbers. If the first digit is not a $9$,
then there are $8$ choices for this first digit. Also, we have
$\binom{4}{1} = 4$ ways of choosing were the $9$ will be, and we
have $9^3$ ways of filling the $3$ remaining spots. Thus in this
case there are $8\cdot 4 \cdot 9^3 =  23328$ such numbers. In total
there are $6561+23328 = 29889$ five-digit positive integers with
exactly one $9$ in their decimal representation.
  \item  We condition on the first digit. If the first digit is a $9$
then one of the remaining four must be a $9$, and the choice of
place can be accomplished in $\binom{4}{1} = 4$ ways. The other
three remaining digits must be different from $9$, giving $4\cdot
9^3 = 2916$ such numbers. If the first digit is not a $9$, then
there are $8$ choices for this first digit. Also, we have
$\binom{4}{2} = 6$ ways of choosing were the two $9$'s will be, and
we have $9^2$ ways of filling the two remaining spots. Thus in this
case there are $8\cdot 6 \cdot 9^2 = 3888$ such numbers. Altogether
there are $2916 + 3888 = 6804$ five-digit positive integers with
exactly two $9$'s in their decimal representation.
\item  Again we condition on the first digit. If the first digit
is a $9$ then two of the remaining four must be $9$'s, and the
choice of place can be accomplished in $\binom{4}{2} = 6$ ways. The
other two remaining digits must be different from $9$, giving
$6\cdot 9^2 = 486$ such numbers. If the first digit is not a $9$,
then there are $8$ choices for this first digit. Also, we have
$\binom{4}{3} = 4$ ways of choosing were the three $9$'s will be,
and we have $9$ ways of filling the remaining spot. Thus in this
case there are $8\cdot 4 \cdot 9 = 288$ such numbers. Altogether
there are $486 + 288 = 774$ five-digit positive integers with
exactly three $9$'s in their decimal representation.
 \item  If the first digit is a $9$ then three of the remaining four
must be $9$'s, and the choice of place can be accomplished in
$\binom{4}{3} = 4$ ways. The other  remaining digit must be
different from $9$, giving $4\cdot 9 =  36$ such numbers. If the
first digit is not a $9$, then there are $8$ choices for this first
digit. Also, we have $\binom{4}{4} = 4$ ways of choosing were the
four $9$'s will be, thus filling all the spots. Thus in this case
there are $8\cdot 1 = 8$ such numbers. Altogether there are $36 + 8
= 44$ five-digit positive integers with exactly three $9$'s in their
decimal representation.

  \item There is obviously  only $1$ such positive integer.

\begin{rem}Observe that $37512 = 29889 + 6804 + 774 + 44 +
1$.\end{rem}

\item  We have $7$ choices for the first digit and $8$ choices for
the remaining $4$ digits, giving $7\cdot 8^4 = 28672$ such integers.


\item  We have $6$ choices for the first digit and $7$ choices for
the remaining $4$ digits, giving $6\cdot 7^4 = 14406$ such integers.


\item  We use inclusion-exclusion. From figure
\ref{fig:5digit_inclusion-exclusion}, the numbers inside the circles
add up to $85854$. Thus the desired number is $90000-85854=4146.$
\end{enumerate}
\end{multicols}

\section*{Homework}\addcontentsline{toc}{section}{Homework}\markright{Homework}
\begin{multicols}{2}\columnseprule 1pt \columnsep 25pt\multicoltolerance=900
\begin{pro}
Verify the following.
\begin{dingautolist}{202}
\item $\binom{20}{3} = 1140$ \item $\binom{12}{4}\binom{12}{6} =
457380$ \item $\dfrac{\binom{n}{1}}{\binom{n}{n-1}} = 1$ \item
$\binom{n}{2} = \dfrac{n(n-1)}{2}$ \item $\binom{6}{1} +
\binom{6}{3} + \binom{6}{6} = 2^5.$ \item $\binom{7}{0} +
\binom{7}{2} + \binom{7}{4} = 2^6 - \binom{7}{6}$
\end{dingautolist}

      \end{pro}
              \begin{pro}
A publisher proposes to issue a set of dictionaries to translate
from any one language to any other. If he confines his system to ten
languages, how many dictionaries must be published?
\begin{answer}$\binom{10}{2} = 45$
\end{answer}
  \end{pro}

\begin{pro}
From a group of $12$ people---$7$ of which are men and $5$
women---in how many ways may choose a committee of $4$ with $1$ man
and $3$ women? \begin{answer} $\binom{7}{1}\binom{5}{3} = (7)(10) =
70 $
\end{answer}
\end{pro}
   \begin{pro}
$N$ friends meet and shake hands with one another. How many
handshakes?
\begin{answer}$\binom{N}{2}$

\end{answer}
   \end{pro}
      \begin{pro}
How many $4$-letter words can be made by taking $4$ letters  of the
word {\bf RETICULA} and permuting them?
\begin{answer}$\binom{8}{4}4! = 1680$
\end{answer}
  \end{pro}
   \begin{pro}[AHSME 1989] Mr. and Mrs. Zeta want to name baby Zeta so
that its monogram (first, middle and last initials) will be in
alphabetical order with no letters repeated. How many such monograms
are possible?
\begin{answer} $\binom{25}{2} = 300$
\end{answer}
  \end{pro}
\begin{pro}
In how many ways can $\{1,2,3,4\}$ be written as the union of two
non-empty, disjoint subsets?
\begin{answer}
Let the subsets be $A$ and $B$. We have either $\card{A} = 1$ or
$\card{A} = 2$. If $\card{A} = 1$ then there are $\binom{4}{1}=4$
ways of choosing its elements and $\binom{3}{3}=1$ ways of choosing
the elements of $B$. If $\card{A} = 2$ then  there are
$\binom{4}{2}=6$ ways of choosing its elements and $\binom{2}{2}=1$
ways of choosing the elements of $B$. Altogether there are $4+6 =
10$ ways.

\end{answer}
\end{pro}

\begin{pro}
How many  lists of $3$ elements taken from the set $\{1,2,3,4,5,6\}$
 list the elements in increasing order?
 \begin{answer}  $\dis{\binom{6}{3} =
20}$
\end{answer}
\end{pro}
\begin{pro} How many times is the digit $3$ listed in the numbers $1$ to
$1000$? \begin{answer} We count those numbers that have exactly
once, twice and three times. There is only one number that has it
thrice (namely 333). Suppose the number $xyz$ is to have the digit 3
exactly twice. We can choose these two positions in $\binom{3}{2}$
ways. The third position can be filled with any of the remaining
nine digits (the digit 3 has already been used). Thus there are
$9\binom{3}{2}$ numbers that the digit 3 exactly twice. Similarly,
there are $9^2 \binom{3}{2}$ numbers that have 3 exactly once. The
total required is hence $3\cdot 1 + 2\cdot 9\cdot\binom{3}{2} + 9^2
\binom{3}{1} = 300.$
\end{answer}
\end{pro}


\begin{pro}
How many subsets of the set $\{a,b,c,d,e\}$ have exactly $3$
elements? \begin{answer} $\dis{\binom{5}{3} = 10}$\end{answer}
\end{pro}

\begin{pro}How many subsets of the set $\{a,b,c,d,e\}$ have an odd number of
elements? \begin{answer} $\binom{5}{1} + \binom{5}{3} + \binom{5}{5}
= 5 + 10 + 1 = 16.$\end{answer}
\end{pro}


    \begin{pro}[AHSME 1994] Nine chairs in a row are to be occupied by
six students and Professors Alpha, Beta and Gamma. These three
professors arrive before the six students and decide to choose their
chairs so that each professor will be between two students. In how
many ways can Professors Alpha, Beta and Gamma choose their chairs?
\begin{answer}  $10\times 3! = 60$
\end{answer}
  \end{pro}
    \begin{pro}
There are $E$ (different) English novels, $F$ (different) French
novels, $S$ (different) Spanish novels, and $I$ (different) Italian
novels on a shelf. How many different permutations are there if
\begin{dingautolist}{202}
\item if there are no restrictions?

\item if all books of the same language must be together?

\item if all the Spanish novels must be together? \item if no two
Spanish novels are adjacent?


\item if all the Spanish novels must be together, and all the
English novels must be together, but no Spanish novel is next to an
English novel?


\end{dingautolist}
\begin{answer}We have
\begin{dingautolist}{202}
\item $(E + F + S + I)!$\item  $4!\cdot E!F!S!I!$\item
 $\binom{E+F+I+1}{1}S!(E + F+ I)!$\item  $\binom{E+F+I+1}{S}S!(E + F+ I)!$\item
  $2!\binom{F+I+1}{2}S!E!(F+ I)!$ \end{dingautolist}
\end{answer}
 \end{pro}

\begin{pro} How many committees of seven with a given chairman can be
selected from twenty people? \begin{answer} We can choose the seven
people in $\binom{20}{7}$ ways. Of the seven, the chairman
can be chosen in seven ways. The answer is thus $$7\binom{20}{7} = 542640.$$ \\
{\em Aliter}: Choose the chairman first. This can be done in twenty
ways. Out of the nineteen remaining people, we just have to choose
six, this can be done in $\binom{19}{6}$ ways. The total number of
ways is hence $20\binom{19}{6} = 542640$.
\end{answer}
\end{pro}
\begin{pro} How many committees of seven with a given chairman and a given
secretary can be selected from twenty people? Assume the chairman
and the secretary are different persons. \begin{answer} We can
choose the seven people in $\binom{20}{7}$ ways. Of these seven
people chosen, we can choose the chairman in seven ways and the
secretary in six ways. The
answer is thus $7\cdot 6\binom{20}{7} = 3255840.$ \\
{\em Aliter}: If one chooses the chairman first, then the secretary
and finally the remaining five people of the committee, this can be
done in $20\cdot 19\cdot \binom{18}{5} = 3255840$ ways.
\end{answer}
\end{pro}
\begin{pro}[AHSME 1990] How many of the numbers $$100, 101, \ldots ,
999, $$ have three different digits in increasing order or in
decreasing order? \begin{answer} For a string of three-digit numbers
to be decreasing, the digits must come from $\{ 0, 1, \ldots , 9\}$
and so there are $\binom{10}{3} = 120$ three-digit numbers with all
its digits in decreasing order.  If the string of three-digit
numbers is increasing, the digits have to come from $\{ 1, 2, \ldots
, 9\}$, thus there are $\binom{9}{3} = 84$ three-digit numbers with
all the digits increasing. The total asked is hence $120 + 84 =
204.$
\end{answer}
\end{pro}
\begin{pro} There are twenty students in a class. In how many ways can
the twenty students take five different tests if four of the
students are to take each test? \begin{answer} We can choose the
four students who are going to take the first test in
$\binom{20}{4}$ ways. From the remaining ones, we can choose
students in $\binom{16}{4}$ ways to take the second test. The third
test can be taken in $\binom{12}{4}$ ways. The fourth in
$\binom{8}{4}$ ways and the fifth in $\binom{4}{4}$ ways. The total
number is thus
$$ \binom{20}{4}\binom{16}{4}\binom{12}{4}\binom{8}{4}\binom{4}{4}.$$
\end{answer}
\end{pro}

\begin{pro} In how many ways can a deck of playing cards be
arranged if no two hearts are adjacent?\begin{answer} We align the
thirty-nine cards which are not hearts first. There are thirty-eight
spaces between them and one at the beginning and one at the end
making a total of forty spaces where the hearts can go. Thus there
are $\binom{40}{13}$ ways of choosing the {\em places} where the
hearts can go. Now, since we are interested in arrangements, there
are $39!$ different configurations of the non-hearts and $13!$
different configurations of the hearts. The total number of
arrangements is thus $\binom{40}{13}39!13!.$
\end{answer}
\end{pro}
\begin{pro} Given a positive integer $n$, find the number of quadruples $(a,
b, c, d,)$ such that $ 0 \leq a \leq b \leq c \leq d \leq n$.
\begin{answer} The equality signs cause us trouble, since allowing them
would entail allowing repetitions in our choices. To overcome that
we establish a one-to-one  correspondence between the vectors $(a,
b, c, d), 0 \leq a \leq b \leq c \leq d \leq n$ and the vectors
$(a', b', c', d'), 0 \leq a' < b' < c' < d' \leq n + 3$. Let $(a',
b', c', d') = (a, b + 1, c + 2, d + 3)$. Now we just have to pick
four different numbers from the set $\{ 0, 1, 2, 3, \ldots , n, n +
1, n + 2, n + 3\}$. This can be done in $\binom{n + 4}{4}$ ways.
\end{answer}
\end{pro}


\begin{pro}There are $T$ books on Theology, $L$ books on Law and $W$
books on Witchcraft on Dr. Faustus' shelf. In how many ways may one
order the books
\begin{dingautolist}{202}
\item there are no constraints in their order? \item all books of
a subject must be together? \item no two books on Witchcraft are
juxtaposed? \item all the books on Witchcraft must be together?
\end{dingautolist}
\begin{answer}We have\begin{dingautolist}{202} \item $(T + L + W)!$ \item $3!T!L!W!
= 6T!L!W!$ \item $\dis{\binom{T + L + 1}{W}(T+L)!W!}$
\item $\dis{\binom{T + L + 1}{1}(T+L)!W!}$
\end{dingautolist}
\end{answer}
\end{pro}


\begin{pro}
From a group of $20$ students, in how many ways may a professor
choose at least one in order to work on a project? \begin{answer}
The required number is $$\binom{20}{1} + \binom{20}{2} + \cdots +
\binom{20}{20} = 2^{20} - \binom{20}{0} = 1048576 - 1 = 1048575.
$$
\end{answer}
\end{pro}

\begin{pro}
From a group of $20$ students, in how many ways may a professor
choose an even number number of them, but at least four in order to
work on a project? \begin{answer} The required number is
$$\binom{20}{4} + \binom{20}{6} + \cdots + \binom{20}{20} = 2^{19} -
\binom{20}{0} - \binom{20}{2} = 524288 - 1-190 = 524097.
$$
\end{answer}
\end{pro}
  \begin{pro}How many permutations of the word
$${\bf CHICHICUILOTE}   $$are there
\begin{dingautolist}{202}
\item if there are no restrictions?





 \item if the word must
start in an {\bf I} and end also in an {\bf I}?





\item
if the word must start in an {\bf I} and end in a {\bf C}?\\




\item if the two {\bf H}'s are adjacent?\\
 \item if the two {\bf H}'s are not adjacent?\\
 \item if the particle {\bf LOTE} must appear, with the letters
in this order?
\end{dingautolist}
\begin{answer} We have \begin{dingautolist}{202}
\item   $\dfrac{13!}{2!3!3!} = 86486400$\item  $\dfrac{11!}{2!3!} =
3326400$\item  $\dfrac{11!}{2!2!2!} = 4989600$\item
$\binom{12}{1}\dfrac{11!}{3!3!} = 13305600$\item
$\binom{12}{2}\dfrac{11!}{3!3!} = 73180800$\item
$\binom{10}{1}\dfrac{9!}{3!3!2!} = 50400$
\end{dingautolist}
\end{answer}
      \end{pro}
      \begin{pro}
There are $M$ men and $W$ women in a group. A committee of $C$
people will be chosen. In how many ways may one do this if
\begin{dingautolist}{202}
\item there are no constraints on the sex  of the committee
members? \item there must be exactly $T$ women? \item A committee
must always include George and Barbara?  \item  A committee must
always exclude George and Barbara?
\end{dingautolist}
Assume George and Barbara form part of the original set of people.
\begin{answer} We have \begin{dingautolist}{202} \item
$\dis{\binom{M + W}{C}}$\item  $\dis{\binom{M}{C -
T}\binom{W}{T}}$\item  $\binom{M + W - 2}{C - 2}$ \item  $\binom{M +
W - 2}{C}$
\end{dingautolist}
\end{answer}
    \end{pro}
         \begin{pro}
There are $M$ men and $W$ women in a group. A committee of $C$
people will be chosen. In how many ways may one do this if George
and Barbara are feuding and will not work together in a committee?
Assume George and Barbara form part of the original set of people.
\begin{answer}$$\binom{M + W}{C} - \binom{M + W - 2}{C-2}=2\binom{M + W-2}{C-1}  + \binom{M + W-2}{C}.$$
\end{answer}
    \end{pro}
         \begin{pro}
Out of $30$ consecutive integers, in how many ways can three be
selected so that their sum be even?
\begin{answer}$2030$
\end{answer}
   \end{pro}
      \begin{pro}
In how many ways may we choose three distinct integers from $\{1,2,
\ldots, 100\}$ so that one of them is the average of the other two?
\begin{answer}$2\binom{50}{2}$
\end{answer}
  \end{pro}
      \begin{pro}How many vectors $(a_1, a_2, \ldots, a_k)$ with integral $$a_i\in \{1,2, \ldots , n\}$$
are there satisfying $$1 \leq a_1 \leq a_2 \leq \cdots \leq a_k \leq
n?$$   \begin{answer} $\binom{n + k -1}{k}$
\end{answer}
 \end{pro}
 \begin{pro}
A square chessboard has $16$ squares ($4$ rows and $4$ columns). One
puts $4$ checkers in such a way that  only one checker can be put in
a square. Determine the number of ways of putting these checkers if
\begin{dingautolist}{202}
\item there must be exactly one checker per row and column. \item
there must be exactly one column without a checker. \item there must
be at least one column without a checker.
\end{dingautolist}
\begin{answer} [1] For the first column one can put any of $4$ checkers, for the second one, any of $3$, etc. hence there
are  $4 \cdot 3 \cdot 2 \cdot 1 = 24$. [2] If there is a column
without a checker then there must be a column with $2$ checkers.
There are $3$ choices for this column. In this column we can put the
two checkers in $\binom{4}{2} = 6$ ways. Thus there are $4 \cdot 3
\binom{4}{2} 4\cdot 4 = 1152$ ways of putting the checkers. [3]  The
number of ways of filling the board with no restrictions is
$\binom{16}{4}$. The number of ways of of of filling the board so
that there is one checker per column is $4^4$. Hence the total is
$\binom{16}{4} - 4^4 = 1564$.
\end{answer}
    \end{pro}
   \begin{pro}
A box contains $4$ red, $5$ white, $6$ blue, and $7$ magenta balls.
In how many of all possible samples of size $5$, chosen without
replacement, will every colour be represented?
\begin{answer} $7560$.
\end{answer}
  \end{pro}
     \begin{pro} In how many ways can eight students be divided into four
indistinguishable teams of two each?\begin{answer} $\frac{1}{4!}
\binom{8}{2} \binom{6}{2} \binom{4}{2}.$
\end{answer}
   \end{pro}
     \begin{pro} How many ways can three boys share fifteen different sized
pears if the youngest gets seven pears and the other two boys get
four each?\begin{answer} $\binom{15}{7} \binom{8}{4}.$
\end{answer}      \end{pro}         \begin{pro} Among the integers $1$ to $10^{10}$, which are there more of:
those in which the digit 1 occurs or those in which it does not
occur?\begin{answer} There are $6513215600$ of former and
$3486784400$ of the latter.
\end{answer}
  \end{pro}
    \begin{pro} Four writers must write a book containing seventeen chapters.
The first and third writers must each write five chapters, the
second must write four chapters, and the fourth must write three
chapters.  How many ways can the book be divided between the
authors?  What if the first and third had to write ten chapters
combined, but it did not matter which of them wrote how many (i.e.
the first could write ten and the third none, the first could write
none and the third one, etc.)?\begin{answer} $\binom{17}{5}
\binom{12}{5} \binom{7}{4} \binom{3}{3}; \binom{17}{3} \binom{14}{4}
2^{10}.$
\end{answer}
   \end{pro}
        \begin{pro} In how many ways can a woman choose three lovers or more from
seven eligible suitors? \begin{answer} $\displaystyle{\sum _{k = 3}
^7 \binom{7}{k}} = 99$
\end{answer}      \end{pro}         \begin{pro}[AIME 1988] One commercially available ten-button lock
may be opened by depressing---in any order---the correct five
buttons. Suppose that these locks are redesigned so that sets of as
many as nine buttons or as few as one button could serve as
combinations.  How many additional combinations would this allow?
\begin{answer} $ 2^{10} - 1 - 1 - \binom{10}{5} = 1024 - 2 - 252 = 770 $
\end{answer}      \end{pro}         \begin{pro} From a set of $n \geq 3$ points on the plane, no three collinear,
\begin{dingautolist}{202}
\item how many straight lines are determined? \item how many
straight lines pass through a particular point? \item how many
triangles are determined? \item how many triangles have a particular
point as a vertex?
\end{dingautolist}\begin{answer}$\binom{n}{2}$;  $n - 1$; $\binom{n}{3}$;  $\binom{n - 1}{2}$

\end{answer}
 \end{pro}
  \begin{pro} In how many ways can you pack twelve books into four parcels if
one parcel has one book, another has five books, and another has two
books, and another has four books?\begin{answer} $\binom{12}{1}
\binom{11}{5} \binom{6}{2} \binom{4}{4}  $
\end{answer}
  \end{pro}
    \begin{pro} In how many ways can a person invite three of his six friends to
lunch every day for twenty days if he has the option of inviting the
same or different friends from previous days?\begin{answer}
$\binom{6}{3}^{20} = 104857600000000000000000000$
\end{answer}
  \end{pro}
     \begin{pro} A committee is to be chosen from a set of nine women and five
men. How many ways are there to form the committee if the committee
has three men and three women?\begin{answer} $\binom{9}{3}
\binom{5}{3} = 840$
\end{answer}
    \end{pro}
          \begin{pro}
At a dance there are $b$ boys and $g$ girls. In how many ways can
they form $c$ couples consisting of different sexes?
\begin{answer}$\binom{b}{c}\binom{g}{c}c!$
\end{answer}
   \end{pro}         \begin{pro} From three Russians, four Americans, and two Spaniards, how many
selections of people can be made, taking at least one of each kind?
\begin{answer} $(2^3 - 1)(2^4 - 1)(2^2-1) = 315$
\end{answer}
   \end{pro}
        \begin{pro}

The positive integer $r$ satisfies
$$\dfrac{1}{\binom{9}{r}} - \dfrac{1}{\binom{10}{r}} =
\dfrac{11}{6\binom{11}{r}}.
$$Find $r$.

  \end{pro}
           \begin{pro}
If $11\binom{28}{2r} = 225\binom{24}{2r - 4}$, find $r$.

  \end{pro}
          \begin{pro} Compute the number of ten-digit numbers which contain only the
digits $1, 2$, and $3$ with the digit $2$ appearing in each number
exactly twice.\begin{answer} $\binom{10}{2} 2^8$

\end{answer}
  \end{pro}
\begin{pro} Prove {\em Pascal's Identity}:
$$ \binom{n}{k} = \binom{n - 1}{k - 1} + \binom{n - 1}{k},$$for
integers $1 \leq k \leq n.$ \begin{answer} We have $$
{\everymath{\displaystyle}\begin{array}{lcl}\binom{n - 1}{k - 1} +
\binom{n - 1}{k} & = &
\frac{(n - 1)!}{(k - 1)!(n - k)!} + \frac{(n - 1)!}{k!(n - k - 1)!} \\
& = & \frac{(n - 1)!}{(n - k - 1)!(k - 1)!}\left( \frac{1}{n - k} + \frac{1}{k}\right) \\
& = & \frac{(n - 1)!}{(n - k - 1)!(k - 1)!}\frac{n}{(n - k)k} \\
& = & \frac{n!}{(n - k)!k!} .\\
& = & \binom{n}{k}.\end{array}}$$ A combinatorial interpretation can
be given as follows.  Suppose we have a bag with $n$ red balls. The
number of ways of choosing $k$ balls is $n$. If we now paint one of
these balls blue, the number of ways of choosing $k$ balls is the
number of ways of choosing balls if we always {\em include} the blue
ball (and this can be done in $\binom{n - 1}{k - 1}$) ways, plus the
number of ways of choosing $k$ balls if we always {\em exclude} the
blue ball (and this can be done in $\binom{n - 1}{k}$ ways).
\end{answer}
\end{pro}


\begin{pro} Give a combinatorial interpretation of {\bf Newton's Identity}: \begin{equation} \binom{n}{r} \binom{r}{k} = \binom{n}{k}
\binom{n - k}{r - k}\end{equation} for $0 \leq k \leq r \leq n$.
\begin{answer} The sinistral side counts the number of ways of selecting $r$
elements from a set of $n$, then selecting $k$ elements from those
$r$.  The dextral side counts how many ways to select the $k$
elements first, then select the remaining $r - k$ elements to be
chosen from the remaining $n - k$ elements.
\end{answer}
\end{pro}
\begin{pro} Give a combinatorial proof that for integer $n \geq 1,$ $$ \binom{2n}{n} =
\sum _{k = 0} ^n \binom{n}{k} ^2 .$$ \begin{answer} The dextral side
sums
$$\binom{n}{0} \binom{n}{0} + \binom{n}{1} \binom{n}{1} +
\binom{n}{2} \binom{n}{2} + \cdots + \binom{n}{n} \binom{n}{n}.$$ By
the symmetry identity, this is equivalent to summing $$\binom{n}{0}
\binom{n}{n} + \binom{n}{1} \binom{n}{n - 1} + \binom{n}{2}
\binom{n}{n - 2} + \cdots + \binom{n}{n} \binom{n}{0}.$$Now consider
a bag with $2n$ balls, $n$ of them red and $n$ of them blue. The
above sum is counting the number of ways of choosing 0 red balls and
$n$ blue balls, 1 red ball and $n - 1$ blue balls, 2 red balls and
$n - 2$ blue balls, etc.. This is clearly the number of ways of
choosing $n$ balls of either colour from the bag, which is
${\binom{2n}{n}}$.
\end{answer}
\end{pro}
   \begin{pro}
In each of the $6$-digit numbers
$$333333, 225522, 118818, 707099,$$each digit in the number
appears at least twice. Find the number of such $6$-digit natural
numbers.
\begin{answer}$11754$
\end{answer}
 \end{pro}
  \begin{pro}
In each of the $7$-digit numbers $$1001011, 5550000, 3838383,
7777777,$$each digit in the number appears at least thrice. Find the
number of such $7$-digit natural numbers.
\begin{answer} $2844$
\end{answer}
 \end{pro}
  \begin{pro}[AIME 1983]  The numbers $1447, 1005$ and $1231$ have
something in common: each is a four-digit number beginning with $1$
that has exactly two identical digits. How many such numbers are
there?
\begin{answer} $432$
\end{answer}
   \end{pro}
       \begin{pro} If there are fifteen players on a baseball team, how many ways
can the coach choose nine players for the starting lineup if it does
not matter which position the players play (i.e., no distinction is
made between player A playing shortstop, left field, or any other
positions as long as he is on the field)?  How many ways are there
if it does matter which position the players play?\begin{answer}
$\binom{15}{9}; 15!/6!$
\end{answer}
     \end{pro}
   \begin{pro}[AHSME 1989] A child has a set of $96$ distinct
blocks. Each block is one of two materials ({\em plastic, wood}),
three sizes ({\em small, medium, large}), four colours ({\em blue,
green, red, yellow}), and four shapes ({\em circle, hexagon, square,
triangle}). How many blocks in the set are different from the ``{\em
plastic medium red circle}'' in exactly two ways? (The ``{\em wood
medium red square}'' is such a block.) \begin{answer} 29.
\end{answer}
  \end{pro}
       \begin{pro}[AHSME 1989] Suppose that $k$ boys and $n - k$ girls line
up in a row. Let $S$ be the number of places in the row where a boy
and a girl are standing next to each other. For example, for the row
$$GBBGGGBGBGGGBGBGGBGG,$$ with $k = 7, n = 20$ we have $S = 12.$
Shew that the average value of $S$ is $\frac{2k(n - k)}{n}$.
      \end{pro}
           \begin{pro} There are four
different kinds of sweets at a sweets store.  I want to buy up to
four sweets (I'm not sure if I want none, one, two, three, or four
sweets) and I refuse to buy more than one of any kind of sweet. How
many ways can I do this?\begin{answer} $2^4$
\end{answer}
   \end{pro}
          \begin{pro} Suppose five people are in a lift. There are eight floors
that the lift stops at. How many distinct ways can the people exit
the lift if either one or zero people exit at each
stop?\begin{answer} $\binom{8}{5}5!$



\end{answer}
      \end{pro}
            \begin{pro} If the natural numbers from $1$ to $222222222$ are written
down in succession, how many $0$'s are written?\begin{answer}
$175308642$

\end{answer}
    \end{pro}
           \begin{pro} In how many ways can we distribute $k$ identical balls into $n$
different boxes so that each box contains at most one ball and no
two consecutive boxes are empty? \begin{answer}Hint: There are $k$
occupied boxes and $n - k$ empty boxes. Align the balls first!
$\binom{k + 1}{n - k}$.
\end{answer}
      \end{pro}
              \begin{pro} In a row of $n$ seats in the doctor's waiting-room $k$ patients
sit down in a particular order from left to right. They sit so that
no two of them are in adjacent seats.
In how many ways could a suitable set of $k$ seats be chosen? \\
\begin{answer} There are $n - k$ empty seats. Sit the people in between
those seats. $\binom{n  - k + 1}{k}$. \end{answer}
\end{pro}
\end{multicols}

\section{Combinations with Repetitions}
\begin{thm}[De Moivre]
Let $n$ be a positive integer. The number of positive integer
solutions to
$$x_1 + x_2 + \cdots + x_r = n$$is
$$\binom{n - 1}{r - 1}.$$
\label{thm:distributions}\end{thm}
\begin{pf}
Write $n$ as
$$n = 1 + 1 + \cdots + 1 + 1,$$where there are $n$ 1s and $n - 1$
$+$s. To decompose $n$ in $r$ summands we only need to choose $r -
1$ pluses from the $n - 1$, which proves the theorem.
\end{pf}
\begin{exa}
In how many ways may we write the number $9$ as the sum of three
positive integer summands? Here order counts, so, for example, $1 +
7 + 1$ is to be regarded different from $7 + 1 + 1$.
\end{exa}
Solution: Notice that this is example \ref{exa:summingto9}. We are
seeking integral solutions to $$ a + b + c = 9,  \ \ \ a>0, b>0,
c>0.
$$ By Theorem \ref{thm:distributions} this is $$\binom{9-1}{3-1} = \binom{8}{2} =  28.   $$

\begin{exa}
In how many ways can $100$ be written as the sum of four positive
integer summands?
\end{exa}
Solution: We want the number of positive integer solutions to
$$a + b + c + d = 100,$$ which by Theorem \ref{thm:distributions}
is $$\binom{99}{3} = 156849.$$


\begin{cor}
Let $n$ be a positive integer. The number of non-negative integer
solutions to
$$y_1 + y_2 + \cdots + y_r = n$$is
$$\binom{n + r - 1}{r - 1}.$$
\end{cor}
\begin{pf}
Put $x_r - 1 = y_r$. Then $x_r \geq 1$. The equation
$$x_1 - 1 + x_2 - 1 + \cdots + x_r - 1 = n$$is equivalent to
$$x_1 + x_2 + \cdots + x_r = n + r,$$which from Theorem
\ref{thm:distributions}, has $$\binom{n + r - 1}{r - 1}$$solutions.
\end{pf}

\begin{exa}Find the number of quadruples $(a, b, c, d)$ of
integers satisfying
$$a + b + c + d  = 100, \ a \geq 30, b > 21, c \geq 1, d \geq 1.$$\end{exa}
Solution: Put $a' + 29 = a, b' +  20 = b.$ Then we want the number
of positive integer solutions to
$$a' + 29 + b' + 21 + c + d = 100,$$ or
$$a' + b' + c + d = 50.$$
By Theorem \ref{thm:distributions} this number is
$$\binom{49}{3} = 18424.$$
\begin{exa}
There are five people in a lift of a building having eight floors.
In how many ways can they choose their floor for exiting the lift?
\end{exa}
Solution: Let $x_i$ be the number of people that floor $i$ receives.
We are looking for non-negative solutions of the equation $$x_1 +
x_2 + \cdots + x_8 = 5.   $$ Putting $y_i = x_i + 1$, then
 $$\begin{array}{lll}x_1 + x_2 + \cdots + x_8 = 5  & \implies &    (y_1-1) + (y_2-1) + \cdots + (y_8-1) = 5\\ & \implies &  y_1 + y_2 + \cdots + y_8 = 13, \end{array}  $$
whence the number sought is the number of positive solutions to
$$ y_1 + y_2 + \cdots + y_8 = 13        $$ which is $\binom{12}{7} = 792.$


\begin{exa}Find the
number of quadruples $(a, b, c, d)$ of non-negative integers which
satisfy the inequality
$$a + b + c + d \leq 2001.$$\end{exa}
Solution: The number of non-negative solutions to $$a + b + c + d
\leq 2001$$ equals the number of solutions to $$a + b + c + d + f =
2001$$where $f$ is a non-negative integer. This number is the same
as the number of positive integer solutions to $$a_1 - 1 + b_1 - 1 +
c_1 - 1 + d_1 - 1 + f_1 - 1 =  2001,$$ which is easily seen to be
$\binom{2005}{4}$.
\begin{exa}\item How many integral solutions to the equation
$$a + b + c + d = 100,$$are there given the following
constraints:
$$1 \leq a \leq 10,\ b \geq 0, \ c \geq 2, 20 \leq d \leq 30 ?$$
\end{exa}Solution: We use Inclusion-Exclusion.
There are $\binom{80}{3} =82160$ integral solutions to $$a + b + c +
d = 100, \ \ a \geq 1, b \geq 0, c \geq 2, d \geq 20.$$ Let $A$ be
the set of solutions with $$ a \geq 11, b \geq 0, c \geq 2, d \geq
20$$ and $B$ be the set of solutions with
$$ a \geq 1, b \geq 0, c \geq 2, d \geq
31.$$ Then $\card{A} = \binom{70}{3}$, $\card{B} = \binom{69}{3}$,
$\card{A \cap B} =  \binom{59}{3}$ and so
$$\card{A \cup B} = \binom{70}{3} + \binom{69}{3} - \binom{59}{3} = 74625.$$
The total number of solutions to
$$a + b + c + d = 100$$with
$$1 \leq a \leq 10,\ b \geq 0, \ c \geq 2, 20 \leq d \leq 30 $$
is thus
$$\binom{80}{3} - \binom{70}{3} - \binom{69}{3} + \binom{59}{3} = 7535.$$

\section*{Homework}\addcontentsline{toc}{section}{Homework}\markright{Homework}
\begin{multicols}{2}\columnseprule 1pt \columnsep 25pt\multicoltolerance=900
\begin{pro}
How many positive integral solutions are there to $$a + b + c =
10?$$
\begin{answer} $36$
\end{answer}
  \end{pro}
   \begin{pro}
Three fair dice, one red, one white, and one blue are thrown. In how
many ways can they land so that their sum be $10$ ?
\begin{answer} $36 - 9 = 25$
\end{answer}
  \end{pro}
   \begin{pro} Adena has twenty indistinguishable pieces of sweet-meats that
she wants to divide amongst her five stepchildren.  How many ways
can she divide the sweet-meats so that each stepchild gets at least
two pieces of sweet-meats?\begin{answer} $\binom{14}{4}$

\end{answer}
    \end{pro}
          \begin{pro}
How many integral solutions are there to the equation $$x_1 + x_2 +
\cdots + x_{100} = n
$$ subject to the constraints $$x_1 \geq 1, x_2 \geq 2, x_3 \geq 3, \ldots , x_{99} \geq 99, x_{100} \geq
100?$$

    \end{pro}
         \begin{pro}[AIME 1998] Find the number of ordered quadruplets $(a, b, c,
d)$ of positive odd integers satisfying $a+b+c+d=98$.
\begin{answer}$\binom{50}{3} = 19600$
\end{answer}
\end{pro}
\end{multicols}













\section{The Binomial Theorem} We recall that the
symbol $$\binom{n}{k} = \frac{n!}{(n - k)!k!}, n, k \in \BBN , 0
\leq k \leq n,$$ counts the number of ways of selecting $k$
different objects from $n$ different objects. Observe that we have
the following {\em absorbtion identity: }
$$\binom{n}{k} = \frac{n}{k}\binom{n - 1}{k - 1}.$$





\begin{exa} Prove {\em Pascal's Identity}:
$$ \binom{n}{k} = \binom{n - 1}{k - 1} + \binom{n - 1}{k},$$for
integers $1 \leq k \leq n.$ \end{exa} Solution: We have $$
{\everymath{\displaystyle}\begin{array}{lcl}\binom{n - 1}{k - 1} +
\binom{n - 1}{k} & = &
\frac{(n - 1)!}{(k - 1)!(n - k)!} + \frac{(n - 1)!}{k!(n - k - 1)!} \\
& = & \frac{(n - 1)!}{(n - k - 1)!(k - 1)!}\left( \frac{1}{n - k} + \frac{1}{k}\right) \\
& = & \frac{(n - 1)!}{(n - k - 1)!(k - 1)!}\frac{n}{(n - k)k} \\
& = & \frac{n!}{(n - k)!k!} .\\
& = & \binom{n}{k}\end{array}}$$
\begin{exa} Prove {\em Newton's Identity}:
$$ \binom{n}{i}\binom{i}{j} = \binom{n}{j}\binom{n - j}{j - i},$$ for integers
$0 \leq j \leq i \leq n$.\end{exa} Solution: We have
$$ \binom{n}{i}\binom{i}{j} = \frac{n!i!}{i!(n - i)!j!(i - j)!} = \frac{n!(n - j)!}{(n - j)!j!(n - i)!(i - j)!} $$
which is the same as$$ \binom{n}{j}\binom{n - j}{i - j}.$$



 Using Pascal's Identity we obtain
{\em Pascal's Triangle.} \renewcommand{\arraystretch}{1.2}
$$\begin{array}{ccccccccccc}
 & & & & & \binom{0}{0} & & & & & \\
 & & & &  \binom{1}{0} & &  \binom{1}{1} & & & &  \\
 & & &  \binom{2}{0} & &  \binom{2}{1} & &  \binom{2}{2} & & &  \\
 & &  \binom{3}{0} & &  \binom{3}{1} & &  \binom{3}{2} & &  \binom{3}{3} & & \\
 &  \binom{4}{0} & &  \binom{4}{1} & &  \binom{4}{2} & &  \binom{4}{3} & &  \binom{4}{4} &  \\
  \binom{5}{0} & &  \binom{5}{1} & &  \binom{5}{2} & &  \binom{5}{3} & &  \binom{5}{4} & &  \binom{5}{5}  \\
\end{array}$$




When the numerical values are substituted, the triangle then looks
like this.
$$
\begin{array}{ccccccccccc}
 & & & & & 1 & & & & &  \\
 & & & & 1 & & 1 & & & &  \\
 & & & 1 & & 2 & & 1 & & & \\
 & & 1 & & 3 & & 3 & & 1 & &  \\
 & 1 & & 4 & & 6 & & 4 & & 1 &  \\
 1 & & 5 & & 10 & & 10 & & 5 & & 1  \\
\end{array}$$





We see from Pascal's Triangle that binomial coefficients are
symmetric.  This symmetry is easily justified by the identity
$\binom{n}{k} = \binom{n}{n - k}$.  We also notice that the binomial
coefficients tend to increase until they reach the middle, and that
they decrease symmetrically. That is, the $\binom{n}{k}$ satisfy
$$\binom{n}{0} < \binom{n}{1} < \cdots < \binom{n}{[n/2] - 1} <
\binom{n}{[n/2]} > \binom{n}{[n/2] + 1} > \binom{n}{[n/2] + 2} >
\cdots > \binom{n}{n - 1} > \binom{n}{n}$$ if $n$ is even, and that
$$\binom{n}{0} < \binom{n}{1} < \cdots <
 \binom{n}{[n/2] - 1} < \binom{n}{[n/2]} = \binom{n}{[n/2] + 1} >
\binom{n}{[n/2] + 2} > \binom{n}{[n /2] + 3} > \cdots > \binom{n}{n
- 1} > \binom{n}{n}$$ for odd $n$. We call this property the {\em
unimodality} of the binomial coefficients. For example, without
finding the exact numerical values we can see that $\binom{200}{17}
< \binom{200}{69}$ and that $\binom{200}{131} = \binom{200}{69} <
\binom{200}{99}$.


We now present some examples on the use of binomial coefficients.

\begin{exa} The {\em Catalan number of order n} is defined as $$
C_n = \frac{1}{n + 1}\binom{2n}{n}.$$ Prove that $C_n$ is an
integer for all natural numbers $n$. \end{exa} Solution: Observe
that $$ \frac{1}{n + 1}\binom{2n}{n} = \binom{2n}{n} -
\binom{2n}{n - 1},$$ the difference of two integers.
\begin{exa}[Putnam 1972] Shew that no four consecutive binomial
coefficients
$$ \binom{n}{r}, \binom{n}{r + 1}, \binom{n}{r + 2}, \binom{n}{r + 3}$$
($n, r$ positive integers and $r + 3 \leq n$) are in arithmetic
progression. \end{exa} Solution: Assume that $a = \binom{n}{r}, a
+ d = \binom{n}{r + 1}, a + 2d = \binom{n}{r + 2}, a + 3d =
\binom{n}{r + 3}$. This yields
$$ 2\binom{n}{r + 1} = \binom{n}{r} + \binom{n}{r + 2},$$or equivalently
$$ 2 = \frac{r + 1}{n - r} + \frac{n - r - 1}{r + 2}\hspace{10mm} (*).$$
This is a quadratic equation in $r$, having $r$ as one of its
roots. The condition that the binomial coefficients are in
arithmetic progression means that $r + 1$ is also a root of $(*)$.
Replacing $r$ by $n - r - 2$ we also obtain $$ 2 = \frac{n - r -
1}{r + 2} + \frac{r + 1}{n - r},$$which is the same as $(*).$ This
means that $n - r - 3$ and $n - r - 2$ are also roots of $(*)$.
Since a quadratic equation can only have two roots, we must have
$r = n - r - 3.$ The four binomial coefficients must then be
$$ \binom{2r + 3}{r}, \ \binom{2r + 3}{r + 1}, \ \binom{2r + 3}{r + 2}, \
\binom{2r + 3}{r + 3}.$$But these cannot be in an arithmetic
progression, since binomial coefficients are unimodal and
symmetric.
\begin{exa} Let $N(a)$ denote the number of solutions to the equation $a =
\binom{n}{k}$ for nonnegative integers $n, k$. For example, $N(1)
= \infty , N(3) = N(5) = 2, N(6) = 3,$ etc. Prove that $N(a) \leq
2 + 2\log _2 a.$ \end{exa} Solution: Let $b$ be the first time
that $\binom{2b}{b} > a$. By the unimodality of the binomial
coefficients, $\binom{i + j}{i} = \binom{i + j}{j}$ is
monotonically increasing in $i$ and $j$. Hence
$$ \binom{b + i + b + j}{b + j} \geq \binom{b + b + j}{b} \geq \binom{2b}{b}
> a$$for all $i, j \geq 0.$ Hence $\binom{i + j}{j} = a$ implies $i < b,$
or $j < b.$ Also, for each fixed value of $i$ (or $j$), $\binom{i
+ j}{i} = a$ has at most one solution. It follows that $N(a) <
2b.$ Since
$$ a \geq \binom{2(b - 1)}{b - 1} \geq 2^{b - 1},$$ it follows that $b
\leq \log _2 a + 1,$ and the statement is proven.





We now use Pascal's Triangle in order to expand the binomial $$(a
+ b)^n.$$



The {\em Binomial Theorem} states that for $n \in\BBZ, n \geq 0,$
$$(1 + x)^n = \sum _{k = 0} ^n \binom{n}{k}x^k .$$As a way of proving this, we observe that expanding
$$\underbrace{(1 + x)(1 + x) \cdots (1 + x)}_{\scriptsize{{\it n} \ {\rm factors}}}$$
consists of adding up all the terms obtained from multiplying
either a $1$ or a $x$ from the first set of parentheses times
either a $1$ or an $x$ from the second set of parentheses etc.  To
get $x^k$, $x$ must be chosen from exactly $k$ of the sets of
parentheses.  Thus the number of $x^k$ terms is $\binom{n}{k}$. It
follows that
$$(1 + x)^n = \binom{n}{0} + \binom{n}{1}x + \binom{n}{2}x^2 + \cdots + \binom{n}{n}x^n =
\sum _{k = 0} ^n \binom{n}{k}x^k.$$
\begin{exa} Prove that $$ \sum _{k = 0} ^n \binom{n}{k} = 2^n .$$\end{exa}
Solution: This follows from letting $x = 1$ in the expansion
$$ (1 + x)^n = \sum _{k = 0} ^n \binom{n}{k}x^k .$$
\begin{exa} Prove that for integer $n \geq 1,$
$$ \sum _{j = i} ^n \binom{n}{j}\binom{j}{i} = \binom{n}{i}2^{n - i},
\,\, i \leq n.$$ \end{exa} Solution: Recall that by Newton's
Identity$$\binom{n}{j}\binom{j}{i} = \binom{n}{i}\binom{n - i}{j -
i}.$$Thus $$\sum _{j = 0} ^n \binom{n}{j}\binom{j}{i} =
\binom{n}{i}\sum _{j = 0} ^n \binom{n - i}{j - i}.$$But upon
re-indexing$$ \sum _{j = 0} ^n \binom{n - i}{j - i} = \sum _{j =
0} ^{n - i} \binom{n - i}{j} =  2^{n - i},$$by the preceding
problem. Thus the assertion follows.


\begin{exa} Prove that
$$\sum _{k \leq n} \binom{m + k}{k} = \binom{n + m + 1}{n}.$$ \end{exa}
Solution: Using Pascal's Identity
$$ {\everymath{\displaystyle}\begin{array}{lcl}
\sum _{k = 0} ^n \binom{k + m}{k} & = &
\binom{0 + m }{-1} + \binom{0 + m}{0} + \binom{1 + m}{1} + \binom{2 + m}{2}  + \binom{3 + m}{3} \\
& & \quad + \cdots + \binom{n - 1 + m}{n - 1} + \binom{n + m}{n} \\
& = & \binom{1 + m}{0} + \binom{1 + m}{1} + \binom{2 + m}{2}  + \binom{3 + m}{3} \\
& & \quad + \cdots + \binom{n - 1 + m}{n - 1} + \binom{n + m}{n} \\
& = & \binom{2 + m}{1} + \binom{2 + m}{2}  + \binom{3 + m}{3} \\
& & \quad + \cdots + \binom{n - 1 + m}{n - 1} + \binom{n + m}{n} \\
& = & \binom{3 + m}{2} + \binom{3 + m}{3} \\
& & \quad + \cdots + \binom{n - 1 + m}{n - 1} + \binom{n + m}{n} \\
& \vdots & \\
& = & \binom{n + m}{n - 1}  + \binom{n + m}{n} \\
& = & \binom{n + m + 1}{n},
\end{array}}$$which is what we wanted.
\begin{exa} Find a closed formula for $$ \sum _{0 \leq k \leq m}
\binom{m}{k}\binom{n}{k} ^{-1}\,\, n \geq m \geq 0. $$ \end{exa}
Solution: Using Newton's Identity,
$$   \sum _{0 \leq k \leq m}
\binom{m}{k}\binom{n}{k} ^{-1}   =  \binom{n}{m}^{-1}\sum  _{0
\leq k \leq m} \binom{n - k}{m - k} .$$ Re-indexing,
$$\sum  _{0 \leq k \leq m} \binom{n - k}{m - k}  = \sum _{k \leq m} \binom{n - m + k}{k} = \binom{n + 1}{m},$$
by the preceding problem. Thus
$$\sum _{0 \leq k \leq m}
\binom{m}{k}\binom{n}{k} ^{-1} = \binom{n + 1}{m}/\binom{n}{m} =
\frac{n + 1}{n + 1 - m}.$$
\begin{exa} Simplify $$ \sum _{0 \leq k
\leq 50} \binom{100}{2k}.$$ \end{exa} Solution: By the Binomial
Theorem
$$ {\everymath{\displaystyle}\begin{array}{lcl}(1 + 1)^{100} & = & \binom{100}{0} + \binom{100}{1} + \binom{100}{2} +  \ldots + \binom{100}{99} + \binom{100}{100} \\
(1 - 1)^{100} & = & \binom{100}{0} - \binom{100}{1} +
\binom{100}{2} -  \ldots - \binom{100}{99} + \binom{100}{100},
\end{array}}$$whence summing both columns
$$ 2^{100} = 2\binom{100}{0} + 2\binom{100}{2} +  \ldots + 2\binom{100}{100}.$$
Dividing by 2, the required sum is thus $2^{99}.$
\begin{exa}Simplify $$\sum _{k = 1} ^{50} \binom{100}{2k - 1}.$$ \end{exa}
Solution: We know that $$\sum _{k = 0} ^{100} \binom{100}{k} =
2^{100}$$ and $$\sum _{k = 0} ^{50} \binom{100}{2k} = 2^{99}.$$The
desired sum is the difference of these two values $2^{100} -
2^{99} = 2^{99}.$
\begin{exa} Simplify$$\sum_ {k = 1}^{10} 2^k \binom{11}{k}.$$ \end{exa}
Solution: By the Binomial Theorem, the complete sum $\sum _{k = 0}
^{11} \binom{11}{k}2^k = 3^{11}$. The required sum lacks the
zeroth term, $\binom{11}{0}2^0 = 1$, and the eleventh term,
$\binom{11}{11}2^{11}$ from this complete sum. The required sum is
thus $3^{11} - 2^{11} - 1$.
\begin{exa} Which coefficient of the expansion of $$ \left(\frac{1}{3} +
\frac{2}{3}x\right)^{10}$$ has the greatest magnitude? \end{exa}
Solution: By the Binomial Theorem,
$$\left(\frac{1}{3} + \frac{2}{3}x\right)^{10} = \sum _{k = 0} ^{10} \binom{10}{k}(1/3)^{k}(2x/3)^{10 - k} =  \sum _{k = 0} ^{10} a_k x^k .$$
We consider the ratios $\frac{a_k}{a_{k - 1}}, k = 1, 2, \ldots
n$. This ratio is seen to be $$ \frac{a_k}{a_{k - 1}} = \frac{2(10
- k + 1)}{k}.$$ This will be $< 1$ if $k < 22/3 < 8.$ Thus $a_0 <
a_1 < a_2 < \ldots < a_7$. If $k > 22/3$, the ratio above will be
$< 1.$ Thus $a_7 > a_8 > a_9 > a_{10}$. The largest term is that
of $k = 7$, i.e. the eighth term.
\begin{exa} At what positive integral value of $x$ is the $x^4$ term in the
expansion of $(2x + 9)^{10}$ greater than the adjacent
terms?\end{exa} Solution: We want to find integral $x$ such that
$$ \binom{10}{4}(2x)^4 (9)^6 \geq \binom{10}{3}(2x)^3 (9)^7,$$ and
$$ \binom{10}{4}(2x)^4 (9)^6  \geq \binom{10}{5}(2x)^5 (9)^5
.$$After simplifying the factorials, the two inequalities sought
are $$ x \geq 18/7 $$ and $$ 15/4 \geq x.$$ The only integral $x$
that satisfies this is $x = 3.$
\begin{exa} Prove that for integer $n \geq 1,$
$$ \sum _{k = 1} ^n k\binom{n}{k} = n2^{n - 1}.$$ \end{exa}
Solution:  Using the absorption identity $$   \sum _{k = 0} ^{n}
n\binom{n - 1}{k - 1} = \sum _{k = 0} ^n k\binom{n}{k},
$$with the convention that $\binom{n - 1}{-1} = 0$. But since
$$\sum _{k = 0} ^{n} \binom{n - 1}{k - 1} = \sum _{k = 0} ^{n - 1}
\binom{n - 1}{k} = 2^{n - 1},$$ we obtain the result once again.

\begin{exa} Find a closed formula for $$ \sum _{k = 0} ^n \frac{1}{k +
1}\binom{n}{k}.$$\end{exa} Solution:  Using the absorption
identity
$$ \sum _{k = 0} ^{n}\frac{1}{k + 1} \binom{n}{k} = \frac{1}{n + 1}\sum _{k = 0} ^n  \binom{n + 1}{k + 1} =
\frac{1}{n + 1}(2^{n + 1} - 1).$$

\begin{exa}Prove that if m, n are nonnegative integers then $$ \binom{n +
1}{m + 1} = \sum _{k = m} ^n \binom{k}{m}.$$ \end{exa} Solution:
Using Pascal's Identity
$$ {\everymath{\displaystyle}\begin{array}{lcl}
\sum _{k = m} ^n \binom{k}{m}  & = & \binom{m}{m + 1} + \binom{m}{m} + \binom{m + 1}{m} + \cdots + \binom{n}{m} \\
& = & \binom{m + 1}{m + 1} + \binom{m + 1}{m} + \binom{m + 2}{m} +
\cdots +
\binom{n}{m} \\
& = & \binom{m + 2}{m + 1} + \binom{m + 2}{m} + \binom{m + 3}{m} +
\cdots
+ \binom{n}{m} \\
& \vdots & \\
& = & \binom{n}{m + 1} + \binom{n}{m} \\
& = & \binom{n + 1}{m + 1}.\end{array}}$$
\begin{exa} Find a closed form for $$ \sum _{k \leq n} k(k + 1).$$ \end{exa}
Solution: Let $$ S = \sum _{k \leq n} k(k + 1).$$ Then
$$ S/2! = \sum _{k \leq n} \frac{k(k + 1)}{2!} = \sum _{k \leq n} \binom{k + 1}{2}.$$
By the preceding problem
$$ \sum _{k \leq n} \binom{k + 1}{2} = \binom{n + 2}{3}.$$We gather that $S = 2\binom{n + 2}{3}
= n(n + 1)(n + 2)/3$.

\section*{Practice}\addcontentsline{toc}{section}{Practice}\markright{Practice}\begin{multicols}{2}\columnseprule 1pt \columnsep 25pt\multicoltolerance=900

\begin{pro} Prove that $$ \sum _{0 \leq k \leq n/2} \binom{n}{2k + 1} =
2^{n - 1}.$$\end{pro}
\begin{pro} Expand $$ \frac{1}{2}(1 + \sqrt{x})^{100} + \frac{1}{2}(1 - \sqrt{x})^{100}.$$\end{pro}

\begin{pro}

Four writers must write a book containing seventeen chapters. The
first and third writers must each write five chapters, the second
must write four chapters, and the fourth writer must write three
chapters. How many ways can the book be written? What if the first
and third writers had to write ten chapters combined, but it did
not matter which of them wrote how many (e.g., the first could
write ten and the third none, the first could write none and the
third one, etc.)?
\begin{answer}
$\binom{17}{5}\binom{12}{5}\binom{7}{4}\binom{3}{3};
\binom{17}{3}\binom{14}{4}2^{10}$
\end{answer}
\end{pro}

\begin{pro} Prove that
$$ \sum _{j_n = 1} ^m \sum _{j_{n - 1} = 1} ^{j_n} \cdots \sum _{k = 1} ^{j_{1}} \ 1 = \binom{n + m}{n + 1}.$$\end{pro}
\begin{pro} The expansion of $(x + 2y)^{20}$ contains two terms with
the same coefficient, $Kx^ay^b$ and $Kx^{a + 1}y^{b - 1}$. Find
a.\end{pro}
\begin{pro} Prove that for $n \in \BBN , n > 1$ the identity
$$ \sum _{k = 1} ^n (-1)^{k - 1}k\binom{n}{k} = 0$$holds true.\end{pro}
\begin{pro} If $n$ is an even natural number, shew that
$$\begin{array}{l}
\frac{1}{1!(n - 1)!} + \frac{1}{3!(n - 3)!} \\ \quad + \frac{1}{5!(n
- 5)!} + \cdots + \frac{1}{(n - 1)!1!}\\ \qquad = \frac{2^{n -
1}}{n!}.\end{array}$$\end{pro}
\begin{pro} Find a closed formula for $$\sum _{0 \leq k \leq n} \, \binom{n - k}{k} (-1)^k .$$
\end{pro}
\begin{pro} What is the exact numerical value of $$ \sum _{k \leq 100}k3^k\binom{100}{k}?$$\end{pro}
\begin{pro} Find a closed formula for $$ \sum _{k = 1} ^n k^2 - k.$$ \end{pro}
\begin{pro} Find a closed formula for
$$ \sum _{0 \leq k \leq n} k \binom{m - k - 1}{n - k - 1}\,\,\, m > n
\geq 0.$$
\begin{answer}
Write $k = m - (m - k)$. Use the absorption identity to evaluate
$$ \sum _{k = 0} ^n (m - k)\binom{m - k - 1}{n - k - 1}.)$$
\end{answer}
\end{pro}
\begin{pro} What is the exact numerical value of
$$ \sum _{k \leq 100} \frac{5^k}{k + 1}\binom{100}{k}?$$\end{pro}
\begin{pro} Find $n$ if $\binom{10}{4} + \binom{10}{3} = \binom{n}{4}.$
\begin{answer} $11$ \end{answer}
\end{pro}
\begin{pro} If
$$\binom{1991}{1} + \binom{1991}{3} + \binom{1991}{5} + \cdots + \binom{1991}{1991} = 2^a,$$
find $a.$
\begin{answer} $a=1990$ \end{answer}
\end{pro}
\begin{pro} True or False:
$\binom{20}{5} = \binom{20}{15}.$
\begin{answer} True. \end{answer}
\end{pro}
\begin{pro} True or False:
$$49\binom{48}{9} = 10\binom{49}{10}.$$
\begin{answer} True. \end{answer}
\end{pro}
\begin{pro} What is the coefficient of $x^{24} y^{24}$ in the expansion
$$(2x^3 + 3y^2)^{20}? $$
\begin{answer}  $\binom{20}{8}(2^8)(3^{12})$ \end{answer}
\end{pro}
\begin{pro} What is the coefficient of $x^{12} y^7$ in the expansion
$$(x^{3/2} + y)^{15}?$$
 \begin{answer} $\binom{15}{8}$ \end{answer}

\end{pro}
\begin{pro} What is the coefficient of $x^4 y^6$ in $$(x\sqrt{2} -
y)^{10}?$$
\begin{answer} $840$
  \end{answer}
\end{pro}
\begin{pro} Shew that the binomial coefficients satisfy the following
hexagonal property: \end{pro}
$$\begin{array}{l}\binom{n - 1}{k - 1}\binom{n}{k + 1}\binom{n + 1}{k}\\ \qquad = \binom{n -
1}{k}\binom{n + 1}{k + 1}\binom{n}{k - 1}. \end{array}$$
\begin{pro}[AIME 1991] In the expansion $$(1 + 0.2)^{1000} =
\displaystyle{\sum_{k = 0} ^{1000} \binom{1000}{k} (0.2)^k},$$
which one of the $1001$ terms is the largest?
\begin{answer} The $166$-th \end{answer}
 \end{pro}

\begin{pro}[Putnam 1971] Shew that for $0 < \epsilon < 1$ the
expression $$ (x + y)^n(x^2 - (2 - \epsilon )xy + y^2)$$is a
polynomial with positive coefficients for integral $n$ sufficiently
large. For $\epsilon = .002$ find the smallest admissible value of
$n$. \end{pro}
\begin{pro} Prove that for integer $n \geq 1,$
$$ \sum _{k = 1} ^n k^3 \binom{n}{k} = n^2 (n + 3)2^{n - 3}.$$ \end{pro}
\begin{pro} Expand and simplify $$ (\sqrt{1 - x^2} + 1)^7 -
(\sqrt{1 - x^2} - 1)^7.$$\end{pro}
\begin{pro} Simplify$$\binom{5}{5} + \binom{6}{5} + \binom{7}{5} + \cdots +
\binom{999}{5}$$
\begin{answer} $\binom{1000}{6}   $ \end{answer}
\end{pro}
\begin{pro} Simplify$$ \binom{15}{1} - \binom{15}{2} + \binom{15}{3} - \binom{15}{4}
\cdots + \binom{15}{13} - \binom{15}{14}            $$
\begin{answer}
$0$, as $\binom{15}{1} = \binom{15}{14}$, $\binom{15}{2} =
\binom{15}{13}$, etc.
\end{answer}
\end{pro}
 \begin{pro} What is the exact numerical value
of
$$\sum _{k = 0} ^{1994} (-1)^{k - 1} \binom{1994}{k} \ ?$$
\begin{answer}
$0$
\end{answer}
\end{pro}
\begin{pro}True or False:
$$\binom{4}{4} + \binom{5}{4} + \cdots + \binom{199}{4} > \binom{16}{16} + \binom{17}{16} + \cdots + \binom{199}{16}.$$
\begin{answer} False. Sinistral side = $\binom{200}{5}$, dextral side =
$\binom{200}{17}$
\end{answer}
\end{pro}
\begin{pro}[AIME 1992] In which row of Pascal's triangle (we start
with zeroth row, first row ,etc.) do three consecutive entries
occur that are in the ratio $3:4:5$?
\begin{answer}
The $62$-nd.
\end{answer}
\end{pro}
\end{multicols}
\section{Multinomial Theorem} If $n, n_1 , n_2 ,
\ldots , n_k$ are nonnegative integers and $n = n_1 + n_2 + \cdots
n_k$ we put
$$ \binom{n}{n_1 , n_2  \cdots n_k } = \frac{n!}{n_1 ! n_2 ! \cdots n_k !}.$$Using the De-Polignac Legendre Theorem, it
is easy to see  that this quantity is an integer. Proceeding in
the same way we proved the Binomial Theorem, we may establish the
{\em Multinomial Theorem:}
$$ (x_1 + x_2 + \cdots + x_k )^n = \sum _{\stackrel{n_1 + n_2 + \cdots + n_k = n}{n_1 , n_ 2 , \ldots , n_k \geq 0}} x_1 ^{n_1}x_2 ^{n_2} \cdots x_k ^{n_k}.$$



We give a few examples on the use of the Multinomial Theorem.



\begin{exa} Determine the coefficient of $x^2 y^3 z^3$ in $$(x + 2y +
z)^8$$.\end{exa} Solution:  By the Multinomial Theorem  $$ (x + 2y
+ z)^{8} = \sum _{\stackrel{n_1, n_2, n_3 \geq 0}{n_1 + n_2 + n_3
= 8}} \binom{8}{n_1, n_2, n_3} x^{n_1}(2y)^{n_2}z^{n_3}.$$This
requires $n_1 = 2, n_2 = 3, n_3 = 3.$ The coefficient sought is
then $2^3 \binom{8}{2, 3, 3}.$
\begin{exa} In $(1 + x^5 + x^9)^{23}$, find the coefficient of
$x^{23}.$\end{exa} Solution:  By the Multinomial Theorem $$ \sum
_{\stackrel{n_1, n_2, n_3 \geq 0}{n_1 + n_2 + n_3 = 23}}
\binom{23}{n_1, n_2, n_3} x^{5n_2 + 9n_3}.$$Since $5n_2 + 9n_3 =
23$ and $n_1 + n_2 + n_3 = 23,$ we must have $n_1 = 20, n_2 = 1,
n_3 = 2.$ The coefficient sought is thus $\binom{23}{20, 1, 2}.$

\begin{exa} How many different terms are there in the expansion of $$ (x +
y + z + w + s + t)^{20} ?$$ \end{exa} Solution: There as many
terms as nonnegative integral solutions of $$ n_1 + n_2 + \cdots +
n_6 = 20.$$But we know that there are $\binom{25}{5}$ of these.
\section*{Practice}\addcontentsline{toc}{section}{Practice}\markright{Practice}\begin{multicols}{2}\columnseprule 1pt \columnsep 25pt\multicoltolerance=900

\begin{pro} How many terms are in the expansion $(x + y + z)^{10}$?
\begin{answer}
$\binom{12}{2}$
\end{answer}
\end{pro}

\begin{pro} Find the coefficient of $x^4$ in the expansion of $$(1 + 3x +
2x^3)^{10} ? $$
\begin{answer} $6\binom{10}{1, 1, 8} + 3^4
\binom{10}{0, 4, 6}.$ \end{answer}
\end{pro}
\begin{pro} Find the coefficient of $x^2 y^3 z^5$ in the expansion of $$(x +
y + z)^{10}?$$ \begin{answer} $\binom{10}{2, 3, 5}$  \end{answer}

\end{pro}
\end{multicols}


\chapter{Equations}
\section{Equations in One Variable}

Let us start with the following example.
\begin{exa}
Solve the equation $2^{|x|} = \sin x^2$.

\end{exa}
Solution: Clearly $x = 0$ is not a solution. Since $2^y > 1$ for
$y > 0,$ the equation does not have a solution.

\begin{exa} Solve the equation $\dis{|x - 3|^{(x^2 - 8x + 15)/(x - 2)} = 1.}$

\end{exa}
Solution: We want either the exponent to be zero, or the base to
be 1. We cannot have, however, $0^0$ as this is undefined. So, $|x
- 3| = 1$ implies $x = 4$ or $x = 2.$ We discard $x = 2$ as the
exponent is undefined at this value. For the exponent we want $x^2
- 8x + 15 = 0$ or $x = 5$ or $x = 3.$ We cannot have $x = 3$ since
this would give $0^0$. So the only solutions are $x = 4$ and $x =
5.$
\begin{exa}  What would be the appropriate  value of $x $if

$$x^{x^{x^{.^{.^{.}}}}} = 2$$ made sense?

\end{exa}
Solution: Since $\dis{x^{x^{x^{.^{.^{.}}}}} = 2}$, we have $x^2 =
2$ (the chain is infinite, so cutting it at one step does not
change the value). Since we want a positive value we must have $x
= \sqrt{2}$.


\begin{exa} Solve $9 + x^{-4} = 10x^{-2}.$  \end{exa}
Solution: Observe that
$$x^{-4}  - 10x^{-2} + 9 = (x^{-2} - 9)(x^{-2} - 1).$$Then
$x = \pm \frac{1}{3}$ and $x = \pm 1.$
\begin{exa} Solve $9^x - 3^{x + 1} - 4 = 0.$\end{exa}
Solution: Observe that $9^x - 3^{x + 1} - 4 = (3^x - 4)(3^x + 1).$
As no real number $x$ satisfies  $3^x + 1 = 0,$ we discard this
factor. So $3^x - 4 = 0$ yields $x = \log _3 4.$
\begin{exa} Solve $$(x - 5)(x - 7)(x + 6)(x + 4) = 504.$$ \end{exa}
Solution: Reorder the factors and multiply in order to obtain
$$(x - 5)(x - 7)(x + 6)(x + 4) = (x - 5)(x + 4)(x - 7)(x + 6) = (x^2 - x - 20)(x^2 - x - 42).$$
Put $y = x^2 - x.$ Then $(y - 20)(y - 42) = 504$, which is to say,
$y^2 - 62y + 336 = (y - 6)(y - 56) = 0.$ Now, $y = 6,\  56,$
implies
$$x^2 - x = 6$$and$$x^2 - x = 56.$$Solving both quadratics, $x = -2, 4, -7, 8.$
\begin{exa} Solve $12x^4 - 56x^3 + 89x^2 - 56x + 12 = 0.$ \end{exa}
Solution: Reordering
\begin{equation}12x^4 + 12  - 56(x^3 + x) + 89x^2 = 0.\end{equation}
Dividing by $x^2,$
$$12(x^2 + \frac{1}{x^2}) - 56(x + \frac{1}{x}) + 89 = 0.$$Put $u = x + 1/x.$
Then $u^2 - 2 = x^2 + 1/x^2$. Using this, (6) becomes $12(u^2 - 2)
- 56u + 89 = 0,$ whence $u = 5/2,\ 13/6.$ From this
$$x + \frac{1}{x} = \frac{5}{2}$$ and  $$x + \frac{1}{x} = \frac{13}{6}.$$
Solving both quadratics we conclude that $x = 1/2, 2, 2/3, 3/2.$

\begin{exa} Find the real solutions to $$x^2 - 5x + 2\sqrt{x^2 - 5x + 3} = 12.$$ \end{exa}
Solution: Observe that $$x^2 - 5x + 3 + 2\sqrt{x^2 - 5x + 3} - 15
= 0.$$ Let $u = x^2 - 5x + 3$  and so $u + 2u^{1/2} - 15 =
(u^{1/2} + 5)(u^{1/2} - 3) = 0$. This means that $u = 9$ (we
discard $u^{1/2} + 5 = 0,$ why?). Therefore $x^2 - 5x + 3 = 9$ or
$x = -1,\ 6.$
\begin{exa} Solve \begin{equation}\sqrt{3x^2 - 4x + 34} - \sqrt{3x^2 - 4x - 11} = 9.\end{equation}\end{exa}
Solution: Notice the trivial identity
\begin{equation} (3x^2 - 4x + 34) - (3x^2 - 4x - 11) = 45. \end{equation}
Dividing each member of (8) by the corresponding members of (7),
we  obtain
\begin{equation}
\sqrt{3x^2 - 4x + 34} + \sqrt{3x^2 - 4x - 11} = 5.
\end{equation}
Adding (7) and (9) $$\sqrt{3x^2 - 4x + 34} = 7,$$from where $x =
-\frac{5}{3}, 3.$
\begin{exa} Solve
$$\sqrt[3]{14 + x} + \sqrt[3]{14 - x} =  4.$$\end{exa}
Solution: Let$u = \sqrt[3]{14 + x}, v = \sqrt[3]{14 - x}.$ Then
$$64 = (u + v)^3 = u^3 + v^3 + 3uv(u + v) = 14 + x + 14 - x + 12(196 - x^2)^{1/3}, $$whence
$$3 = (196 - x^2)^{1/3}, $$which upon solving yields $x = \pm 13.$
\begin{exa}  Find the exact value of $\cos 2\pi/5.$\end{exa}
Solution: Using the identity
$$\cos (u \pm v) = \cos u\cos v \mp \sin u\sin v$$twice, we obtain
\begin{equation}\cos 2\theta = 2\cos ^2\theta - 1\end{equation}and
\begin{equation}\cos 3\theta = 4\cos ^3 \theta - 3\cos \theta .\end{equation}
Let $x = \cos 2\pi /5.$ As $\cos 6\pi /5 = \cos 4\pi /5,$ thanks
to  (5) and (6), we see that $x$ satisfies the equation
$$4x^3 - 2x^2 - 3x + 1 = 0,$$which is to say $$(x - 1)(4x^2 + 2x - 1) =  0.$$
As $x = \cos 2\pi /5 \neq 1$, and  $\cos 2\pi /5 > 0$, $x$
positive root of the quadratic equation $4x^2 + 2x - 1 = 0,$ which
is to say
$$\cos \frac{2\pi}{5} = \frac{\sqrt{5} - 1}{4}.$$
\begin{exa} How many real numbers $x$ satisfy
$$\sin x = \frac{x}{100} ?$$\end{exa}
Solution: Plainly $x = 0$ is a solution. Also, if $x > 0$ is a
solution, so is $-x < 0$. So, we can restrict ourselves to
positive solutions.


If $x$ is a solution then $|x| = 100|\sin x| \leq 100$. So we can
further restrict $x$ to the interval $]0; 100].$ Decompose  $]0;
100]$ into  $2\pi$-long  intervals (the last interval is shorter):
$$]0; 100] = ]0; 2\pi ]\  \cup \ ]2\pi ; 4\pi ]\ \cup \ ]4\pi ; 6\pi ]\ \cup\  \cdots \ \cup \ ]28\pi ; 30\pi]\  \cup\ ]30\pi ; 100].$$
From the graphs of  $y = \sin x, y = x/100$ we that the interval
$]0; 2\pi]$ contains only one solution. Each interval of the form
$]2\pi k;\ 2(k+ 1)\pi ], k = 1, 2, \ldots , 14$ contains two
solutions. As $31\pi < 100$, the interval $]30\pi ; 100]$ contains
a full wave, hence it contains two solutions. Consequently, there
are $1 + 2\cdot 14 + 2 = 31$ positive solutions, and hence, 31
negative solutions. Therefore, there is a total of $31 + 31 + 1 =
63$ solutions.

\section*{Practice}\addcontentsline{toc}{section}{Practice}\markright{Practice}\begin{multicols}{2}\columnseprule 1pt \columnsep 25pt\multicoltolerance=900


\begin{pro} Solve for $x$ $$2\sqrt{\frac{x}{a}} + 3\sqrt{\frac{a}{x}} = \frac{b}{a} + \frac{6a}{b}.$$
\end{pro}
\begin{pro} Solve $$(x - 7)(x - 3)(x + 5)(x + 1) = 1680.$$ \end{pro}
\begin{pro}Solve $$x^4 + x^3 - 4x^2 + x + 1 = 0.$$ \end{pro}
\begin{pro} Solve the equation
$$2^{\sin ^2x} + 5\cdot 2^{\cos ^2x} = 7.$$
\begin{answer} $\cos ^2 x = 1 - \sin ^2 x$\end{answer}
\end{pro}
\begin{pro} If the equation
$\dis{\sqrt{x + \sqrt{x + \sqrt{x + \sqrt{\cdots}}}} = 2}$ made
sense, what would be the value of $x$?

\end{pro}

\begin{pro} How many real solutions are there to
$$\sin x = \log_e x ?$$
\begin{answer} $\log_e x > 1$ if $x > e$\end{answer}
\end{pro}
\begin{pro} Solve the equation
$$|x + 1| - |x| + 3|x - 1| - 2|x - 2| = x + 2.$$\end{pro}
\begin{pro} Find the real roots of
$$\sqrt{x + 3 - 4\sqrt{x - 1}} + \sqrt{x + 8 - 6\sqrt{x - 1}} = 1.$$\end{pro}
\begin{pro} Solve the equation
$$6x^4  - 25x^3 + 12x^2 + 25x + 6 = 0.$$\end{pro}
\begin{pro} Solve the equation
$$x(2x + 1)(x - 2)(2x - 3) = 63.$$\end{pro}
\begin{pro}
Find the value of
$$\sqrt{30\cdot 31\cdot 32\cdot 33 + 1}.$$
\end{pro}
\begin{pro} Solve
$$\frac{x + \sqrt{x^2 - 1}}{x - \sqrt{x^2 - 1}} + \frac{x - \sqrt{x^2 - 1}}{x + \sqrt{x^2 - 1}} = 98.$$\end{pro}
\begin{pro} Find a real solution to
$$(x^2 - 9x - 1)^{10} + 99x^{10} = 10x^9(x^2 - 1).$$\end{pro}
Hint: Write this equation as
$$(x^2 - 9x - 1)^{10} - 10x^9(x^2 - 9x - 1) + 9x^{10} = 0.$$
\begin{pro} Find the real solutions to
$$\underbrace{\sqrt{x + 2\sqrt{x + 2\sqrt{x + \cdots + 2\sqrt{x + 2\sqrt{3x}}}} }}_{n \ {\rm \ radicals}} = x.$$
\end{pro}

\begin{pro}Solve the equation
$$\cfrac{1}{1 + \cfrac{1}{1 + \cfrac{1}{1 + \cfrac{\vdots}{1 + \cfrac{1}{x}}}}} = x.$$where the fraction is
repeated $n$ times.
\end{pro}
\begin{pro}
Solve for $x$
$$\sqrt{x + \sqrt{x + 11}} + \sqrt{x + \sqrt{x - 11}} = 4.$$
\end{pro}
\end{multicols}
\section{Systems of Equations}
\begin{exa} Solve the system of equations
$$ \begin{array}{ccc}
x + y + u & = & 4, \\
y +  u + v & = & -5, \\
u + v + x & = & 0,\\
v + x + y & = & -8. \end{array}$$ \end{exa} Solution: Adding all
the equations and dividing by 3,
$$x + y + u + v = -3.$$This implies
$$ \begin{array}{ccc}
4 + v & = & -3, \\
-5 + x & = & -3, \\
0 + y & = & -3,\\
-8 + u & = & -3, \end{array}$$ whence $x = 2, y = -3, u = 5, v =
-7.$
\begin{exa} Solve the system
$$(x + y)(x + z) = 30,$$
$$(y + z)(y + x) = 15,$$
$$(z + x)(z + y) = 18.$$
\end{exa}
Solution: Put $u = y + z, v = z + x, w = x + y.$ The system
becomes
\begin{equation}vw = 30,\  wu = 15,\  uv = 18. \end{equation}
Multiplying all of these equations we obtain $u^2v^2w^2 = 8100$,
that is, $uvw = \pm 90.$ Dividing each of the equations in  (7),
we gather $u = 3, v = 6, w = 5,$ or $u = -3, v = -6, w = - 5.$
This yields
$$\begin{array}{lllllll}
y + z & = & 3, & {\rm or} & y + z & = &  -3, \\
z + x & = & 6, & {\rm or}  & z + x & = & -6, \\
x + y & = & 5, &    {\rm or}      & x + y & = & - 5, \\
\end{array}$$
whence $x = 4,\ y = 1, \ z = 2$ or $x = -4,\ y = -1,\ z = -2.$.

\section*{Practice}\addcontentsline{toc}{section}{Practice}\markright{Practice}\begin{multicols}{2}\columnseprule 1pt \columnsep 25pt\multicoltolerance=900
\begin{pro} Let  a,b, c be real constants, $abc \neq 0.$ Solve
$$x^2 - (y - z)^2 = a^2,$$
$$y^2 - (z - x)^2 = b^2,$$
$$z^2 - (x - y)^2 = c^2.$$\end{pro}
\begin{pro} Solve
$$x^3 + 3x^2y + y^3 = 8,$$
$$2x^3 - 2x^2y + xy^2 = 1.$$
\begin{answer}Let $y = mx$ and divide the equations obtained and solve for
$m$.  \end{answer}
\end{pro}

\begin{pro} Solve the system
$$x + 2 + y + 3 + \sqrt{(x + 2)(y + 3)} = 39,$$
$$(x + 2)^2 + (y + 3)^2 + (x + 2)(y + 3)= 741.$$\end{pro}
\begin{answer} Put $u = x + 2, v = y + 3.$ Divide one equation by the
other. \end{answer}
\begin{pro} Solve the system
$$x^4 + y^4 = 82,$$ $$x - y = 2.$$
\begin{answer} Let $u = x + y, v = x - y.$
\end{answer}
\end{pro}

\begin{pro} Solve the system
$$x_1x_2 = 1, \ x_2x_3 = 2,\ \ldots , \ x_{100}x_{101} = 100,\ x_{101}x_1 = 101.$$
\end{pro}
\begin{pro} Solve the system
$$x^2 - yz = 3,$$
$$y^2 - zx = 4,$$
$$z^2 - xy = 5.$$
\end{pro}
\begin{pro} Solve  the system
$$2x + y + z + u  = -1$$
$$x + 2y + z + u  = 12$$
$$x + y + 2z + u  = 5$$
$$x + y + z + 2u  = -1$$

\end{pro}
\begin{pro} Solve the system
$$x^2 + x + y = 8,$$
$$y^2 + 2xy + z = 168,$$
$$z^2 + 2yz + 2xz = 12480.$$
\end{pro}
\end{multicols}
\section{Remainder and Factor Theorems} The {\em
Division Algorithm} for polynomials states that if the polynomial
$p(x)$ is divided by $a(x)$ then there exist polynomials  $q(x),
r(x)$  with
\begin{equation}
p(x) = a(x)q(x) + r(x)\end{equation} and  $0 \leq$ degree $r(x) <$
degree $a(x)$. For example, if $x^5 + x^4 + 1$ is divided by $x^2
+ 1$ we obtain
$$x^5 + x^4 + 1 = (x^3 + x^2 - x - 1)(x^2 + 1) + x + 2,$$and so the quotient is
$q(x) = x^3 + x^2 - x - 1$ and the remainder is $r(x) = x + 2.$
\begin{exa} Find the remainder when $(x + 3)^5 + (x + 2)^8 + (5x + 9)^{1997}$ is divided by
$x + 2.$ \end{exa} Solution: As we are dividing by a polynomial of
degree 1, the remainder is a polynomial of degree 0, that is, a
constant. Therefore, there is a polynomial  $q(x)$ and a constant
$r$ with
$$
(x + 3)^5 + (x + 2)^8 + (5x + 9)^{1997} = q(x)(x + 2) + r
$$Letting  $x = - 2$ we obtain
$$(-2 + 3)^5 + (-2 + 2)^8 + (5(-2) + 9)^{1997} = q(-2)(-2 + 2) + r = r. $$As the sinistral side
is 0 we deduce that the remainder $r = 0$.
\begin{exa} A polynomial leaves remainder $-2$ upon division by $x - 1$ and remainder
$-4$ upon division by $x + 2.$ Find the remainder when this
polynomial is divided by $x^2 + x - 2$.  \end{exa} Solution: From
the given information, there exist polynomials $q_1(x), q_2(x)$
with $p(x) = q_1(x)(x - 1) - 2$ and $p(x) = q_2(x)(x + 2) - 4.$
Thus $p(1) = -2$ and $p(-2) = - 4.$ As $x^2 + x - 2 = (x - 1)(x +
2)$ is a polynomial of degree 2 the remainder $r(x)$ upon dividing
$p(x)$ by $x^2 + x - 1$ is of degree 1 or less, that is $r(x) = ax
+ b$ for some constants $a, b$ which we must determine. By the
Division Algorithm,
$$p(x) = q(x)(x^2 + x - 1) + ax + b.$$Hence
$$- 2 = p(1) = a + b$$and $$- 4 = p(-2) = -2a + b.$$From these equations we deduce that
$a = 2/3, b = -8/3.$ The remainder sought is $r(x) = 2x/3 - 8/3.$
\begin{exa}
Let $f(x) = x^4 + x^3 + x^2 + x + 1.$ Find the remainder when
$f(x^5)$ is divided by $f(x).$
\end{exa}
Solution: Observe that $f(x)(x - 1) = x^5 - 1$ and
$$f(x^5) = x^{20} + x^{15} + x^{10} + x^5 + 1 = (x^{20} - 1) + (x^{15} - 1) + (x^{10} - 1)
+ (x^5 - 1) + 5.$$ Each of the summands in parentheses is
divisible by $x^5 - 1$ and, a fortiori, by $f(x).$ The remainder
sought is thus 5.



Using the Division Algorithm we may derive the following theorem.
\begin{thm}
{\bf Factor Theorem} The polynomial $p(x)$ is divisible by $x - a$
if and only if $p(a) = 0.$
\end{thm}
{\bf Proof}  As $x - a$ is a polynomial of degree 1, the remainder
after diving $p(x)$ by $x - a$ is a polynomial of degree 0, es
that is, a constant. Therefore
$$p(x) = q(x)(x - a) + r.$$From this we gather that  $p(a) = q(a)(a - a) + r = r,$
from where the theorem easily follows.
\begin{exa} If $p(x)$  is a cubic polynomial with $p(1) = 1, p(2) = 2, p(3) = 3, p(4) = 5,$
find $p(6)$. \end{exa} Solution: Put $g(x) = p(x) - x$. Observe
that $g(x)$ is a polynomial of degree 3 and that  $g(1) = g(2) =
g(3) = 0.$ Thus $g(x) = c(x - 1)(x - 2)(x - 3)$ for some constant
$c$ that we must determine. Now, $g(4) = c(4 - 1)(4 - 2) (4 - 3) =
6c$ and $g(4) = p(4) - 4 = 1,$ whence $c = 1/6.$ Finally
$$p(6) = g(6) + 6 = \frac{(6 - 1)(6 - 2)(6 - 3)}{6} + 6 = 16.$$
\begin{exa}  The polynomial $p(x)$ has integral coefficients and $p(x) = 7$ for four different
values of $x$. Shew that $p(x)$ never equals $14$.  \end{exa}
Solution: The polynomial $g(x) = p(x) - 7$ vanishes at the 4
different integer values $a, b, c, d$. In virtue of the Factor
Theorem, $$g(x) = (x - a)(x - b)(x - c)(x - d)q(x),$$ where $q(x)$
is a polynomial with integral coefficients. Suppose that $p(t) =
14$ for some integer $t$.  Then $g(t) = p(t) - 7 = 14 - 7 = 7.$ It
follows that
$$7 = g(t) = (t - a)(t - b)(t - c)(t - d)q(t),$$that is, we have factorised 7 as the product of
at least 4 different factors, which is impossible since  7 can be
factorised as $7(-1)1$, the product of at most 3 distinct integral
factors. From this contradiction we deduce that such an integer
$t$ does not exist.
\section*{Practice}\addcontentsline{toc}{section}{Practice}\markright{Practice}\begin{multicols}{2}\columnseprule 1pt \columnsep 25pt\multicoltolerance=900


\begin{pro}
If $p(x)$ is a polynomial of degree n such that $p(k) = 1/k, k =
1, 2, \ldots , n + 1$, find $p(n + 2)$.
\end{pro}
\begin{pro}
The polynomial $p(x)$ satisfies $p(-x) = -p(x).$ When $p(x)$ is
divided by $x - 3$ the remainder is $6$. Find the remainder when
$p(x)$ is divided by $x^2 - 9$.
\end{pro}
\end{multicols}
\section{Vi\`{e}te's Formulae} Let us consider first the following example.
\begin{exa}
Expand the product $$(x + 1)(x - 2)(x + 4)(x - 5)(x + 6).$$
\end{exa}
Solution: The product is a polynomial of degree 5. To obtain the
coefficient of $x^5$ we take an $x$ from each of the five
binomials. Therefore, the coefficient of $x^5$ is 1. To form the
$x^4$ term, we take an  $x$ from 4 of the binomials and a constant
from the remaining binomial. Thus the coefficient of $x^4$ is
$$1 - 2 + 4 - 5 + 6 = 4.$$To form the coefficient of  $x^3$ we take three $x$
from 3 of the binomials and two constants from the remaining
binomials. Thus the coefficient of $x^3$ is
$$(1)(-2) + (1)(4) + (1)(-5) + (1)(6) + (-2)(4) + (-2)(-5) + (-2)(6)$$
$$+ (4)(-5) + (4)(6)
+ (-5)(6) = -33.$$ Similarly, the coefficient of $x^2$ is
$$(1)(-2)(4) + (1)(-2)(-5) + (1)(-2)(6) + (1)(4)(-5) + (1)(4)(6) + (-2)(4)(-5)$$
$$ +
(-2)(4)(6) + (4)(-5)(6) = -134$$and the coefficient of  $x$ is
$$(1)(-2)(4)(-5) + (1)(-2)(4)(6) + (1)(-2)(-5)(6) + (1)(4)(-5)(6) + (-2)(4)(-5)(6)  = 172.$$
Finally, the constant term is $(1)(-2)(4)(-5)(6) = 240.$ The
product sought is thus
$$x^5 + 4x^4 - 33x^3 - 134x^2 + 172x + 240.$$

\bigskip

From the preceding example, we see that each summand of the
expanded product has ``weight'' 5, because of the five given
binomials we either take the  $x$  or take the constant.

\bigskip


If $a_0 \neq 0$ and $$a_0x^n + a_{1}x^{n - 1} + a_2x^{n - 2} +
\cdots + a_{n - 1}x + a_n$$ is a polynomial with roots $\alpha _1
, \alpha _2 , \ldots , \alpha _n$ then we may write
$$
a_0x^n + a_{1}x^{n - 1} + a_2x^{n - 2} + \cdots + a_{n - 1}x + a_n
= a_0(x - \alpha _1)(x - \alpha _2)(x - \alpha _3)\cdots (x -
\alpha _{n - 1})(x - \alpha _n) .$$ From this we deduce the {\em
Vi\`{e}te Formul\ae:}
$$-\frac{a_1}{a_0} = \sum _{k = 1} ^n \alpha _k,$$
$$
\frac{a_2}{a_0} = \sum _{1 \leq j < k \leq n} \alpha _j \alpha _k,
$$
$$-\frac{a_3}{a_0} = \sum _{1 \leq j < k < l \leq n}  \alpha _j\alpha _k \alpha _l,
$$
$$\frac{a_4}{a_0} = \sum _{1 \leq j < k < l < s \leq n}  \alpha _j\alpha _k \alpha _l\alpha _s,
$$
$$..........$$
$$..........$$
$$...........$$
$$(-1)^n\frac{a_n}{a_0} = \alpha _1\alpha _2\cdots \alpha _n.$$
\begin{exa}
Find the sum of the roots, the sum of the roots taken two at a
time, the sum of the square of the roots and the sum of the
reciprocals of the roots of
$$2x^3 - x + 2 = 0.$$
\end{exa}
Solution: Let $a, b, c$ be the roots of $2x^3 - x + 2 = 0$. From
the Vi\`{e}te  Formul\ae \ the sum of the roots is
$$a + b + c = -\frac{0}{2} = 0$$ and the sum of the roots taken two at a time is
$$ab + ac + bc = \frac{-1}{2}.$$To find $a^2 + b^2 + c^2$ we observe that

$$a^2 + b^2 + c^2 = (a + b + c)^2 - 2(ab + ac + bc).$$
Hence
$$a^2 + b^2 + c^2 = 0^2 - 2(-1/2) = 1.$$
Finally, as $abc = -2/2 = -1$, we gather that
$$\frac{1}{a} + \frac{1}{b}+ \frac{1}{c} = \frac{ab + ac + bc}{abc} = \frac{-1/2}{-1} = 1/2.$$
\begin{exa}
Let $\alpha , \beta , \gamma$ be the roots of $x^3 - x^2 + 1 = 0$.
Find
$$\frac{1}{\alpha ^2} + \frac{1}{\beta ^2} + \frac{1}{\gamma ^2}.$$
\end{exa}

Solution: From $x^3 - x^2 + 1 = 0$ we deduce that $1/x^2 = 1  -
x$. Hence
$$
\frac{1}{\alpha ^2} + \frac{1}{\beta ^2} + \frac{1}{\gamma ^2} =
(1 - \alpha) + (1 - \beta) + (1 - \gamma) = 3 - (\alpha + \beta +
\gamma) = 3 - 1 = 2. $$


\bigskip


Together with the Vi\`{e}te Formul\ae \ we also have the {\em
Newton-Girard Identities} for the sum of the powers $s_k = \alpha _1
^k + \alpha _2 ^k + \cdots + \alpha _n ^k$ of the roots:
$$a_0s_1 + a_1 = 0,$$
$$a_0s_2 + a_1s_1 + 2a_2 = 0,$$
$$a_0s_3 + a_1s_2 + a_2s_1 + 3a_3 = 0,$$etc..
\begin{exa}If  $a, b , c$ are the roots of $x^3 - x^2 + 2 = 0$, find
$$a^2 + b^2 + c^2$$
$$a^3 + b^3 + c^3$$ and  $$a^4 + b^4 + c^4.$$\end{exa}
Solution: First observe that
$$a^2 + b^2 + c^2 = (a + b + c)^2 - 2(ab + ac + bc) = 1^2 - 2(0) = 1.$$
As $x^3 = x^2 - 2,$ we gather
$$a^3 + b^3 + c^3 = a^2 - 2 + b^2 - 2 + c^2 - 2 = a^2 + b^2 + c^2 - 6 = 1 - 6 = -5. $$
Finally, from $x^3 = x^2 - 2$ we obtain $x^4 = x^3 - 2x$, whence
$$a^4 + b^4 + c^4 = a^3 - 2a + b^3 - 2b + c^3 - 2c = a^3 + b^3 + c^3 - 2(a + b + c) = -5 - 2(1) = -7.$$
\begin{exa}[USAMO 1973] Find all solutions (real or complex) of the system
$$ x + y + z = 3,$$
$$x^2 + y^2 + z^2 = 3,$$
$$x^3 + y^3 + z^3 = 3.$$

\end{exa}
Solution: Let $x, y , z$ be the roots of
$$p(t) = (t - x)(t - y)(t - z) = t^3 - (x + y + z)t^2 + (xy + yz + zx)t - xyz.$$
Now $xy + yz + zx = (x + y + z)^2/2 - (x^2 + y^2 + z^2)/2 = 9/2 -
3/2 = 3$ and from
$$x^3 + y^3 + z^3 - 3xyz = (x + y + z)(x^2 + y^2 + z^2 - xy - yz - zx)$$ we gather that
 $xyz = 1.$ Hence
$$p(t) = t^3 - 3t^2 + 3t - 1 = (t - 1)^3.$$
Thus $x = y = z = 1$ is the only solution of the given system.

\clearpage

\section*{Practice}\addcontentsline{toc}{section}{Practice}\markright{Practice}\begin{multicols}{2}\columnseprule 1pt \columnsep 25pt\multicoltolerance=900

\begin{pro}
Suppose that
$$\begin{array}{l}x^n + a_1x^{n - 1} + a_2x^{n - 2} + \cdots + a_n \\ \qquad = (x + r_1)(x + r_2)\cdots (x + r_n)\end{array}$$
where $r_1, r_2, \ldots , r_n$ are real numbers. Shew that
$$(n - 1)a_1 ^2 \geq 2na_2.$$
\end{pro}
\begin{pro}[USAMO 1984] The product of the roots of
$$x^4 - 18x^3 + kx^2 + 200x - 1984 = 0$$is $-32$. Determine  $k$.

\end{pro}
\begin{pro}
The equation $x^4 - 16x^3 + 94x^2 + px + q = 0$ has two double
roots. Find $p + q.$
\end{pro}
\begin{pro}
If $\alpha _1, \alpha _2 , \ldots , \alpha _{100}$ are the roots
of
$$x^{100} - 10x + 10 = 0,$$find the sum
$$\alpha _1 ^{100} + \alpha _2 ^{100} + \cdots + \alpha _{100} ^{100}.$$
\end{pro}
\begin{pro}
Let $\alpha , \beta , \gamma$ be the roots of $x^3 - x - 1 = 0.$
Find $$\frac{1}{\alpha ^3} + \frac{1}{\beta ^3} + \frac{1}{\gamma
^3}$$y
$$\alpha ^5 + \beta ^5 + \gamma ^5.$$
\end{pro}
\begin{pro} The real numbers $\alpha , \beta$ satisfy
$$\alpha ^3 - 3\alpha ^2 + 5\alpha - 17 = 0,$$
$$\beta ^3 - 3\beta ^2 + 5\beta + 11 = 0.$$Find $\alpha + \beta .$
\end{pro}
\end{multicols}

\section{Lagrange's Interpolation}
\begin{exa}
Find a cubic polynomial $p(x)$ vanishing at $x = 1, 2, 3$ and
satisfying $p(4) = 666.$
\end{exa}
Solution: The polynomial must be of the form $p(x) = a(x - 1)(x -
2)(x - 3)$, where $a$ is a constant. As $666 = p(4) = a(4 - 1)(4 -
2)(4 - 3) = 6a, \ a = 111.$ The desired polynomial is therefore
$p(x) = 111(x - 1)(x - 2)(x - 3).$
\begin{exa}
Find a cubic polynomial $p(x)$ satisfying $p(1) = 1, p(2) = 2,
p(3) = 3, p(4) = 5$.
\end{exa}
Solution: We shall use the following method due to Lagrange. Let
$$p(x) = a(x) + 2b(x) + 3c(x) + 5d(x),$$where $a(x), b(x), c(x), d(x)$ are cubic polynomials with
the following properties: $a(1) = 1$ and $a(x)$ vanishes when $x =
2, 3, 4$;$b(2) = 1$ and $b(x)$ vanishes when $x = 1, 3, 4$; $c(3)
= 1$ and $c(3) = 1$ vanishes when $x = 1, 2, 4$, and finally,
$d(4) = 1$, $d(x)$ vanishing at $x = 1, 2, 3.$


Using the technique of the preceding example, we find
$$a(x) = -\frac{(x - 2)(x - 3)(x - 4)}{6},$$
$$b(x) = \frac{(x - 1)(x - 3)(x - 4)}{2},$$
$$c(x) = -\frac{(x - 1)(x - 2)(x - 4)}{2}$$y
$$d(x) = \frac{(x - 1)(x - 2)(x - 3)}{6}.$$
Thus
$$p(x) = -\frac{1}{6}\cdot (x - 2)(x - 3)(x - 4) + (x - 1)(x - 3)(x - 4)$$
$$ \hspace{2in} - \frac{3}{2}\cdot (x - 1)(x - 2)(x - 4) + \frac{5}{6}(x - 1)(x - 2)(x - 3).$$
It is left to the reader to verify that the polynomial satisfies
the required properties.

\section*{Practice}\addcontentsline{toc}{section}{Practice}\markright{Practice}
\begin{multicols}{2}\columnseprule 1pt \columnsep 25pt\multicoltolerance=900

\begin{pro}  Find a polynomial $p(x)$ of degree $4$ with
$p(1) = 1, p(2) = 2, p(3) = 3, p(4) = 4, p(5) = 5.$\end{pro}
\begin{pro}  Find a polynomial $p(x)$ of degree $4$ with
$p(1) = -1, p(2) = 2, p(-3) = 4, p(4) = 5, p(5) = 8.$\end{pro}
\end{multicols}











\chapter{Inequalities}
\section{Absolute Value}
\begin{df}[The Signum (Sign) Function] Let $x$ be a real number. We
define $ \signum{x}= \left\{
\begin{tabular}{ll}
$-1$       & \rm{if} \ $x < 0$, \\
$0$ & \rm{if} \ $x = 0$,\\
$+1$       & \rm{if} \ $x > 0$. \\
\end{tabular}
\right. $
\end{df}
\begin{lem}\label{lem:signum-is-multiplicative}
The signum function is multiplicative, that is, if $(x,y)\in\BBR^2$
then $\signum{x\cdot y}=\signum{x}\signum{y}$.
\end{lem}
\begin{pf}
Immediate from the definition of signum.
\end{pf}
\begin{df}[Absolute Value] Let $x\in\BBR$. The {\em absolute value}
of $x$ is defined and denoted by $$ \absval{x}=\signum{x}x. $$

\end{df}
\begin{thm}\label{thm:absval-properties}
Let $x\in\BBR$. Then
\begin{enumerate}
\item  $ |x| =
\left\{
\begin{array}{ll}
-x       & \mathrm{if} \ x < 0, \\
x & \mathrm{if} \ x \geq 0.
\end{array}
\right.$
\item $\absval{x}\geq 0$,
\item $\absval{x}=\max (x,-x)$,

\item $\absval{-x} = \absval{x}$,

\item \label{eq:abs_val_interval} $-\absval{x} \leq x \leq \absval{x}$.
\item $\sqrt{x^2} = |x| $
\item $|x|^2 = |x^2| = x^2$
\item $x = \signum{x}\absval{x}$
\end{enumerate}
\end{thm}
\begin{pf} These are immediate from the definition of $\absval{x}$.
\end{pf}
\begin{thm}
$(\forall (x, y)\in \BBR^2)$,
$$\absval{xy} =\absval{x}\absval{y}. $$
\end{thm}
\begin{pf}
We have $$\absval{xy} = \signum{xy}xy = \left(\signum{x}x\right)
\left(\signum{y}y\right) = \absval{x}\absval{y},  $$where we have
used  Lemma \ref{lem:signum-is-multiplicative}.
\end{pf}
\begin{thm}\label{thm:|x|within_t}Let $t\geq 0$. Then
$$|x| \leq t \iff -t \leq x \leq t.  $$
\end{thm}
\begin{pf}
Either $\absval{x} =x$ or $\absval{x}=-x$. If   $\absval{x} =x$,
$$\absval{x}\leq t \iff x\leq t \iff -t \leq 0\leq x \leq t.   $$
If   $\absval{x} =-x$,
$$\absval{x}\leq t \iff -x\leq t \iff -t \leq x \leq 0 \leq t.   $$
\end{pf}


\begin{thm}If $(x,y)\in\BBR^2$,
$\max (x, y) = \dfrac{x+y+\absval{x-y}}{2}$ and $\min (x, y) =
\dfrac{x+y-\absval{x-y}}{2}$.
\end{thm}
\begin{pf}Observe that $\max (x, y) + \min (x, y) = x+y$, since one
of these quantities must be the maximum and the other the minimum,
or else, they are both equal.

\bigskip

Now, either $\absval{x-y} = x-y$, and so $x\geq y$, meaning that
$\max (x, y)-\min (x, y) = x-y$, or $\absval{x-y} = -(x-y)=y-x$,
which means that $y\geq x$ and so $\max (x, y)-\min (x, y) = y-x$.
In either case we get $\max(x,y)-\min (x, y)=\absval{x-y}$. Solving
now the system of equations $$\begin{array}{lll}\max (x, y) + \min
(x, y) & = & x+y\\  \max(x,y)-\min (x, y) & = &\absval{x-y},\\
\end{array}
$$for $\max(x,y)$ and $\min (x,y)$ gives the result.
\end{pf}
\section{Triangle Inequality}
\begin{thm}[Triangle Inequality] Let $(a, b)\in \BBR^2$. Then
\begin{equation}\fcolorbox{blue}{white}{ $|a + b| \leq |a| + |b|$.}\end{equation}
\label{tri_ineq}
\end{thm}
\begin{pf}
From \ref{eq:abs_val_interval} in Theorem
\ref{thm:absval-properties}, by addition,
$$-|a| \leq a \leq |a| $$to$$-|b| \leq b \leq |b| $$we obtain
$$-(|a| + |b| ) \leq a + b \leq (|a| + |b|),$$whence the theorem follows by applying Theorem \ref{thm:|x|within_t}.
\end{pf}
By induction, we obtain the following generalisation to $n$ terms.
\begin{cor}\label{cor:triangle-ineq}
Let $x_1, x_2, \ldots , x_n$ be real numbers. Then $$ \absval{x_1 +
x_2 + \cdots + x_n}\leq \absval{x_1}+\absval{x_2}+\cdots +
\absval{x_n}.
$$
\end{cor}
\begin{pf}
We apply Theorem \ref{tri_ineq} $n-1$ times
$$ \begin{array}{lll} \absval{x_1 +
x_2 + \cdots + x_n}& \leq & \absval{x_1} +\absval{ x_2 + \cdots
x_{n-1}+ x_n}\\
& \leq & \absval{x_1} +\absval{x_2} +\absval{ x_3 + \cdots
x_{n-1}+ x_n}\\
& \vdots & \\
& \leq & \absval{x_1}+\absval{x_2}+\cdots + \absval{x_{n-1}+x_n}\\ &
\leq & \absval{x_1}+\absval{x_2}+\cdots +
\absval{x_{n-1}}+\absval{x_n}.
\end{array}$$


\end{pf}
\begin{cor}
 Let $(a, b)\in \BBR^2$. Then
\begin{equation} \fcolorbox{blue}{white}{$ \absval{|a| - |b|}  \leq \absval{a - b}$}.\end{equation}
\label{tri_ineq_2}
\end{cor}
\begin{pf}
We have $$|a| = |a - b + b| \leq |a - b| + |b|,$$giving
$$|a| - |b| \leq |a - b|.$$Similarly,
$$|b| = |b - a + a| \leq |b - a| + |a| = |a - b| + |a|,$$gives
$$|b| - |a| \leq |a - b|\implies -\absval{a-b} \leq \absval{a}-\absval{b}.$$Thus
$$-\absval{a-b} \leq \absval{a}-\absval{b}  \leq  \absval{a-b},  $$
and we now apply Theorem \ref{thm:|x|within_t}.
\end{pf}
\section{Rearrangement Inequality}

\begin{df}
Given a set of real numbers $\{x_1, x_2, \ldots , x_n\}$ denote by
$$\check{x}_1\geq  \check{x}_2\geq  \cdots \geq \check{x}_n$$ the
decreasing rearrangement of the $x_i$ and denote by
$$\hat{x}_1\leq  \hat{x}_2\leq  \cdots \leq \hat{x}_n$$the
increasing rearrangement of the $x_i$.
\end{df}
\begin{df}
Given two sequences  of real numbers $\{x_1, x_2, \ldots , x_n\}$
and $\{y_1, y_2, \ldots , y_n\}$ of the same length $n$, we say that
they are {\em similarly sorted} if they are both increasing or both
decreasing, and {\em differently sorted} if one is increasing and
the other decreasing..
\end{df}
\begin{exa}
The sequences $1 \leq 2 \leq \cdots \leq n$ and $1^2 \leq 2^2 \leq
\cdots \leq n^2$ are similarly sorted, and the sequences
$\dfrac{1}{1^2} \geq \dfrac{1}{2^2} \geq \cdots \geq \dfrac{1}{n^2}$
and $1^3 \leq 2^3 \leq \cdots \leq n^3$ are differently sorted.
\end{exa}
\begin{thm}[Rearrangement Inequality]\label{thm:rearrangement-ineq}
Given sets of real numbers $\{a_1, a_2, \ldots , a_n\}$ and $\{b_1,
b_2, \ldots , b_n\}$ we have $$ \sum _{1\leq k\leq
n}\check{a}_k\hat{b}_k \leq \sum _{1\leq k \leq n} a_kb_k \leq \sum
_{1\leq k \leq n}\hat{a}_k\hat{b}_k.
$$Thus the sum $ \sum _{1\leq k \leq n} a_kb_k $ is minimised when
the sequences are differently sorted, and maximised when the
sequences are similarly sorted.
\end{thm}
\begin{rem}
Observe that $$\sum _{1\leq k \leq n}\hat{a}_k\hat{b}_k = \sum
_{1\leq k \leq n}\check{a}_k\check{b}_k.  $$
\end{rem}
\begin{pf}
Let $\{\sigma (1),\sigma(2), \ldots , \sigma(n)\}$ be a reordering
of $\{1,2, \ldots , n\}$. If there are two sub-indices   $i, j$,
such that the sequences pull in opposite directions, say, $a_i
>a_j$ and $b_{\sigma (i)}< b_{\sigma (j)}$, then consider the sums
$$ \begin{array}{lll}S & = & a_1b_{\sigma (1)}+a_2b_{\sigma (2)}+\cdots +a_ib_{\sigma (i)}+\cdots +a_jb_{\sigma (j)}+\cdots + a_nb_{\sigma (n)}\\
S' & = & a_1b_{\sigma (1)}+a_2b_{\sigma (2)}+\cdots +a_ib_{\sigma (j)}+\cdots +a_jb_{\sigma (i)}+\cdots + a_nb_{\sigma (n)}\\
 \end{array}$$
 Then $$ S'-S = (a_i-a_j)(b_{\sigma(j)}-b_{\sigma (i)}) >0. $$This
 last inequality shews that  the closer the $a$'s and the $b$'s are to pulling in the same direction the larger the sum
 becomes. This proves the result.
\end{pf}
\section{Mean Inequality}
\begin{thm}[Arithmetic Mean-Geometric Mean Inequality]\label{thm:AMGM-ineq}
Let $a_{1}, \dots, a_{n}$ be positive real numbers. Then their
geometric mean is at most their arithmetic mean, that is,
    $$
   \sqrt[n]{a_{1}\cdots a_{n}} \leq  \dfrac{a_{1} + \cdots + a_{n}}{n},
    $$
    with equality if and only if $a_{1} = \cdots = a_{n}$.
\end{thm}
We will provide multiple proofs of this important inequality. Some
other proofs will be found in latter chapters.
\begin{f-pf}
Our first  proof uses the Rearrangement Inequality (Theorem
\ref{thm:rearrangement-ineq}) in a rather clever way. We may assume
that the $a_k$ are strictly positive. Put
$$x_1=\dfrac{a_1}{(a_1a_2\cdots a_n)^{1/n}}, \quad x_2=\dfrac{a_1a_2}{(a_1a_2\cdots
a_n)^{2/n}},\quad  \ldots ,\quad  x_n=\dfrac{a_1a_2\cdots
a_n}{(a_1a_2\cdots a_n)^{n/n}}=1,
$$
and$$ y_1 = \dfrac{1}{x_1}, \quad y_2 = \dfrac{1}{x_2},\quad \ldots
,\quad y_n = \dfrac{1}{x_n}=1.
$$
Observe that for $2 \leq k \leq n$, $$x_ky_{k-1} =
\dfrac{a_1a_2\cdots a_k}{(a_1a_2\cdots a_n)^{k/n}}\cdot
\dfrac{(a_1a_2\cdots a_n)^{(k-1)/n}} {a_1a_2\cdots a_{k-1}} =
\dfrac{a_k}{(a_1a_2\cdots a_n)^{1/n}}.$$

The $x_k$ and $y_k$ are differently sorted, so by virtue of the
Rearrangement Inequality we gather
$$\begin{array}{lll} 1+1+\cdots + 1 & = & x_1y_1+x_2y_2+\cdots + x_ny_n \\
& \leq  & x_1y_n+x_2y_{1}+\cdots + x_ny_{n-1} \\
& = & \dfrac{a_1}{(a_1a_2\cdots a_n)^{1/n}}+
\dfrac{a_2}{(a_1a_2\cdots a_n)^{1/n}} + \cdots +
\dfrac{a_n}{(a_1a_2\cdots a_n)^{1/n}},
 \end{array}$$
 or $$ n \leq \dfrac{a_1+a_2+\cdots + a_n}{(a_1a_2\cdots a_n)^{1/n}},
 $$from where we obtain the result.
\end{f-pf}
\begin{s-pf}
This second proof is a clever induction argument due to Cauchy. It
proves the inequality first for powers of $2$ and then interpolates
for numbers between consecutive powers of $2$.

\bigskip
 Since the square of a
real number is always positive, we have, for positive real numbers
$a, b$
$$ (\sqrt{a}-\sqrt{b})^2 \geq 0 \implies \sqrt{ab} \leq \dfrac{a+b}{2},
$$proving the inequality for $k=2$. Observe that equality happens if and only if $a=b$. Assume now that the inequality
is valid for $k=2^{n-1}>2$. This means that for any positive real
numbers $x_1, x_2, \ldots , x_{2^{n-1}} $ we have
\begin{equation}\label{eq:amgm-n-1}
\left(x_1x_2\cdots x_{2^{n-1}}\right)^{1/2^{n-1}} \leq
\dfrac{x_1+x_2+\cdots +x_{2^{n-1}}}{2^{n-1}}.
\end{equation}
Let us prove the inequality for $2k=2^n$. Consider any any positive
real numbers $y_1, y_2, \ldots , y_{2^{n}}$. Notice that there are
$2^n-2^{n-1} = 2^{n-1}(2-1) = 2^{n-1}$ integers in the interval
$\lcrc{2^{n-1}+1}{2^n}$. We have
$$\begin{array}{lll}\left(y_1y_2\cdots y_{2^{n}}\right)^{1/2^{n}} & = &
\sqrt{\left(y_1y_2\cdots
y_{2^{n-1}}\right)^{1/2^{n-1}}\left(y_{2^{n-1}+1}\cdots
y_{2^{n}}\right)^{1/2^{n-1}}} \\
& \leq & \dfrac{\left(y_1y_2\cdots
y_{2^{n-1}}\right)^{1/2^{n-1}}+\left(y_{2^{n-1}+1}\cdots
y_{2^{n}}\right)^{1/2^{n-1}}}{2}\\
& \leq & \dfrac{\dfrac{y_1+y_2+\cdots
+y_{2^{n-1}}}{2^{n-1}}+\dfrac{y_{2^{n-1}+1}+\cdots
+y_{2^{n}}}{2^{n-1}}}{2}\\
& = & \dfrac{y_1 + \cdots + y_{2^n}}{2^n},
\end{array}$$
where the first inequality follows by the Case $n=2$ and the second
by the induction hypothesis (\ref{eq:amgm-n-1}). The theorem is thus
proved for powers of $2$.

\bigskip

Assume now that $2^{n-1}<k<2^n$, and consider the $k$ positive real
numbers $a_1, a_2, \ldots , a_k$. The trick is to pad this
collection of real numbers up to the next highest power of $2$, the
added real numbers being the average of the existing ones. Hence
consider the $2^n$ real numbers
$$a_1, a_2, \ldots , a_k, a_{k+1}, \ldots , a_{2^n}  $$
with $a_{k+1}= \ldots = a_{2^n}=\dfrac{a_1+ a_2+ \cdots + a_k}{k}$.
Since we have already proved the theorem for $2^n$ we have
$$ \left(a_1a_2\cdots a_k\left(\dfrac{a_1+ a_2+ \cdots + a_k}{k}\right)^{2^n-k}\right)^{1/2^n} \leq
\dfrac{a_1+ a_2+ \cdots + a_k+(2^n-k)\left(\dfrac{a_1+ a_2+ \cdots +
a_k}{k}\right)}{2^n},$$whence $$\left(a_1a_2\cdots
a_k\right)^{1/2^n}\left(\dfrac{a_1+ a_2+ \cdots +
a_k}{k}\right)^{1-k/2^n} \leq \dfrac{k\dfrac{a_1+ a_2+ \cdots +
a_k}{k}+(2^n-k)\left(\dfrac{a_1+ a_2+ \cdots + a_k}{k}\right)}{2^n},
$$which implies
$$\left(a_1a_2\cdots
a_k\right)^{1/2^n}\left(\dfrac{a_1+ a_2+ \cdots +
a_k}{k}\right)^{1-k/2^n} \leq \left(\dfrac{a_1+ a_2+ \cdots +
a_k}{k}\right),
$$


Solving for $\dfrac{a_1+ a_2+ \cdots + a_k}{k}$ gives the desired
inequality.
\end{s-pf}
\begin{t-pf}
As in the second proof, the Case $k=2$ is easily established. Put $$
A_k = \dfrac{a_1+a_2+\cdots + a_k}{k}, \qquad G_k =
\left(a_1a_2\cdots a_k\right)^{1/k}.$$Observe that
$$ a_{k+1} = (k+1)A_{k+1}-kA_k. $$
The inductive hypothesis is that $A_k \geq G_k$ and we must shew
that $A_{k+1}\geq G_{k+1}$. Put
$$A= \dfrac{a_{k+1}+(k-1)A_{k+1}}{k}, \qquad G=\left(a_{k+1}A_{k+1} ^{k-1}\right)^{1/k}.
$$By the inductive hypothesis $A \geq G$.
Now, $$ \dfrac{A+A_k}{2} =
\dfrac{\dfrac{(k+1)A_{k+1}-kA_k+(k-1)A_{k+1}}{k}+A_k}{2}=A_{k+1}.
$$Hence
$$\begin{array}{lll} A_{k+1} & = & \dfrac{A+A_k}{2}\\
& \geq & \left(AA_k\right)^{1/2} \\
& \geq & \left(GG_k\right)^{1/2}. \\
& = & \left(G_{k+1} ^{k+1}A_{k+1} ^{k-1}\right)^{1/2k}\\
\end{array}$$ We have established that $$A_{k+1} \geq \left(G_{k+1} ^{k+1}A_{k+1} ^{k-1}\right)^{1/2k}\implies A_{k+1}\geq G_{k+1},
$$completing the induction.
\end{t-pf}
\begin{fo-pf}
We will make a series of substitutions that preserve the sum
$$a_1+a_2+\cdots +a_n$$while strictly increasing the product $$a_1a_2\cdots a_n. $$ At the
end, the $a_i$ will all be equal and the arithmetic mean $A$ of the
numbers will be equal to their geometric mean $G$. If the $a_i$
where all $>A$ then $\dfrac{a_1+a_2+\cdots
+a_n}{n}>\dfrac{nA}{n}=A$, impossible. Similarly, the $a_i$ cannot
be all $<A$. Hence there must exist two indices say $i, j$, such
that $a_i < A < a_j$. Put $a_i '=A$, $a_j'=a_i+a_j-A$. Observe that
$a_i+a_j=a_i'+a_j'$, so replacing the original $a$'s with the primed
$a$'s does not alter the arithmetic mean. On the other hand,
$$
 a_i'a_j'=   A\left(a_i + a_j - A\right) = a_ia_j + \left(a_j - A\right)\left(A - a_i\right) > a_ia_j
$$since $a_{j}-A>0$ and $A-a_i>0$.

\bigskip
This change has replaced one of the $a$'s by a quantity equal to the
arithmetic mean, has not changed the arithmetic mean, and made the
geometric mean larger. Since there at most $n$ $a$'s to be replaced,
the procedure must eventually terminate when all the $a$'s are equal
(to their arithmetic mean). Strict inequality hence holds if when at
least two of the $a$'s are unequal.
\end{fo-pf}



\begin{thm}[Cauchy-Bunyakovsky-Schwarz Inequality]\label{thm:CBS-ineq}
Let $x_k, y_k$ be real numbers, $1 \leq k \leq n$. Then
$$\absval{\sum _{k = 1} ^n x_ky_k} \leq \left(\sum _{k = 1} ^n x_k
^2 \right)^{1/2}\left(\sum _{k = 1} ^n y_k ^2 \right)^{1/2},   $$
with equality if and only if
$$(a_1,a_2,\ldots , a_n)=t(b_1,b_2,\ldots , b_n)
$$for some real constant $t$.
\end{thm}
\begin{f-pf}The inequality follows at once from Lagrange's Identity
$$ \left(\sum _{k=1} ^nx_ky_k\right)^2
=\left(\sum _{k=1} ^nx_k ^2\right)\left(\sum _{k=1} ^ny_k
^2\right)-\sum _{1\leq k <j\leq n}(x_ky_j-x_jy_k)^2$$ (Theorem
\ref{thm:lagranges-id}), since $\sum _{1\leq k <j\leq
n}(x_ky_j-x_jy_k)^2\geq 0$.
\end{f-pf}
\begin{s-pf}
Put $\dis{a = \sum _{k = 1} ^n x_k ^2}$, $\dis{b = \sum _{k = 1} ^n
x_ky_k }$, and $\dis{c = \sum _{k = 1} ^n y_k ^2}$. Consider the
quadratic polynomial
$$at^2 +
bt + c= t^2\sum _{k = 1} ^n x_k ^2 - 2t \sum _{k = 1} ^n x_ky_k +
\sum _{k = 1} ^n y_k ^2 = \sum _{k = 1} ^n (tx_k - y_k)^2  \geq 0,
$$ where the inequality follows because a sum of squares of real numbers is being summed. Thus this  quadratic polynomial is positive for all
real $t$, so it must have complex roots. Its discriminant $b^2 -
4ac$ must be non-positive, from where we gather
$$4\left(\sum _{k = 1} ^n x_ky_k\right)^2 \leq 4\left(\sum _{k = 1} ^n x_k
^2 \right)\left(\sum _{k = 1} ^n y_k ^2 \right),   $$which gives the
inequality
\end{s-pf}
For our third proof of the CBS Inequality we need the following
lemma.

\begin{lem}\label{lem:for-CBS}
For $(a,b,x,y)\in \BBR^4$ with $x>0$ and $y>0$ the following
inequality holds: $$\dfrac{a^2}{x}+\dfrac{b^2}{y}\geq
\dfrac{(a+b)^2}{x+y}.
$$Equality holds if and only if $\dfrac{a}{x} = \dfrac{b}{y}$.
\end{lem}
\begin{pf}
Since the square of a real number is always positive, we have
$$\begin{array}{lll}(ay-bx)^2\geq 0& \implies & a^2y^2-2abxy+b^2x^2\geq 0\\ & \implies & a^2y(x+y)+b^2x(x+y) \geq
(a+b)^2xy\\
& \implies & \dfrac{a^2}{x}+\dfrac{b^2}{y}\geq
\dfrac{(a+b)^2}{x+y}.\\
\end{array}$$ Equality holds if and only if the first inequality is $0$.\end{pf}
\begin{rem}
Iterating the result on Lemma \ref{lem:for-CBS},
$$\dfrac{a_1 ^2}{b_1} + \dfrac{a_2 ^2}{b_2} + \cdots + \dfrac{a_n ^2}{b_n} \geq \dfrac{(a_1+a_2+\cdots + a_n)^2}{b_1+b_2+\cdots + b_n}, $$
with equality if and only if
$\dfrac{a_1}{b_1}=\dfrac{a_2}{b_2}=\cdots = \dfrac{a_n}{b_n}.$
\end{rem}
\begin{t-pf}
By the preceding remark, we have
$$\begin{array}{lll}x_1 ^2 + x_2 ^2 + \cdots + x_n ^2 & = & \dfrac{x_1 ^2 y_1 ^2}{y_1 ^2}+ \dfrac{x_2 ^2 y_2 ^2}{y_2 ^2}+\cdots +  \dfrac{x_n ^2 y_n ^2}{y_n ^2}
\\ & \geq & \dfrac{(x_1y_1+x_2y_2+\cdots + x_ny_n)^2}{y_1 ^2+y_2 ^2 +
\cdots + y_n ^2},
\end{array}
$$and upon rearranging, CBS is once again obtained.\end{t-pf}
\begin{thm}[Minkowski's Inequality]\label{thm:minkowski-ineq}
Let $x_k, y_k$ be any real numbers. Then
$$ \left(\sum _{k=1} ^n (x_k+y_k)^2\right)^{1/2}\leq  \left(\sum _{k=1} ^n x_k ^2\right)^{1/2} +  \left(\sum _{k=1} ^n y_k ^2\right)^{1/2}. $$
\end{thm}
\begin{pf}
We have
$$ \begin{array}{lll}\sum _{k=1} ^n (x_k+y_k)^2 & = & \sum _{k=1} ^n x_k ^2+2\sum _{k=1} ^n x_ky_k      +\sum _{k=1} ^ny_k ^2\\
& \leq & \sum _{k=1} ^n x_k ^2+2\left(\sum _{k=1} ^n x_k
^2\right)^{1/2}\left(\sum _{k=1} ^ny_k ^2\right)^{1/2} +\sum _{k=1} ^ny_k ^2\\
& = & \left(\left(\sum _{k=1} ^nx_k ^2\right)^{1/2} +\left(\sum
_{k=1} ^ny_k ^2\right)^{1/2} \right)^2,\end{array}$$where the
inequality follows from the CBS Inequality.\end{pf}
\subsection*{Practice}\addcontentsline{toc}{subsection}{Practice}
\begin{multicols}{2}\columnseprule 1pt \columnsep
25pt\multicoltolerance=900
\begin{pro}
Let $x, y$ be real numbers. Then $$ 0 \leq  x < y \iff x^2 < y^2. $$
\end{pro}
\begin{pro}Let $t \geq 0$. Prove that
$$|x| \geq t \iff (x \geq t) \quad \mathrm{or}\quad (x\leq -t).  $$
\end{pro}
\begin{pro}
Let $(x, y)\in \BBR^2$. Prove that $\max (x, y)=-\min (-x, -y)$.
\end{pro}

\begin{pro}
Let $x, y , z$ be real numbers. Prove that
$$
\max (x, y, z)   =   x + y + z - \min (x, y) - \min (y, z)  - \min
(z, x) + \min (x, y , z). $$
\end{pro}

\begin{pro}Let $(x_1,x_2, \ldots, x_n)\in\BBR^n$ be such that $$ x_1 ^2+x_2 ^2+\cdots +x_n ^2=
x_1 ^3+x_2 ^3+\cdots +x_n ^3=x_1 ^4+x_2 ^4+\cdots +x_n ^4.
$$Prove that $x_k\in \{0,1\}$.
\begin{answer}
The given equalities entail t $$ \sum _{k=1} ^n (x_k ^2-x_k)^2=0.
$$A sum of squares is $0$ if and only if every term is $0$. This
gives the result.
\end{answer}
\end{pro}
\begin{pro}Let $n \geq 2$ an integer. Let $(x_1,x_2, \ldots, x_n)\in\BBR^n$ be such that $$ x_1 ^2+x_2 ^2+\cdots +x_n ^2=x_1x_2+x_2x_3+\cdots + x_{n-1}x_n+x_nx_1  .
$$Prove that $x_1=x_2=\cdots=x_n$.
\begin{answer}
The given equality entails that $$
\dfrac{1}{2}\left((x_1-x_2)^2+(x_2-x_3)^2+\cdots +
(x_{n-1}-x_n)^2+(x_n-x_1)^2\right)=0.
$$A sum of squares is $0$ if and only if every term is $0$. This
gives the result.
\end{answer}
\end{pro}

\begin{pro}
If $b>0$ and $B>0$ prove that
$$ \dfrac{a}{b}<\dfrac{A}{B}\implies \dfrac{a}{b}<\dfrac{a+A}{b+B}<\dfrac{A}{B}.
$$Further, if
$p$ and $q$ are positive integers such that
$$ \dfrac{7}{10} < \dfrac{p}{q} < \dfrac{11}{15},
$$what is the least value of $q$? \\
\begin{answer}
Since $aB<Ab$ one has $a(b+B)=ab +aB <ab+Ab = (a+A)b$ so
$\dfrac{a}{b}<\dfrac{a+A}{b+B}$. Similarly $B(a+A) = aB+AB<Ab+AB =
A(b+B)$ and so $\dfrac{a+A}{b+B}<\dfrac{A}{B}$.

\bigskip
We have $$ \dfrac{7}{10} < \dfrac{11}{15} \implies \dfrac{7}{10} <
\dfrac{18}{25} < \dfrac{11}{15} \implies \dfrac{7}{10}
<\dfrac{25}{35} < \dfrac{18}{25} < \dfrac{11}{15}.    $$ Since
$\dfrac{25}{35} = \dfrac{5}{7},$ we have $q \leq 7$. Could it be
smaller? Observe that $\dfrac{5}{6} > \dfrac{11}{15} $ and that
$\dfrac{4}{6} < \dfrac{7}{10}$. Thus by considering the cases with
denominators $q = 1, 2, 3, 4, 5, 6$, we see that no such fraction
lies in the desired interval. The smallest denominator is thus $7$.
\end{answer}
\end{pro}

\begin{pro}
Let $a<b$. Demonstrate that $$\absval{x-a}<\absval{x-b} \iff
x<\dfrac{a+b}{2}.
$$
\end{pro}
\begin{pro}
Prove that if $r \geq s \geq t$ then
$$
r^2 - s^2 + t^2  \geq (r - s + t)^2. $$
\begin{answer}  We have
$$ (r - s + t)^2 - t^2 = (r - s + t - t)(r - s + t + t) = (r - s)(r - s + 2t).$$
Since $t - s \leq 0,$ $r - s + 2t = r + s + 2(t - s) \leq r + s$ and
so
$$(r - s + t)^2 - t^2 \leq (r - s)(r + s) = r^2 - s^2$$which gives
$$(r - s + t)^2 \leq r^2 - s^2 + t^2.$$
\end{answer}
\end{pro}

\begin{pro}
Assume that $a_k, b_k, c_k, k = 1, \ldots, n$, are positive real
numbers. Shew that
$$\left(\sum _{k = 1} ^n a_kb_kc_k\right)^{4}
\leq \left(\sum _{k = 1} ^n a_k ^4\right)\left(\sum _{k = 1} ^n b_k
^4\right) \left(\sum _{k = 1} ^n c_k
^2\right)^{2}.$$\begin{answer}Using the CBS Inequality (Theorem
\ref{thm:CBS-ineq}) on $\sum _{k = 1} ^n (a_kb_k)c_k$ once we obtain
$$\sum _{k = 1} ^n a_kb_kc_k
\leq \left(\sum _{k = 1} ^n a_k ^2b_k ^2\right)^{1/2} \left(\sum _{k
= 1} ^n c_k ^2\right)^{1/2}.
$$Using CBS again on $\left(\sum _{k = 1} ^n a_k ^2b_k ^2\right)^{1/2}$ we obtain
$$
\begin{array}{lll}
\sum _{k = 1} ^n a_kb_kc_k  & \leq &
 \left(\sum _{k = 1} ^n a_k ^2 b_k ^2\right)^{1/2}
\left(\sum _{k = 1} ^n c_k ^2\right)^{1/2} \\
  & \leq & \left(\sum _{k = 1} ^n a_k ^4\right)^{1/4}
\left(\sum _{k = 1} ^n b_k ^4\right)^{1/4}
\left(\sum _{k = 1} ^n c_k ^2\right)^{1/2}, \\
\end{array}
$$which gives the required inequality.
\end{answer}
\end{pro}
\begin{pro}
Prove that for integer $n>1$, $$ n!<\left(\dfrac{n+1}{2}\right)^n.
$$
\begin{answer}
This follows directly from the AM-GM Inequality applied to
$1,2,\ldots , n$:
$$ n!^{1/n} (1\cdot 2 \cdots n)^{1/n}< \dfrac{1+2+\cdots + n}{n} = \dfrac{n+1}{2},
$$where strict inequality follows since the factors are unequal for
$n>1$.
\end{answer}
\end{pro}
\begin{pro}
Prove that for integer $n>2$, $$ n^{n/2}<n!.
$$
\begin{answer}
First observe that for integer $k$, $1< k < n$,\quad
$k(n-k+1)=k(n-k)+k>1(n-k)+k= n$. Thus
$$ n!^2 = (1\cdot n)(2\cdot (n-1))(3\cdot (n-2))\cdots ((n-1)\cdot 2)(n\cdot 1)>n\cdot n\cdot n\cdots n=n^n.  $$
\end{answer}
\end{pro}

\begin{pro}\label{pro:sum-squares-ineq}
Prove that $\forall (a, b, c)\in\BBR^3$,
$$ a^2+b^2+c^2\geq ab + bc +ca. $$
\begin{answer}
Assume without loss of generality that  $a\geq b \geq c$. Then
$a\geq b \geq c$ is similarly sorted as itself, so by the
Rearrangement Inequality
$$ a^2 + b^2 + c^2=aa+bb+cc \geq ab + bc + ca. $$
This also follows directly from the identity
$$ a^2+b^2+c^2-ab-bc-ca = \left(a-\dfrac{b+c}{2}\right)^2+\dfrac{3}{4}\left(b-c\right)^2. $$
One can also use the AM-GM Inequality thrice:
$$a^2+b^2\geq 2ab; \quad  b^2+c^2\geq 2bc;\quad c^2+a^2\geq 2ca,
$$and add.
\end{answer}
\end{pro}
\begin{pro}
Prove that $\forall (a, b, c)\in\BBR^3$, with $a\geq 0$, $b \geq 0$,
$c\geq 0$, the following inequalities hold:
$$ a^3+b^3+c^3\geq \max(a^2b + b^2c +c^2a, a^2c+b^2a+c^2b), $$
$$ a^3+b^3+c^3\geq 3abc,  $$ $$ a^3+b^3+c^3\geq \dfrac{1}{2}\left(a^2(b+c) + b^2(c+a) +
c^2(a+b)\right).  $$
\begin{answer}
Assume without loss of generality that  $a\geq b \geq c$. Then
$a\geq b \geq c$ is similarly sorted as $a^2\geq b^2 \geq c^2$, so
by the Rearrangement Inequality
$$  a^3+b^3+c^3= aa^2 + bb^2 + cc^2 \geq a^2b + b^2c + c^2a,$$
and
$$  a^3+b^3+c^3= aa^2 + bb^2 + cc^2 \geq a^2c + b^2a + c^2b.$$
Upon adding $$  a^3+b^3+c^3= aa^2 + bb^2 + cc^2 \geq
\dfrac{1}{2}\left(a^2(b+c) + b^2(c+a) + c^2(a+b)\right).$$Again, if
$a\geq b \geq c$ then $$ab\geq ac \geq bc, $$ thus
$$ a^3+b^3+c^3=  \geq a^2b + b^2c + c^2a = (ab)a+(bc)b+(ac)c\geq (ab)c+(bc)a+(ac)b=3abc.   $$
This last inequality also follows directly from the AM-GM
Inequality, as
$$ (a^3b^3c^3)^{1/3}\leq \dfrac{a^3+b^3+c^3}{3}, $$or from the
identity
$$ a^3+b^3+c^3-3abc= (a+b+c)(a^2+b^2+c^2-ab-bc-ca), $$and the
inequality of problem \ref{pro:sum-squares-ineq}.
\end{answer}
\end{pro}
\begin{pro}[Chebyshev's Inequality] Given sets of real numbers $\{a_1, a_2, \ldots , a_n\}$ and $\{b_1,
b_2, \ldots , b_n\}$ prove that $$\dfrac{1}{n} \sum _{1\leq k\leq
n}\check{a}_k\hat{b}_k \leq \left(\dfrac{1}{n}\sum _{1\leq k \leq n}
a_k\right)\left(\dfrac{1}{n}\sum _{1\leq k \leq n}b_k \right)\leq
\dfrac{1}{n}\sum _{1\leq k \leq n}\hat{a}_k\hat{b}_k.
$$
\begin{answer}
We apply $n$ times the Rearrangement Inequality
$$\begin{array}{lllll}\check{a}_1\hat{b}_1+
\check{a}_2\hat{b}_2+\cdots + \check{a}_n\hat{b}_n &  \leq  &
a_1b_1+a_2b_2+ \cdots + a_nb_n & \leq & \hat{a}_1\hat{b}_1+
\hat{a}_2\hat{b}_2+\cdots + \hat{a}_n\hat{b}_n\\
\check{a}_1\hat{b}_1+ \check{a}_2\hat{b}_2+\cdots +
\check{a}_n\hat{b}_n &  \leq  & a_1b_2+a_2b_3+ \cdots + a_nb_1 &
\leq & \hat{a}_1\hat{b}_1+
\hat{a}_2\hat{b}_2+\cdots + \hat{a}_n\hat{b}_n\\
\check{a}_1\hat{b}_1+ \check{a}_2\hat{b}_2+\cdots +
\check{a}_n\hat{b}_n &  \leq  & a_1b_3+a_2b_4+ \cdots + a_nb_2 &
\leq & \hat{a}_1\hat{b}_1+
\hat{a}_2\hat{b}_2+\cdots + \hat{a}_n\hat{b}_n\\
& & \vdots & & \\
\check{a}_1\hat{b}_1+ \check{a}_2\hat{b}_2+\cdots +
\check{a}_n\hat{b}_n &  \leq  & a_1b_n+a_2b_1+ \cdots + a_nb_{n-1} &
\leq & \hat{a}_1\hat{b}_1+
\hat{a}_2\hat{b}_2+\cdots + \hat{a}_n\hat{b}_n\\
\end{array}$$
Adding we obtain the desired inequalities.

\end{answer}
\end{pro}
\begin{pro}
If $x > 0$, from
$$\sqrt{x + 1} - \sqrt{x} = \frac{1}{\sqrt{x + 1} + \sqrt{x}},$$prove that
$$\frac{1}{2\sqrt{x + 1}} < \sqrt{x + 1} - \sqrt{x} < \frac{1}{2\sqrt{x}}.$$
Use this to prove that if  $n > 1$ is a positive integer, then
$$2\sqrt{n + 1} - 2 < 1 + \frac{1}{\sqrt{2}} + \frac{1}{\sqrt{3}} + \cdots + \frac{1}{\sqrt{n}}  < 2\sqrt{n} - 1$$
\end{pro}
\begin{pro}
If $0 < a \leq b$, shew that
$$\frac{1}{8}\cdot\frac{(b - a)^2}{b} \leq \frac{a + b}{2} - \sqrt{ab} \leq \frac{1}{8}\cdot\frac{(b - a)^2}{a} $$
\begin{answer}
Use the fact that $(b - a)^2 = (\sqrt{b} - \sqrt{a})^2(\sqrt{b} +
\sqrt{a})^2$.
\end{answer}
\end{pro}
\begin{pro}
Shew that $$\frac{1}{2}\cdot\frac{3}{4}\cdot\frac{5}{6}\cdots
\frac{9999}{10000} < \frac{1}{100}.$$ \begin{answer} Let
$$A = \frac{1}{2}\cdot\frac{3}{4}\cdot\frac{5}{6}\cdots \frac{9999}{10000} $$
and
$$ B =\frac{2}{3}\cdot\frac{4}{5}\cdot\frac{6}{7}\cdots\frac{10000}{10001}.$$

\bigskip

Clearly, $x^2 - 1 < x^2$  for all real numbers $x$. This implies
that
$$\frac{x - 1}{x} < \frac{x}{x + 1}$$ whenever these four quantities are
positive. Hence
$${\everymath{\displaystyle}\begin{array}{ccc}
{1}/{2} & < & {2}/{3} \\
{3}/{4} & < & {4}/{5} \\
{5}/{6} & < & {6}/{7} \\
\vdots & \vdots & \vdots \\
{9999}/{10000} & < & {10000}/{10001} \\
\end{array} }$$
As all the numbers involved are positive, we multiply both columns
to obtain
$$\frac{1}{2}\cdot\frac{3}{4}\cdot\frac{5}{6}\cdots \frac{9999}{10000}
<
\frac{2}{3}\cdot\frac{4}{5}\cdot\frac{6}{7}\cdots\frac{10000}{10001},$$
or $A < B.$ This yields $A^2 = A\cdot A < A\cdot B.$ Now
$$A\cdot B = \frac{1}{2}\cdot\frac{2}{3}\cdot\frac{3}{4}\cdot\frac{4}{5}\cdot\frac{5}{6}\cdot\frac{6}{7}
\cdot\frac{7}{8}\cdots\frac{9999}{10000}\cdot\frac{10000}{10001} =
\frac{1}{10001},$$ and consequently, $A^2 < A\cdot B = 1/10001.$ We
deduce that $A < 1/\sqrt{10001} < 1/100.$
\end{answer}
\end{pro}
\begin{pro}
Prove that for all $x>0$,
$$\sum _{k=1} ^n \dfrac{1}{(x+k)^2}<\dfrac{1}{x}-\dfrac{1}{x+n}.  $$
\begin{answer}
Observe that  for $k\geq 1$, $(x+k)^2>(x+k)(x+k-1)$ and so $$
\dfrac{1}{(x+k)^2} <\dfrac{1}{(x+k)(x+k-1)}  =
\dfrac{1}{x+k-1}-\dfrac{1}{x+k}.
$$Hence
$$\begin{array}{lll}\dfrac{1}{(x+1)^2} + \dfrac{1}{(x+2)^2} + \dfrac{1}{(x+3)^2} + \cdots + \dfrac{1}{(x+n-1)^2} + \dfrac{1}{(x+n)^2}
&  <  & \dfrac{1}{x(x+1)} +
\dfrac{1}{(x+1)(x+2)}+\dfrac{1}{(x+2)((x+3))}\\ & &  + \cdots +
\dfrac{1}{(x+n-2)(x+n-1)} +
\dfrac{1}{(x+n-1)(x+n)}\\
& = &
\dfrac{1}{x}-\dfrac{1}{x+1}+\dfrac{1}{x+1}-\dfrac{1}{x+2}+\dfrac{1}{x+2}-\dfrac{1}{x+3}\\
& & +\cdots
+\dfrac{1}{x+n-2}-\dfrac{1}{x+n-1}+\dfrac{1}{x+n-1}-\dfrac{1}{x+n}\\
& = & \dfrac{1}{x}-\dfrac{1}{x+n}.
 \end{array}$$
\end{answer}
\end{pro}
\begin{pro}
Let $x_i\in\BBR$ such that $\sum _{i=1}\absval{x_i}=1$ and $\sum
_{i=1}x_i=0$. Prove that $$ \absval{\sum _{i=1} ^n \dfrac{x_i}{i}}
\leq \dfrac{1}{2}\left(1-\dfrac{1}{n}\right).
$$
\begin{answer}
For $1 \leq i \leq n$, we have
$$\absval{\dfrac{2}{i}-1-\dfrac{1}{n}} \leq 1 - \dfrac{1}{n} \iff \left(\dfrac{2}{i}-\left(1+\dfrac{1}{n}\right)\right)^2\leq \left(1-\dfrac{1}{n}\right)^2
\iff
\dfrac{4}{i^2}-\dfrac{4}{i}\left(1+\dfrac{1}{n}\right)+\dfrac{4}{n}\leq
0 \iff \dfrac{(i-n)(i-1)}{i^2n}\leq 0.$$Thus
$$\absval{\sum _{i=1} ^n \dfrac{x_i}{i}} = \dfrac{1}{2}\absval{\sum _{i=1} ^n \left(\dfrac{2}{i}-\left(1+\dfrac{1}{n}\right)\right)x_i},
$$as $\sum
_{i=1}x_i=0$. Now
$$ \absval{\sum _{i=1} ^n \left(\dfrac{2}{i}-\left(1+\dfrac{1}{n}\right)\right)x_i} \leq
\sum _{i=1} ^n \absval{ \dfrac{2}{i}-1-\dfrac{1}{n}}\absval{x_i}\leq
\left(1-\dfrac{1}{n}\right)\sum _{i=1} ^n\absval{x_i} =
\left(1-\dfrac{1}{n}\right).
$$
\end{answer}

\end{pro}
\begin{pro} Let $n$ be a strictly positive integer.
Let $x_i \geq 0$. Prove that $$ \prod _{k=1} ^n (1+x_k) \geq 1 +
\sum _{k=1} ^n x_k.
$$When does equality hold?
\begin{answer}
Expanding the product
$$\prod _{k=1} ^n (1+x_k) = 1 +
\sum _{k=1} ^n x_k  + \sum _{1 \leq i <j\leq n} ^n x_ix_j+\cdots
\geq  1 + \sum _{k=1} ^n x_k,
$$since the $x_k\geq 0$. When $n=1$ equality is obvious. When $n>1$
equality is achieved when $\sum _{1 \leq i <j\leq n} ^n x_ix_j=0$.
\end{answer}
\end{pro}

\begin{pro}[Nesbitt's Inequality] Let $a, b, c$ be strictly positive
real numbers. Then $$
\dfrac{a}{b+c}+\dfrac{b}{c+a}+\dfrac{c}{a+b}\geq \dfrac{3}{2}.
$$
\begin{answer}
Assume $a\geq b \geq c$. Put $s=a+b+c$. Then $$-a \leq -b \leq -c
\implies s-a\leq s-b\leq s-c\implies \dfrac{1}{s-a}\geq
\dfrac{1}{s-b}\geq \dfrac{1}{s-c}$$ and so the sequences $a, b, c$
and $\dfrac{1}{s-a},\dfrac{1}{s-b}, \dfrac{1}{s-c}$ are similarly
sorted. Using the Rearrangement Inequality twice: $$ \dfrac{a}{s-a}+
\dfrac{b}{s-b}+\dfrac{c}{s-c}\geq
\dfrac{a}{s-c}+\dfrac{b}{s-a}+\dfrac{c}{s-b}; \qquad \dfrac{a}{s-a}+
\dfrac{b}{s-b}+\dfrac{c}{s-c}\geq
\dfrac{a}{s-b}+\dfrac{b}{s-c}+\dfrac{c}{s-a}.
$$Adding these two inequalities
$$ 2\left( \dfrac{a}{s-a}+
\dfrac{b}{s-b}+\dfrac{c}{s-c}\right) \geq
\dfrac{b+c}{s-a}+\dfrac{c+a}{s-b}+\dfrac{c+a}{s-c},
$$whence
$$ 2\left( \dfrac{a}{b+c}+
\dfrac{b}{c+a}+\dfrac{c}{a+b}\right) \geq 3,
$$from where the result follows.
\end{answer}
\end{pro}
\begin{pro}
Let $a, b, c$ be positive real numbers. Prove that $$
(a+b)(b+c)(c+a)\geq 8abc.
$$
\begin{answer}
From the AM-GM Inequality, $$ a+b \geq 2\sqrt{ab}; \quad  b+c \geq
2\sqrt{bc}; c+a \geq 2\sqrt{ca}, $$and the desired inequality
follows upon multiplication of these three inequalities.
\end{answer}
\end{pro}
\begin{pro}[IMO, 1978]Let $a_k$ be a sequence of pairwise distinct
positive integers. Prove that $$\sum _{k=1} ^n \dfrac{a_k}{k^2}\geq
\sum _{k=1} ^n \dfrac{1}{k}.
$$
\begin{answer}
By the Rearrangement inequality
$$\sum _{k=1} ^n \dfrac{a_k}{k^2}\geq
\sum _{k=1} ^n \dfrac{\check{a}_k}{k^2}\geq \sum _{k=1} ^n
\dfrac{1}{k},
$$as $\check{a}_k \geq k$, the $a$'s being pairwise distinct
positive integers.
\end{answer}
\end{pro}
\begin{pro}[Harmonic
Mean-Geometric Mean Inequality] Let $x_i>0$ for $1 \leq i \leq n$.
Then
$$ \dfrac{n}{\dfrac{1}{x_1}+\dfrac{1}{x_2}+\cdots + \dfrac{1}{x_n}} \leq (x_1x_2\cdots x_n)^{1/n},  $$
with equality iff $x_1=x_2=\cdots =x_n$.
\begin{answer}By the AM-GM Inequality,
$$ \left(\dfrac{1}{x_1}\dfrac{1}{x_2}\cdots \dfrac{1}{x_n}\right)^{1/n}\leq \dfrac{\dfrac{1}{x_1}+\dfrac{1}{x_2}+\cdots + \dfrac{1}{x_n}}{n},  $$
whence the inequality.
\end{answer}
\end{pro}
\begin{pro}[Arithmetic
Mean-Quadratic Mean Inequality] Let $x_i\geq 0$ for $1 \leq i \leq
n$. Then
$$ \dfrac{x_1+x_2+\cdots + x_n}{n} \leq \left(\dfrac{x_1 ^2+x_2 ^2+\cdots + x_n ^2}{n}\right)^{1/2},  $$
with equality iff $x_1=x_2=\cdots =x_n$.
\begin{answer}By the CBS Inequality,
$$ \left(1\cdot x_1 + 1\cdot x_2 + \cdots + 1\cdot x_n\right)^2\leq \left(1^2+1^2+ \cdots + 1^2\right)\left(x_1 ^2 + x_2 ^2 + \cdots + x_n ^2\right),
$$which gives the desired inequality.
\end{answer}
\end{pro}

\begin{pro} Given a set of real numbers $\{a_1, a_2, \ldots , a_n\}$
prove that there is an index $m\in \{0,1,\ldots , n\}$ such that $$
\absval{\sum _{1\leq k \leq m} a_k - \sum _{m<k \leq n}a_k}\leq \max
_{1\leq k \leq n}\absval{a_k}.$$If $m=0$ the first sum is to be
taken as $0$ and if $m=n$ the second one will be taken as $0$.
\begin{answer}
Put $$  T_m = \sum _{1\leq k \leq m} a_k - \sum _{m<k \leq
n}a_k.$$Clearly $T_0 = -T_n$. Since the sequence $T_0, T_1, \ldots ,
T_n$ changes signs, choose an index $p$ such that $T_{p-1}$ and
$T_p$ have different signs. Thus either $T_{p-1}-T_p = 2|a_p|$ or
$T_p-T_{p-1}=2|a_p|$.  We claim that
$$ \min \left(\absval{T_{p-1}}, \absval{T_p}\right)=  \leq \max _{1\leq k \leq n}\absval{a_k}. $$

For, if contrariwise both $\absval{T_{p-1}} >  \max _{1\leq k \leq
n}\absval{a_k}$ and $\absval{T_p }>  \max _{1\leq k \leq
n}\absval{a_k}$, then $2|a_p|=|T_{p-1}-T_p|>2\max _{1\leq k \leq
n}\absval{a_k}$, a contradiction.
\end{answer}
\end{pro}
\end{multicols}




\Closesolutionfile{probsemans}
\appendix
\renewcommand{\chaptername}{Appendix}
\chapter{Answers, Hints, and Solutions}

   \input{probsemans1}
\printindex


\end{document}
